\section{A General Framework}
\label{sec:a-general-framework}

We now develop a general framework for discussing
algebraically-indexed types and their relational interpretations. We
have chosen a particular order for our presentation: presenting the
semantics of types before we present the syntax of programs and their
typing relation. We have deliberately selected this order to emphasise
the importance of establishing what types \emph{mean} over the
language used to write programs.

We first present, in \autoref{sec:algebraically-indexed-types}, the
syntax of the language of algebraically-indexed types. Our general
framework is parameterised by a choice of multi-sorted algebraic
theory to be used for indexing types, and the collections of primitive
types and
operations. %FIXME: refer forwards to the names of the parameters
We define substitution of index terms, and what it means for two types
to be equal. Equality on types is generated from the chosen
multi-sorted algebraic theory.

We next present the semantics of types in
\autoref{sec:semantics-algebraically-indexed-types}. Types are first
interpreted without reference to the indexing terms: the
\emph{index-erasure} interpretation. The index-erasure interpretation
provides an ``underlying'' semantics to each type (and term), which we
think of as an abstraction of the actual operational behaviour of a
term. By erasing all the index terms, we can immediately see that the
run-time behaviour of programs cannot depend on index terms. We then
define the \emph{relational interpretation}, which interprets every
type as a binary relation on its index-erasure interpretation. The
relational interpretation acts as ``commentary'' on the underlying
interpretation, constraining the index-erasure interpretations that
are allowed. In particular, we use the relational interpretation to
enforce the invariance properties that we indentified in
\autoref{sec:motivating-examples}.

Throughout this section, we use Kennedy's original units of measure
system as a running example in order to provide a concrete example of
our general framework, and also to demonstrate that we have in fact
generalised Kennedy's original development.

\subsection{Algebraically-Indexed Types}
\label{sec:algebraically-indexed-types}

The indexing terms and types of an instantiation of our general
framework are derived from the following data:
\begin{enumerate}
\item A collection $\SortSet$ of index sorts. We will use the
  meta-syntactic variables $s,s_1,s_2,...$ and so on to stand for
  arbitrary sorts taken from $\SortSet$.
\item A collection $\IndexOpSet$ of index operations, with a function
  $\indexOpArity : \IndexOpSet \to \SortSet^* \times \SortSet$ (we use
  the notation $A^*$ to denote the set of lists of elements of some
  set $A$).
\item A collection $\PrimTypeSet$ of primitive types, with a function
  $\primTyArity : \PrimTypeSet \to \SortSet^*$, describing the sorts
  of the arguments of each primitive type.
\end{enumerate}

\begin{example}[Origin Invariance]
  \fixme{Do this, then update the units of measure example}
\end{example}

\begin{example}[Units of Measure]
  To instantiate our general framework as Kennedy's original units
  of measure system, we would take the following settings. For the
  collection of sorts we take $\SortSet = \{ \mathsf{unit} \}$. For
  the collection of index operations, we take $\mathit{IndexOp} = \{1,
  -\cdot-, -^{-1}\}$, with:
  \begin{eqnarray*}
    \indexOpArity{(1)} &=& ([],\mathsf{unit}) \\
    \indexOpArity{(-\cdot-)} & = & ([\mathsf{unit},\mathsf{unit}],\mathsf{unit}) \\
    \indexOpArity{(-^{-1})} & = & ([\mathsf{unit}],\mathsf{unit})
  \end{eqnarray*}
  The intended interpretation of these operations is group unit, group
  multiplication and group inverse, respectively. When we discuss
  equational theories for index expressions in
  \autoref{sec:type-equality} below, we will impose the abelian group
  laws on these operations.

  The primitive types for the units of measure example are taken to be
  $\PrimTypeSet = \{ \mathsf{num} \}$, with
  $\primTyArity(\mathsf{num}) = [\mathsf{unit}]$.
\end{example}

We assume a countably infinite collection of index variable names $i,
i_1, i_2,$ and so on. \emph{Index contexts} $\Delta = i_1 \mathord:
s_1, ..., i_n \mathord: s_n$ are lists of index expression
variable/sort pairs, such that all the index variable names are
distinct.

\begin{figure*}[t]
  \centering
  \textbf{Index contexts}\\

  $\Delta = i_1 \mathord: s_1, ..., i_n \mathord: s_n$, where each $s_i \in \SortSet$ and no variable name is repeated.

  \bigskip

  \textbf{Well-sorted index expressions}
  \begin{mathpar}
    \inferrule* [right=IVar]
    {i : s \in \Delta}
    {\Delta \vdash i : s}
    
    \inferrule* [right=IOp]
    {\indexOp{f} \in \mathit{IndexOp} \\
      \indexOpArity(\indexOp{f}) = ([s_1,...,s_n], s) \\
      \{\Delta \vdash e_j : s_j\}_{1 \leq j \leq n}}
    {\Delta \vdash \indexOp{f}(e_1, ..., e_n) : s}
  \end{mathpar}

  \bigskip

  \textbf{Well-indexed types}
  \begin{mathpar}
    \inferrule* [right=TyPrim]
    {\tyPrimNm{X} \in \mathit{PrimType} \\\\
      \primTyArity(\tyPrimNm{X}) = [s_1,...,s_n] \\
      \{\Delta \vdash e_j : s_j\}_{1\leq j \leq n}}
    {\Delta \vdash \tyPrim{X}{e_1,...,e_n} \isType}

    \inferrule* [right=TyUnit]
    { }
    {\Delta \vdash \tyUnit \isType}

    \inferrule* [right=TyArr]
    {\Delta \vdash A \isType \\ \Delta \vdash B \isType}
    {\Delta \vdash A \tyArr B \isType}

    \inferrule* [right=TyTuple]
    {\Delta \vdash A \isType \\ \Delta \vdash B \isType}
    {\Delta \vdash A \tyProduct B \isType}

    \inferrule* [right=TySum]
    {\Delta \vdash A \isType \\ \Delta \vdash B \isType}
    {\Delta \vdash A + B \isType}
    
    \inferrule* [right=TyForall] %FIXME: macroize forall
    {\Delta, i : s \vdash A \isType}
    {\Delta \vdash \forall i \mathord: s. A \isType}
  \end{mathpar}
  \caption{Index expressions and types}
  \label{fig:indexes-and-types}
\end{figure*}

Given the above data, the rules in \autoref{fig:indexes-and-types}
generate two judgements: well-sorted index terms $\Delta \vdash e : s$
and well-indexed types $\Delta \vdash A \isType$. Since index
variables may appear in types, types are judged to be well-indexed
with respect to an index context $\Delta$. The rules for well-sorted
index terms are particularly simple: either an index term is a
variable that appears in the context (rule \TirName{IVar}), or it is
an application of an index operation taken from $\IndexOpSet$ to other
index terms (rule \TirName{IOp}). The rules for well-indexed types
include the usual rules for constructing types of the simply-typed
$\lambda$-calculus with unit, sum and tuple types (rules
\TirName{TyUnit}, \TirName{TyArr}, \TirName{TyTuple} and
\TirName{TySum}). The rule \TirName{TyPrim} allows us to form, from a
a primitive type $\tyPrimNm{X}$ and appropriately sorted index terms
$e_1,...,e_n$, the well-indexed type $\tyPrim{X}{e_1,...,e_n}$. The
rule \TirName{TyForall} permits the formation of universally
quantified types, where the universal quantification ranges over all
index terms of some sort.

\subsubsection{Simultaneous Substitutions of Index Terms}
\label{sec:simultaneous-substitution}

It will be technically convenient to express substitution of index
terms in our framework in terms of simultaneous substitutions.  Given
a pair of index contexts $\Delta$ and $\Delta' = i_1 \mathord: s_1,
..., i_n \mathord: s_n$, a \emph{simultaneous substitution} $\Delta
\vdash \sigma \Rightarrow \Delta'$ is a sequence of terms $\sigma =
(e_1,...,e_n)$ such that for each $j$ such that $1 \leq j \leq n$,
$\Delta \vdash e_j : s_j$. Given a simultaneous substitution $\Delta
\vdash \sigma = (e_1,...,e_n) \Rightarrow \Delta'$ and a variable $i_j
\mathord: s_j$ in $\Delta'$, we write $\sigma(i_j)$ for the index term
$e_j$.

Given a context $\Delta = i_1\mathord:s_1,...,i_n\mathord:s_n$, and a
variable/type pair $i\mathord:s$ such that $i$ does not appear in
$\Delta$, we define the \emph{projection} simultaneous substitution
$\pi_{i\mathord:s} : \Delta,i\mathord:s \Rightarrow \Delta$ as
$\pi_{i\mathord:s} = (i_1,...,i_n)$. The subscript on
$\pi_{i\mathord:s}$ is the variable/type pair that is being discarded.

For a simultaneous substitution $\Delta \vdash \sigma \Rightarrow
\Delta'$ where $\Delta' = i_1\mathord:s_1,...,i_n\mathord:s_n$ and a
variable/type pair $i\mathord:s$ such that $i$ does not appear in
either $\Delta$ or $\Delta'$, then we can form the \emph{lifted}
simultaneous substitution $\Delta,i\mathord:s \vdash
\sigma_{i\mathord:s} = (\sigma(i_1), ..., \sigma(i_n), i) \Rightarrow
\Delta',i\mathord:s$. Note that we have implicity used the fact that
well-sortedness of index terms is preserved by addition of extra items
to the context.

Application of a simultaneous substitution $\Delta \vdash \sigma
\Rightarrow \Delta'$ to a well-sorted index expression $\Delta' \vdash
e : s$ yields a well-sorted index expression $\Delta \vdash \sigma^*e
: s$. The expression $\sigma^*e$ is defined by the following two
clauses:
\begin{mathpar}
  \sigma^*i \isDefinedAs \sigma(i)

  \sigma^*(\indexOp{f}(e_1,...,e_n)) \isDefinedAs \indexOp{f}(\sigma^*e_1, ..., \sigma^*e_n)
\end{mathpar}
Similarly, given a well-indexed type $\Delta' \vdash A \isType$, we
can apply $\sigma$ to $A$ to produce a new type $\sigma^*A$. The
application of a simultaneous substitution to a type is defined
by the following clauses:
\begin{mathpar}
  \sigma^*(\tyPrim{X}{e_1,...,e_n}) \isDefinedAs \tyPrim{X}{\sigma^*e_1,...,\sigma^*e_n}

  \sigma^*\tyUnit \isDefinedAs \tyUnit

  \sigma^*(A \tyArr B) \isDefinedAs \sigma^*A \tyArr \sigma^*B

  \sigma^*(A \tyProduct B) \isDefinedAs \sigma^*A \tyProduct \sigma^*B

  \sigma^*(A + B) \isDefinedAs \sigma^*A + \sigma^*B

  \sigma^*(\forall i\mathord:s.A) \isDefinedAs \forall i\mathord:s.\sigma_{i\mathord:s}^*A
\end{mathpar}
\begin{lemma}
  Let $\sigma : \Delta \Rightarrow \Delta'$ be a simultaneous
  substitution.
  \begin{enumerate}
  \item If $\Delta' \vdash e : s$, then $\Delta \vdash \sigma^*e : s$; and
  \item If $\Delta' \vdash A \isType$, then $\Delta \vdash \sigma^*A
    \isType$.
  \end{enumerate}
\end{lemma}

The \emph{composition} of two simultaneous substitutions $\Delta
\vdash \sigma \Rightarrow \Delta'$ and $\Delta' \vdash \sigma'
\Rightarrow \Delta''$, where $\sigma' = (e'_1,...,e'_n)$, is defined
as $\Delta \vdash \sigma' \circ \sigma \isDefinedAs (\sigma^*e'_1,
..., \sigma^*e'_n) \Rightarrow \Delta''$.

\subsubsection{Index Term Equality and Type Equality}
\label{sec:type-equality}

Much of the power of indexing types by the terms of an algebraic
theory comes from the equations of the theory, and the fact that these
equations are inherited by the types. In the units of measure example,
the well-indexed types $\tyPrim{X}{u_1\cdot u_2}$ and $\tyPrim{X}{u_2
  \cdot u_1}$ are considered equal by the type system, due to the
commutativity of the $-\cdot-$ operation in the theory of abelian
groups.

In our general framework, the equations between types are derived from
a set $\IndexAxiomSet$ of axioms $\Delta \vdash e \stackrel{ax}\equiv
e' : s$ between well-sorted index terms. The constraint that axioms
are well-sorted means that both $\Delta \vdash e : s$ and $\Delta
\vdash e' : s$ for all axioms $(\Delta \vdash e \stackrel{ax}\equiv e'
: s) \in \IndexAxiomSet$.

Given a set $\IndexAxiomSet$ of axioms, we generate the judgmental
equality between index terms $\Delta \vdash e \equiv e' : s$ by the
following rules, which allow us to use substitution instances of
axioms, and ensure that judgmental equality is a congruence relation
with respect to the index operations in $\IndexOpSet$:
\begin{mathpar}
  \inferrule*
  {(\Delta \vdash e \stackrel{ax}\equiv e' : s) \in \IndexAxiomSet \\
    \sigma : \Delta' \Rightarrow \Delta}
  {\Delta' \vdash \sigma^*e \equiv \sigma^*e' : s}

  \inferrule*
  {\{\Delta \vdash e_j \equiv e'_j : s_j\}_{1\leq j\leq n}}
  {\Delta \vdash \indexOp{f}(e_1, ..., e_n) \equiv \indexOp{f}(e'_1, ..., e'_n) : s}
\end{mathpar}
We also assume the standard reflexivity, symmetry and transitivity
rules for equality.

\begin{example}[Units of Measure]
  For the units of measure system, seen as an instantiation of our
  general framework, the set $\IndexAxiomSet$ consists of the abelian
  group axioms:
  \begin{displaymath}
    \begin{array}{l}
      u : \mathsf{unit} \vdash u \cdot 1 \stackrel{ax}\equiv u : \mathsf{unit} \\
      u_1, u_2, u_3 : \mathsf{unit} \vdash u_1 \cdot (u_2 \cdot u_3) \stackrel{ax}\equiv (u_1 \cdot u_2) \cdot u_3 : \mathsf{unit} \\
      u_1, u_2 : \mathsf{unit} \vdash u_1 \cdot u_2 \stackrel{ax}\equiv u_2 \cdot u_1 : \mathsf{unit} \\
      u : \mathsf{unit} \vdash u \cdot u^{-1} \stackrel{ax}\equiv 1 : \mathsf{unit} \\
    \end{array}
  \end{displaymath}
\end{example}

The judgmental equality $\Delta \vdash e \equiv e' : s$ on index
terms generates the judgmental equality $\Delta \vdash A \equiv B
\isType$ on types. The basic rule generating judgmental equality on
types states that two applications of primitive types are equal if
their index term arguments are equal:
\begin{displaymath}
  \inferrule*
  {\{ \Delta \vdash e_j \equiv e'_j : s_j\}_{1\leq j \leq n}}
  {\Delta \vdash \tyPrim{X}{e_1,...,e_n} \equiv \tyPrim{X}{e'_1,...,e'_n} \isType}
\end{displaymath}
The rest of the rules for judgmental equality on types ensure that it
is a congruence relation on types, and that it is an equivalence
relation (i.e.~judgmental equality on types is reflexive, symmetric
and transitive). For example, for universally quantified types we have
the following congruence rule:
\begin{displaymath}
  \inferrule*
  {\Delta, i : s \vdash A \equiv B \isType}
  {\Delta \vdash \forall i\mathord:s.A \equiv \forall i\mathord:s.B \isType}
\end{displaymath}
The congruence rules for the other type formers are similar.

The judgmental equalities on index terms and on types are preserved by
substitution:
\begin{lemma}
  Let $\sigma : \Delta \Rightarrow \Delta'$ be a simultaneous
  substitution.
  \begin{enumerate}
  \item If $\Delta' \vdash e \equiv e' : s$ then $\Delta \vdash
    \sigma^*e \equiv \sigma^*e' : s$; and
  \item If $\Delta' \vdash A \equiv B \isType$ then $\Delta \vdash
    \sigma^*A \equiv \sigma^* B \isType$.
  \end{enumerate}
\end{lemma}

We define a pair of simultaneous substitutions $\sigma, \sigma' :
\Delta \Rightarrow \Delta'$ to be judgmentally equal if their
component terms are judgmentally equal in the context $\Delta$: i.e.~
if $\Delta \vdash e_j \equiv e'_j : s_j$, for all $j$. We write
$\Delta \vdash \sigma \equiv \sigma' \Rightarrow \Delta'$ when two
simultaneous substitutions are judgmentally equal.

%%%%%%%%%%%%%%%%%%%%%%%%%%%%%%%%%%%%%%%%%%%%%%%%%%%%%%%%%%%%%%%%%%%%%%%%%%%%%%
%%%%%%%%%%%%%%%%%%%%%%%%%%%%%%%%%%%%%%%%%%%%%%%%%%%%%%%%%%%%%%%%%%%%%%%%%%%%%%
\subsection{Semantics of Algebraically-Indexed Types}
\label{sec:semantics-algebraically-indexed-types}

Having defined the language of algebraically-indexed types that we
will consider, we now turn to the denotational semantics of these
types. In \autoref{sec:index-erasure-semantics} we define an
\emph{index-erasure} interpretation of types that interprets every
well-indexed type as a set, ignoring the indexing by algebraic
terms. In the index erasure semantics, the semantics of a well-indexed
type $\tyPrim{X}{e_1,...,e_n}$ is determined solely by the primitive
type $\tyPrimNm{X}$ and not by the index terms $e_1,...,e_n$. Building
on the index-erasure semantics, we define in
\autoref{sec:relational-semantics} the \emph{relational}
interpretation of types.

% For both of our semantics of types, we state and prove two properties:
% types that are syntactically equal have equal denotations, and that
% substitution of index terms is interpreted via composition. These
% properties ensure that our semantics of types is well-behaved.

\subsubsection{The Index-Erasure Interpretation of Types}
\label{sec:index-erasure-semantics}

For each primitive type $\tyPrimNm{X} \in \PrimTypeSet$, we assume
that $\tyPrimNm{X}$ is assigned a set $\tyPrimSem{\tyPrimNm{X}}$. We
extend this to assign a set to every well-formed type by induction on
the type structure by the following clauses.
\begin{displaymath}
  \begin{array}{@{\hspace{0.2em}}c@{\hspace{0em}}c}
    \begin{array}{l}
      \tySem{\tyUnit} \isDefinedAs \{*\} \\
      \tySem{A \tyProduct B} \isDefinedAs \tySem{A} \times \tySem{B} \\
      \tySem{A \tyArr B} \isDefinedAs \tySem{A} \to \tySem{B} \\
    \end{array}
    &
    \begin{array}{l}
      \tySem{A + B} \isDefinedAs \tySem{A} + \tySem{B} \\
      \tySem{\tyPrim{X}{e_1,...,e_n}} \isDefinedAs \tyPrimSem{\tyPrimNm{X}} \\
      \tySem{\forall i\mathord:s. A} \isDefinedAs \tySem{A}
    \end{array}
  \end{array}
\end{displaymath}
The interpretation of the unit type is a chosen one element set, and
the interpretations of the function, tuple and sum types is simply by
the corresponding construction on sets. The interpretation of the
universal quantifier simply ignores the indexing: the interpretation
of the type $\forall i\mathord:s.A$ is exactly the interpretation of
the type $A$.

\begin{example}[Units of Measure] Kennedy's original units of measure
  system seen as an instantiation of our system uses the assignment
  $\tyPrimSem{\mathsf{unit}} \isDefinedAs \mathbb{Q}$, where
  $\mathbb{Q}$ is the set of rational numbers.
\end{example}

Since the index-erasure interpretation has completely ignores the
index expressions and type equality is defined as an extension of
index equality, it is straightforward to prove that equal types have
equal denotations when interpreted in the index-erasure semantics, and
that substitution of index terms has no effect on the index-erasure
interpretation of types:
\begin{lemma}\label{lem:tyeqsubst-erasure}
  \begin{enumerate}
  \item If $\Delta \vdash A \equiv B \isType$ then $\tySem{A} =
    \tySem{B}$; and
  \item If $\Delta \vdash A \isType$ and $\sigma : \Delta' \Rightarrow
    \Delta$, then $\tySem{\sigma^*A} = \tySem{A}$.
  \end{enumerate}
\end{lemma}

\subsubsection{The Relational Interpretation of Types}
\label{sec:relational-semantics}

We now define the relational semantics of each well-indexed type
$\Delta \vdash A \isType$ as some binary relation on the index-erasure
interpretation of $A$. For a set $X$, we write $\Rel(X)$ for the set
of binary relations $R \subseteq X \times X$ on $X$.

For the unit, tuple, sum and function types the we will define the
relational interpretation as a standard logical relation. The
relational interpretations of primitive types with index arguments and
the universally quantified types requires an interpretation of index
contexts. We assign interpretations to index contexts in terms of
\emph{relational environments}, which we now define.

A relation environment for a context $\Delta$ is a
(dependently typed) function that, for each primitive type
$\tyPrimNm{X}$ and instantiation of its arguments $\sigma$, assigns a
binary relation $\rho\ \tyPrimNm{X}\ \sigma$ on the index-erasure
interpretation of $\tyPrimNm{X}$:
\begin{displaymath}
  \rho : (\tyPrimNm{X} \in \PrimTypeSet) \to (\Delta \Rightarrow \primTyArity(\tyPrimNm{X})) \to \Rel(\tyPrimSem{\tyPrimNm{X}})
\end{displaymath}
Every relation environment must respect index term equality: for any
$\tyPrimNm{X} \in \PrimTypeSet$ and pair of simultaneous substitutions
$\sigma : \Delta \Rightarrow \primTyArity(\tyPrimNm{X})$ and $\sigma'
: \Delta \Rightarrow \primTyArity(\tyPrimNm{X})$ such that $\Delta
\vdash \sigma \equiv \sigma' \Rightarrow \primTyArity(\tyPrimNm{X})$, % FIXME: make sure the notation for ss equality is consistent
then $\rho\ \tyPrimNm{X}\ \sigma = \rho\ \tyPrimNm{X}\ \sigma'$.

% FIXME: an example?

We write $\mathrm{RelEnv}(\Delta)$ for the set of all relation
environments for a context $\Delta$. Given a relational environment
$\rho \in \mathrm{RelEnv}(\Delta')$ and a simultaneous substitution
$\sigma : \Delta \Rightarrow \Delta'$, we can derive the composed
relational environment (with an abuse of notation) $\rho \circ \sigma
\in \mathrm{RelEnv}(\Delta)$ as $(\rho \circ \sigma)\ \tyPrimNm{X}\
\sigma' = \rho\ \tyPrimNm{X}\ (\sigma \circ \sigma')$.

% FIXME: fix the notation here for interpretation of index contexts:
% should use some sort of semantic brackets

As we stated above, index contexts $\Delta$ are interpreted as certain
sets of relation environments, i.e.~ subsets of
$\mathrm{RelEnv}(\Delta)$. Depending on the primitive operations that
we assume for our system, we can have different sets of relational
environments that enforce different invariants. For the general
framework, we assume that, for each index context $\Delta$, we are
given a set of relation environments $\relEnv{E}(\Delta)$. Note that
this assignment of sets of relation environments to index contexts is
not assumed to be compositional: it is not necessarily the case that
the set of relational environments $\relEnv{E}(\Delta_1,\Delta_2)$ is
defined in terms of $\relEnv{E}(\Delta_1)$ and
$\relEnv{E}(\Delta_2)$. However, it must satisfy the following two
properties:
\begin{enumerate}
\item Closure under simultaneous substitution of index terms: if $\rho
  \in \relEnv{E}(\Delta)$ and $\sigma : \Delta' \Rightarrow \Delta$,
  then $\rho \circ \sigma \in \relEnv{E}(\Delta')$.
\item If we have a pair of relation environments $\rho_1 \in
  \relEnv{E}(\Delta', i\mathord:s')$ and $\rho_2 \in
  \relEnv{E}(\Delta)$, along with an simultaneous substitution $\sigma
  : \Delta' \Rightarrow \Delta$, such that the outer edges of the
  following diagram commute, for all primitive types $\tyPrimNm{X} \in
  \mathit{PrimType}$:
  %FIXME: check that these compositions are the right way round
  %FIXME: re-do this as a set of equations
  %FIXME: give a better English description of why this is needed
  \begin{displaymath}
    \xymatrix{
      {\Delta \Rightarrow \primTyArity(\tyPrimNm{X})} \ar[r]^(.45){- \circ \pi_{i\mathord:s'}} \ar[d]_{- \circ \sigma}
      &
      {\Delta,i\mathord:s' \Rightarrow \primTyArity(\tyPrimNm{X})} \ar[d]^{- \circ \sigma_{s'}} \ar@/^/[rdd]^{\rho_1 \tyPrimNm{X}}
      \\
      {\Delta' \Rightarrow \primTyArity(\tyPrimNm{X})} \ar[r]^(.45){- \circ \pi_{i\mathord:s'}} \ar@/_/[rrd]_{\rho_2\ \tyPrimNm{X}}
      &
      {\Delta,i\mathord:s' \Rightarrow \primTyArity(\tyPrimNm{X})} \ar@{.>}[dr]_{\rho\ \tyPrimNm{X}}
      \\
      &
      &
      {\Rel(\tyPrimSem{\tyPrimNm{X}})}
    }
  \end{displaymath}
  Then there exists a relation environment $\rho \in
  \relEnv{E}(\Delta,i\mathord:s')$ (the dotted arrow) such that the
  two triangles in the bottom right of the diagram commute.
  % FIXME: mention that we already know that the square commutes
\end{enumerate}

Both of these conditions ensure that the relational interpretation of
types that we define below behaves correctly with respect to
simultaneous substitution of index terms
(\lemref{lem:tyeqsubst-relational}, below). The second condition may
seem mysterious at first, but it is essential for proving that the
relational interpretation of universal quantification types behaves
correctly with respect to application of simultaneous substitution.

\begin{example}[Units of Measure]
  We give two examples of assignments of sets of relational
  environments to index contexts in the units of measure example. The
  second example is a refinment of the first. FIXME: some general
  words about relational environments in these cases.
  \begin{enumerate}
  \item FIXME: assign a positive scaling factor to each variable in an
    index context, extend homomorphically to all index terms, and then
    use this to define a scaling relation
  \item FIXME: given an index context, assume a subgroup and a
    homomorphism into the scaling factors
  \end{enumerate}
\end{example}

Given a relation environment $\rho \in \relEnv{E}(\Delta)$, we define
the set of extensions $\extends{\rho}{i\mathord:s}$ of $\rho$ by an
additional index variable $i\mathord:s$ to be
$\extends{\rho}{i\mathord:s} \isDefinedAs \{ \rho' \in
\relEnv{E}(\Delta,i\mathord:s) \sepbar \rho' \circ \pi^*_{i\mathord:s}
= \rho \}$. The set of extensions of a relational environment will be
used in the interpretation of the universal quantification type.

A relational environment $\rho \in \relEnv{E}(\Delta)$ is extended to
a relation interpretation of any type $\Delta \vdash A \isType$ by
induction on its derivation: % Thus, we define a relation
% $\rsem{A}{\relEnv{E}}{\rho} \in \Rel(\tySem{A})$ for each well-indexed
% type.
\begin{eqnarray*}
  \rsem{\tyUnit}{\relEnv{E}}\rho & \isDefinedAs & \{(*,*)\} \\
  \rsem{\tyPrim{X}{e_1,...,e_n}}{\relEnv{E}}\rho & \isDefinedAs & \rho\ {\tyPrimNm{X}}\ (e_1,...,e_n) \\
  \rsem{A \tyArr B}{\relEnv{E}}\rho & \isDefinedAs & \rsem{A}{\relEnv{E}}\rho \relArrow \rsem{B}{\relEnv{E}}\rho \\
  \rsem{A \tyProduct B}{\relEnv{E}}\rho & \isDefinedAs & \rsem{A}{\relEnv{E}}\rho \relTimes \rsem{B}{\relEnv{E}}\rho \\
  \rsem{A + B}{\relEnv{E}}\rho & \isDefinedAs & \rsem{A}{\relEnv{E}}\rho \relSum \rsem{B}{\relEnv{E}}\rho \\
  \rsem{\forall i\mathord:s.A}{\relEnv{E}}\rho & \isDefinedAs & \bigcap\{ \rsem{A}{\relEnv{E}}\rho' \sepbar \rho' \in \extends{\rho}{i\mathord:s} \}
\end{eqnarray*}
In this definition we have made use of the following three
constructions on binary relations. If $R \in \Rel(X)$ and $S \in
\Rel(Y)$, then $R \relArrow S \in \Rel(X \to Y)$ is defined as $\{
(f_1,f_2) \sepbar \forall (a_1,a_2) \in R.\ (f_1a_1,f_2a_2) \in S
\}$. With the same assumptions on $R$ and $S$, the relation $R
\relTimes S \in \Rel(X \times Y)$ is defined as $\{
((a_1,b_1),(a_2,b_2)) \sepbar (a_1,a_2) \in R \land (b_1,b_2) \in S
\}$. Finally, the relation $R \relSum S \in \Rel(X + Y)$ is defined as
$\{ (\mathrm{inl}\ x, \mathrm{inl}\ x') \sepbar (x,x') \in R \} \cup
\{ (\mathrm{inr}\ y, \mathrm{inr}\ y') \sepbar (y,y') \in S \}$.

The following lemma states that the relation interpretation of types
that we have defined in this section behaves well: the first part of
the lemma states that two types that are judgmentally equal are given
equal relational interpretations, and the second part states that
substitution of index terms in types can be interpreted by the
composition of relational environments with simultaneous
substitutions. Note that both parts of the lemma depend on
\lemref{lem:tyeqsubst-erasure} to be ``well-typed''.
\begin{lemma}\label{lem:tyeqsubst-relational}
  \begin{enumerate}
  \item If $\Delta \vdash A \equiv B \isType$, then for all $\rho \in
    \relEnv{E}(\Delta)$, $\rsem{A}{\relEnv{E}}{\rho} =
    \rsem{B}{\relEnv{E}}{\rho}$;
  \item If $\Delta' \vdash A \isType$ and $\sigma : \Delta \Rightarrow
    \Delta'$, then for all $\rho \in \relEnv{E}(\Delta)$,
    $\rsem{\sigma^*A}{\relEnv{E}}\rho = \rsem{A}{\relEnv{E}}(\rho
    \circ \sigma)$.
  \end{enumerate}
\end{lemma}

%%%%%%%%%%%%%%%%%%%%%%%%%%%%%%%%%%%%%%%%%%%%%%%%%%%%%%%%%%%%%%%%%%%%%%%%%%%%%%
%%%%%%%%%%%%%%%%%%%%%%%%%%%%%%%%%%%%%%%%%%%%%%%%%%%%%%%%%%%%%%%%%%%%%%%%%%%%%%
\subsection{Well-typed Programs}
\label{sec:well-typed-programs}

We now present the rules for well-typed programs over the collection
of types we generated in
\autoref{sec:algebraically-indexed-types}. Each well-type program is
assigned an index-erasure semantics, building on the index-erasure
semantics of types we defined in
\autoref{sec:index-erasure-semantics}. Our main result
(\thmref{thm:abstraction}) is that the index-erasure semantics is
every well-typed program is related to itself in the relational
interpretation of its type: this is the abstraction theorem for every
calculus in our general framework. In the next section we will use
this theorem to derive several interesting results about the programs
expressible in instances of our general framework.

Well-typed programs are defined with respect to well-indexed typing
contexts, which are in turn defined with respect to an index
context. Well-indexed typing contexts with respect to an index context
$\Delta$ are sequences of variable/type pairs with no repeated
variable names such that each type is well-indexed with respect to
$\Delta$. Formally, well-indexed typing contexts are defined by the
following two rules:
\begin{mathpar}
  \inferrule*
  { }
  {\Delta \vdash \epsilon \isCtxt}

  \inferrule*
  {\Delta \vdash \Gamma \isCtxt \\ \Delta \vdash A \isType \\ x \not\in \Gamma}
  {\Delta \vdash \Gamma, x : A \isCtxt}
\end{mathpar}
Substitution by a simultaneous substitution extends to typing contexts
by performing substitution pointwise on each individual type.

The typing rules for our system define the well-typed terms with
respect to an index context $\Delta$ and a type context $\Delta \vdash
\Gamma \isCtxt$. The judgement $\Delta; \Gamma \vdash M : A$ is
defined in \autoref{fig:programs}. The equational theory on types is
incorporated into the type system via the rule \TirName{TyEq}, which
allows for a term that is judged to have type $A$ to also have any
other equal type $B$.
\begin{figure*}[t]
  \centering
  \begin{mathpar}
    \inferrule* [right=Var]
    {\Delta \vdash \Gamma \isCtxt \\ x : A \in \Gamma}
    {\Delta; \Gamma \vdash x : A}

    \inferrule* [right=TyEq]
    {\Delta; \Gamma \vdash M : A \\ \Delta \vdash A \equiv B \isType}
    {\Delta; \Gamma \vdash M : B}

    \inferrule* [right=Unit]
    { }
    {\Delta; \Gamma \vdash * : 1}

    \inferrule* [right=Pair]
    {\Delta; \Gamma \vdash M : A \\
      \Delta; \Gamma \vdash N : B}
    {\Delta; \Gamma \vdash (M, N) : A \tyProduct B}

    \inferrule* [right=Proj1]
    {\Delta; \Gamma \vdash M : A \tyProduct B}
    {\Delta; \Gamma \vdash \pi_1 M : A}

    \inferrule* [right=Proj2]
    {\Delta; \Gamma \vdash M : A \tyProduct B}
    {\Delta; \Gamma \vdash \pi_2 M : B}

    \inferrule* [right=Inl]
    {\Delta; \Gamma \vdash M : A}
    {\Delta; \Gamma \vdash \mathrm{inl}\ M : A + B}

    \inferrule* [right=Inr]
    {\Delta; \Gamma \vdash M : B}
    {\Delta; \Gamma \vdash \mathrm{inr}\ M : A + B}

    \inferrule* [right=Case]
    {\Delta; \Gamma \vdash M : A + B \\
      \Delta; \Gamma, x : A \vdash N_1 : C \\
      \Delta; \Gamma, y : B \vdash N_2 : C}
    {\Delta; \Gamma \vdash \textrm{case}\ M\ \textrm{of}\ \textrm{inl}\ x.N_1; \textrm{inr}\ y.N_2 : C}

    \inferrule* [right=Abs]
    {\Delta; \Gamma, x : A \vdash M : B}
    {\Delta; \Gamma \vdash \lambda x.M : A \tyArr B}

    \inferrule* [right=App]
    {\Delta; \Gamma \vdash M : A \tyArr B \\
      \Delta; \Gamma \vdash N : A}
    {\Delta; \Gamma \vdash M N : B}

    \inferrule* [right=UnivAbs]
    {\Delta, i : s; \pi_{i\mathord:s}^*\Gamma \vdash M : A}
    {\Delta; \Gamma \vdash \Lambda i. M : \forall i\mathord:s. A}

    \inferrule* [right=UnivApp]
    {\Delta; \Gamma \vdash M : \forall i\mathord:s. A \\ \Delta \vdash e : s}
    {\Delta; \Gamma \vdash M [e] : (\id_\Delta, e)^*A}
  \end{mathpar}
  
  \caption{Well-typed Programs}
  \label{fig:programs}
\end{figure*}

\subsubsection{The Index-Erasure Interpretation of Programs}
\label{sec:erasure-semantics-programs}

% FIXME: do we need to prove something about the effect
% of substitution on the erasure semantics of contexts?

We assign an index-erasure semantics to any well-indexed typing
context $\Delta \vdash \Gamma \isCtxt$ by induction on its structure:
$\ctxtSem{\epsilon} = \{*\}$ and $\ctxtSem{\Gamma, x : A} =
\ctxtSem{\Gamma} \times \tySem{A}$. For a well-typed program $\Delta;
\Gamma \vdash M : A$, we define the \emph{erasure interpretation} as a
function $\tmSem{e} : \ctxtSem{\Gamma} \to \tySem{A}$, completely
ignoring the indexing information. In light of
\lemref{lem:tyeqsubst-erasure}, we define this function directly on
the syntax of well-typed terms, rather than on the typing
derivations. We define $\tmSem{e}$ by the following clauses:
\begin{displaymath}
  \begin{array}{@{\hspace{0em}}l@{\hspace{0.5em}}c@{\hspace{0.5em}}l}
    \tmSem{x}\eta & \isDefinedAs & \eta_x \\
    \tmSem{(M,N)}\eta & \isDefinedAs & (\tmSem{M}\eta, \tmSem{N}\eta) \\
    \tmSem{\pi_1M}\eta & \isDefinedAs & \pi_1(\tmSem{M}\eta) \\
    \tmSem{\pi_2M}\eta & \isDefinedAs & \pi_2(\tmSem{M}\eta) \\
    \tmSem{\mathrm{inl}\ M}\eta & \isDefinedAs & \mathrm{inl}\ (\tmSem{M}\eta) \\
    \tmSem{\mathrm{inr}\ M}\eta & \isDefinedAs & \mathrm{inr}\ (\tmSem{M}\eta) \\
    \left\llbracket
      \begin{array}{l}
        \textrm{case}\ M\ \textrm{of}\\
        \textrm{inl}\ x.N_1;\\
        \textrm{inr}\ y.N_2
      \end{array}\right\rrbracket^{\mathit{tm}}\eta & \isDefinedAs &
    \left\{
      \begin{array}{ll}
        \tmSem{N_1}(\eta,a) & \textrm{if }\tmSem{M}\eta = \mathrm{inl}(a) \\
        \tmSem{N_2}(\eta,b) & \textrm{if }\tmSem{M}\eta = \mathrm{inr}(b)
      \end{array}
    \right. \\
    \tmSem{\lambda x. M}\eta & \isDefinedAs & \lambda v. \tmSem{M}(\eta, v) \\
    \tmSem{M N}\eta & \isDefinedAs & (\tmSem{M}\eta) (\tmSem{N}\eta) \\
    \tmSem{\Lambda i.\ M}\eta & \isDefinedAs & \tmSem{M}\eta \\
    \tmSem{M[e]}\eta & \isDefinedAs & \tmSem{M}\eta    
  \end{array}
\end{displaymath}
In the above, we have written $\eta_x$ to denote the appropriate
projection from $\eta$ to get the value in the environment
corresponding to the variable $x$ in the context.

\subsubsection{The Abstraction Theorem}
\label{sec:abstraction-theorem}

We now state and prove the abstraction theorem for well-typed programs
in our system. To state and prove this theorem for open terms, we
first need to extend the relational interpretation of types to typing
contexts. The relational interpretation of contexts is defined by the
following clauses, where we have used the pairing operation
$-\relTimes-$ defined in \autoref{sec:relational-semantics}.
\begin{eqnarray*}
  \rsem{\epsilon}{\relEnv{E}}{\rho} & \isDefinedAs & \{(*,*)\} \\
  \rsem{\Gamma, x : A}{\relEnv{E}}{\rho} & \isDefinedAs & \rsem{\Gamma}{\relEnv{E}}\rho \relTimes \rsem{A}{\relEnv{E}}\rho
\end{eqnarray*}
The relational interpretation of contexts inherits from the relational
interpretation of types the property of interpreting the application
of simultaneous substitutions as composition:
\begin{lemma}\label{lem:ctxtsubst-rel}
  If $\Delta \vdash' \Gamma \isCtxt$ and $\sigma : \Delta \Rightarrow
  \Delta'$, then for all $\rho \in \relEnv{E}(\Delta)$,
  $\rsem{\sigma^*\Gamma}{\relEnv{E}}\rho =
  \rsem{\Gamma}{\relEnv{E}}{(\rho \circ \sigma)}$.
\end{lemma}

\begin{theorem}[Abstraction]\label{thm:abstraction}
  If $\Delta; \Gamma \vdash M : A$, then for all $\rho \in
  \relEnv{E}(\Delta)$ and $\eta_1, \eta_2 \in \ctxtSem{\Gamma}$ such
  that $(\eta_1, \eta_2) \in \rsem{\Gamma}{\relEnv{E}}\rho$, we have
  $(\tmSem{M}\eta_1, \tmSem{M}\eta_2) \in \rsem{A}{\relEnv{E}}\rho$.
\end{theorem}

\begin{proof}
  By induction on the derivation of $\Delta; \Gamma \vdash M : A$. The
  three interesting cases follow:
  \begin{description}
  \item[Case \TirName{TyEq}] This case follows directly from
    \lemref{lem:tyeqsubst-relational}.
  \item[Case \TirName{UnivAbs}] In this case, $M = \Lambda i.M'$ and
    $A = \forall i\mathord:s. A'$. We know from the induction
    hypothesis that for all $\rho' \in \relEnv{E}(\Delta, i:s)$ and
    $(\eta_1,\eta_2) \in \rsem{\pi_{i:s}^*\Gamma}{\relEnv{E}}{\rho'}$,
    we have $(\tmSem{M'}\eta_1,\tmSem{M'}\eta_2) \in
    \rsem{A'}{\relEnv{E}}{\rho'}$. We are given $\rho \in
    \relEnv{E}(\Delta)$, $(\eta_1,\eta_2) \in
    \rsem{\Gamma}{\relEnv{E}}\rho$ and $\rho'' \in
    \extends{\rho}{i:s}$. Since $\rho'' \circ \pi^*_{i:s} = \rho$, we
    have $(\eta_1,\eta_2) \in \rsem{\Gamma}{\relEnv{E}}\rho$ implies
    $(\eta_1,\eta_2) \in \rsem{\Gamma}{\relEnv{E}}{(\rho'' \circ
      \pi^*_{i:s})}$, and therefore $(\eta_1,\eta_2) \in
    \rsem{\pi_{i:s}^*\Gamma}{\relEnv{E}}{\rho''}$ by \lemref{lem:ctxtsubst-rel}. So we
    can apply the induction hypothesis to get $(\tmSem{M'}\eta_1,
    \tmSem{M'}\eta_2) \in \rsem{A'}{\relEnv{E}}{\rho''}$, and thus
    $(\tmSem{\Lambda i.M'}\eta_1,\tmSem{\Lambda i.M'}\eta_2) \in
    \rsem{A'}{\relEnv{E}}\rho''$, by the definition of $\tmSem{\Lambda
      i. M'}$. Hence $(\tmSem{\Lambda i.M'}\eta_1,\tmSem{\Lambda
      i.M'}\eta_2) \in \rsem{\forall
      i\mathord:s. A'}{\relEnv{E}}\rho$, as required.
  \item[Case \TirName{UnivApp}] In this case, $M = M'[e]$ and $A =
    (\id_\Delta, e)^*A'$. By the induction hypothesis, and the
    definition of $\rsem{\forall i\mathord:s.A'}{\relEnv{E}}\rho$, we
    know that for all $\rho' \in \extends{\rho}{i:s}$,
    $(\tmSem{M'}\eta_1, \tmSem{M'}\eta_2) \in
    \rsem{A'}{\relEnv{E}}{\rho'}$. If we let $\rho' = \rho \circ
    (\id_\Delta, e)^*$, we know that $\rho' \in \extends{\rho}{i:s}$,
    because $\rho' \circ \pi^*_{i:s} = \rho \circ (\id_\Delta, e)^*
    \circ \pi^*_{i:s} = \rho$. Hence we may deduce that
    $(\tmSem{M'}\eta_1,\tmSem{M'}\eta_2) \in
    \rsem{A'}{\relEnv{E}}{(\rho \circ (\id_\Delta,e)^*)}$ and thence,
    by \lemref{lem:ctxtsubst-rel}, conclude that $(\tmSem{M'}\eta_1,
    \tmSem{M'}\eta_2) \in \rsem{(\id_\Delta,e)^*A'}{\relEnv{E}}\rho$,
    as required.
  \end{description}
  The remaining cases are standard for proofs of the fundamental lemma
  of logical relations for simply-typed systems.
\end{proof}

%%% Local Variables:
%%% TeX-master: "paper"
%%% End:
