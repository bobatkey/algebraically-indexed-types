\section{Motivating Examples}

We motivate this paper's goal of investigating algebraically-indexed
types and their relational interpretations by presenting several
examples. Our main set of examples comprise a novel type system for
writing programs that manipulate geometric data, while maintaining
geometric invariants through the types. \fixme{and the information
  flow examples}

\subsection{2-Dimensional Geometry and Origin Invariance}

When writing programs that manipulate geometric data, the basic data
structure is the $n$-tuple of real numbers. In the 2-dimensional case,
tuples $\vec{v} = (x,y)$ may be called upon to represent both
\emph{points} -- offsets from some origin -- and \emph{vectors} --
offsets in their own right. The notions of point and vector are very
different despite their common representation. By representing a point
concretely as an offset from an origin, we inadvertently fix that
origin as privileged.

Putting this in concrete terms, if we have two origins, $(0,0)$ and
$(10,20)$, then the tuple $(1,1)$ with respect to the first origin and
$(11,21)$ with respect to the second origin represent the \emph{same}
point. It is merely an artifact of our representation of points as
pairs of numbers that these appear to be different. When we write
programs that manipulate data that represents points, we ought to
follow the general principle that our programs are invariant with
respect to changes in the choice of origin. On the other hand, vectors
are not intended to be regarded as invariant under change of choice of
origin. Vectors represent offsets, or movements, and the vector
$(0,0)$ always represents the zero offset.
% However, they are intended to be invariant under change of basis: the
% vector $(2,3)$, with respect to the basis $(1,0),(0,1)$, and the
% vector $(1,1)$, with respect to the basis $(2,0),(0,3)$, should be
% regarded as the same vector. Invariance under change of basis also
% applies to the representation of points. We therefore refine the
% principle of invariance under change of origin for points to also
% include invariant under change of basis for points and vectors.

% A particular choice of origin and basis is called a \emph{frame}
% (\cite{gallier11geometric}, Chapter 2). In general, the technique of
% maintaining geometric invariants such as frame invariance has been
% referred to as \emph{coordinate-free geometry}: we should aim to
% think in terms of abstract concepts such as points and vectors,
% rather than their actual representation as tuples of numbers.



Invariance under change of representation immediately recalls
Reynolds' fable concerning two professors teaching the theory of
complex numbers \cite{reynolds83types}. One professor chooses
rectangular coordinates ($x + iy$), while the other chooses polar
coordinates ($\alpha\cos\theta + i\alpha\sin\theta$). Happily, after
learning the basic operations on complex numbers in the two
representations, the two classes can interact because the theory of
complex numbers is invariant with respect to the choice of
representation. Reynolds formalises the idea of invariance with
respect to changes of representation by using the notion of
preservation of relations. If a binary relation $R$ relates two
representations of a concept---for example the rectangular and polar
representations of the same complex number are related---then a
program that manipulates complex numbers at a level of abstraction
above their specific representation should preserve the relation
$R$. For example, if $f$ be a program taking complex numbers to
complex numbers that is intended to respect invariance under change of
representation. Then if $c$ is a complex number in rectangular form,
and $c'$ is a complex number in polar form such that $R$ relates them,
which we write as $(c,c') \in R$, then it should be the case that
$(f(c), f(c')) \in R$.

We now apply Reynolds' relational approach to defining invariance
under changes of representation to the geometric setting. Following
Reynolds, we express the fact that two tuples in $\mathbb{R}^2$ are
representatives of the same point under change of origin by defining a
family of binary relations over $\mathbb{R}^2$. Our family of binary
relations is parameterised by a particular choice of origin, and by
quantifying over all choices of origin we will be able to ensure that
our programs are invariant with respect to the particular choice of
origin. We define a $\mathbb{R}^2$-parameterised family of binary
relations $\{ R_{\vec{o}} \subseteq \mathbb{R}^2 \times \mathbb{R}^2
\}_{\vec{o} \in \mathbb{R}^2}$ as follows:
\begin{displaymath}
  R_{\vec{o}} = \{ (\vec{v}, \vec{v'}) \sepbar \vec{v'} = \vec{v} + \vec{o} \}
\end{displaymath}

Now consider a function $f$ that, for example, takes two tuples in
$\mathbb{R}^2$ and returns a single tuple in $\mathbb{R}^2$. We intend
that all these tuples represent points with respect to the same
origin, but that the function itself should be invariant with respect
to the choice of origin. We use Reynolds' idea of preservation of
relations to formalise this: for any $\vec{o} \in \mathbb{R}^2$
(i.e.~any choice of origin), the function $f$ should satisfy the
following statement:
\begin{equation}\label{eq:f-preserve-rel-frame}
  \forall (\vec{v_1},\vec{v'_1}) \in R_{\vec{o}}, (\vec{v_2},\vec{v'_2}) \in R_{\vec{o}}. (f(\vec{v_1}, \vec{v_2}), f(\vec{v'_1}, \vec{v'_2})) \in R_{\vec{o}}
\end{equation}
Unfolding the definition of $R_{\vec{o}}$, this statement is
equivalent to the following statement expressing invariance of $f$
with respect to changes of origin: for all $\vec{o} \in \mathbb{R}^2$,
\begin{equation}\label{eq:f-invariant-frame}
  \forall \vec{v_1}, \vec{v_2}.\ f(\vec{v_1} + \vec{o},\vec{v_2} + \vec{o}) = f(\vec{v_1},\vec{v_2}) + \vec{o}.
\end{equation}
So Reynolds' preservation of relations, when instantiated with the
family of relations $R$, yields exactly the geometric property of
invariance under change of origin.

% An example function $f$ that satisfies
% \statementref{eq:f-invariant-frame} is the following function that
% computes a particular affine combination of two points by working
% directly on their coordinate representation:
% \begin{displaymath}
%   f(\vec{v_1}, \vec{v_2}) = \frac{1}{2}\vec{v_1} + \frac{1}{2}\vec{v_2}.
% \end{displaymath}
% In general, affine combinations of points $\lambda_1\vec{v_1} +
% \lambda_2\vec{v_2}$, where $\lambda_1 + \lambda_2 = 1$, satisfy
% \statementref{eq:f-invariant-frame}. Affine combination is one of the
% fundamental building blocks of affine geometry -- the properties of
% points invariant under invertible affine maps
% (\cite{gallier11geometric}, Chapter 2). If we drop the condition that
% $\lambda_1 + \lambda_2 = 1$, then we are dealing with linear
% combinations of vectors, and we are no longer invariant with respect
% to changes of frame. However, linear combinations are invariant with
% respect to change of basis. We can represent changes of basis as
% linear invertible maps $B : \mathbb{R}^2 \to \mathbb{R}^2$. The set of
% all such maps forms the general linear group $\GL(2)$. We now define
% another family of relations $\{R_{\texttt{vec}}(B) \subseteq
% \mathbb{R}^2 \times \mathbb{R}^2 \}_{B \in \GL(2)}$ that relates two
% points up to change of basis:
% \begin{displaymath}
%   R_{\texttt{vec}}(B) = \{ (\vec{v_1},\vec{v_2}) \sepbar B\vec{v_2} = \vec{v_1} \}.
% \end{displaymath}
% Now the functions $f_{\lambda_1\lambda_2}(\vec{v_1},\vec{v_2}) =
% \lambda_1\vec{v_1} + \lambda_2\vec{v_2}$, for arbitrary $\lambda_1$
% and $\lambda_2$, do preserve the relations $R_{\texttt{vec}}(B)$, for
% all $B \in \GL(2)$. Unfolding the definition of $R_{\texttt{vec}}(B)$
% in the analogous statement to \statementref{eq:f-preserve-rel-frame}
% for $R_{\texttt{vec}}$ instead of $R_{\texttt{pt}}$, we can see that
% preservation of the relations $R_{\texttt{vec}}$ characterises the
% functions that are invariant under change of basis. For all $B \in \GL(2)$, we have
% \begin{displaymath}
%   \forall \vec{v_1}, \vec{v_2}.\ f(B\vec{v_1},B\vec{v_2}) = B(f(\vec{v_1},\vec{v_2})),
% \end{displaymath}
% and this is exactly the property of invariance under change of basis
% we required above of programs manipulating vectors.

% Note that the family of relations $R_{\texttt{vec}}$ is just
% $R_{\texttt{pt}}$ when restricted to elements of the group
% $\GL(2)$. In the type system we introduce in the next section, we
% combine points and vectors into the same data type. Whether it
% represents a point or a vector depends on the group of geometric
% transformations that we expect it to be invariant under.

% FIXME: vectors are not invariant under change of origin, so they are
% represented by the relation $R_0$, which is just equality. But they,
% and points are invariant under change of basis. However, as we shall
% see below, not all operations are invariant under all changes of
% basis. In particular the dot product is only invariant under
% orthogonal transformations.

\subsection{A Type System for Origin Invariance}
\label{sec:type-system-geom-intro}

Reynolds further insight was to see how preservation of relations by
the denotational interpretations of programs can automatically be
established by following a typing discipline. In Reynolds' case, the
typing discipline was the polymorphic $\lambda$-calculus where new
types can be constructed by universal quantification over all
types. Universal type quantification is commonly written $\forall
\alpha. A$, which denotes a type that may be instantiated with any
other type $B$ by replacing the type variable $\alpha$ with $B$ in
$A$.

In terms of relations, Reynolds interprets universal quantification
over types as quantification over binary relations between denotations
of types. In the statements of geometric invariance that we stated in
the previous section (e.g., \statementref{eq:f-invariant-frame}), we
did not quantify over all relations. Instead, we quantified over all
choices of origin and used the choice to select a relation from the
family $\{R_{\vec{o}}\}$. Driven by this observation, we introduce
quantification over choice of origin into the language of types.  We
use the notation $\forall t \mathord: T_2. A$ for quantification over
all 2-dimensional translations (i.e.~choices of origin) $t$. We refer
to $T_2$ as the \emph{sort} of $t$.

% As we proceed, we will introduce the sorts representing geometric
% groups that we will be using in addition to $\GA(2)$ and $\GL(2)$.

The sort $T_2$ represents an abelian group, so we can combine its
elements using the usual group operations. We write operations
additively, using $e_1 + e_2$ for the group operation, $-e$ for
inverse and $0$ for the unit.  We also regard expressions built from
variables and the group operations up to the group axioms. So, for
example, $e_1 + (e_2 + e_3)$ and $(e_1 + e_2) + e_3$ are to be
regarded as equivalent.

Our language of types also includes the common type constructors for
function types, $A \to B$, sum types $A + B$, unit type
$\mathsf{Unit}$ and tuple types $A \times B$. We also assume a
primitive type $\mathsf{real}$, used to represent scalars. As we noted
at the start of the previous section, the central data structure in
geometric applications is the tuple of real numbers for representing
points and vectors. However, we cannot simply express this as the type
$\mathsf{real} \times \mathsf{real}$ because it will not have the
correct relational interpretation. (Two elements of type
$\mathsf{real}$ will be related if and only if they are equal, and
hence two elements of $\mathsf{real} \times \mathsf{real}$ will be
related if and only if they are equal, by Reynolds' definition of the
relational interpretation of tuple types.) Therefore, we introduce a
new type parameterised by expressions $e$ of sort $T_2$ to represent
points (and later, vectors): $\mathsf{vec}\langle e \rangle$. We have
used the name $\mathsf{vec}$, even though we have taken pains to
separate geometric points and vectors, to recall the computer science
notion of vector as a sequence of values of homogeneous type which has
a known length (in this case two).

% FIXME: Denotational semantics of $\mathsf{vec}$? and transformation
% quantification?

Equipped with quantification over the group $T_2$ of changes of
origin, and the parameterised types $\mathsf{vec}\langle \cdot
\rangle$, we can now express origin invariance properties as types. In
the previous section, we discussed functions $f : \mathbb{R}^2 \times
\mathbb{R}^2 \to \mathbb{R}^2$ that have the property that they
preserve all changes of origin. We can now express this property as a
type:
\begin{displaymath}
  \mathrm{f} : \forall t \mathord: T_2.\ \mathsf{vec}\langle t \rangle \times \mathsf{vec}\langle t \rangle \to \mathsf{vec}\langle t \rangle
\end{displaymath}
FIXME: forward references...

\subsection{Affine and Vector Operations}
\label{sec:affine-vector-ops}

As we stated in the previous section, the primitive operation of
affine geometry is affine combination of points: $\lambda_1\vec{v_1} +
\lambda_2\vec{v_2}$, where $\lambda_1 + \lambda_2 = 1$. Geometrically,
affine combination can be interpreted as interpolation between the
points represented by $\vec{v_1}$ and $\vec{v_2}$. We add affine
combination of points to our calculus, with the following typing and
intended denotation:
\begin{displaymath}
  \begin{array}{l}
    \mathrm{affineCombination} :\forall t \mathord: T_2.\ \mathsf{vec}\langle t \rangle \to \mathsf{real} \to \mathsf{vec}\langle t \rangle \to \mathsf{vec}\langle t \rangle \\
    \sem{\mathrm{affineCombination}}\ (x_1,y_1)\ r\ (x_2,y_2) = \\
    \hspace{3cm}((1-r)x_1 + tx_2, (1-r)y_1 + ty_2)
  \end{array}
\end{displaymath}
By defining the primitive function $\mathrm{affineCombination}$ to
take a single $\mathsf{real}$ parameter $t$, we can easily ensure that
we are taking the affine combination of two representatives of points,
and not just an arbitrary linear combination. The intended denotation
in the second line does not take any argument corresponding to the
universally quantified $t$ in the type. Type-level variables are
erased at runtime.

% In Linear Algebra notation, we could have written the right-hand side
% as $(1-t)\vec{v_1} + t\vec{v_2}$, where $\vec{v_1} = (x_1,y_1)$ and
% $\vec{v_2} = (x_2,y_2)$. 

\begin{example}
  The evaluation of quadratic B\'{e}zier curves (B\'{e}zier curves
  with two endpoints and a single control point) can be expressed
  using the affine combination primitive and three steps of De
  Casteljau's algorithm:
  \begin{displaymath}
    \begin{array}{@{}l}
      \mathrm{quadB\acute{e}zier} : \forall t \mathord:T_2.\ \mathsf{vec}\langle t \rangle \to \mathsf{vec}\langle t \rangle \to \mathsf{vec}\langle t \rangle \to \mathsf{real} \to \mathsf{vec}\langle t \rangle \\
      \mathrm{quadB\acute{e}zier}\ [t]\ p_0\ p_1\ p_2\ s = \\
      \quad
      \begin{array}{l@{\hspace{-0.002cm}}l}
        \mathrm{affineCombination}\ [F]\ & (\mathrm{affineCombination}\ [F]\ p_0\ s\ p_1) \\
        & s \\
        & (\mathrm{affineCombination}\ [F]\ p_1\ s\ p_2)
      \end{array}
    \end{array}
  \end{displaymath}
  For two endpoints $p_0$ and $p_2$ and a control point $p_1$, an
  application $\mathrm{quadB\acute{e}zier}\ p_0\ p_1\ p_2\ s$ for $s
  \in [0,1]$ gives us the point on the curve at ``time'' $s$. Note
  that the type of $\mathrm{quadB\acute{e}zier}$ tells us immediately
  that it preserves all changes of origin
\end{example}

Affine combination only provides us with an operation on points. We
are also interested in operations on the vectors representing offsets
between points. We now examine the correct types to assign to the
vector space operations of addition of vectors, negation of vectors,
multiplication by a scalar and the zero vector. The typings of these
operations will make use of the abelian group structure of the sort of
translations $T_2$.

Vector operations are not invariant under change of origin. The
obvious type assignment for vector addition is therefore:
\begin{displaymath}
  (+) : \mathsf{vec}\langle 0 \rangle \to \mathsf{vec}\langle 0 \rangle \to \mathsf{vec}\langle 0 \rangle
\end{displaymath}
(we write binary operators intended to be used infix in parentheses
when not appearing in infix position, following the Haskell
syntax.)

However, it is possible to give vector addition a more precise type
that describes its effect on the chosen origin. 
\begin{displaymath}
  (+) : \forall t_1, t_2 \mathord: T_2.\ \mathsf{vec}\langle t_1\rangle \to \mathsf{vec}\langle t_2 \rangle \to \mathsf{vec}\langle t_1 + t_2 \rangle
\end{displaymath}
We can also negate vectors, yielding a vector which points in the
opposite direction. Negation negates translation arguments:
\begin{displaymath}
  \mathrm{negate} : \forall t \mathord: T_2.\ \mathsf{vec}\langle t \rangle \to \mathsf{vec}\langle -t \rangle
\end{displaymath}
With the primitive operations of addition and negation of vectors, we
can define the derived operation of subtraction:
\begin{displaymath}
  \begin{array}{l}
    (-) : \forall t_1,t_2 \mathord:T_2.\ \mathsf{vec}\langle t_1\rangle \to \mathsf{vec}\langle t_2\rangle \to \mathsf{vec}\langle t_1 - t_2 \rangle \\
    (-)\ [t_1]\ [t_2]\ p_1\ p_2 = p_1 + \mathrm{negate}\ p_2
  \end{array}
\end{displaymath}

Given two points that have been constrained to be invariant with
respect to the same change of origin---i.e.~two values of type
$\mathsf{vec}\langle t \rangle$---we can compute their offset, which
is a vector expressed with respect to the null change of origin. The
offset operation is a special case of vector subtraction:
\begin{displaymath}
  \begin{array}{l}
    \mathrm{offset} : \forall t \mathord:T_2.\ \mathsf{vec}\langle t \rangle \to \mathsf{vec}\langle t \rangle \to \mathsf{vec}\langle 0 \rangle \\
    \mathrm{offset}\ [t]\ p_1\ p_2 = p_1 - p_2
  \end{array}
\end{displaymath}
The type of the vector addition operation can be specialised to the
case of moving a point by a vector:
\begin{displaymath}
  \begin{array}{l}
    \mathrm{moveBy} : \forall t \mathord:T_2.\ \mathsf{vec}\langle t \rangle \to \mathsf{vec}\langle 0 \rangle \to \mathsf{vec}\langle t \rangle \\
    \mathrm{moveBy}\ [t]\ p\ v = p + v
  \end{array}
\end{displaymath}
The types we assign to the remaining vector space primitives, the zero
vector and multiplication by a scalar, do not describe any interesting
effect on translations%\footnote{but they could...}
. The zero vector has the type:
\begin{displaymath}
  \mathrm{0} : \mathsf{vec}\langle 0 \rangle
\end{displaymath}
Intuitively, there cannot be a zero vector invariant under any change
of origin, because this would entail picking a privileged origin. For
multiplication of a vector by a scalar, we assign the following type,
which is again not invariant under change of origin.
\begin{displaymath}
  (*) : \mathsf{real} \to \mathsf{vec}\langle 0 \rangle \to \mathsf{vec}\langle 0 \rangle
\end{displaymath}

\begin{example}
  The vector space operators and the properties that follow from their
  types allow us to establish a useful type isomorphism. Consider
  functions with types following the schema:
  \begin{displaymath}
    \tau_{n} \isDefinedAs \forall t\mathord:T_2.\ \underbrace{\mathsf{vec}\langle t \rangle \to ... \mathsf{vec}\langle t \rangle}_{n+1\textrm{ times}} \to \mathsf{real}
  \end{displaymath}
  Just by looking at the types $\tau_{n}$, we know that their
  inhabitants will be invariant under change of origin, due to the
  quantification over all $t$ in $T_2$. Therefore, we may as well pick
  one of the input points as the origin and assume that all the other
  points are defined with respect to this chosen origin. This is
  similar to the common mathematical step of stating that, ``without
  loss of generality'', we may pick some point in a description of a
  problem to be the origin, as long as the problem statement is
  invariant under translation.

  The types $\tau_{n}$ are isomorphic to the corresponding types
  $\sigma_{n}$:
  \begin{displaymath}
    \sigma_{n} \isDefinedAs \underbrace{\mathsf{vec}\langle 0 \rangle \to ... \mathsf{vec}\langle 0 \rangle}_{n\textrm{ times}} \to \mathsf{real}
  \end{displaymath}
  We demonstrate this isomorphism formally in \autoref{sec:FIXME}.

  % The isomorphism between these two families of types can be witnessed
  % by the following two functions.
  % \begin{displaymath}
  %   \begin{array}{l}
  %     i_{n,A} : \tau_{n,A} \to \sigma_{n,A} \\
  %     i_{n,A}\ f\ [B]\ (p_1, p_2, ..., p_n) = f\ [B \ltimes 0]\ (0, p_1, ..., p_n) \\
  %     \\
  %     i^{-1}_{n,A} : \sigma_{n,A} \to \tau_{n,A} \\
  %     i^{-1}_{n,A}\ g\ [F]\ (p_0, ..., p_n) = g\ [\pi_L(F)]\ (p_1-p_0, ..., p_n-p_0)
  %   \end{array}
  % \end{displaymath}
  % The direction defined by the function $i_{n,A}$ treats the inputs
  % $p_1,...,p_n$ as vector offsets from the origin $0$. The direction
  % witnessed by $i^{-1}_{n,A}$ picks the first point to act as the
  % origin, and uses the operation of vector subtraction we defined
  % above to turn each of the other points into offsets from this
  % point. Note that this isomorphism is not unique: for each $n$ we can
  % pick any of the $n+1$ inputs to act as the distinguished
  % origin. Thus the families of types $\tau_{n,A}$ and $\sigma_{n,A}$
  % are isomorphic, but not uniquely isomorphic.
\end{example}

\begin{example}
  An inexpressible type
  \begin{displaymath}
    \forall t \mathord: T_2.\ \vec{t + t} \to \vec{t}
  \end{displaymath}
  FIXME: explain that it is possible to show that this type is
  uninhabited with the given operations. We show this in general 
\end{example}

\subsection{Generalising Origin Invariance}
\label{sec:motivation-generalising}

\begin{enumerate}
\item Basis invariance
\item Orthogonal transform invariance for dot product
\item Rotational invariance (the type of the cross-product)
\item Scale invariance (units of measure)
\end{enumerate}

In some cases, we can use the invariance as a programming tool
(e.g. type isomorphisms), but at others we can only use it from the
outside.

\subsection{Information Flow}
\label{sec:information-flow}

\begin{enumerate}
\item Think of information flow as passing around unforgable tokens
  \begin{enumerate}
  \item Give an example (given a piece of data protected by a token
    and a token, can we extract the data?)
  \item Need to make sure that tokens are unforgable
  \end{enumerate}
\item Could either enforce unforgability of tokens by some dynamic
  means, or by static type-based means
  \begin{enumerate}
  \item Leave the dynamic method as vague
  \item Ingredients of the static method: run-time tokens are just
    dummy values ($*$), the token type is parameterised by logical
    formulae, and the relational meaning is derived from an assignment
    of boolean values to atoms.
  \item Explain this as invariance under changes of satisfaction of
    the participants. Relate to descriptions of logics as rules that
    don't care about truth, only its preservation.
  \end{enumerate}
\item Cite Sabelfeld and Sands: ``A PER Model of Secure Information
  Flow in Sequential Programs''
\end{enumerate}

%%% Local Variables:
%%% TeX-master: "paper"
%%% End:
