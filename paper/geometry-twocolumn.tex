\section{Geometry via Algebraically Indexed Types}
\label{sec:motivating-examples}

We motivate our investigation of algebraically indexed types and their
relational interpretations by developing a novel type system for
programs that manipulate two-dimensional geometric data. Geometry is
rich with operations that are invariant under transformation: affine
operations are invariant under change of origin, vector space
operations are invariant under change of basis, and dot product is
invariant under orthogonal changes of basis. On the other hand, some
geometric operations are interestingly variant under transformation.
For example, cross products vary with scalings of the plane. We
can incorporate (in)variance information about geometric primitives
into type systems via algebraically indexed types.

\subsection{Origin Invariance and Representation Independence}\label{sec:oiri}

The $n$-tuple of numbers is the basic data structure for programs that
manipulate geometric data. Pairs, i.e., $2$-tuples, 
%The basic data structure used in programs that manipulate geometric data is
%the $n$-tuple of numbers. In the 2-dimensional case, tuples 
$\vec{v} = (x,y)$ serve double duty, representing
%are called upon to represent both 
both \emph{points}---offsets from some origin---and
\emph{vectors}---offsets in their own right.  Despite their common
representation, points and vectors are very different, and
distinguishing between them is the key feature of \emph{affine
  geometry} (see, for example, Chapter 2 of
\cite{gallier11geometric}). Nevertheless, computational geometry
libraries traditionally either leave it to the programmer to maintain
the distinction between points and vectors, or else use different
abstract types for points and vectors to enforce it.
%(one might say that the standard mathematical formulation of affine
%spaces uses the abstract types approach). 
In this paper we investigate a more sophisticated approach based on 
types indexed by change of origin transformations.

This approach regards the difference between points and vectors as a
change of data representation. For example, if $(0,0)$ and $(10,20)$
are two origins, then the tuple $(1,1)$ with respect to $(0,0)$ the
tuple $(11,21)$ with respect to $(10,20)$ represent \emph{the same
  point} because they have the same displacement from these two
origins, respectively.
%It is merely an artifact of the representation of points as pairs of
%numbers that they appear to be different.
This suggests that programs that manipulate %data representing points
should be invariant with respect to changes in the choice of
origin. Program sthat manipulate vectors, on the other hand, should
not be invariant under change of origin. Different vectors represent
different offsets, and the vector $(0,0)$ always represents the zero
offset.

Invariance under change of representation immediately recalls
Reynolds' fable about two professors teaching the theory of complex
numbers \cite{reynolds83types}. One professor represents complex
numbers using rectangular coordinates ($x + iy$), while the other
represents them using polar coordinates ($\alpha\cos\theta +
i\alpha\sin\theta$). Happily, after learning the basic operations on
complex numbers in the two representations, the two classes can
interact because the theory of complex numbers is invariant under the
choice of representation. Reynolds formalises the idea of invariance
under changes of representation as preservation of relations.  For
example, if a binary relation $R$ relates the rectangular and polar
representations of complex numbers, then a program that manipulates
complex numbers at a level of abstraction above their specific
representation should preserve $R$. %More precisely, if
%$f$ is a program that is invariant under the choice of representation
%of complex numbers,
%%taking complex numbers to complex numbers that respects invariance
%%under change of representation. Then 
%$c$ is a complex number in rectangular form, $c'$ is a complex number
%in polar form, and $R$ is a relation that relates the rectangluar and
%polar representations of complex numbers, then $(c,c') \in R$ implies
%$(f(c), f(c')) \in R$.

Reynolds' relational approach can be applied in the geometric setting
to show how quantifying over all changes of origin ensures the
invariance of programs under any particular choice of origin. For
this, we first define a family of binary relations on $\mathbb{R}^2$
that is indexed by changes of origin. Changes of origin are
represented by vectors in $\mathbb{R}^2$, and form a group
$\Transl{2}$ of translations under addition. The $\Transl{2}$-indexed
family of binary relations $\{ R_{\vec{t}} \subseteq \mathbb{R}^2
\times \mathbb{R}^2 \}_{\vec{t} \in \Transl{2}}$ is then defined by
%\begin{displaymath}
$R_{\vec{t}} = \{ (\vec{v}, \vec{v'}) \sepbar \vec{v'} = \vec{v} + \vec{t} \}$.
%\end{displaymath}
We then consider a function $f$ that takes as input two tuples in
$\mathbb{R}^2$ and returns a single tuple in $\mathbb{R}^2$. We intend
that the tuples all represent points with respect to the same origin,
and that $f$ is invariant under the choice of origin.  Reynolds'
relational approach formalises this intention precisely: for any
$\vec{t} \in \Transl{2}$: % (i.e.~any change of origin), the function
%$f$
%should satisfy the following statement:
\begin{equation}\label{eq:f-preserve-rel-frame}
  \forall (\vec{v_1},\vec{v'_1}) \in R_{\vec{t}},
  (\vec{v_2},\vec{v'_2}) \in R_{\vec{t}}. (f(\vec{v_1}, \vec{v_2}),
  f(\vec{v'_1}, \vec{v'_2})) \in R_{\vec{t}}
\end{equation}
Unfolding the definition of $R_{\vec{t}}$ gives the equivalent formulation:
%\statementref{eq:f-preserve-rel-frame} is equivalent to: 
%for all $\vec{t} \in \Transl{2}$,
\begin{equation}\label{eq:f-invariant-frame}
  \forall \vec{v_1}, \vec{v_2}.\ f(\vec{v_1} + \vec{t},\vec{v_2} +
  \vec{t}) = f(\vec{v_1},\vec{v_2}) + \vec{t}.
\end{equation}
Thus, Reynolds' preservation of relations, when instantiated with the
family of relations $\{R_{\vec{t}}\}$, yields exactly the geometric
property of invariance under change of origin.

% An example function $f$ that satisfies
% \statementref{eq:f-invariant-frame} is the following function that
% computes a particular affine combination of two points by working
% directly on their coordinate representation:
% \begin{displaymath}
%   f(\vec{v_1}, \vec{v_2}) = \frac{1}{2}\vec{v_1} + \frac{1}{2}\vec{v_2}.
% \end{displaymath}
% In general, affine combinations of points $\lambda_1\vec{v_1} +
% \lambda_2\vec{v_2}$, where $\lambda_1 + \lambda_2 = 1$, satisfy
% \statementref{eq:f-invariant-frame}. Affine combination is one of the
% fundamental building blocks of affine geometry -- the properties of
% points invariant under invertible affine maps
% (\cite{gallier11geometric}, Chapter 2). If we drop the condition that
% $\lambda_1 + \lambda_2 = 1$, then we are dealing with linear
% combinations of vectors, and we are no longer invariant with respect
% to changes of frame. However, linear combinations are invariant with
% respect to change of basis. We can represent changes of basis as
% linear invertible maps $B : \mathbb{R}^2 \to \mathbb{R}^2$. The set of
% all such maps forms the general linear group $\GL(2)$. We now define
% another family of relations $\{R_{\texttt{vec}}(B) \subseteq
% \mathbb{R}^2 \times \mathbb{R}^2 \}_{B \in \GL(2)}$ that relates two
% points up to change of basis:
% \begin{displaymath}
%   R_{\texttt{vec}}(B) = \{ (\vec{v_1},\vec{v_2}) \sepbar B\vec{v_2} = \vec{v_1} \}.
% \end{displaymath}
% Now the functions $f_{\lambda_1\lambda_2}(\vec{v_1},\vec{v_2}) =
% \lambda_1\vec{v_1} + \lambda_2\vec{v_2}$, for arbitrary $\lambda_1$
% and $\lambda_2$, do preserve the relations $R_{\texttt{vec}}(B)$, for
% all $B \in \GL(2)$. Unfolding the definition of $R_{\texttt{vec}}(B)$
% in the analogous statement to \statementref{eq:f-preserve-rel-frame}
% for $R_{\texttt{vec}}$ instead of $R_{\texttt{pt}}$, we can see that
% preservation of the relations $R_{\texttt{vec}}$ characterises the
% functions that are invariant under change of basis. For all $B \in \GL(2)$, we have
% \begin{displaymath}
%   \forall \vec{v_1}, \vec{v_2}.\ f(B\vec{v_1},B\vec{v_2}) = B(f(\vec{v_1},\vec{v_2})),
% \end{displaymath}
% and this is exactly the property of invariance under change of basis
% we required above of programs manipulating vectors.

% Note that the family of relations $R_{\texttt{vec}}$ is just
% $R_{\texttt{pt}}$ when restricted to elements of the group
% $\GL(2)$. In the type system we introduce in the next section, we
% combine points and vectors into the same data type. Whether it
% represents a point or a vector depends on the group of geometric
% transformations that we expect it to be invariant under.

% FIXME: vectors are not invariant under change of origin, so they are
% represented by the relation $R_0$, which is just equality. But they,
% and points are invariant under change of basis. However, as we shall
% see below, not all operations are invariant under all changes of
% basis. In particular the dot product is only invariant under
% orthogonal transformations.

\subsection{A Type System for Change of Origin Invariance}
\label{sec:type-system-geom-intro}

Reynolds also showed how a type discipline can be used to establish
that (the denotational interpretations of) programs preserve
relations. For Reynolds, the type discipline of interest was that of
the polymorphic $\lambda$-calculus, which supports the construction of
new types by universal quantification over types.
%, in which new types can be
%constructed by universal quantification over all types.
In terms of relations, Reynolds interprets universal quantification
over types as quantification over binary relations between denotations
of types. By contrast, in our statements of geometric invariance in
\autoref{sec:type-system-geom-intro} we did not quantify over all
relations, but instead quantified over all changes of origin and used
a specific choice of origin to select a relation from the family
$\{R_{\vec{t}}\}$. This suggests introducing quantification over
changes of origin into the language of types. We use the notation
$\forall t \mathord: \SynTransl{2}. A$ for quantification over all
2-dimensional translations (i.e.,~choices of origin) $t$, and refer to
$\SynTransl{2}$ as the \emph{sort} of $t$. Note the difference in
fonts used to distinguish the semantic group $\Transl{2}$ from the
syntactic sort $\SynTransl{2}$. We use a similar convention
below, too.

Since the sort $\SynTransl{2}$ represents an abelian group, we can
combine its elements using the usual group operations. We write
operations additively, using $e_1 + e_2$ for the group operation, $-e$
for inverse and $0$ for the unit.  We also regard expressions built
from variables and the group operations up to the abelian group
axioms. For example, we regard $e_1 + (e_2 + e_3)$ and $(e_1 + e_2) +
e_3$ as equivalent.

Our language of types includes the unit type $\tyPrimNm{unit}$ and,
for all types $A$ and $B$, the function type $A \to B$, the sum type
$A + B$, and the tuple type $A \times B$. We also assume a primitive
type $\tyPrimNm{real}$, used to represent scalars. Although tuples of
real numbers represent points and vectors in geometric applications,
we cannot express this via the type
%central data structure in geometric applications is the tuple of real
%numbers for representing points and vectors, we cannot simply express
%this as the type 
$\tyPrimNm{real} \times \tyPrimNm{real}$. Indeed,
%because this type does not have the correct relational
%interpretation. (
two elements of type $\tyPrimNm{real}$ are related if and only if they
are equal and, by Reynolds' interpretation of tuple types, two
elements of $\tyPrimNm{real} \times \tyPrimNm{real}$ are also related
if and only if they are equal. But since this does not give the
correct relational interpretations for points and vectors, we
introduce a new type $\tyPrim{vec}{e}$, indexed by expressions $e$ of
sort $\SynTransl{2}$, to represent them.  The index $e$ represents the
displacement by change of origin of a point of this type.  Although we
have taken pains to distinguish geometric points and vectors, we use
the name $\tyPrimNm{vec}$ for both to recall the computer science
notion of vector as a homogeneous sequence of values with a known
length (in this case, 2).

As is standard for parametricity, every type has two interpretations:
an index-erasure interpretation that ignores the indexing expression,
and a relational interpretation as a binary relation on the
index-erasure interpretation. To give such interpretations for the
types $\tyPrim{vec}{e}$ and $\forall t\mathord:\SynTransl{2}.A$,
we assume for now that we can map each expression $e$ of sort
$\SynTransl{2}$ to an element $\sem{e}\rho$ of the group $\Transl{2}$
using some environment $\rho$ that interprets $e$'s free
variables. The (simplified) index-erasure and relation interpretations
of $\tyPrim{vec}{e}$ are:
\begin{displaymath}
  \begin{array}{l@{\hspace{0.5em}=\hspace{0.5em}}l}
    \tySem{\tyPrim{vec}{e}} & \mathbb{R}^2
    \\ \rsem{\tyPrim{vec}{e}}{}\rho & R_{\sem{e}\rho} = \{
    (\vec{v},\vec{v'}) \sepbar \vec{v'} = \vec{v} + \sem{e}\rho \}
  \end{array}
\end{displaymath}
The index-erasure and relational interpetations of 
$\forall t\mathord:\SynTransl{2}.A$ are:
%is simply the interpretation of the
%underlying type $A$. The relational interpretation intersects the
%relational interpretations of the type $A$ under all extensions of the
%relational environment $\rho$:
\begin{displaymath}
  \begin{array}{l@{\hspace{0.5em}=\hspace{0.5em}}l}
    \tySem{\forall t\mathord:\SynTransl{2}.A} & \tySem{A}
    \\ \rsem{\forall t\mathord:\SynTransl{2}.A}{}\rho & \bigcap\{
    \rsem{A}{}{\rho'} \sepbar \rho' \in
    \extends{\rho}{t\mathord:\SynTransl{2}} \}
  \end{array}
\end{displaymath}
The exact definition of $\extends{\rho}{t\mathord:\SynTransl{2}}$
depends on which (possibly non-compositional) interpretation of
relational environments we use. For now we can think of environments
as mappings from free variables of sort $\SynTransl{2}$ to the group
$\Transl{2}$, and of extension as adding extra components to
mappings. The index-erasure and relational interpretations are given
formally in Sections~\ref{sec:index-erasure-semantics}
%the relational interpretation is in general more complex, due to the
%possibility of non-compositional interpretations of free index
%variables. The relational interpretation will be presented in
and~\ref{sec:relational-semantics}. %, respectively.

At the end of \autoref{sec:oiri} we considered functions $f :
\mathbb{R}^2 \times \mathbb{R}^2 \to \mathbb{R}^2$ that preserve all
changes of origin. This property of $f$ can be expressed in terms of
types by
%\begin{displaymath}
$  \mathrm{f} : \forall t \mathord: \SynTransl{2}.\ \tyPrim{vec}{t}
  \times \tyPrim{vec}{t} \to \tyPrim{vec}{t}$.
%\end{displaymath}
Spelling out the relational interpretation of this type using the
definitions above and the standard relational interpretations for
tuple and function types, we recover
\statementref{eq:f-invariant-frame} exactly.

\subsection{Affine and Vector Operations}
\label{sec:affine-vector-ops}

Invariance under change of origin is the key feature of affine
geometry, whose central operation is the affine combination of points:
$\lambda_1\vec{v_1} + \lambda_2\vec{v_2}$, where $\lambda_1 +
\lambda_2 = 1$.  Geometrically, this can be interpreted as
interpolation between the points represented by $\vec{v_1}$ and
$\vec{v_2}$.  We add affine combination of points to our calculus as
follows: % typing and intended denotation:for
\begin{displaymath}
  \begin{array}{l}
    \mathrm{affComb} :\forall t \mathord:
    \SynTransl{2}.\ \tyPrim{vec}{t} \to \tyPrimNm{real} \to
    \tyPrim{vec}{t} \to \tyPrim{vec}{t} \\ 
    \sem{\mathrm{affComb}}\ \vec{v_1}\ r\ \vec{v_2} = (1-r)\vec{v_1} +
    r\vec{v_2} 
\end{array}
\end{displaymath}
It can be verified by hand that %the intended denotation
$\sem{\mathrm{affComb}}$ is invariant under all changes of origin, as
dictated by its type.
% By defining the primitive function $\mathrm{affComb}$ to take a
% single $\tyPrimNm{real}$ parameter $t$, we can easily ensure that we
% are taking the affine combination of two representatives of points,
% and not just an arbitrary linear combination.

\begin{example}
  The evaluation of quadratic B\'{e}zier curves (B\'{e}zier curves
  with two endpoints and a single control point) can be expressed
  using the affine combination primitive as follows: % and three steps
  % of De Casteljau's algorithm:
  \begin{displaymath}
    \begin{array}{@{}l}
      \mathrm{quadB\acute{e}zier} : \forall t
      \mathord:\SynTransl{2}.\ \tyPrim{vec}{t} \mathord\to \tyPrim{vec}{t}\mathord\to
      \tyPrim{vec}{t} \mathord\to \tyPrimNm{real} \mathord\to \tyPrim{vec}{t}
      \\ \mathrm{quadB\acute{e}zier}\ [t]\ p_0\ p_1\ p_2\ s = \\ \quad
      \mathrm{affComb}\ [t]\ (\mathrm{affComb}\ [t]\ p_0\ s\ p_1)\ s\ (\mathrm{affComb}\ [t]\ p_1\ s\ p_2)
      \\
    \end{array}
  \end{displaymath}
  For two endpoints $p_0$ and $p_2$, a control point $p_1$, and $s \in
  [0,1]$, an application
  $\mathrm{quadB\acute{e}zier}\ p_0\ p_1\ p_2\ s$ gives the point on
  the curve at ``time'' $s$.  The type of
  $\mathrm{quadB\acute{e}zier}$ immediately tells us that it preserves
  all changes of origin.
\end{example}
% Affine combination only provides us with an operation on points. We
% are also interested in operations on the vectors representing offsets
% between points. We now examine the correct types to assign to the
% vector space operations of addition of vectors, negation of vectors,
% multiplication by a scalar and the zero vector. The typings of these
% operations will make use of the abelian group structure of change of
% origin translations.

The obvious type for vector addition is
%\begin{displaymath}
$  (+) : \tyPrim{vec}{0} \to \tyPrim{vec}{0} \to \tyPrim{vec}{0}$.
%\end{displaymath}
%(we write binary operators intended to be used infix in parentheses
%when not appearing in infix position, following the Haskell
%syntax.)
But we can reflect the fact that $(+)$ is not invariant under change
of origin by giving it a more precise type that reflects how it varies
with change of origin:
\begin{displaymath}
  (+) : \forall t_1, t_2 \mathord: \SynTransl{2}.\ \tyPrim{vec}{t_1}
  \to \tyPrim{vec}{t_2} \to \tyPrim{vec}{t_1 + t_2}
\end{displaymath}
Intuitively, this type says that if the first input vector has been
displaced by $t_1$ and the second by $t_2$, then their sum is
displaced by $t_1 + t_2$. We can also negate vectors, yielding a
vector which points in the opposite direction. Negation negates
translation arguments:
\begin{displaymath}
  \mathrm{negate} : \forall t \mathord:
  \SynTransl{2}.\ \tyPrim{vec}{t} \to \tyPrim{vec}{-t}
\end{displaymath}
Finally, with the primitive operations of addition and negation of
vectors we can define the derived operation of subtraction:
\begin{displaymath}
  \begin{array}{l}
    (-) : \forall t_1,t_2 \mathord:\SynTransl{2}.\ \tyPrim{vec}{t_1}
    \to \tyPrim{vec}{t_2} \to \tyPrim{vec}{t_1 - t_2}
    \\ (-)\ [t_1]\ [t_2]\ p_1\ p_2 = p_1 + \mathrm{negate}\ p_2
  \end{array}
\end{displaymath}

Given two points that are invariant with respect to the same change of
origin---i.e.~two values of type $\tyPrim{vec}{t}$---we can use
subtraction to compute their offset:
\begin{displaymath}
  \begin{array}{l}
    \mathrm{offset} : \forall t
    \mathord:\SynTransl{2}.\ \tyPrim{vec}{t} \to \tyPrim{vec}{t} \to
    \tyPrim{vec}{0} \\ \mathrm{offset}\ [t]\ p_1\ p_2 = p_1 - p_2
  \end{array}
\end{displaymath}
The result is a vector expressed with respect to the null change of
origin: note how the algebraic structure on the indexing theory
induces type equalities that can be used to simplify the type of the
result of $\mathrm{offset}$ from $\tyPrim{vec}{t-t}$ to
$\tyPrim{vec}{0}$. The type of $(+)$ can also be specialised to the
case of moving a point by a vector:
\begin{displaymath}
  \begin{array}{l}
    \mathrm{moveBy} : \forall t
    \mathord:\SynTransl{2}.\ \tyPrim{vec}{t} \to \tyPrim{vec}{0} \to
    \tyPrim{vec}{t} \\ \mathrm{moveBy}\ [t]\ p\ v = p + v
  \end{array}
\end{displaymath}
The types we assign to the remaining vector space primitives, namely
$\mathrm{0} : \tyPrim{vec}{0}$ for
the zero vector and $(*) : \tyPrimNm{real} \to
    \tyPrim{vec}{0} \to \tyPrim{vec}{0}$ for
multiplication by a scalar, do not describe any
interesting effects on translations.
%\begin{displaymath}
%  \begin{array}{l@{\hspace{0.5em}:\hspace{0.5em}}l@{\hspace{4em}}l@{\hspace{0.5%em}:\hspace{0.5em}}l}
%    \mathrm{0} & \tyPrim{vec}{0}& (*) & \tyPrimNm{real} \to
%    \tyPrim{vec}{0} \to \tyPrim{vec}{0}
%  \end{array}
%\end{displaymath}

% \fixme{If we had dependent types, could the affComb operation be
%   derived from an appropriately dependently typed scaling operation?}

\begin{example}\label{ex:type-iso}
  The vector space operators and the properties that follow from their
  types allow us to establish a useful type isomorphism. Consider
  functions with types following the schema:
  \begin{displaymath}
    \tau_{n} \isDefinedAs \forall
    t\mathord:\SynTransl{2}.\ \underbrace{\tyPrim{vec}{t} \to ... \to
      \tyPrim{vec}{t}}_{n+1\textrm{ times}} \to \tyPrimNm{real}
  \end{displaymath}
  Just by looking at the types $\tau_{n}$, we know that their
  inhabitants will be invariant under change of origin because of the
  quantification over all $t$ in $\SynTransl{2}$. So we may as well
  choose one of the input points as the origin and assume that all the
  other points are defined with respect to it.  This formalises the
  common mathematical practice of stating that ``without loss of
  generality'' we can take some point in a description of a problem to
  be the origin provided the problem statement is invariant under
  translation. Each type $\tau_{n}$ is isomorphic to the corresponding
  type $\sigma_{n}$:
  \begin{displaymath}
    \sigma_{n} \isDefinedAs \underbrace{\tyPrim{vec}{0} \to
      ... \tyPrim{vec}{0}}_{n\textrm{ times}} \to \tyPrimNm{real}
  \end{displaymath}
  We demonstrate these isomorphisms formally in
  \autoref{sec:instantiations}, in the more general setting of types
  indexed by abelian groups.
  % The isomorphism between these two families of types can be witnessed
  % by the following two functions.
  % \begin{displaymath}
  %   \begin{array}{l}
  %     i_{n,A} : \tau_{n,A} \to \sigma_{n,A} \\
  %     i_{n,A}\ f\ [B]\ (p_1, p_2, ..., p_n) = f\ [B \ltimes 0]\ (0, p_1, ..., p_n) \\
  %     \\
  %     i^{-1}_{n,A} : \sigma_{n,A} \to \tau_{n,A} \\
  %     i^{-1}_{n,A}\ g\ [F]\ (p_0, ..., p_n) = g\ [\pi_L(F)]\ (p_1-p_0, ..., p_n-p_0)
  %   \end{array}
  % \end{displaymath}
  % The direction defined by the function $i_{n,A}$ treats the inputs
  % $p_1,...,p_n$ as vector offsets from the origin $0$. The direction
  % witnessed by $i^{-1}_{n,A}$ picks the first point to act as the
  % origin, and uses the operation of vector subtraction we defined
  % above to turn each of the other points into offsets from this
  % point. Note that this isomorphism is not unique: for each $n$ we can
  % pick any of the $n+1$ inputs to act as the distinguished
  % origin. Thus the families of types $\tau_{n,A}$ and $\sigma_{n,A}$
  % are isomorphic, but not uniquely isomorphic.
\end{example}

\begin{example}\label{ex:uninhabited-type}
  So far we have emphasised the derivation of properties, or ``free
  theorems'', of programs from their types. But using more refined
  relational interpretations of types we can also show that certain
  types are uninhabited. For example, the type $\forall t \mathord:
  \SynTransl{2}.\ \tyPrim{vec}{t + t} \to \tyPrim{vec}{t}$ has no
  quantifier-free inhabitants\footnote{We define quantifier-free
    programs in the statement of
    \thmref{thm:abstraction}.}. Intuitively, this is because we cannot
  remove the extra $t$ in $\tyPrim{vec}{t+t}$ using the
  vector operations. We formalise this indefinability result in
  \autoref{sec:types-indexed-abelian-groups-indef} using a more
  sophisticated relational interpretation of free translation
  variables than the one sketched above.
\end{example}

\subsection{Change of Basis Invariance}
\label{sec:motivation-generalising}

Although vector addition, negation, and scaling are not invariant
under change of origin, they are invariant under change of basis. As
with origin invariance, we can express basis invariance as
preservation of relations indexed by changes of basis. Change of basis
is achieved by applying an invertible linear map, and the collection of
all such maps on $\mathbb{R}^2$ forms the \emph{General Linear} group
$\GL{2}$ which we represent in our language by a new indexing sort
$\SynGL{2}$ with (non-abelian) group structure that we will write
multiplicatively. We then extend $\tyPrimNm{vec}$ to allow indices of
sort $\SynGL{2}$, as well as $\SynTransl{2}$, so that
$\tyPrim{vec}{B,t}$ is a vector that varies with change of basis $B$
and change of origin $t$. Formally, the index-erasure and (simplified)
relational semantics of $\tyPrim{vec}{B,t}$ are given by:
\begin{displaymath}
  \begin{array}{l@{\hspace{0.5em}=\hspace{0.5em}}l}
    \tySem{\tyPrim{vec}{e_B,e_t}} & \mathbb{R}^2
    \\ \rsem{\tyPrim{vec}{e_B,e_t}}{}\rho & \{ (\vec{v},\vec{v'})
    \sepbar \vec{v'} = (\sem{e_B}\rho)\vec{v} + \sem{e_t}\rho \}
  \end{array}
\end{displaymath}

\paragraph{Affine Geometry} An \emph{affine transformation} 
an invertible linear map together with a translation. We can assign
types to all the primitive affine and vector space operations
indicating how they they behave with respect to affine
transformations:
\begin{eqnarray*}
  \mathrm{affComb} & : &
  \begin{array}[t]{@{}l}
    \forall B \mathord: \SynGL{2}, t \mathord: \SynTransl{2}.\\
    \hspace{0.2cm} \tyPrim{vec}{B,t} \to \tyPrimNm{real} \to
    \tyPrim{vec}{B,t} \to \tyPrim{vec}{B,t}
  \end{array}
  \\
  (+) & : &
  \begin{array}[t]{@{}l}
    \forall B \mathord: \SynGL{2}, t_1,t_2 \mathord: \SynTransl{2}. \\
    \hspace{0.2cm}\tyPrim{vec}{B,t_1} \to \tyPrim{vec}{B,t_2} \to
    \tyPrim{vec}{B,t_1 + t_2}
  \end{array}
  \\
  \mathrm{negate} & : & \forall B \mathord: \SynGL{2}, t \mathord: \SynTransl{2}.\ \tyPrim{vec}{B,t} \to \tyPrim{vec}{B,-t} \\
  0 & : & \forall B \mathord: \SynGL{2}.\ \tyPrim{vec}{B,0} \\
  (*) & : & \forall B \mathord: \SynGL{2}.\ \tyPrimNm{real} \to \tyPrim{vec}{B,0} \to \tyPrim{vec}{B,0}
\end{eqnarray*}

\paragraph{Euclidean Geometry} Euclidean geometry extends affine
geometry with the \emph{dot product}, or \emph{inner product},
operation of two vectors. The dot product is defined 
%on the coordinate representation 
by
%\begin{displaymath}
$  (x_1,y_1) \cdot (x_2,y_2) = x_1x_2 + y_1y_2$.
%\end{displaymath}
% The dot product is used to define properties of vectors such as their
% \emph{norm} $||\vec{v}|| = \vec{v}\cdot\vec{v}$, which is the square
% of the length of the vector $\vec{v}$, and the notion of
% orthogonality: two vectors $\vec{v_1}$ and $\vec{v_2}$ are orthogonal
% if $\vec{v_1}\cdot\vec{v_2} = 0$.
To %incorporate the operation of dot product into our calculus, we
must assign it a type we note that, although dot product is not
invariant under $\GL{2}$ or $\Transl{2}$, it
% that we have considered so far. However,e  the dot product 
is invariant under the subgroup $\Orth{2}$ of $\GL{2}$ of
\emph{orthogonal} linear transformations, i.e., the subgroup of
invertible linear maps whose matrix representations' transposes are
equal to their inverses. We thus introduce a new sort $\SynOrth{2}$ of
orthogonal transformations, and overload the multiplicative group
operations for inhabitants of $\SynOrth{2}$. Further assuming a
function $\iota_O$ that takes $e : \SynOrth{2}$ to $\iota_O(e) :
\SynGL{2}$ we assign dot product this type:
%a type to the dot product, using quantification over $\SynOrth{2}$:
\begin{displaymath}
  (\cdot) : \forall O \mathord: \SynOrth{2}.\ \tyPrim{vec}{\iota_OO, 0} \to \tyPrim{vec}{\iota_OO, 0} \to \tyPrimNm{real}
\end{displaymath}

\begin{figure}[t]
  \centering
  \begin{eqnarray*}
    0   &:& \forall s \mathord:\SynGL{1}.\ \tyPrim{real}{s} \\
    1   &:& \tyPrim{real}{1} \\
    (+) &:& \forall s \mathord:\SynGL{1}.\ \tyPrim{real}{s} \to \tyPrim{real}{s} \to \tyPrim{real}{s} \\
    (-) &:& \forall s \mathord:\SynGL{1}.\ \tyPrim{real}{s} \to \tyPrim{real}{s} \to \tyPrim{real}{s} \\
    (*) &:& \forall s_1,s_2 \mathord:\SynGL{1}.\ \tyPrim{real}{s_1} \to \tyPrim{real}{s_2} \to \tyPrim{real}{s_1s_2} \\
    (/) &:&
    \begin{array}[t]{@{}l@{}l}
      \forall s_1,s_2 \mathord:\SynGL{1}.\ \tyPrim{real}{s_1}\ & \to \tyPrim{real}{s_2} \\
      &\to \tyPrim{real}{s_1s_2^{-1}} + \tyPrimNm{unit} \\
    \end{array}\\
    \mathrm{abs} &:& \forall s \mathord:\SynGL{1}.\ \tyPrim{real}{s} \to \tyPrim{real}{|s|} %\\
%    \mathrm{sqrt} &:& \tyPrim{real}{1} \to \tyPrim{real}{1}
  \end{eqnarray*}
  \caption{Operations on scaled real numbers}
  \label{fig:real-ops}
\end{figure}

The cross product of two vectors is defined on coordinate
representations as $(x_1,y_1) \times (x_2,y_2) = x_1y_2 - x_2y_1$.
Geometrically, the cross product is the signed area of the
parallelogram described by the pair of input vectors.  Under change of
basis by an invertible linear transformation $B$, the cross product of
two vectors varies with the determinant of $B$. This corresponds to
scaling the plane by the change of basis transformation, so we augment
our calculus with a new sort $\SynGL{1}$ of scale factors
(i.e.,~$1$-dimensional invertible linear maps). Semantically,
$\SynGL{1}$ ranges over the non-zero real numbers and forms an abelian
group which we write multiplicatively. We also add two new operations:
determinant, $\det B$, which takes an inhabitant of $\SynGL{2}$ to its
determinant in $\SynGL{1}$, and absolute value, $|e|$, which takes
scaling factors to scaling factors. We also refine the type
$\tyPrimNm{real}$ of real numbers so that it is indexed by the sort
$\SynGL{1}$: $\tyPrim{real}{e}$. The old type $\tyPrimNm{real}$ is
then just $\tyPrim{real}{1}$, and the full collection of operations on
real numbers indexed by scaling factors is shown in
\autoref{fig:real-ops}. We can thus assign cross product the type:
%With the additional sort $\SynGL{1}$ and the refined type of real
%numbers, we can assign a type to the cross product:
\begin{displaymath}
  (\times) : \forall B \mathord: \SynGL{2}.\ \tyPrim{vec}{B,0} \to
  \tyPrim{vec}{B,0} \to \tyPrim{real}{\det B} 
\end{displaymath}
Since the absolute value of the determinant of an orthogonal
transformation is always $1$, we assume $|\det (\iota_O O)| = 1$ to
hold.

\begin{example}\label{ex:area-of-triangle-1}
  We can use the operations of this subsection to
  compute the area of a triangle. We have:
  \begin{displaymath}
    \begin{array}{@{}l}
      \mathrm{area} : \forall B\mathord:\SynGL{2},
      t\mathord:\SynTransl{2}.\ \\
      \hspace{0.8cm}\tyPrim{vec}{B, t} \to \tyPrim{vec}{B, t} \to
      \tyPrim{vec}{B, t} \to \tyPrim{real}{|\det B|}
      \\ \mathrm{area}\ [B]\ [t]\ p_1\ p_2\ p_3 = \frac{1}{2} *
      \mathrm{abs}\ ((p_2 - p_1) \times (p_3 - p_1))
    \end{array}
  \end{displaymath}
  The calculation is performed in several steps, each of which removes
  some of the symmetry described by the type of
  $\mathrm{area}$. First, the two offset vectors $p_2 - p_1$ and $p_3
  - p_1$ are computed. These operations remove the effect of
  translations on the result in exactly the same way as the type
  isomorphism in \exref{ex:type-iso}. Next, we compute the cross
  product of the two vectors is computed, which gives the area of the
  parallelogram described by the sides of the triangle and has type
  $\tyPrim{real}{\det B}$. This removes some of the symmetry due to
  invertible linear maps, but the cross product still varies with the
  sign of the determinant. We remove this symmetry as well using
  $\mathrm{abs}$. This gives a value of type $\tyPrim{real}{|\det B|}$
  which we multiply by $\frac{1}{2}$ to recover the area of the
  triangle rather than that of the whole parallelogram. If we
  specialise $\mathrm{area}$ to just orthogonal transformations, the
  assumption $|\det (\iota_OO)| = 1$ gives the following type:
  \begin{displaymath}
    \begin{array}{@{}l}
      \mathrm{area} : \forall O\mathord:\SynOrth{2},
      t\mathord:\SynTransl{2}.\ \\
      \hspace{0.8cm}\tyPrim{vec}{\iota_OO, t} \to
      \tyPrim{vec}{\iota_OO, t} \to \tyPrim{vec}{\iota_OO, t} \to
      \tyPrim{real}{1}
    \end{array}
  \end{displaymath}
  This type shows that the area of a triangle is invariant under
  orthogonal transformations and translations. Combinations of such
  transformations are {\em isometries}, i.e., distance preserving
  maps.
\end{example}

% \begin{enumerate}
% \item Possibly do a $n$-body gravity simulator as an example?
% \item Have a special sort and type for rotations. An operation for
%   computing the rotation between two points, and for applying a
%   rotation to a vector. This will allow for more free theorems...
% \end{enumerate}

\subsection{Scale Invariance and Dimensional Analysis}
\label{sec:scale-invariance}

Indexing types by scaling factors brings us to the original
inspiration for the current work: Kennedy's interpretation of his
units of measure type system via scaling invariance
\cite{kennedy97relational}. Kennedy shows how interpreting types in
terms of scaling invariance brings the techniques of dimensional
analysis to bear on programming. The types of the real number
arithmetic operations in \autoref{fig:real-ops} are exactly the types
Kennedy assigns in his units of measure system, except for that of the
absolute value operation. Semantically, our type indexes by
%our assigned type is that, semantically, we are indexing by
non-zero scaling factors, whereas Kennedy's indexes by strictly
positive ones. We examine the typing of the absolute value operation
further in \autoref{sec:types-indexed-abelian-groups}.

In our two-dimensional setting we can add to Kennedy's one-dimensional
scaling invariance an operation $\iota_1$ that, semantically, takes
scale factors in $\GL{1}$ to invertible linear maps in $\GL{2}$,
i.e., takes numbers $s$ to matrices $\left(
  \begin{smallmatrix}s & 0 \\ 0 & s\end{smallmatrix}\right)$.  This
operation satisfies the equation $\det (\iota_1 s) = s^2$, indicating
that scaling the plane by $s$ in both directions scales areas by
$s^2$. %, as we now illustrate.

\begin{example}\label{ex:area-of-triangle-2}
  Just as we specialised the type of the $\mathrm{area}$ function to
  orthogonal transformations in \exref{ex:area-of-triangle-1}, we can
  also specialise $\mathrm{area}$'s type to scaling
  transformations. This yields the type:
  \begin{displaymath}
    \begin{array}{@{}l}
      \mathrm{area} : \forall s\mathord:\SynGL{1}, t\mathord:\SynTransl{2}.\ \\
      \hspace{0.8cm} \tyPrim{vec}{\iota_1s, t} \to \tyPrim{vec}{\iota_1s, t} \to \tyPrim{vec}{\iota_1s, t} \to \tyPrim{real}{s^2}
    \end{array}
  \end{displaymath}
  As expected, the area of a triangle varies with the square of
  scalings of the plane, and this is reflected in the type.
\end{example}

Linear maps of the form $\left(
  \begin{smallmatrix}s & 0 \\ 0 & s\end{smallmatrix}\right)$, as
generated by $\iota_1$, commute with all other invertible linear
maps. We thus require
$(\iota_1 s)B = B(\iota_1 s)$ to hold.
The scaling maps $\left(
  \begin{smallmatrix}s & 0 \\ 0 & s\end{smallmatrix}\right)$
are precisely the elements of $\GL{2}$ that commute with all others;
these form the \emph{centre} of $\GL{2}$. If we keep track of
scalings, then we can assign the following more precise types to
scalar multiplcation and dot product:
\begin{displaymath}
\begin{array}{@{}l@{}l}
  (*) : \forall s \mathord: \SynGL{1}, B \mathord:
  \SynGL{2}.\ & \tyPrim{real}{s} \to \tyPrim{vec}{B,0} \to
  \tyPrim{vec}{\iota_1(s)B,0}\\
%\end{displaymath}
%and dot product:
%\begin{displaymath}
%  \begin{array}{l@{}l}
    (\cdot) : \forall s \mathord: \SynGL{1}, O \mathord: \SynOrth{2}.\ & \tyPrim{vec}{\iota_1(s)\iota_O(O),0} \to\\
    &\tyPrim{vec}{\iota_1(s)\iota_O(O),0} \to \tyPrim{real}{s^2}
  \end{array}
\end{displaymath}

\begin{example}
  With the operations in \autoref{fig:real-ops}, it is not possible to
  write a quantifier-free program with the following type that is not
  constantly zero:
  \begin{displaymath}
    \forall s \mathord: \SynGL{2}.\ \tyPrim{real}{s^2} \to \tyPrim{real}{s}
  \end{displaymath}
  This was shown by Kennedy for his units of measure system
  \cite{kennedy97relational}.  In particular, it is not possible to
  write a square root function with the above type.  The
  indefinability of square root is similar to the uninhabitation of the
  type in \exref{ex:uninhabited-type}.

  In \autoref{sec:types-indexed-abelian-groups-indef} we revisit
  Kennedy's proof and show that even if we add square root as a
  primitive operation---with the type above---then it is still not
  possible to construct the cube root function. The indefinability of
  cube root is related to the impossibility of trisecting an arbitrary
  angle by ruler and compass constructions.
%  is of interest due to its relevance to the classical problem of
%  trisecting an angle by ruler and compass constructions.
\end{example}


%%% Local Variables:
%%% TeX-master: "paper"
%%% End:

