\section{Proofs for \autoref{sec:a-general-framework}}

This appendix includes proofs for some of the lemmas and theorems in
the main text. For reference, we include the rules for judgmental type
equality here:
\begin{mathpar}
  \inferrule* [right=TyEqPrim]
  {\primTyArity(\tyPrimNm{X}) = (s_1,...,s_n) \\
    \{\Delta \vdash e_i \equiv e'_i : s_i\}_{1\leq i \leq n}}
  {\Delta \vdash \tyPrim{X}{e_1,...,e_n} \equiv \tyPrim{X}{e'_1,..,e'_n} \isType}
    
  \inferrule* [right=TyEqUnit]
  { }
  {\Delta \vdash \tyUnit \equiv \tyUnit \isType}
  
  \inferrule* [right=TyEqArr]
  {\Delta \vdash A \equiv A' \isType \\ \Delta \vdash B \equiv B' \isType}
  {\Delta \vdash A \tyArr B \equiv A' \tyArr B' \isType}
  
  \inferrule* [right=TyEqProd]
  {\Delta \vdash A \equiv A' \isType \\ \Delta \vdash B \equiv B' \isType}
  {\Delta \vdash A \tyProduct B \equiv A' \tyProduct B' \isType}
  
  \inferrule* [right=TyEqSum]
  {\Delta \vdash A \equiv A' \isType \\ \Delta \vdash B \equiv B' \isType}
  {\Delta \vdash A + B \equiv A' + B' \isType}
  
  \inferrule* [right=TyEqForall]
  {\Delta, i \mathord: s \vdash A \equiv A' \isType}
  {\Delta \vdash \forall i \mathord: s. A \equiv \forall i \mathord: s. A' \isType}

  \inferrule* [right=TyEqForall]
  {\Delta, i \mathord: s \vdash A \equiv A' \isType}
  {\Delta \vdash \exists i \mathord: s. A \equiv \exists i \mathord: s. A' \isType}
\end{mathpar}

\begin{restateLemma}[\lemref{lem:tyeqsubst-erasure}]
  \begin{enumerate}
  \item If $\Delta \vdash A \equiv B \isType$ then $\tySem{A} =
    \tySem{B}$; and
  \item If $\Delta \vdash A \isType$ and $\sigma : \Delta' \Rightarrow
    \Delta$, then $\tySem{\sigma^*A} = \tySem{A}$.
  \end{enumerate}
\end{restateLemma}
\begin{proof}
  \begin{enumerate}
  \item By induction on the derivation of $\Delta \vdash A \equiv B
    \isType$:
    \begin{description}
    \item[Case \TirName{TyEqUnit}] In this case, $A = B = \tyUnit$, so
      $\tySem{A} = \tySem{B}$.
    \item[Case \TirName{TyEqPrim}] We have $A =
      \tyPrim{X}{e_1,...,e_n}$ and $B =
      \tyPrim{X}{e'_1,...,e'_n}$. The erasure semantics ignores the
      index arguments, so $\tySem{A} = \tySem{\tyPrim{X}{e_1,...,e_n}}
      = \tyPrimSem{X} = \tySem{\tyPrim{X}{e'_1,...,e'_n}} =
      \tySem{B}$, as required.
    \item[Case \TirName{TyEqArr}] Here, $A = A_1 \tyArr A_2$ and $B =
      B_1 \tyArr B_2$, and by the induction hypothesis we know that
      $\tySem{A_1} = \tySem{B_1}$ and $\tySem{A_2} =
      \tySem{B_2}$. Hence $\tySem{A} = \tySem{A_1 \tyArr A_2} =
      \tySem{A_1} \to \tySem{A_2} = \tySem{B_1} \to \tySem{B_2} =
      \tySem{B_1 \tyArr B_2} = \tySem{B}$.
    \item[Case \TirName{TyEqProd}] In this case, $A = A_1 \tyProduct
      A_2$ and $B = B_1 \tyProduct B_2$, and by the induction
      hypothesis we know that $\tySem{A_1} = \tySem{B_1}$ and
      $\tySem{A_2} = \tySem{B_2}$. Hence $\tySem{A} = \tySem{A_1
        \tyProduct A_2} = \tySem{A_1} \times \tySem{A_2} = \tySem{B_1}
      \times \tySem{B_2} = \tySem{B_1 \tyProduct B_2} = \tySem{B}$.
    \item[Case \TirName{TyEqSum}] In this case, $A = A_1 + A_2$ and $B
      = B_1 + B_2$, and by the induction hypothesis we know that
      $\tySem{A_1} = \tySem{B_1}$ and $\tySem{A_2} =
      \tySem{B_2}$. Hence $\tySem{A} = \tySem{A_1 + A_2} = \tySem{A_1}
      + \tySem{A_2} = \tySem{B_1} + \tySem{B_2} = \tySem{B_1 + B_2} =
      \tySem{B}$.
    \item[Case \TirName{TyEqForall}] We have $A = \forall
      i\mathord:s.A'$ and $B = \forall i\mathord:s.B'$. By the
      induction hypothesis we know that $\tySem{A'} = \tySem{B'}$, and
      hence $\tySem{A} = \tySem{\forall i\mathord:s. A'} = \tySem{A'}
      = \tySem{B'} = \tySem{\forall i\mathord:s. B'} = \tySem{B}$.
    \item[Case \TirName{TyEqExist}] We have $A = \exists
      i\mathord:s.A'$ and $B = \exists i\mathord:s.B'$. By the
      induction hypothesis we know that $\tySem{A'} = \tySem{B'}$, and
      hence $\tySem{A} = \tySem{\exists i\mathord:s. A'} = \tySem{A'}
      = \tySem{B'} = \tySem{\exists i\mathord:s. B'} = \tySem{B}$.
    \end{description}
  \item By induction on the derivation of $\Delta \vdash A \isType$.
    \begin{description}
    \item[Case \TirName{TyUnit}] Directly from the definition of
      substitution on types of the form $\tyUnit$, we have
      $\tySem{\sigma^*\tyUnit} = \tySem{\tyUnit}$, as required.
    \item[Case \TirName{TyPrim}] Since the index-erasure semantics
      ignores index expressions:
      $\tySem{\sigma^*\tyPrim{X}{e_1,...,e_n}} =
      \tySem{\tyPrim{X}{\sigma^*e_1,...,\sigma^*e_n}} = \tyPrimSem{X}
      = \tySem{\tyPrim{X}{e_1,...,e_n}}$.
    \item[Case \TirName{TyArr}] We have $A = A_1 \tyArr A_2$, and by
      the induction hypothesis, we know that $\tySem{\sigma^*A_1} =
      \tySem{A_1}$ and $\tySem{\sigma^*B_1} = \tySem{B_1}$. Thus, by
      the definition of substitution on arrow types, we have
      $\tySem{\sigma^*(A_1 \tyArr A_2)} = \tySem{\sigma^*A_1 \tyArr
        \sigma^*A_2} = \tySem{\sigma^*A_1} \to \tySem{\sigma^*A_2} =
      \tySem{A_1} \to \tySem{A_2} = \tySem{A_1 \tyArr A_2}$, as
      required.
    \item[Case \TirName{TyProd}] We have $A = A_1 \tyProduct A_2$, and
      by the induction hypothesis, we know that $\tySem{\sigma^*A_1} =
      \tySem{A_1}$ and $\tySem{\sigma^*A_2} = \tySem{A_2}$. Thus, by
      the definition of substitution on product types, we have
      $\tySem{\sigma^*(A_1 \tyProduct A_2)} = \tySem{\sigma^*A_1
        \tyProduct \sigma^*A_2} = \tySem{\sigma^*A_1} \times
      \tySem{\sigma^*A_2} = \tySem{A_1} \times \tySem{A_2} =
      \tySem{A_1 \tyProduct A_2}$, as required.
    \item[Case \TirName{TySum}] We have $A = A_1 + A_2$, and by the
      induction hypothesis, we know that $\tySem{\sigma^*A_1} =
      \tySem{A_1}$ and $\tySem{\sigma^*A_2} = \tySem{A_2}$. Thus, by
      the definition of substitution on sum types, we have
      $\tySem{\sigma^*(A_1 + A_2)} = \tySem{\sigma^*A_1 + \sigma^*A_2}
      = \tySem{\sigma^*A_1} + \tySem{\sigma^*A_2} = \tySem{A_1} +
      \tySem{A_2} = \tySem{A_1 + A_2}$, as required.
    \item[Case \TirName{TyForall}] We have $A = \forall
      i\mathord:s.A'$, and by the induction hypothesis we know that,
      for all $\sigma'$, $\tySem{\sigma'^*A'} =
      \tySem{A'}$. Therefore, from the definition of substitution for
      $\forall$-types, we have $\tySem{\sigma^*\forall i\mathord:s.A'}
      = \tySem{\forall i\mathord:s.\sigma_{i\mathord:s}^{*}A'} =
      \tySem{\sigma_{i\mathord:s}^{*}A'} = \tySem{A'} = \tySem{\forall
        i\mathord:s.A'}$, where we have relied on the fact that the
      index-erasure semantics of types ignores the index
      quantification.
    \item[Case \TirName{TyExist}] We have $A = \exists
      i\mathord:s.A'$, and by the induction hypothesis we know that,
      for all $\sigma'$, $\tySem{\sigma'^*A'} =
      \tySem{A'}$. Therefore, from the definition of substitution for
      $\exists$-types, we have $\tySem{\sigma^*\exists i\mathord:s.A'}
      = \tySem{\exists i\mathord:s.\sigma_{i\mathord:s}^{*}A'} =
      \tySem{\sigma_{i\mathord:s}^{*}A'} = \tySem{A'} = \tySem{\exists
        i\mathord:s.A'}$, where we have relied on the fact that the
      index-erasure semantics of types ignores the index
      quantification.
    \end{description}
  \end{enumerate}
\end{proof}

\begin{restateLemma}[\lemref{lem:tyeqsubst-relational}]
  \begin{enumerate}
  \item If $\Delta \vdash A \equiv B \isType$, then for all $\rho \in
    \relEnv{E}(\Delta)$, $\rsem{A}{\relEnv{E}}{\rho} =
    \rsem{B}{\relEnv{E}}{\rho}$;
  \item If $\Delta' \vdash A \isType$ and $\sigma : \Delta \Rightarrow
    \Delta'$, then for all $\rho \in \relEnv{E}(\Delta)$,
    $\rsem{\sigma^*A}{\relEnv{E}}\rho = \rsem{A}{\relEnv{E}}(\rho
    \circ \sigma)$.
  \end{enumerate}
\end{restateLemma}
\begin{proof}
  \begin{enumerate}
  \item By induction on the derivation of $\Delta \vdash A \equiv B
    \isType$.
    \begin{description}
    \item[Case \TirName{TyEqUnit}] In this case $A = B = \tyUnit$, so
      the statement is trivially statisfied.
    \item[Case \TirName{TyEqPrim}] In this case we have $A =
      \tyPrim{X}{e_1,...,e_n}$ and $B = \tyPrim{X}{e'_1,...,e'_n}$,
      with $\Delta \vdash e_j \equiv e'_j : s_j$. By ou requirement
      that relation environments preserve equality of simultaneous
      substitutions we have that $\rho_{\tyPrimNm{X}}(e_1,...,e_n) =
      \rho_{\tyPrimNm{X}}(e'_1,...,e'_n)$ and hence
      $\rsem{\tyPrim{X}{e_1,...,e_n}}{\relEnv{E}}\rho =
      \rsem{\tyPrim{X}{e'_1,...,e'_n}}{\relEnv{E}}\rho$.
    \item[Case \TirName{TyEqArr}] We have $A = A_1 \tyArr A_2$ and $B
      = B_1 \tyArr B_2$, with $\Delta \vdash A_1 \equiv B_1 \isType$
      and $\Delta \vdash A_2 \equiv B_2 \isType$. By the induction
      hypothesis, we know that $\rsem{A_1}{\relEnv{E}}\rho =
      \rsem{B_1}{\relEnv{E}}\rho$ and $\rsem{A_2}{\relEnv{E}}\rho =
      \rsem{B_2}{\relEnv{E}}\rho$. Thus we may reason as follows:
      $\rsem{A_1 \tyArr A_2}{\relEnv{E}}\rho =
      \rsem{A_1}{\relEnv{E}}\rho \relArrow \rsem{A_2}{\relEnv{E}}\rho
      = \rsem{B_1}{\relEnv{E}}\rho \relArrow
      \rsem{B_2}{\relEnv{E}}\rho = \rsem{B_1 \tyArr
        B_2}{\relEnv{E}}\rho$, as required.
    \item[Case \TirName{TyEqProd}] We have $A = A_1 \tyProduct A_2$
      and $B = B_1 \tyProduct B_2$, with $\Delta \vdash A_1 \equiv B_1
      \isType$ and $\Delta \vdash A_2 \equiv B_2 \isType$. By the
      induction hypothesis, we know that $\rsem{A_1}{\relEnv{E}}\rho =
      \rsem{B_1}{\relEnv{E}}\rho$ and $\rsem{A_2}{\relEnv{E}}\rho =
      \rsem{B_2}{\relEnv{E}}\rho$. Therefore, we can reason as
      follows: $\rsem{A_1 \tyProduct A_2}{\relEnv{E}}\rho =
      \rsem{A_1}{\relEnv{E}}\rho \relTimes \rsem{A_2}{\relEnv{E}}\rho
      = \rsem{B_1}{\relEnv{E}}\rho \relTimes
      \rsem{B_2}{\relEnv{E}}\rho = \rsem{B_1 \tyProduct
        B_2}{\relEnv{E}}\rho$, as required.
    \item[Case \TirName{TyEqSum}] We have $A = A_1 + A_2$ and $B = B_1
      + B_2$, with $\Delta \vdash A_1 \equiv B_1 \isType$ and $\Delta
      \vdash A_2 \equiv B_2 \isType$. By the induction hypothesis, we
      know that $\rsem{A_1}{\relEnv{E}}\rho =
      \rsem{B_1}{\relEnv{E}}\rho$ and $\rsem{A_2}{\relEnv{E}}\rho =
      \rsem{B_2}{\relEnv{E}}\rho$. Therefore, we can reason as
      follows: $\rsem{A_1 + A_2}{\relEnv{E}}\rho =
      \rsem{A_1}{\relEnv{E}}\rho \relSum \rsem{A_2}{\relEnv{E}}\rho =
      \rsem{B_1}{\relEnv{E}}\rho \relSum \rsem{B_2}{\relEnv{E}}\rho =
      \rsem{B_1 + B_2}{\relEnv{E}}\rho$, as required.
    \item[Case \TirName{TyEqForall}] We have $A = \forall
      i\mathord:s. A'$ and $B = \forall i\mathord:s.B'$, with $\Delta,
      i \mathord: s \vdash A' \equiv B' \isType$. By the induction hypothesis,
      we know that, for all $\rho'$, $\rsem{A'}{\relEnv{E}}\rho' =
      \rsem{B'}\rho'$. We reason as follows:
      \begin{eqnarray*}
        &    & (x_1,x_2) \in \rsem{\forall i\mathord:s. A'}{\relEnv{E}}\rho \\
        &\iff& \forall \rho' \in \extends{\rho}{i\mathord:s}.\ (x_1,x_2) \in \rsem{A'}{\relEnv{E}}\rho' \\
        &\iff& \forall \rho' \in \extends{\rho}{i\mathord:s}.\ (x_1,x_2) \in \rsem{B'}{\relEnv{E}}\rho' \\
        &\iff& (x_1,x_2) \in \rsem{\forall i\mathord:s. B'}{\relEnv{E}}\rho
      \end{eqnarray*}
      Thus, by extensionality, we have $\rsem{\forall
        i\mathord:s.A'}\rho \equiv \rsem{\forall
        i\mathord:s.B'}{\relEnv{E}}\rho$, as required.
    \item[Case \TirName{TyEqExist}] We have $A = \exists
      i\mathord:s. A'$ and $B = \exists i\mathord:s.B'$, with $\Delta,
      i \mathord: s \vdash A' \equiv B' \isType$. By the induction hypothesis,
      we know that, for all $\rho'$, $\rsem{A'}{\relEnv{E}}\rho' =
      \rsem{B'}\rho'$. We reason as follows:
      \begin{eqnarray*}
        &    & (x_1,x_2) \in \rsem{\exists i\mathord:s. A'}{\relEnv{E}}\rho \\
        &\iff& \exists \rho' \in \extends{\rho}{i\mathord:s}.\ (x_1,x_2) \in \rsem{A'}{\relEnv{E}}\rho' \\
        &\iff& \exists \rho' \in \extends{\rho}{i\mathord:s}.\ (x_1,x_2) \in \rsem{B'}{\relEnv{E}}\rho' \\
        &\iff& (x_1,x_2) \in \rsem{\exists i\mathord:s. B'}{\relEnv{E}}\rho
      \end{eqnarray*}
      Thus, by extensionality, we have $\rsem{\exists
        i\mathord:s.A'}\rho \equiv \rsem{\exists
        i\mathord:s.B'}{\relEnv{E}}\rho$, as required.
    \end{description}
  \item By induction on the derivation of $\Delta \vdash A \isType$. We
    analyse each case in turn:
    \begin{description}
    \item[Case \TirName{TyUnit}] Directly from the definition of
      substitution on types of the form $\tyUnit$, and the fact that
      $\rsem{\tyUnit}{\relEnv{E}}\rho_1 =
      \rsem{\tyUnit}{\relEnv{E}}\rho_2$ for any pair $\rho_1,\rho_2$
      of relation environments, we have
      $\rsem{\sigma^*\tyUnit}{\relEnv{E}}\rho =
      \rsem{\tyUnit}{\relEnv{E}}{(\rho \circ \sigma^*)}$, as required.
    \item[Case \TirName{TyPrim}] We have $A =
      \tyPrim{X}{e_1,...,e_n}$, therefore
      $\rsem{\sigma^*A}{\relEnv{E}}\rho =
      \rsem{\sigma^*\tyPrim{X}{e_1,...,e_n}}{\relEnv{E}}\rho =
      \rsem{\tyPrim{X}{\sigma^*e_1,...,\sigma^*e_n}}{\relEnv{E}}\rho =
      \rho_{\tyPrimNm{X}}(\sigma^*e_1,...,\sigma^*e_n) =
      \rsem{\tyPrim{X}{e_1,...,e_n}}{\relEnv{E}}{(\rho \circ
        \sigma^*)}$, as required.
    \item[Case \TirName{TyArr}] In this case, $A = A_1 \tyArr A_2$ and
      by the induction hypothesis, we know that
      $\rsem{\sigma^*A_1}{\relEnv{E}}\rho =
      \rsem{A_1}{\relEnv{E}}{(\rho \circ \sigma^*)}$ and
      $\rsem{\sigma^*A_2}{\relEnv{E}}\rho =
      \rsem{A_2}{\relEnv{E}}{(\rho \circ \sigma^*)}$. Accordingly, we
      reason as follows: $\rsem{\sigma^*(A_1 \tyArr
        A_2)}{\relEnv{E}}\rho = \rsem{\sigma^*A_1 \tyArr
        \sigma^*A_2}{\relEnv{E}}\rho =
      \rsem{\sigma^*A_1}{\relEnv{E}}\rho \relArrow
      \rsem{\sigma^*A_2}{\relEnv{E}}\rho =
      \rsem{A_1}{\relEnv{E}}{(\rho \circ \sigma^*)} \relArrow
      \rsem{A_2}{\relEnv{E}}{(\rho \circ \sigma^*)} = \rsem{A_1 \tyArr
        A_2}{\relEnv{E}}{(\rho \circ \sigma^*)}$, as required.
    \item[Case \TirName{TyProd}] In this case, $A = A_1 \tyProduct
      A_2$ and by the induction hypothesis, we know that
      $\rsem{\sigma^*A_1}{\relEnv{E}}\rho =
      \rsem{A_1}{\relEnv{E}}{(\rho \circ \sigma^*)}$ and
      $\rsem{\sigma^*A_2}{\relEnv{E}}\rho =
      \rsem{A_2}{\relEnv{E}}{(\rho \circ \sigma^*)}$. Accordingly, we
      reason as follows: $\rsem{\sigma^*(A_1 \tyProduct
        A_2)}{\relEnv{E}}\rho = \rsem{\sigma^*A_1 \tyProduct
        \sigma^*A_2}{\relEnv{E}}\rho =
      \rsem{\sigma^*A_1}{\relEnv{E}}\rho \relTimes
      \rsem{\sigma^*A_2}{\relEnv{E}}\rho =
      \rsem{A_1}{\relEnv{E}}{(\rho \circ \sigma^*)} \relTimes
      \rsem{A_2}{\relEnv{E}}{(\rho \circ \sigma^*)} = \rsem{A_1
        \tyProduct A_2}{\relEnv{E}}{(\rho \circ \sigma^*)}$, as
      required.
    \item[Case \TirName{TySum}] In this case, $A = A_1 + A_2$ and by
      the induction hypothesis, we know that
      $\rsem{\sigma^*A_1}{\relEnv{E}}\rho =
      \rsem{A_1}{\relEnv{E}}{(\rho \circ \sigma^*)}$ and
      $\rsem{\sigma^*A_2}{\relEnv{E}}\rho =
      \rsem{A_2}{\relEnv{E}}{(\rho \circ \sigma^*)}$. Accordingly, we
      reason as follows: $\rsem{\sigma^*(A_1 + A_2)}{\relEnv{E}}\rho =
      \rsem{\sigma^*A_1 + \sigma^*A_2}{\relEnv{E}}\rho =
      \rsem{\sigma^*A_1}{\relEnv{E}}\rho \relSum
      \rsem{\sigma^*A_2}{\relEnv{E}}\rho =
      \rsem{A_1}{\relEnv{E}}{(\rho \circ \sigma^*)} \relSum
      \rsem{A_2}{\relEnv{E}}{(\rho \circ \sigma^*)} = \rsem{A_1 +
        A_2}{\relEnv{E}}{(\rho \circ \sigma^*)}$, as required.
    \item[Case \TirName{TyForall}] We have $A = \forall
      i\mathord:s.A'$. We prove the required equation by demonstrating
      two inclusions. From left-to-right, we know that
      \begin{equation}
        \label{eq:tysubst-rel-forall1}
        (x_1,x_2) \in \rsem{\sigma^*\forall i\mathord:s. A'}{\relEnv{E}}\rho
      \end{equation}
      and we must show that $(x_1, x_2) \in \rsem{\forall
        i\mathord:s. A'}{\relEnv{E}}{(\rho \circ \sigma)}$. From
      (\ref{eq:tysubst-rel-forall1}) and the definition of
      substitution, we know that
      \begin{displaymath}
        \forall \rho' \in \extends{\rho}{i\mathord:s}.\ (x_1,x_2) \in \rsem{\sigma_{i\mathord:s}^*A'}{\relEnv{E}}\rho'
      \end{displaymath}
      By the induction hypothesis, we can deduce that
      \begin{equation}
        \label{eq:tysubst-rel-forall2}
        \forall \rho' \in \extends{\rho}{i\mathord:s}.\ (x_1,x_2) \in \rsem{A'}{\relEnv{E}}{(\rho' \circ \sigma_{i\mathord:s})}
      \end{equation}
      To show the required inclusion, we must show that for all
      $\rho_1 \in \extends{\rho \circ \sigma}{i\mathord:s}$, it is the
      case that $(x_1,x_2) \in \rsem{A'}{\relEnv{E}}{\rho_1}$. We now
      make use of the pushout property of our family $\relEnv{E}$ of
      relation environments: we have a $\rho \in \relEnv{E}(\Delta)$
      and $\rho_1 \in \relEnv{E}(\Delta, i\mathord:s)$ such that
      $\rho_1 \circ \pi_{i\mathord:s} = \rho \circ \sigma$ (by the
      assumption that $\rho_1 \in \extends{\rho \circ
        \sigma}{i\mathord:s}$), so we can obtain a $\rho' \in
      \relEnv{E}(\Delta',i\mathord:s)$ such that $\rho' \circ
      \pi_{i\mathord:s} = \rho$ (so $\rho' \in
      \extends{\rho}{i\mathord:s}$), and $\rho' \circ \sigma =
      \rho_1$. Therefore we use (\ref{eq:tysubst-rel-forall2}) to
      deduce that $(x_1,x_2) \in \rsem{A'}{\relEnv{E}}{\rho_1}$, and
      the inclusion from left-to-right is shown.

      From right-to-left, we must now show
      (\ref{eq:tysubst-rel-forall2}) under the assumption that
      \begin{equation}
        \label{eq:tysubst-rel-forall3}
        \forall \rho_1 \in \extends{\rho \circ \sigma}{i\mathord:s}.\ (x_1,x_2) \in \rsem{A'}{\relEnv{E}}{\rho_1}.
      \end{equation}
      Given $\rho' \in \extends{\rho}{i\mathord:s}$, we let $\rho_1 =
      \rho' \circ \sigma$. Now, $\rho_1 \circ \pi_{i\mathord:s} =
      \rho' \circ \sigma \circ \pi_{i\mathord:s} = \rho' \circ
      \pi_{i\mathord:s} \circ \sigma_{i\mathord:s} = \rho \circ
      \sigma_{i\mathord:s}$. Thus $\rho_1 \in \extends{\rho \circ
        \sigma}{i\mathord:s}$ and we may use
      (\ref{eq:tysubst-rel-forall3}) to deduce that $(x_1,x_2) \in
      \rsem{A'}{\relEnv{E}}{(\rho' \circ \sigma_{i\mathord:s})}$, as
      required.
    \item[Case \TirName{TyExist}] We have $A = \exists
      i\mathord:s.A'$. We prove the required equation by demonstrating
      two inclusions. From left-to-right, we know that
      \begin{equation}
        \label{eq:tysubst-rel-exist1}
        (x_1,x_2) \in \rsem{\sigma^*\exists i\mathord:s. A'}{\relEnv{E}}\rho
      \end{equation}
      and we must show that $(x_1,x_2) \in \rsem{\exists
        i\mathord:s. A'}{\relEnv{E}}{(\rho \circ \sigma)}$. From
      (\ref{eq:tysubst-rel-exist1}) and the definition of
      substitution, we know that
      \begin{displaymath}
        \exists \rho' \in \extends{\rho}{i\mathord:s}.\ (x_1,x_2) \in \rsem{\sigma_{i\mathord:s}^*A'}{\relEnv{E}}{\rho'}
      \end{displaymath}
      By the induction hypothesis, we can deduce that
      \begin{equation}
        \label{eq:tysubst-rel-exist2}
        \exists \rho' \in \extends{\rho}{i\mathord:s}.\ (x_1,x_2) \in \rsem{A'}{\relEnv{E}}{(\rho' \circ \sigma_{i\mathord:s})}
      \end{equation}
      To show the required inclusion, we must show that there exists a
      $\rho_1 \in \extends{\rho \circ \sigma}{i\mathord:s}$ such that
      $(x_1,x_2) \in \rsem{A'}{\relEnv{E}}{\rho_1}$. Let $\rho' \in
      \extends{\rho}{i\mathord:s}$ be the witness from
      (\ref{eq:tysubst-rel-exist2}), and set $\rho_1 = \rho' \circ
      \sigma_{i\mathord:s}$. Then $\rho_1 \in \extends{\rho \circ
        \sigma}{i\mathord:s}$ because $\rho_1 \circ \pi_{i\mathord:s}
      = \rho' \circ \sigma_{i\mathord:s} \circ \pi_{i\mathord:s} =
      \rho' \circ \pi_{i\mathord:s} \circ \sigma = \rho \circ
      \sigma$. Moreover, directly from (\ref{eq:tysubst-rel-exist2}),
      $(x_1,x_2) \in \rsem{A'}{\relEnv{E}}{\rho_1}$, as required.

      From right-to-left, we must now show
      (\ref{eq:tysubst-rel-exist2}) under the assumption that
      \begin{equation}
        \label{eq:tysubst-rel-exist3}
        \exists \rho_1 \in \extends{\rho \circ \sigma}{i\mathord:s}.\ (x_1,x_2) \in \rsem{A'}{\relEnv{E}}{\rho_1}
      \end{equation}
      We make use of the pushout property we assumed of relational
      environments.  We have $\rho_1 \in \relEnv{E}(\Delta,
      i\mathord:s)$ with $\rho_1 \circ \pi_{i\mathord:s} = \rho \circ
      \sigma$. Thus we obtain a $\rho' \in
      \relEnv{E}(\Delta',i\mathord:s)$ such that $\rho' \circ
      \pi_{i\mathord:s} = \rho$ (i.e.~$\rho' \in
      \extends{\rho}{i\mathord:s}$), and $\rho' \circ \sigma =
      \rho_1$. By (\ref{eq:tysubst-rel-exist3}), $(x_1,x_2) \in
      \rsem{A'}{\relEnv{E}}{(\rho' \circ \sigma)}$, and the inclusion
      from right-to-left is shown.
    \end{description}
  \end{enumerate}
\end{proof}

For reference, we state the complete index-erasure semantics of terms
here. For a well-typed program $\Delta; \Gamma \vdash M : A$, we
define the \emph{erasure interpretation} as a function $\tmSem{M} :
\ctxtSem{\Gamma} \to \tySem{A}$ by the following clauses:
\begin{displaymath}
  \begin{array}{@{\hspace{0em}}l@{\hspace{0.5em}}c@{\hspace{0.5em}}l}
    \tmSem{x}\eta & \isDefinedAs & \eta_x \\
    \tmSem{(M,N)}\eta & \isDefinedAs & (\tmSem{M}\eta, \tmSem{N}\eta) \\
    \tmSem{\pi_1M}\eta & \isDefinedAs & \pi_1(\tmSem{M}\eta) \\
    \tmSem{\pi_2M}\eta & \isDefinedAs & \pi_2(\tmSem{M}\eta) \\
    \tmSem{\mathrm{inl}\ M}\eta & \isDefinedAs & \mathrm{inl}\ (\tmSem{M}\eta) \\
    \tmSem{\mathrm{inr}\ M}\eta & \isDefinedAs & \mathrm{inr}\ (\tmSem{M}\eta) \\
    \left\llbracket
      \begin{array}{l}
        \textrm{case}\ M\ \textrm{of}\\
        \textrm{inl}\ x.N_1;\\
        \textrm{inr}\ y.N_2
      \end{array}\right\rrbracket^{\mathit{tm}}\eta & \isDefinedAs &
    \left\{
      \begin{array}{ll}
        \tmSem{N_1}(\eta,a) & \textrm{if }\tmSem{M}\eta = \mathrm{inl}(a) \\
        \tmSem{N_2}(\eta,b) & \textrm{if }\tmSem{M}\eta = \mathrm{inr}(b)
      \end{array}
    \right. \\
    \tmSem{\lambda x. M}\eta & \isDefinedAs & \lambda v. \tmSem{M}(\eta, v) \\
    \tmSem{M N}\eta & \isDefinedAs & (\tmSem{M}\eta) (\tmSem{N}\eta) \\
    \tmSem{\Lambda i.\ M}\eta & \isDefinedAs & \tmSem{M}\eta \\
    \tmSem{M[e]}\eta & \isDefinedAs & \tmSem{M}\eta \\
    \tmSem{\langle[e], M\rangle}\eta & \isDefinedAs & \tmSem{M}\eta \\
    \tmSem{\mathrm{let}\langle[i], x\rangle = M\ \mathrm{in}\ N}\eta & \isDefinedAs & \tmSem{N}(\eta, \tmSem{M}\eta)
  \end{array}
\end{displaymath}
In the above, we have written $\eta_x$ to denote the appropriate
projection from $\eta$ to get the value in the environment
corresponding to the variable $x$ in the context.

\begin{restateTheorem}[\thmref{thm:abstraction}]
  If $\Delta; \Gamma \vdash M : A$, then for all $\rho \in
  \relEnv{E}(\Delta)$ and $\eta_1, \eta_2 \in \ctxtSem{\Gamma}$ such
  that $(\eta_1, \eta_2) \in \rsem{\Gamma}{\relEnv{E}}\rho$, we have
  $(\tmSem{M}\eta_1, \tmSem{M}\eta_2) \in \rsem{A}{\relEnv{E}}\rho$.
\end{restateTheorem}

\begin{proof}
  By induction on the derivation of $\Delta; \Gamma \vdash M : A$. The
  three interesting cases follow:
  \begin{description}
  \item[Case \TirName{TyEq}] This case follows directly from
    \lemref{lem:tyeqsubst-relational}.
  \item[Case \TirName{UnivAbs}] In this case, $M = \Lambda i.M'$ and
    $A = \forall i\mathord:s. A'$. We know from the induction
    hypothesis that for all $\rho' \in \relEnv{E}(\Delta,
    i\mathord:s)$ and $(\eta_1,\eta_2) \in
    \rsem{\pi_{i\mathord:s}^*\Gamma}{\relEnv{E}}{\rho'}$, we have
    $(\tmSem{M'}\eta_1,\tmSem{M'}\eta_2) \in
    \rsem{A'}{\relEnv{E}}{\rho'}$. We are given $\rho \in
    \relEnv{E}(\Delta)$, $(\eta_1,\eta_2) \in
    \rsem{\Gamma}{\relEnv{E}}\rho$ and $\rho'' \in
    \extends{\rho}{i\mathord:s}$. Since $\rho'' \circ
    \pi_{i\mathord:s} = \rho$, we have $(\eta_1,\eta_2) \in
    \rsem{\Gamma}{\relEnv{E}}\rho$ implies $(\eta_1,\eta_2) \in
    \rsem{\Gamma}{\relEnv{E}}{(\rho'' \circ \pi_{i\mathord:s})}$, and
    therefore $(\eta_1,\eta_2) \in
    \rsem{\pi_{i\mathord:s}^*\Gamma}{\relEnv{E}}{\rho''}$ by
    \lemref{lem:ctxtsubst-rel}. So we can apply the induction
    hypothesis to get $(\tmSem{M'}\eta_1, \tmSem{M'}\eta_2) \in
    \rsem{A'}{\relEnv{E}}{\rho''}$, and thus $(\tmSem{\Lambda
      i.M'}\eta_1,\tmSem{\Lambda i.M'}\eta_2) \in
    \rsem{A'}{\relEnv{E}}\rho''$, by the definition of $\tmSem{\Lambda
      i. M'}$. Hence $(\tmSem{\Lambda i.M'}\eta_1,\tmSem{\Lambda
      i.M'}\eta_2) \in \rsem{\forall
      i\mathord:s. A'}{\relEnv{E}}\rho$, as required.
  \item[Case \TirName{UnivApp}] In this case, $M = M'[e]$ and $A =
    (\id_\Delta, e)^*A'$. By the induction hypothesis, and the
    definition of $\rsem{\forall i\mathord:s.A'}{\relEnv{E}}\rho$, we
    know that for all $\rho' \in \extends{\rho}{i\mathord:s}$,
    $(\tmSem{M'}\eta_1, \tmSem{M'}\eta_2) \in
    \rsem{A'}{\relEnv{E}}{\rho'}$. If we let $\rho' = \rho \circ
    (\id_\Delta, e)$, we know that $\rho' \in
    \extends{\rho}{i\mathord:s}$, because $\rho' \circ
    \pi_{i\mathord:s} = \rho \circ (\id_\Delta, e) \circ
    \pi_{i\mathord:s} = \rho$. Hence we may deduce that
    $(\tmSem{M'}\eta_1,\tmSem{M'}\eta_2) \in
    \rsem{A'}{\relEnv{E}}{(\rho \circ (\id_\Delta,e))}$ and thence, by
    \lemref{lem:ctxtsubst-rel}, conclude that $(\tmSem{M'}\eta_1,
    \tmSem{M'}\eta_2) \in \rsem{(\id_\Delta,e)^*A'}{\relEnv{E}}\rho$,
    as required.
  \item[Case \TirName{ExPack}] In this case, $M = \langle[e],
    M'\rangle$ and $A = \exists i\mathord:s.A'$. We know from the
    ind.~hyp.~that $(\tmSem{M'}\eta_1, \tmSem{M'}\eta_2) \in
    \rsem{(\id_\Delta, e)^*A'}{\relEnv{E}}\rho$. By
    \lemref{lem:tyeqsubst-relational} on substitutions, we can deduce
    that $(\tmSem{M'}\eta_1, \tmSem{M'}\eta_2) \in
    \rsem{A'}{\relEnv{E}}{(\rho \circ (\id_\Delta, e))}$. Since $(\rho
    \circ (\id_\Delta, e)) \in \extends{\rho}{i\mathord:s}$, we have
    $(\tmSem{M'}\eta_1, \tmSem{M'}\eta_2) \in \rsem{\exists
      i\mathord:s.A'}{\relEnv{E}}{\rho}$, as required.
  \item[Case \TirName{ExUnpack}] In this case, $M = \mathrm{let}\langle
    [i], x\rangle = M'\ \mathrm{in}\ N$. Let $\exists i\mathord:s. A'$ be the
    existential type that is being eliminated. By the induction
    hypothesis we know that there exists a $\rho' \in \extends{\rho}{i\mathord:s}$ such that
    \begin{equation}
      \label{eq:absthm-exunpack1}
      (\tmSem{M'}\eta_1, \tmSem{M'}\eta_2) \in \rsem{A'}{\relEnv{E}}{\rho'}
    \end{equation}
    and for all $\rho'' \in \relEnv{E}(\Delta,i\mathord:s)$ and
    $(\eta'_1,\eta'_2) \in \rsem{\pi_{i\mathord:s}^*\Gamma, x :
      A'}{\relEnv{E}}{\rho''}$,
    \begin{displaymath}
      (\tmSem{N}\eta'_1, \tmSem{N}\eta'_2) \in \rsem{\pi_{i\mathord:s}^*B}{\relEnv{E}}{\rho''}.
    \end{displaymath}
    We let $\rho'' = \rho'$ and set $\eta'_1 = (\eta_1,
    \tmSem{M'}\eta_1)$ and $\eta'_2 = (\eta_2,
    \tmSem{M'}\eta_2)$. This choice of $\eta'_1$ and $\eta'_2$ are
    related in the first component because $(\eta_1, \eta_2) \in
    \rsem{\Gamma}{\relEnv{E}}{\rho} = \rsem{\Gamma}{\relEnv{E}}{(\rho'
      \circ \pi_{i\mathord:s})} =
    \rsem{\pi_{i\mathord:s}^*\Gamma}{\relEnv{E}}{\rho'}$ by the
    induction hypothesis and \lemref{lem:ctxtsubst-rel}, and in the
    second component by (\ref{eq:absthm-exunpack1}). Thus we obtain:
    \begin{displaymath}
      (\tmSem{N}(\eta_1,\tmSem{M'}\eta_1),
       \tmSem{N}(\eta_2,\tmSem{M'}\eta_2)) \in \rsem{\pi_{i\mathord:s}^*B}{\relEnv{E}}{\rho'}
    \end{displaymath}
    Now, by \lemref{lem:tyeqsubst-relational},
    $\rsem{\pi_{i\mathord:s}^*B}{\relEnv{E}}{\rho'} =
    \rsem{B}{\relEnv{E}}{(\rho' \circ \pi_{i\mathord:s})} =
    \rsem{B}{\relEnv{E}}{\rho}$. By the definition of the
    index-erasure semantics of $\mathrm{unpack}$ terms, we are done.
  \end{description}
  The remaining cases are standard for proofs of the fundamental lemma
  of logical relations for simply-typed systems.
\end{proof}

\section{Proofs for Abelian Groups}
\subsection{Abelian Group Indexed Types}
\label{sec:abelian-group-indexed-types}

\newcommand{\Grp}{\mathit{Grp}}

Primitive operations $\Gamma_\Grp$:
\begin{displaymath}
  \begin{array}{r@{\hspace{0.5em}:\hspace{0.5em}}l}
    1 & \tyPrim{G}{1} \\
    (\cdot) & \forall a,b\mathord:\mathsf{G}.\ \tyPrim{G}{a} \to \tyPrim{G}{b} \to \tyPrim{G}{a b} \\
    \ ^{-1} & \forall a\mathord:\mathsf{G}.\ \tyPrim{G}{a} \to \tyPrim{G}{a^{-1}}
  \end{array}
\end{displaymath}

For the index-erasure interpretation, we assume a group $(G, 1, \cdot,
\ ^{-1})$. Let $Z(G)$ be the centre of the $G$: i.e.,~the set of
elements that commute with every other element in $G$. We set
$\tyPrimSem{G} = G$. The primitive operations in $\Gamma_\Grp$ are
interpreted just as the unit, multiplication and inverse operations
of $G$. For each index context $\Delta$, the set
$\relEnv{E}_\Grp(\Delta)$ of relational environments is defined to be
$\{ \rho_{(H,h)} \sepbar H\textrm{ is a subgroup of }(\Delta
\Rightarrow \mathsf{G})/\equiv, h\textrm{ is a group homomorphism }H
\to Z(G) \}$, where
\begin{displaymath}
  \rho_{(H,h)}\ \tyPrimNm{G}\ (e) = \left\{
    \begin{array}{ll}
      \{ \} & \textrm{if}\ e\not\in H \\
      \{ (x, h(e)\cdot x) \sepbar x \in G \} & \textrm{if}\ e\in H
    \end{array}
  \right.
\end{displaymath}

\begin{lemma}
  For all $\Delta$ and $\rho \in \relEnv{E}_\Grp(\Delta)$,
  $(\eta_\Grp,\eta_\Grp) \in \rsem{\Gamma_\Grp}{\relEnv{E}}\rho$.
\end{lemma}

\begin{theorem}
  Let $m$ and $n$ be natural numbers. For all $n$-by-$m$-matricies of
  natural numbers $A$ and $m$-by-$1$ matricies of integers $b$,
  there exists a program $M$ such that
  \begin{equation}
    \label{eq:fo-group-type}
    \Gamma_\Grp \vdash M : \forall i_1,\dots,i_m\mathord:\mathsf{G}.\ \tyPrim{G}{e_1} \to \dots \to \tyPrim{G}{e_n} \to \tyPrim{G}{e}
  \end{equation}
  where $e_j = i_1^{A_{j1}}\dots i_m^{A_{jm}}$ and $e = i_1^{b_1}\dots
  i_m^{b_m}$, if and only if there is a solution $\vec{x} =
  (x_1,\cdots,x_n)$ in the integers to the equation $A x^\top = b$.
\end{theorem}

\fixme{Define shorthand notation $x^z$}

\begin{proof}
  (If) The program $\Lambda i_1\dots i_m.\ \lambda g_1\dots g_n.\
  g_1^{x_1}\dots g_n^{x_n}$ satisfies the typing judgment
  (\ref{eq:fo-group-type}).

  (Only if) Assume that there is a program $M$ satisfying the typing
  judgment (\ref{eq:fo-group-type}). By \thmref{thm:abstraction} we
  know that for all relational environments $\rho \in
  \relEnv{E}_\Grp(i_1,\dots,i_m)$, and all $g_j,g'_j \in G$,
  \begin{displaymath}
    \begin{array}{l}
      (g_1,g'_1) \in \rho\ \tyPrimNm{G}\ (e_1) \land \dots \land (g_n,g'_n) \in \rho\ \tyPrimNm{G}\ (e_n) \Rightarrow \\
      \quad (\tmSem{M}g_1\dots g_n, \tmSem{M}g'_1\dots g'_n) \in \rho\ \tyPrimNm{G}\ (e)
    \end{array}
  \end{displaymath}
  We select the relational environments
  \begin{displaymath}
    \rho\ \tyPrimNm{G}\ (e) = \left\{
      \begin{array}{ll}
        \{(g,g) \sepbar g \in \mathcal{G} \} &
        \begin{array}[t]{@{}l}
          \textrm{if}\ e = e_1^{x_1}\dots e_n^{x_n} \\
          \textrm{ for some }x_1,\dots,x_n \in \mathbb{Z}
        \end{array}
        \\
        \{\} & \textrm{otherwise}
      \end{array}
    \right.
  \end{displaymath}
  Now, for any $g_1,\dots,g_m \in \mathcal{G}$, we know that, for all
  $j$, $(g_j,g_j) \in \rho\ \tyPrimNm{G}\ (e_j)$, so we know that
  $(\tySem{M}g_1\dots g_n, \tySem{M}g_1 \dots g_n) \in \rho\
  \tyPrimNm{G}\ (e)$. Thus there exist integers $x_1,...,x_n$ such
  that $e = e_1^{x_1}\dots e_n^{x_n}$. But we also know that $e =
  i_1^{b_1}...i_m^{b_m}$. By the cancellation property of free groups,
  we learn that $x_1,...,x_n$ is the solution we need.
\end{proof}

\begin{enumerate}
\item Can add any other operations that distribute over the group
  multiplication.
\item What about partial division operations? Make division just
  return $0$.
\item Adding square root as an operation. Need to check that all the
  other operations preserve the new relational environments. Square
  root is also supported by the plain relational environments, so the
  type isomorphism result still holds.
\end{enumerate}



\subsubsection{Type isomorphisms}
\label{sec:abelian-group-type-isos}

\begin{enumerate}
\item This only needs the simpler relational environments, but needs
  the group used for the index-erasure semantics to be abelian.
\item One direction is easy, but the other requires the use of the
  free theorem for the type.
\item There are two version: one with some fixed ground type as the
  result, and one with the same type as the result, but with the same
  parameter.
\item Integrates into the Smith Normal Form thing too, since
  invertible integer-valued matrices can be turned into type
  isomorphisms.
\item Partiality?
\end{enumerate}

\fixme{Define what a type isomorphism is}

\begin{theorem}
  For all natural numbers $n$, the types
  \begin{displaymath}
    \tau_{n} \isDefinedAs \forall a\mathord:\mathsf{G}.\ \underbrace{\tyPrim{G}{a} \to ... \to \tyPrim{G}{a}}_{n+1\textrm{ times}} \to \tyPrim{G}{a}
  \end{displaymath}
  and
  \begin{displaymath}
    \sigma_{n} \isDefinedAs \underbrace{\tyPrim{G}{1} \to ... \tyPrim{G}{1}}_{n\textrm{ times}} \to \tyPrim{G}{1}
  \end{displaymath}
  are isomorphic.
\end{theorem}

\begin{proof}
  We define programs $\Gamma_\Grp \vdash M : \tau_n \to \sigma_n$ and
  $\Gamma_\Grp \vdash M^{-1} : \sigma_n \to \tau_n$ as follows:
  \begin{displaymath}
    M = \lambda f. \lambda g_1 \dots g_n.\ f\ [1]\ 1\ g_1\ \dots\ g_n
  \end{displaymath}
  and
  \begin{displaymath}
    M^{-1} = \lambda f. \Lambda a.\ \lambda g_0 \dots g_n.\ (f\ (g_1 g^{-1}_0)\ \dots\ (g_n g^{-1}_0)) g_0
  \end{displaymath}
  In one direction, the proof that these are mutually inverse is
  straightfoward, using only the fact that $G$ is a group. For any
  program $\Gamma_\Grp \vdash f : \sigma_n$, we have
  \begin{displaymath}
    \begin{array}{cl}
        & \tmSem{M\ (M^{-1}\ f)}\eta_\Grp \\
      = & (\tmSem{M}\eta_{\Grp})\ (\lambda g_0 \dots g_n.\ (\tmSem{f}\eta_\Grp\ (g_1 g^{-1}_0)\ \dots\ (g_n g^{-1}_0)) g_0) \\
      = & \lambda g_1\dots g_n.\ \tmSem{f}\eta_\Grp\ (g_1 1^{-1})\ \dots\ (g_n 1^{-1})) 1) \\
      = & \lambda g_1\dots g_n.\ \tmSem{f}\eta_\Grp\ g_1\ \dots\ g_n \\
      = & \tmSem{f}\eta_\Grp
    \end{array}
  \end{displaymath}
  as required. In the other direction, we need to use the abstraction
  theorem. For any program $\Gamma_\Grp \vdash f : \tau_n$, we have
  \begin{displaymath}
    \begin{array}{cl}
        & \tmSem{M^{-1}\ (M\ f)}\eta_\Grp \\
      = & (\tmSem{M^{-1}}\eta_\Grp)\ (\lambda g_1\dots g_n.\ (\tmSem{f}\eta_\Grp\ 1\ g_1\ \dots\ g_n)) \\
      = & \lambda g_0\dots g_n.\ (\tmSem{f}\eta_\Grp\ 1\ (g_1g^{-1}_0)\ \dots\ (g_ng^{-1}_0))g_0
    \end{array}
  \end{displaymath}
  By \thmref{thm:abstraction}, we know that the following holds of
  $\tmSem{f}$:
  \begin{displaymath}
    \forall a.\ \forall g_0 \dots g_n.\ \tmSem{f}\eta_\Grp\ (g_0a)\ \dots\ (g_na) = (\tmSem{f}\eta_\Grp\ g_0\ \dots\ g_n)a
  \end{displaymath}
  As a special case, we set $a = g^{-1}_0$, and learn that
  \begin{displaymath}
    \forall g_0 \dots g_n.\ \tmSem{f}\eta_\Grp\ 1\ (g_1g^{-1}_0) \dots\ (g_ng^{-1}_0) = (\tmSem{f}\eta_\Grp\ g_0\ \dots\ g_n)g^{-1}_0.
  \end{displaymath}
  By functional extensionality, we are done.
\end{proof}


\section{Proofs for \autoref{sec:monoid-indexed-types}}

\begin{restateTheorem}[\thmref{thm:monoid-indefinability}]
  Let $m$ and $n$ be natural numbers. For all $n$-by-$m$-matricies of
  natural numbers $A$ and $m$-by-$1$ matricies of natural numbers $b$,
  there exists a program:
  \begin{equation}
    \label{eq-appendix:fo-monoid-type}
    ; \Gamma_\Mon \vdash M : \forall i_1,\dots,i_m\mathord:\mathsf{M}.\ \tyPrim{M}{e_1} \to \dots \to \tyPrim{M}{e_n} \to \tyPrim{M}{e}
  \end{equation}
  where $e_j = i_1^{A_{j1}}\dots i_m^{A_{jm}}$ and $e = i_1^{b_1}\dots
  i_m^{b_m}$, iff there is a solution $\vec{x} = (x_1,...,x_n)$ in the
  natural numbers to $A x^\top = b$.
\end{restateTheorem}

\begin{proof}
  (If) The program $\Lambda i_1\dots i_m.\ \lambda z_1\dots z_n.\
  z_1^{x_1}\dots z_n^{x_n}$ satisfies the typing judgment
  (\ref{eq-appendix:fo-monoid-type}).

  (Only if) Assume that there is a program $M$ satisfying the typing
  judgment (\ref{eq-appendix:fo-monoid-type}). By
  \thmref{thm:abstraction} we know that for all relational
  environments $\rho \in \relEnv{E}(i_1,\dots,i_m)$, and all $z_j,z'_j
  \in \mathcal{M}$,
  \begin{displaymath}
    \begin{array}{l}
      (z_1,z'_1) \in \rho\ \tyPrimNm{M}\ (e_1) \land \dots \land (z_n,z'_n) \in \rho\ \tyPrimNm{M}\ (e_n) \Rightarrow \\
      \quad (\tmSem{M}z_1\dots z_n, \tmSem{M}z'_1\dots z'_n) \in \rho\ \tyPrimNm{M}\ (e)
    \end{array}
  \end{displaymath}
  We make use of relation environments generated by the set
  $\{e_1,...,e_n\}$, and homomorphism $h(e) = 1$. Thus the relational
  environments will effectively have the following form:
  \begin{displaymath}
    \rho\ \tyPrimNm{M}\ (e) = \left\{
      \begin{array}{ll}
        \{(z,z) \sepbar z \in \mathcal{M} \} &
        \begin{array}[t]{@{}l}
          \textrm{if}\ e = e_1^{x_1}\dots e_n^{x_n} \\
          \textrm{ for some }x_1,\dots,x_n \in \mathbb{N}
        \end{array}
        \\
        \{\} & \textrm{otherwise}
      \end{array}
    \right.
  \end{displaymath}
  Now, for any $z_1,\dots,z_m \in \mathcal{M}$, we know that, for all
  $j$, $(z_j,z_j) \in \rho\ \tyPrimNm{M}\ (e_j)$, so we know that
  $(\tySem{M}z_1\dots z_n, \tySem{M}z_1 \dots z_n) \in \rho\
  \tyPrimNm{M}\ (e)$. Thus there exist $x_1,...,x_n$ such that $e =
  e_1^{x_1}\dots e_n^{x_n}$. But we also know that $e =
  i_1^{b_1}...i_m^{b_m}$. By the cancellation property of free
  monoids, we learn that $x_1,...,x_n$ is the solution we need.
\end{proof}


%%% Local Variables:
%%% TeX-master: "paper"
%%% End:
