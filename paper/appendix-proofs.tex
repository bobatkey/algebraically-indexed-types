\section{Details of proofs}

\begin{proof}\textbf{[Of \lemref{lem:tyeq-erasure}]}
  By induction on the derivation of $\Delta \vdash A \equiv B
  \isType$:
  \begin{description}
  \item[Case \TirName{TyEqUnit}] In this case, $A = B = \tyUnit$, so
    $\tySem{A} = \tySem{B}$.
  \item[Case \TirName{TyEqPrimIdx}] We have $A = \tyPrim{X}{e_1}$ and
    $B = \tyPrim{X}{e_2}$, for some $e_1$ and $e_2$ of the same
    sort. The erasure semantics ignores the index arguments, so
    $\tySem{A} = \tySem{\tyPrim{X}{e_1}} = \tyPrimSem{X} =
    \tySem{\tyPrim{X}{e_2}} = \tySem{B}$, as required.
  \item[Case \TirName{TyEqPrimNoIdx}] We have $A = \tyPrimNm{X}$ and
    $B = \tyPrimNm{X}$. So $\tySem{A} = \tyPrimSem{X} = \tySem{B}$, as
    required.
  \item[Case \TirName{TyEqArr}] Here, $A = A_1 \tyArr A_2$ and $B =
    B_1 \tyArr B_2$, and by the induction hypothesis we know that
    $\tySem{A_1} = \tySem{B_1}$ and $\tySem{A_2} = \tySem{B_2}$. Hence
    $\tySem{A} = \tySem{A_1 \tyArr A_2} = \tySem{A_1} \to \tySem{A_2}
    = \tySem{B_1} \to \tySem{B_2} = \tySem{B_1 \tyArr B_2} =
    \tySem{B}$.
  \item[Case \TirName{TyEqProd}] In this case, $A = A_1 \tyProduct
    A_2$ and $B = B_1 \tyProduct B_2$, and by the induction hypothesis
    we know that $\tySem{A_1} = \tySem{B_1}$ and $\tySem{A_2} =
    \tySem{B_2}$. Hence $\tySem{A} = \tySem{A_1 \tyProduct A_2} =
    \tySem{A_1} \times \tySem{A_2} = \tySem{B_1} \times \tySem{B_2} =
    \tySem{B_1 \tyProduct B_2} = \tySem{B}$.
  \item[Case \TirName{TyEqSum}] In this case, $A = A_1 + A_2$ and $B =
    B_1 + B_2$, and by the induction hypothesis we know that
    $\tySem{A_1} = \tySem{B_1}$ and $\tySem{A_2} = \tySem{B_2}$. Hence
    $\tySem{A} = \tySem{A_1 + A_2} = \tySem{A_1} + \tySem{A_2} =
    \tySem{B_1} + \tySem{B_2} = \tySem{B_1 + B_2} = \tySem{B}$.
  \item[Case \TirName{TyEqForall}] We have $A = \forall
    i\mathord:s.A'$ and $B = \forall i\mathord:s.B'$. By the induction
    hypothesis we know that $\tySem{A'} = \tySem{B'}$, and hence
    $\tySem{A} = \tySem{\forall i\mathord:s. A'} = \tySem{A'} =
    \tySem{B'} = \tySem{\forall i\mathord:s. B'} = \tySem{B}$.
  \end{description}
\end{proof}

\begin{proof}\textbf{[Of \lemref{lem:tysubst-erasure}]}
  By induction on the derivation of $\Delta \vdash A \isType$.
  \begin{description}
  \item[Case \TirName{TyUnit}] Directly from the definition of
    substitution on types of the form $\tyUnit$, we have
    $\tySem{\sigma^*\tyUnit} = \tySem{\tyUnit}$, as required.
  \item[Case \TirName{TyPrimIdx}] Since the index-erasure semantics
    ignores all index expessions, we have
    $\tySem{\sigma^*\tyPrim{X}{e}} = \tySem{\tyPrim{X}{\sigma^*e}} =
    \tyPrimSem{X} = \tySem{\tyPrim{X}{e}}$.
  \item[Case \TirName{TyPrimNoIdx}] Similar to the previous case.
  \item[Case \TirName{TyArr}] We have $A = A_1 \tyArr A_2$, and by the
    induction hypothesis, we know that $\tySem{\sigma^*A_1} =
    \tySem{A_1}$ and $\tySem{\sigma^*B_1} = \tySem{B_1}$. Thus, by the
    definition of substitution on arrow types, we have
    $\tySem{\sigma^*(A_1 \tyArr A_2)} = \tySem{\sigma^*A_1 \tyArr
      \sigma^*A_2} = \tySem{\sigma^*A_1} \to \tySem{\sigma^*A_2} =
    \tySem{A_1} \to \tySem{A_2} = \tySem{A_1 \tyArr A_2}$, as
    required.
  \item[Case \TirName{TyProd}] We have $A = A_1 \tyProduct A_2$, and
    by the induction hypothesis, we know that $\tySem{\sigma^*A_1} =
    \tySem{A_1}$ and $\tySem{\sigma^*A_2} = \tySem{A_2}$. Thus, by the
    definition of substitution on product types, we have
    $\tySem{\sigma^*(A_1 \tyProduct A_2)} = \tySem{\sigma^*A_1
      \tyProduct \sigma^*A_2} = \tySem{\sigma^*A_1} \times
    \tySem{\sigma^*A_2} = \tySem{A_1} \times \tySem{A_2} = \tySem{A_1
      \tyProduct A_2}$, as required.
  \item[Case \TirName{TySum}] We have $A = A_1 + A_2$, and by the
    induction hypothesis, we know that $\tySem{\sigma^*A_1} =
    \tySem{A_1}$ and $\tySem{\sigma^*A_2} = \tySem{A_2}$. Thus, by the
    definition of substitution on sum types, we have
    $\tySem{\sigma^*(A_1 + A_2)} = \tySem{\sigma^*A_1 + \sigma^*A_2} =
    \tySem{\sigma^*A_1} + \tySem{\sigma^*A_2} = \tySem{A_1} +
    \tySem{A_2} = \tySem{A_1 + A_2}$, as required.
  \item[Case \TirName{TyForall}] We have $A = \forall i\mathord:s.A'$,
    and by the induction hypothesis we know that, for all $\sigma'$,
    $\tySem{\sigma'^*A'} = \tySem{A'}$. Therefore, from the definition
    of substitution for $\forall$-types, we have
    $\tySem{\sigma^*\forall i\mathord:s.A'} = \tySem{\forall
      i\mathord:s.\sigma_s^{*}A'} = \tySem{\sigma_s^{*}A'} = \tySem{A'}
    = \tySem{\forall i\mathord:s.A'}$, where we have relied on the
    fact that the index-erasure semantics of types ignores the index
    quantification.
  \end{description}
\end{proof}

\begin{proof}\textbf{[Of \lemref{lem:tyeq-rel}]}
  By induction on the derivation of $\Delta \vdash A \equiv B \isType$.
  \begin{description}
  \item[Case \TirName{TyEqUnit}] In this case $A = B = \tyUnit$, so
    the statement is trivially statisfied.
  \item[Case \TirName{TyEqPrimIdx}] In this case we have $A =
    \tyPrim{X}{e_1}$ and $B = \tyPrim{X}{e_2}$, with $\Delta \vdash
    e_1 \equiv e_2 : s$. The index expressions $e_1$ and $e_2$ are in
    the same equivalence class, so by the definition of relation
    environments we have that $\rho_{\tyPrimNm{X}}(e_1) =
    \rho_{\tyPrimNm{X}}(e_2)$ and hence
    $\rsem{\tyPrim{X}{e_1}}{\relEnv{E}}\rho =
    \rsem{\tyPrim{X}{e_2}}{\relEnv{E}}\rho$.
  \item[Case \TirName{TyEqPrimNoIdx}] This case is similar to the case
    for \TirName{TyEqUnit}.
  \item[Case \TirName{TyEqArr}] We have $A = A_1 \tyArr A_2$ and $B =
    B_1 \tyArr B_2$, with $\Delta \vdash A_1 \equiv B_1 \isType$ and
    $\Delta \vdash A_2 \equiv B_2 \isType$. By the induction
    hypothesis, we know that $\rsem{A_1}{\relEnv{E}}\rho =
    \rsem{B_1}{\relEnv{E}}\rho$ and $\rsem{A_2}{\relEnv{E}}\rho =
    \rsem{B_2}{\relEnv{E}}\rho$. Thus we may reason as follows:
    $\rsem{A_1 \tyArr A_2}{\relEnv{E}}\rho =
    \rsem{A_1}{\relEnv{E}}\rho \relArrow \rsem{A_2}{\relEnv{E}}\rho =
    \rsem{B_1}{\relEnv{E}}\rho \relArrow \rsem{B_2}{\relEnv{E}}\rho =
    \rsem{B_1 \tyArr B_2}{\relEnv{E}}\rho$, as required.
  \item[Case \TirName{TyEqProd}] We have $A = A_1 \tyProduct A_2$ and
    $B = B_1 \tyProduct B_2$, with $\Delta \vdash A_1 \equiv B_1
    \isType$ and $\Delta \vdash A_2 \equiv B_2 \isType$. By the
    induction hypothesis, we know that $\rsem{A_1}{\relEnv{E}}\rho =
    \rsem{B_1}{\relEnv{E}}\rho$ and $\rsem{A_2}{\relEnv{E}}\rho =
    \rsem{B_2}{\relEnv{E}}\rho$. Therefore, we can reason as follows:
    $\rsem{A_1 \tyProduct A_2}{\relEnv{E}}\rho =
    \rsem{A_1}{\relEnv{E}}\rho \relTimes \rsem{A_2}{\relEnv{E}}\rho =
    \rsem{B_1}{\relEnv{E}}\rho \relTimes \rsem{B_2}{\relEnv{E}}\rho =
    \rsem{B_1 \tyProduct B_2}{\relEnv{E}}\rho$, as required.
  \item[Case \TirName{TyEqSum}] We have $A = A_1 + A_2$ and $B = B_1 +
    B_2$, with $\Delta \vdash A_1 \equiv B_1 \isType$ and $\Delta
    \vdash A_2 \equiv B_2 \isType$. By the induction hypothesis, we
    know that $\rsem{A_1}{\relEnv{E}}\rho =
    \rsem{B_1}{\relEnv{E}}\rho$ and $\rsem{A_2}{\relEnv{E}}\rho =
    \rsem{B_2}{\relEnv{E}}\rho$. Therefore, we can reason as follows:
    $\rsem{A_1 + A_2}{\relEnv{E}}\rho = \rsem{A_1}{\relEnv{E}}\rho
    \relSum \rsem{A_2}{\relEnv{E}}\rho = \rsem{B_1}{\relEnv{E}}\rho
    \relSum \rsem{B_2}{\relEnv{E}}\rho = \rsem{B_1 +
      B_2}{\relEnv{E}}\rho$, as required.
  \item[Case \TirName{TyEqForall}] We have $A = \forall
    i\mathord:s. A'$ and $B = \forall i\mathord:s.B'$, with $\Delta, i
    : s \vdash A' \equiv B' \isType$. By the induction hypothesis,
    we know that for all $\rho'$, $\rsem{A'}{\relEnv{E}}\rho' =
    \rsem{B'}\rho'$. We reason as follows:
    \begin{eqnarray*}
      &    & (x_1,x_2) \in \rsem{\forall i\mathord:s. A'}{\relEnv{E}}\rho \\
      &\iff& \forall \rho' \in \extends{\rho}{i:s}.\ (x_1,x_2) \in \rsem{A'}{\relEnv{E}}\rho' \\
      &\iff& \forall \rho' \in \extends{\rho}{i:s}.\ (x_1,x_2) \in \rsem{B'}{\relEnv{E}}\rho' \\
      &\iff& (x_1,x_2) \in \rsem{\forall i\mathord:s. B'}{\relEnv{E}}\rho
    \end{eqnarray*}
    Thus, by extensionality, we have $\rsem{\forall
      i\mathord:s.A'}\rho \equiv \rsem{\forall
      i\mathord:s.B'}{\relEnv{E}}\rho$, as required.
  \end{description}
\end{proof}

\begin{proof}\textbf{[Of \lemref{lem:tysubst-rel}]}
  The first part of the lemma statement is proved by induction on the
  derivation of $\Delta \vdash A \isType$. We analyse each case in
  turn:
  \begin{description}
  \item[Case \TirName{TyUnit}] Directly from the definition of
    substitution on types of the form $\tyUnit$, and the fact that
    $\rsem{\tyUnit}{\relEnv{E}}\rho_1 =
    \rsem{\tyUnit}{\relEnv{E}}\rho_2$ for any pair $\rho_1,\rho_2$ of
    relation environments, we have
    $\rsem{\sigma^*\tyUnit}{\relEnv{E}}\rho =
    \rsem{\tyUnit}{\relEnv{E}}{(\rho \circ \sigma^*)}$, as required.
  \item[Case \TirName{TyPrimIdx}] We have $A = \tyPrim{X}{e}$, therefore
    $\rsem{\sigma^*A}{\relEnv{E}}\rho =
    \rsem{\sigma^*\tyPrim{X}{e}}{\relEnv{E}}\rho =
    \rsem{\tyPrim{X}{\sigma^*e}}{\relEnv{E}}\rho =
    \rho_{\tyPrimNm{X}}(\sigma^*e) =
    \rsem{\tyPrim{X}{e}}{\relEnv{E}}{(\rho \circ \sigma^*)}$, as
    required.
  \item[Case \TirName{TyPrimNoIdx}] Similar to the case for
    \TirName{TyUnit}.
  \item[Case \TirName{TyArr}] In this case, $A = A_1 \tyArr A_2$ and
    by the induction hypothesis, we know that $\rsem{\sigma^*A_1}{\relEnv{E}}\rho
    = \rsem{A_1}{\relEnv{E}}{(\rho \circ \sigma^*)}$ and $\rsem{\sigma^*A_2}{\relEnv{E}}\rho =
    \rsem{A_2}{\relEnv{E}}{(\rho \circ \sigma^*)}$. Accordingly, we reason as
    follows: $\rsem{\sigma^*(A_1 \tyArr A_2)}{\relEnv{E}}\rho = \rsem{\sigma^*A_1
      \tyArr \sigma^*A_2}{\relEnv{E}}\rho = \rsem{\sigma^*A_1}{\relEnv{E}}\rho \relArrow
    \rsem{\sigma^*A_2}{\relEnv{E}}\rho = \rsem{A_1}{\relEnv{E}}{(\rho \circ \sigma^*)}
    \relArrow \rsem{A_2}{\relEnv{E}}{(\rho \circ \sigma^*)} = \rsem{A_1 \tyArr
      A_2}{\relEnv{E}}{(\rho \circ \sigma^*)}$, as required.
  \item[Case \TirName{TyProd}] In this case, $A = A_1 \tyProduct A_2$
    and by the induction hypothesis, we know that
    $\rsem{\sigma^*A_1}{\relEnv{E}}\rho = \rsem{A_1}{\relEnv{E}}{(\rho
      \circ \sigma^*)}$ and $\rsem{\sigma^*A_2}{\relEnv{E}}\rho =
    \rsem{A_2}{\relEnv{E}}{(\rho \circ \sigma^*)}$. Accordingly, we
    reason as follows: $\rsem{\sigma^*(A_1 \tyProduct
      A_2)}{\relEnv{E}}\rho = \rsem{\sigma^*A_1 \tyProduct
      \sigma^*A_2}{\relEnv{E}}\rho =
    \rsem{\sigma^*A_1}{\relEnv{E}}\rho \relTimes
    \rsem{\sigma^*A_2}{\relEnv{E}}\rho = \rsem{A_1}{\relEnv{E}}{(\rho
      \circ \sigma^*)} \relTimes \rsem{A_2}{\relEnv{E}}{(\rho \circ
      \sigma^*)} = \rsem{A_1 \tyProduct A_2}{\relEnv{E}}{(\rho \circ
      \sigma^*)}$, as required.
  \item[Case \TirName{TySum}] In this case, $A = A_1 + A_2$
    and by the induction hypothesis, we know that
    $\rsem{\sigma^*A_1}{\relEnv{E}}\rho = \rsem{A_1}{\relEnv{E}}{(\rho
      \circ \sigma^*)}$ and $\rsem{\sigma^*A_2}{\relEnv{E}}\rho =
    \rsem{A_2}{\relEnv{E}}{(\rho \circ \sigma^*)}$. Accordingly, we
    reason as follows: $\rsem{\sigma^*(A_1 +
      A_2)}{\relEnv{E}}\rho = \rsem{\sigma^*A_1 +
      \sigma^*A_2}{\relEnv{E}}\rho =
    \rsem{\sigma^*A_1}{\relEnv{E}}\rho \relSum
    \rsem{\sigma^*A_2}{\relEnv{E}}\rho = \rsem{A_1}{\relEnv{E}}{(\rho
      \circ \sigma^*)} \relSum \rsem{A_2}{\relEnv{E}}{(\rho \circ
      \sigma^*)} = \rsem{A_1 + A_2}{\relEnv{E}}{(\rho \circ
      \sigma^*)}$, as required.
  \item[Case \TirName{TyForall}] We have $A = \forall
    i\mathord:s.A'$. We prove the required equation by demonstrating
    two inclusions. From left-to-right, we know that
    \begin{equation}
      \label{eq:tysubst-rel-forall1}
      (x_1,x_2) \in \rsem{\sigma^*\forall i\mathord:s. A'}{\relEnv{E}}\rho
    \end{equation}
    and we must show that $(x_1, x_2) \in \rsem{\forall
      i\mathord:s. A'}{\relEnv{E}}{(\rho \circ \sigma^*)}$. From
    (\ref{eq:tysubst-rel-forall1}) and the definition of substitution,
    we know that
    \begin{displaymath}
      \forall \rho' \in \extends{\rho}{i:s}.\ (x_1,x_2) \in \rsem{\sigma_s^*A'}{\relEnv{E}}\rho'
    \end{displaymath}
    By the induction hypothesis, we can deduce that
    \begin{equation}
      \label{eq:tysubst-rel-forall2}
      \forall \rho' \in \extends{\rho}{i:s}.\ (x_1,x_2) \in \rsem{A'}{\relEnv{E}}{(\rho' \circ \sigma_s^{*})}
    \end{equation}
    To show the required inclusion, we must show that for all $\rho_1
    \in \extends{\rho \circ \sigma^*}{i:s}$, it is the case that
    $(x_1,x_2) \in \rsem{A'}{\relEnv{E}}{\rho_1}$. We now make use of
    the pushout property of our family $\relEnv{E}$ of relation
    environments: we have a $\rho \in \relEnv{E}(\Delta)$ and $\rho_1
    \in \relEnv{E}(\Delta, i:s)$ such that $\rho_1 \circ \pi_{i:s} =
    \rho \circ \sigma^*$ (by the assumption that $\rho_1 \in
    \extends{\rho \circ \sigma^*}{i:s}$), so we can obtain a $\rho'
    \in \relEnv{E}(\Delta',i:s)$ such that $\rho' \circ \pi_{i:s} =
    \rho$ (so $\rho' \in \extends{\rho}{i:s}$), and $\rho' \circ
    \sigma^* = \rho_1$. Therefore we use
    (\ref{eq:tysubst-rel-forall2}) to deduce that $(x_1,x_2) \in
    \rsem{A'}{\relEnv{E}}{\rho_1}$, and the inclusion from
    left-to-right is shown.

    From right-to-left, we must now show
    (\ref{eq:tysubst-rel-forall2}) under the assumption that
    \begin{equation}
      \label{eq:tysubst-rel-forall3}
      \forall \rho_1 \in \extends{\rho \circ \sigma^*}{i:s}. (x_1,x_2) \in \rsem{A'}{\relEnv{E}}{\rho_1}.
    \end{equation}
    Given $\rho' \in \extends{\rho}{i:s}$, we let $\rho_1 = \rho'
    \circ \sigma^*$. Now, $\rho_1 \circ \pi_{i:s} = \rho' \circ
    \sigma^* \circ \pi_{i:s} = \rho' \circ \pi_{i:s} \circ \sigma_s^{*}
    = \rho \circ \sigma_s^{*}$. Thus $\rho_1 \in \extends{\rho \circ
      \sigma^*}{i:s}$ and we may use (\ref{eq:tysubst-rel-forall3}) to
    deduce that $(x_1,x_2) \in \rsem{A'}{\relEnv{E}}{(\rho' \circ \sigma_s^{*})}$,
    as required.
  \end{description}
  The second part of the lemma statement follows directly by induction
  on the derivation of $\Delta \vdash \Gamma \isCtxt$ and the first
  part. The step case of the induction is very similar to the
  \TirName{TyProd} case above.
\end{proof}
