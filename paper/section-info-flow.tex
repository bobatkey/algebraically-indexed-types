\section{Logical Information Flow}
\label{sec:information-flow}

\newcommand{\infoflow}{\mathit{Log}}

We now apply our general framework to types that are indexed by
logical propositions. By including a primitive type that represents
logical truth, we can recover--through a construction due to Tse and
Zdancewic \cite{tse04translating}--strong information flow properties
of programs. As a result of our general framework being parameterised
by the choice of equational theory, we can alter the logic that we use
for reasoning about type equality, and hence alter the information
flow properties of the system. 

We first recall the concept of information flow. A function $f : A
\times B \to C$ is said to not allow information to flow from its
second argument to the output if for all $b,b' \in B$ and all $a \in
A$, $f(a,b) = f(a,b')$. If we think of the $B$ argument as
representing high-security information, then we have stated that $f$
does not allow the high-security input to flow to the low security
output. Information flow can be seen as a kind of invariance property
of programs, and so our relational interpretation of types is well
tailored to proving this kind of property.

As described by Sabelfeld and Sands \cite{sabelfeld01per}, information
flow can be captured semantically by partial equivalence relations
(PERs). Abadi, Banerjee, Heintze and Riecke \cite{abadi99core} built a
\emph{Core Calculus for Dependency}, using a type system based around
a security level indexed monad $T_lA$, using PERs to prove the
information flow properties. Tse and Zdancewic \cite{tse04translating}
translated Abadi \emph{et al.}'s calculus into System F\footnote{Tse
  and Zdancewic's translation did not satisfy all the properties that
  they claimed, as pointed out by Shikuma and Igarashi
  \cite{shikuma08proving}. However, this problem is not relevant to
  our discussion here.} translating the monadic type $T_lA$ to
$\alpha_l \to A$ for some free type variable $\alpha_l$, and using
Reynolds' abstraction theorem to prove information flow
properties. For example, if the type variable $\alpha_H$ represents
high-level information, then the non-interference property of the
function $f$ could be expressed by the System F type $A \to (\alpha_H
\to B) \to C$. If a program cannot generate a value of type
$\alpha_H$, then it cannot access the value of $B$, and hence is
insensitive to the actual value. Relationships between security levels
are captured by postulating functions $\alpha_{l_1} \to \alpha_{l_2}$
whenever $l_1$ is a lower security level than $l_2$.

Using algebraic indexing, we refine Tse and Zdancewic's translation by
replacing each type variable $\alpha_l$ with a primitive type
$\tyPrim{T}{\phi_l}$ of representations of the truth of a logical
proposition $\phi_l$ that stands for the security level $l$. The
relationships between security levels are now replaced by logical
entailment, so we only have functions of type $\tyPrim{T}{\phi_{l_1}} \to
\tyPrim{T}{\phi_{l_2}}$ when $\phi_{l_1}$ entails
$\phi_{l_2}$. %Inspired by the PER interpretation of information flow,
We instantiate our general relational framework to interpret
$\tyPrim{T}{\phi}$ as the identity relation if $\phi$ is true and the
empty relation if $\phi$ is false. We shall see that this recovers the
information flow properties of Abadi {\em et al.} and Tse and
Zdancewic.

\paragraph{Instantiation of the General Framework}
We assume a single indexing sort $\mathsf{prop}$ and assume the
operations and equations of boolean algebra. Thus we have constants
$\top$, $\bot$ and binary operators $\land$, $\lor$ with the axioms of
a bounded lattice, and a unary complementation operator $\lnot$. We
will use $\phi, \psi$ to stand for index expressions of sort
$\mathsf{prop}$. We use an equational presentation of boolean algebras
to fit with our general framework, but note that we can define an
order on index expressions as $\phi \leq \psi$ when $\phi = \phi \land
\psi$.

We have a single primitive type $\tyPrimNm{T}$, with
$\primTyArity(\tyPrimNm{T}) = [\mathsf{prop}]$ and index-erasure
semantics $\tyPrimSem{\tyPrimNm{X}} = \{ * \}$. Thus values of type
$\tyPrim{T}{\phi}$ have no run-time content; their only meaning is
given by the relational semantics. For the model of the indexing
theory, we take an arbitrary boolean algebra $L$. The relational
interpretation of the truth representation type is
$\rsem{\tyPrimNm{T}}(x) = \{ (*,*) \sepbar x = \top \}$, where $\top$
is the top element of the boolean algebra $L$. The primitive
operations $\Gamma_\infoflow$ reflect logical consequence:
\begin{displaymath}
  \begin{array}{l}
  \begin{array}{@{}c@{\hspace{2em}}c}
    \mathrm{truth} : \tyPrim{T}{\top} &
    \mathrm{and}   : \forall p, q\mathord:\mathsf{prop}.\ \tyPrim{T}{p} \to \tyPrim{T}{q} \to \tyPrim{T}{p \land q}
  \end{array} \\
  \mathrm{up} : \forall p, q\mathord:\mathsf{prop}.\ \tyPrim{T}{p \land q} \to \tyPrim{T}{p}
\end{array}
\end{displaymath}
The combination of the \TirName{TyEq} rule and the primitive
$\mathrm{up}$ operation allow for logical entailment to be reflected
in programs: if we have $\phi \leq \psi$ and $M : \tyPrim{T}{\phi}$
then $\mathrm{up}\ M : \tyPrim{T}{\psi}$. Each of the primitive
operations has a trivial interpretation, due to the index-erasure
interpretation of $\tyPrim{T}{\phi}$ as a one-element set, giving an
environment $\eta_\infoflow \in \ctxtSem{\Gamma}$. Less trivially, we
have this lemma:
\begin{lemma}\label{lem:environments-information-flow}
  For all $\Delta$ and $\rho \in \rsem{\Delta}$, $(\eta_\infoflow, \eta_\infoflow) \in \rsem{\Gamma_\infoflow}\rho$.
\end{lemma}

\paragraph{Information Flow} We think of logical expressions as
``composite principals''. That is, propositional variables representing
atomic principals that are combined using the logical connectives. We
interpret ``truth'' for principals as stating that a principal is true
when satisfied with the current state of affairs. Thus a relationship
$\phi \leq \psi$ indicates that satisfaction of the composite
principal $\phi$ implies satisfaction of the composite principal
$\psi$. In terms of security levels, the ordering is reversed: if a
high security principal is satisfied, then all of their subordinates
must also be satisfied.

We adapt Tse and Zdancewic's translation of Abadi \emph{et al.}'s monadic
type to our setting. We define a type abbreviation $T_\phi A =
\tyPrim{T}{\phi} \to A$, where $A$ is a type and $\phi$ is an
expression of sort $\mathsf{prop}$. For every $\phi$, we can endow the
types $T_\phi-$ with the structure of a monad. This is due to the fact
that it is an instance of the ``environment'' (or ``reader'') monad
\cite{jones95functional}. We read the types $T_\phi A$ as data of type
$A$ ``protected'' by the principal $\phi$.

As a consequence of the relational interpretation given above, it
follows that if we have an index expression $\phi$ that is interpreted
as some value other than $\top$ in a relational environment $\rho$,
then for all $x,x' \in \tySem{T_\phi\tyPrimNm{bool}}$, we have $(x,x')
\in \rsem{T_\phi\tyPrimNm{bool}}{\rho}$. Thus if a principal is
dissatisfied (i.e.,~$\phi \not= \top$), then data protected by this
principal is indistinguishable from any other data, and a program
cannot get access to the exact value. From this observation, and
\thmref{thm:abstraction}, we obtain the following information flow
result:

\begin{theorem}
  Let $\phi$ and $\psi$ be index expressions of sort $\mathsf{prop}$
  in some indexing context $\Delta$, such that $\psi \not\leq
  \phi$. Then for all terms:
  \begin{displaymath}
    \Delta; \Gamma_\infoflow,\Gamma \vdash M : T_\phi\tyPrimNm{bool} \to T_\psi\tyPrimNm{bool}
  \end{displaymath}
  and all terms $N_1,N_2$ of type $T_\phi\tyPrimNm{bool}$, $M\ N_1
  \stackrel{ctx}\approx M\ N_2$.  Thus there is no information flow
  from $M$'s input to its output.
\end{theorem}
Note that if $\psi \leq \phi$, then it is always possible to write the
identity function with this type, using the $\mathrm{up}$ operation.
The theorem also holds if we move to logics other than boolean
logic. For example, if our equational theory models intuitionistic
logic by taking the axioms of Heyting algebras, then the same non-flow
property for programs of type $T_{p\lor\lnot p}\tyPrimNm{bool} \to
T_\top\tyPrimNm{bool}$ holds, due to the lack of excluded middle. If
we take linear logic, then programs of type $T_{p\otimes
  p}\tyPrimNm{bool} \to T_p\tyPrimNm{bool}$ have no information flow
from their input, due to non-provability of $p \vdash p \otimes p$.


%%% Local Variables:
%%% TeX-master: "paper"
%%% End:
