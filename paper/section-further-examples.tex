\section{Logical Information Flow}
\label{sec:information-flow}

\begin{enumerate}
\item Make sure the logic is at least a meet semi-lattice
\item 
\end{enumerate}

\begin{enumerate}
\item Also makes use of the relational aspect, the previous stuff
  could make do with unary logical predicates as the interpretations
  of types, but the information flow properties make use of the
  relational interpretation.
\item Can also obtain a type non-inhabitation result based on logical
  derivability
\end{enumerate}

\fixme{
  \begin{enumerate}
  \item Make the definition of control of information flow clearer
  \item Make the example a bit more compelling
  \item Put in some examples of composite principals formed from
    boolean logic, and the use of boolean algebra to do reasoning in
    the types. List the laws of boolean algebra, and the laws of
    Heyting algebra. Need to explain the difference between 
  \item Possibly use singleton types to be able to do more interesting
    security policies?
  \item Put in the formal statement of the information flow property
    and the proof.
  \end{enumerate}
}

An extreme example of invariance under change of representation is the
tracking of information flow through programs. If a program does not
depend upon a particular piece of information, then it will be
invariant under \emph{all} changes of representation of this
information. As thoroughly described by Sabelfeld and Sands
\cite{sabelfeld01per}, information flow can be captured semantically
through the use of partial equivalence relations (PERs). Interpreting
types as PERs forms an instance of Reynolds' approach to abstraction
and representation independence. Abadi, Banerjee, Heintze and Riecke
\cite{abadi99core} built a \emph{Core Calculus for Dependency}, using
a type system based around a security level indexed monad $T_lA$. Tse
and Zdancewic \cite{tse04translating} showed that it is possible to
translate Abadi et al.'s calculus into System F, translating the
monadic type $T_lA$ to $\alpha_l \to A$ for some free type variable
$\alpha_l$, and making direct use of Reynolds' abstraction theorem to
prove information flow properties. We now show how a refined variant
of Tse and Zdancewic's translation can be expressed via
algebraically-indexed types.

We assume a single indexing sort $\mathsf{principal}$, intended to
represent (possibly composite) principals in a system. Principals are
combined using the connectives of boolean logic ($\top$, $\bot$,
$\lor$, $\land$, $\lnot$). For the equational theory of principals, we
assume all the axioms of boolean algebra, and semantically we
interpret composite principals as boolean values: either true ($\top$)
or false ($\bot$). We introduce a single primitive type, indexed by
expressions of sort $\mathsf{principal}$, $\tyPrim{T}{e}$, with the
following interpretations:
\begin{displaymath}
  \begin{array}{l@{\hspace{0.5em}=\hspace{0.5em}}l}
    \tySem{\tyPrim{T}{e}} & \{*\} \\
    \rsem{\tyPrim{T}{e}}{}\rho & \left\{
      \begin{array}{ll}
        \{(*,*)\} & \textrm{if }\sem{e}\rho = \top \\
        \{\}      & \textrm{if }\sem{e}\rho = \bot
      \end{array}
      \right.
  \end{array}
\end{displaymath}
The relational interpretation of $\tyPrim{T}{e}$ forces this type to
only be inhabited when the boolean expression $e$ evaluates to true
under the environment $\rho$. To write programs, we assume the
following primitive operations:
\begin{displaymath}
  \begin{array}{l}
  \begin{array}{@{}c@{\hspace{2em}}c}
    \mathrm{truth} : \tyPrim{T}{\top} &
    \mathrm{and}   : \forall p, q\mathord:\mathsf{prn}.\ \tyPrim{T}{p} \to \tyPrim{T}{q} \to \tyPrim{T}{p \land q}
  \end{array} \\
  \mathrm{coerce} : \forall p, q\mathord:\mathsf{prn}.\ \tyPrim{T}{p \land q} \to \tyPrim{T}{p}
\end{array}
\end{displaymath}
We can now adapt Tse and Zdancewic's translation of Abadi et al.'s
monadic type to our setting. We define a type synonym, for each type
$A$ and expression $e$ of sort $\mathsf{principal}$:
\begin{displaymath}
  T_eA = \tyPrim{T}{e} \to A
\end{displaymath}
For every principal $e$, we can endow the types $T_e-$ with the
structure of a monad. This is due to the fact that it is an instance
of the ``environment'' (or ``reader'') monad \cite{jones95functional}.

\begin{example}
  Suppose we have a program with the following type:
  \begin{displaymath}
    \forall p,q\mathord:\mathsf{principal}. \tyPrim{T}{e} \to T_p(\tyPrimNm{unit} + \tyPrimNm{unit}) \to T_q(\tyPrimNm{unit} + \tyPrimNm{unit})
  \end{displaymath}
  where $e$ is some boolean algebra expression involving $p$ and
  $q$. We will show in \autoref{sec:instantiations} that programs with
  this type are non-constant (i.e.~depend on their input) if and only
  if the formula $e \Rightarrow q \Rightarrow p$ is valid in boolean
  logic.
\end{example}

\begin{enumerate}
\item Treat the information flow thing as a type isomorphism, in the
  same way that Tse and Zdancewic do
  \begin{displaymath}
    \begin{array}{l}
    \forall \vec{p}\mathord:\mathsf{principal}.\ T_{e_1}\tyPrimNm{bool} \to T_{e_2}\tyPrimNm{bool}\\
    \hspace{4em}\cong \\
    \left\{
      \begin{array}{ll}
        \tyPrimNm{bool} \tyArr \tyPrimNm{bool} & \textrm{if }e_2 \vdash e_1 \\
        \tyPrimNm{bool} & \textrm{if }e_2 \not\vdash e_1
      \end{array}
    \right.
  \end{array}
  \end{displaymath}
  where $\tyPrimNm{bool} = \tyUnit + \tyUnit$.
\item mention the issue with bogus tokens if we allow for a
  call-by-name style of non-termination
\item if we axiomatise intuitionistic logic instead of
  boolean/classical logic, we get a different type isomorphism.
\item The isomorphism relies upon $e_1 \vdash e_2 \Leftrightarrow e_1
  \lor e_2 = e_1$.
\item Relevant logic? Look at the Wikipedia page to see how to do
  residuated (commutative) monoids on lattices with just
  equations. With this, we will be able to do substructural logics
  too. As long as we know that a particular axiomatisation is sound
  and complete for some reasoning system, then we can state and prove
  the non-interference isomorphism above.
\item Compare to Tse and Zdancewic's addition of axioms to the logic.
\item Internalising equality? The logical structure can then be used
  on types as well as everything else. Need the special rules for
  equality... (see the Fomega paper).
\end{enumerate}

% Go a step further by considering equivalence relations on word32, and
% their lattice structure. Link this to quantative information
% flow. Will also need to include singleton types.

%%% Local Variables:
%%% TeX-master: "paper"
%%% End:
