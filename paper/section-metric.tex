\section{Distance-Indexed Types}
\label{sec:metric-types}

\newcommand{\Metric}{\mathit{Met}}
\newcommand{\metricSort}{\mathsf{R}^{>0}}

The previous two sections have presented examples where the relational
interpretations of primitive types relates a pair of elements by some
transformation if applying the transformation to the first element
yields the second. Thus, the free theorems that we derived directly
take the form of ``invariance'' properties, where some equation holds
between two terms. In this section, we examine another instantiation
of our general framework that relates values when they are within a
certain distance. The free theorems that we obtain inform us of the
effect that programs have on the distances between values. For
example, a program $M$ of type:
\begin{displaymath}
  \forall \epsilon_1,\epsilon_2\mathord:\metricSort.\ \tyPrim{real}{\epsilon_1} \to \tyPrim{real}{\epsilon_2} \to \tyPrim{real}{\epsilon_1 + \epsilon_2}
\end{displaymath}
must satisfy the property that for all $\epsilon_1, \epsilon_2 > 0$
and $x, x', y, y' \in \mathbb{R}$,
\begin{displaymath} % FIXME: include \eta_\Metric
  \begin{array}{l}
    \textrm{if }|x - x'| < \epsilon_1 \textrm{ and } |y - y'| < \epsilon_2 \textrm{ then}\\
    \hspace{2em}|\tmSem{M}\ x\ y - \tmSem{M}\ x'\ y'| < \epsilon_1 + \epsilon_2
  \end{array}
\end{displaymath}
A type system with a relational interpretation that tracks distances
between values has %previously 
been investigated by Reed and Pierce
\cite{reed10distance} in the setting of differential privacy. Their
system uses a linear type discipline to ensure that all programs are
non-expansive (i.e., the distance between the outputs is no more greater
than the distance between the inputs). Here, we instantiate our general
framework %from \autoref{sec:a-general-framework} %in order 
to express
non-expansivity, and more, via algebraically indexed types.

\paragraph{Instantiation of the General Framework}
We assume a single indexing sort $\metricSort$ %, intended to 
to represent positive, non-zero real numbers. For the index
operations, we assume the operations $\min, \max, +$ and
multiplication by constant reals. There is a single primitive type
$\tyPrimNm{real}$ with $\primTyArity(\tyPrimNm{real}) =
[\metricSort]$. The primitive operations $\Gamma_{\Metric}$ are as
follows, where $c$ stands for arbitrary real-valued constants:
\begin{displaymath}
  \begin{array}{r@{\hspace{0.5em}:\hspace{0.5em}}l}
    \underline{c} & \forall \epsilon\mathord:\metricSort.\ \tyPrim{real}{\epsilon} \\
    (+) & \forall \epsilon_1, \epsilon_2\mathord:\metricSort.\ \tyPrim{real}{\epsilon_1} \to \tyPrim{real}{\epsilon_2} \to \tyPrim{real}{\epsilon_1 + \epsilon_2} \\
    (-) & \forall \epsilon_1, \epsilon_2\mathord:\metricSort.\ \tyPrim{real}{\epsilon_1} \to \tyPrim{real}{\epsilon_2} \to \tyPrim{real}{\epsilon_1 + \epsilon_2} \\
    \underline{c} * & \forall \epsilon\mathord:\metricSort.\ \tyPrim{real}{\epsilon} \to \tyPrim{real}{c\epsilon} \\
    \mathrm{coerce} & \forall \epsilon_1,\epsilon_2\mathord:\metricSort.\ \tyPrim{real}{\epsilon_1} \to \tyPrim{real}{\max(\epsilon_1,\epsilon_2)}
  \end{array}
\end{displaymath}
We assume that the index-erasure semantics of the $\tyPrimNm{real}$
type is just the set $\mathbb{R}$, so all except the last operation
have straightforward interpretations. The $\mathrm{coerce}$ operation
is interpreted just as the identity function. The index-erasure
interpretations of the primitive operations are collected together
into an environment $\eta_\Metric \in \ctxtSem{\Gamma_\Mon}$.

For the families of sets of relational environments
$\relEnv{E}_\Metric$, we use the general construction in
\autoref{sec:constr-rel-env} with the following interpretations of the
primitive type $\tyPrimNm{real}$: $R^\bullet_{\tyPrimNm{real}} = \Eq_\mathbb{R}$ and $R_{\tyPrimNm{real}}(\epsilon) = \{ (x,x') \sepbar |x - x'| < \epsilon\}$.
%\begin{displaymath} %FIXME: define \Eq earlier on
%  \begin{array}{l@{\hspace{2em}}l}
%    R^\bullet_{\tyPrimNm{real}} = \Eq_\mathbb{R} &
%    R_{\tyPrimNm{real}}(\epsilon) = \{ (x,x') \sepbar |x - x'| < \epsilon \}
%  \end{array}
%\end{displaymath}
We have:
\begin{lemma}
  For all $\Delta$ and $\rho \in \relEnv{E}_\Metric(\Delta)$,
  $(\eta_\Metric, \eta_\Metric) \in
  \rsem{\Gamma}{\relEnv{E}_\Metric}{\rho}$.
\end{lemma}

\begin{example}[Uniform Continuity]
  %By making use of 
Using %the 
existential types, % in our framework, 
we can
  state the standard 
$\epsilon$-$\delta$ definition of
  \emph{uniform} continuity as a type:
  \begin{displaymath}
    \forall \epsilon \mathord: \metricSort.\ \exists \delta\mathord: \metricSort.\ \tyPrim{real}{\delta} \to \tyPrim{real}{\epsilon}
  \end{displaymath}
  Given a program $M$ with this type, we obtain from
  \thmref{thm:abstraction} a free theorem, which is exactly the
  definition of uniform continuity:
  \begin{displaymath}
    \forall \epsilon \mathord> 0. \exists \delta \mathord> 0. \forall x,x'. |x - x'| < \delta \Rightarrow |\tmSem{M}\ x - \tmSem{M}\ x'| < \epsilon
  \end{displaymath}
  This definition differs from the $\epsilon$-$\delta$
  definition of (regular) continuity by the order of quantification: 
  %For continuity, 
  there, $\forall x$ comes before $\exists \delta$, so 
  the distance 
  $\delta$ may depend on %the value of 
  $x$. We suspect that to
  express standard continuity as a type would require some form of
  type dependency. Chaudhuri, Gulwani and Lublinerman
  \cite{chaudhuri10continuity} have given a program logic based
  approach to verifying the continuity of programs.
\end{example}

% \begin{example}[Differential Privacy]
%   Relative error? Probability monads?

%   Don't have space to develop this fully...

%   $\circ \tyPrimNm{real}$ -- probability distributions on the real numbers

%   Need to think about this...
%   \begin{displaymath}
%     \mathsf{noise} : \tyPrim{$\circ$real}{1}
%   \end{displaymath}
% \end{example}

% \begin{mathpar}
%   % commutativity
%   \min(\epsilon_1, \epsilon_2) = \min(\epsilon_2, \epsilon_1)

%   \max(\epsilon_1, \epsilon_2) = \max(\epsilon_2, \epsilon_1)

%   % associativity
%   \max(\epsilon_1, \max(\epsilon_2, \epsilon_3)) = \max(\max(\epsilon_1, \epsilon_2), \epsilon_3)

%   \min(\epsilon_1, \min(\epsilon_2, \epsilon_3)) = \min(\min(\epsilon_1, \epsilon_2), \epsilon_3)

%   % absorption
%   \max(\epsilon_1, \min(\epsilon_1, \epsilon_2)) = \epsilon_1

%   \min(\epsilon_1, \max(\epsilon_1, \epsilon_2)) = \epsilon_1
% \end{mathpar}

%%% Local Variables:
%%% TeX-master: "paper"
%%% End:

