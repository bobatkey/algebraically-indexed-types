\section{Distance-Indexed Types}
\label{sec:metric-types}

\newcommand{\Metric}{\mathit{Met}}
\newcommand{\metricSort}{\mathsf{R}^{>0}}

The type system for geometry we discussed in
\autoref{sec:motivating-examples} and \autoref{sec:instantiations}
made use of a relational interpretation of primitive types that
relates pairs of elements by some transformation if applying the
transformation to the first element of a pair yields the second. Thus,
the free theorems that we derived directly take the form of
``invariance'' properties, where some equation holds between two
terms. In this section, we examine another instantiation of our
general framework that relates values when they are within a certain
distance. The free theorems that we obtain inform us of the effect
that programs have on the distances between values. For example, a
program $M$ of type:
\begin{displaymath}
  \forall \epsilon_1,\epsilon_2\mathord:\metricSort.\ \tyPrim{real}{\epsilon_1} \to \tyPrim{real}{\epsilon_2} \to \tyPrim{real}{\epsilon_1 + \epsilon_2}
\end{displaymath}
must satisfy the property that for all $\epsilon_1, \epsilon_2 > 0$
and $x, x', y, y' \in \mathbb{R}$:
\begin{displaymath}
  \begin{array}{l}
    \textrm{if }\abs{x - x'} < \epsilon_1 \textrm{ and } \abs{y - y'} < \epsilon_2 \textrm{ then}\\
    \hspace{2em}\abs{\tmSem{M}\ \eta_\Metric\ x\ y - \tmSem{M}\ \eta_\Metric\ x'\ y'} < \epsilon_1 + \epsilon_2
  \end{array}
\end{displaymath}

\paragraph{Instantiation of the General Framework}
We assume a single indexing sort $\metricSort$ %, intended to 
to represent positive, non-zero real numbers. For the index
operations, we assume the operations $\min, \max, +$ and
multiplication by constant reals. There is a single primitive type
$\tyPrimNm{real}$ with $\primTyArity(\tyPrimNm{real}) =
[\metricSort]$. The primitive operations $\Gamma_{\Metric}$ are as
follows, where $c$ stands for arbitrary real-valued constants:
\begin{displaymath}
  \begin{array}{r@{\hspace{0.5em}:\hspace{0.5em}}l}
    \underline{c} & \forall \epsilon\mathord:\metricSort.\ \tyPrim{real}{\epsilon} \\
    (+) & \forall \epsilon_1, \epsilon_2\mathord:\metricSort.\ \tyPrim{real}{\epsilon_1} \to \tyPrim{real}{\epsilon_2} \to \tyPrim{real}{\epsilon_1 + \epsilon_2} \\
    (-) & \forall \epsilon_1, \epsilon_2\mathord:\metricSort.\ \tyPrim{real}{\epsilon_1} \to \tyPrim{real}{\epsilon_2} \to \tyPrim{real}{\epsilon_1 + \epsilon_2} \\
    \underline{c} * & \forall \epsilon\mathord:\metricSort.\ \tyPrim{real}{\epsilon} \to \tyPrim{real}{c\epsilon} \\
    \mathrm{up} & \forall \epsilon_1,\epsilon_2\mathord:\metricSort.\ \tyPrim{real}{\epsilon_1} \to \tyPrim{real}{\max(\epsilon_1,\epsilon_2)}
  \end{array}
\end{displaymath}
We assume that the index-erasure semantics of the $\tyPrimNm{real}$
type is just the set $\mathbb{R}$, so all except the last operation
have straightforward interpretations. The $\mathrm{up}$ operation is
interpreted just as the identity function. The index-erasure
interpretations of the primitive operations are collected together
into an environment $\eta_\Metric \in \ctxtSem{\Gamma_\Metric}$.

For the relational intepretation we construct a model of the indexing
theory by interpreting $\metricSort$ with strictly positive real
numbers. We set $\rsem{\tyPrimNm{real}}(\epsilon) = \{ (x,x') \sepbar
\abs{x - x'} < \epsilon \}$.
\begin{lemma}\label{lem:metric-environments}
  $(\eta_\Metric, \eta_\Metric) \in \rsem{\Gamma_\Metric}{}*$.
\end{lemma}

\paragraph{Uniform continuity}
  %By making use of 
Using %the
existential types, % in our framework,
%we can state 
the standard $\epsilon$-$\delta$ definition of
\emph{uniform} continuity can be expressed as %a type:
%\begin{displaymath}
$  \forall \epsilon \mathord: \metricSort.\ \exists \delta\mathord: \metricSort.\\ \tyPrim{real}{\delta} \to \tyPrim{real}{\epsilon}$.
%\end{displaymath}
For any program $M$ of this type, %we obtain from
\thmref{thm:abstraction} gives a free theorem that is exactly 
%the definition of 
uniform continuity:
\begin{displaymath}
  \forall \epsilon \mathord> 0. \exists \delta \mathord> 0. \forall x,x'. \abs{x - x'} \mathord< \delta \Rightarrow \abs{\tmSem{M}\eta_\Metric~x - \tmSem{M}\eta_\Metric~x'} \mathord< \epsilon
\end{displaymath}
This definition differs from the $\epsilon$-$\delta$ definition of
(regular) continuity by the order of quantification:
  %For continuity, 
there, $\forall x$ comes before $\exists \delta$, so the distance
$\delta$ may depend on %the value of
$x$. We suspect that to express standard continuity as a type would
require some form of type dependency. Chaudhuri, Gulwani and
Lublinerman \cite{chaudhuri10continuity} have given a program logic
based approach to verifying the continuity of programs.

\paragraph{Function Sensitivity}
A type system with a relational interpretation that tracks distances
between values has %previously
been investigated by Reed and Pierce \cite{reed10distance} in the
setting of differential privacy. Their system uses a linear type
discipline to ensure that all programs are $c$-sensitive (i.e., the
distance between the outputs is no greater than $c$ times the
distance between the inputs, for some constant $c$). We can express
their central concept of $c$-sensitivity (for functions on the reals)
as an algebraically-indexed type: $\forall
\epsilon\mathord:\metricSort.\ \tyPrim{real}{\epsilon} \to
\tyPrim{real}{\frac{1}{c}\epsilon}$. Investigating the precise
connection between their system and ours is left to future work.


% \begin{example}[Differential Privacy]
%   Relative error? Probability monads?

%   Don't have space to develop this fully...

%   $\circ \tyPrimNm{real}$ -- probability distributions on the real numbers

%   Need to think about this...
%   \begin{displaymath}
%     \mathsf{noise} : \tyPrim{$\circ$real}{1}
%   \end{displaymath}
% \end{example}

% \begin{mathpar}
%   % commutativity
%   \min(\epsilon_1, \epsilon_2) = \min(\epsilon_2, \epsilon_1)

%   \max(\epsilon_1, \epsilon_2) = \max(\epsilon_2, \epsilon_1)

%   % associativity
%   \max(\epsilon_1, \max(\epsilon_2, \epsilon_3)) = \max(\max(\epsilon_1, \epsilon_2), \epsilon_3)

%   \min(\epsilon_1, \min(\epsilon_2, \epsilon_3)) = \min(\min(\epsilon_1, \epsilon_2), \epsilon_3)

%   % absorption
%   \max(\epsilon_1, \min(\epsilon_1, \epsilon_2)) = \epsilon_1

%   \min(\epsilon_1, \max(\epsilon_1, \epsilon_2)) = \epsilon_1
% \end{mathpar}

%%% Local Variables:
%%% TeX-master: "paper"
%%% End:

