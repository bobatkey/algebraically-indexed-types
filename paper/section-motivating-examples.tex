\section{Motivating Examples}
\label{sec:motivating-examples}

In this section we present three motivating examples that illustrate
why algebraically indexed types are an interesting object of study,
and why relational semantics is a useful and powerful way of studying
the semantics of algebraically indexed types. The common theme running
through each of the examples we present in this section is the use of
algebraic indexing to capture the invariance \emph{and} variance of
programs with respect to changes in their environment. The first
example that we consider is Kennedy's original units of measure
system.

\subsection{Units of Measure and Scale Invariance}
\label{sec:units-of-measure-example-intro}

FIXME:
\begin{enumerate}
\item Describe the units of measure system
\item Describe the concept of scale invariance
\item Describe the consequences of scale invariance
\end{enumerate}

\subsection{Vector Spaces and Frame Invariance}
\label{sec:vector-spaces-intro}

FIXME: integrate a cut-down version of the geometry-invariance note
here.

\subsection{Security Types and Satisfaction Invariance}
\label{sec:security-types-intro}

FIXME: re-do the security types example from the previous draft, but
making sure to re-phrase everything in terms of ``satisfaction''.

\subsection{Taking Stock}
\label{sec:taking-stock}

Main points to get across:
\begin{enumerate}
\item Algebraic indexing captures both the invariance properties as
  well as the variance properties of the action of programs.
\item Relational semantics provides a powerful way to connect the
  invariance and variance properties to types.
\item The assignment of relational semantics to algebraic indexed may
  not necessarily be compositional.
\end{enumerate}
