\section{Motivating Examples}
\label{sec:motivating-examples}

In this section we present three motivating examples that illustrate
why algebraically indexed types are an interesting object of study,
and why relational semantics is a useful and powerful way of studying
the semantics of algebraically indexed types. The common theme running
through each of the examples we present in this section is the use of
algebraic indexing to capture the invariance \emph{and} variance of
programs with respect to changes in their environment. The first
example that we consider is Kennedy's original units of measure
system.

\subsection{Units of Measure and Scale Invariance}
\label{sec:units-of-measure-example-intro}

Kennedy's units of measure system enforces, through a static typing
discipline, a semantic notion of unit correctness. Idea: units of
measure $\Rightarrow$ scale invariance $\Rightarrow$ representation
independence $\Rightarrow$ Reynolds-style parametricity.

To illustrate the semantic formulation of unit correctness, we present
a small example.  Let $f$ be a function that takes pairs of real
numbers to a real number:
\begin{displaymath}
  f : \mathbb{R} \times \mathbb{R} \to \mathbb{R},
\end{displaymath}
and let us assume that the first input to the function should denote
some quantity of distance, the second input should denote some
quantity of time, and that the output should denote some quantity of
distance over time. Intuitively, these unit constraints mean that $f$
may only do something like divide the first argument by the second,
and possibly multiply by some dimensionless constant. But how do we
state such conditions formally in terms of $f$?

One way to state unit correctness is in terms of preservation of
scaling. If we scale all distance inputs by some factor $k_d$, and all
time inputs by some factor $k_t$, then the output of $f$ should
correspondingly scale by the factor $k_d/k_t$ (since the supposed
units of the output is distance over time). In symbols:
\begin{displaymath}
  \forall k_d, k_t, x_d, x_t.\ f(k_dx_d, k_tx_t) = \frac{k_d}{k_t}f(x_d,x_t)
\end{displaymath}


Another way of looking at scale invariance is in terms of invariance
under change of representation. The number $5$, with respect to 

\begin{enumerate}
\item Semantic correctness for units of measure is enforced by scaling
  invariance
\item Systematise this by using Reynolds' idea that semantic type
  abstraction is formulated as preservation of relations. In the units
  of measure case, we assign a relational meaning to every unit
  expression.
\item Consequences...
\end{enumerate}

\subsection{Affine Spaces, Vector Spaces and Frame Invariance}
\label{sec:vector-spaces-intro}

An affine space is a vector space that has ``forgotten its origin''. 

FIXME: integrate a cut-down version of the geometry-invariance note
here.

\subsection{Security Types and Satisfaction Invariance}
\label{sec:security-types-intro}

FIXME: re-do the security types example from the previous draft, but
making sure to re-phrase everything in terms of ``satisfaction''.

% \subsection{Monadic Effects}
% \label{sec:monadic-effects}

% Maybe this will work...

\subsection{Taking Stock}
\label{sec:taking-stock}

Main points to get across:
\begin{enumerate}
\item Algebraic indexing captures both the invariance properties as
  well as the variance properties of the action of programs.
\item Relational semantics provides a powerful way to connect the
  invariance and variance properties to types.
\item The assignment of relational semantics to algebraic indexed may
  not necessarily be compositional.
\end{enumerate}


%%% Local Variables:
%%% TeX-master: "paper"
%%% End:
