\section{Introduction}
\label{sec:introduction}

% Possible angles of attack:
% \begin{itemize}
% \item Parametric polymorphic types allow us to prevent
%   over-specification of the behaviour of programs. For instance, the
%   type $\forall \alpha. [\alpha] \to [\alpha]$ is a generalisation of
%   the types $[\mathsf{int}] \to [\mathsf{int}]$ and $[\mathsf{char}]
%   \to [\mathsf{char}]$. Either of the latter two types over-specify
%   the behaviour of the function.
% \item There are other cases of programs that are over-specified. The
%   leading example we just below is of geometric programs that
%   manipulate coordinate data. Often, programs that manipulate
%   coordinate data are insensitive to geometric transformations. For
%   example, a program that computes the area of a triangle described by
%   three points is insensitive to translations or rotations applied to
%   all three points.
% \item 
% \end{itemize}

% There are three main points to get across:
% \begin{enumerate}
% \item Why algebraically indexed types?
% \item Why relational parametricity?
% \item Why study them together?
% \end{enumerate}

Key lessons:

\paragraph{Relational meanings are induced by indexes}

\paragraph{Indexes can usefully have algebraic structure}

\paragraph{Relational meanings can be non-compositional}

\subsection{Contributions}
\label{sec:contributions}

\begin{itemize}
\item Formulation of a general class of type systems that can either
  be used as programming languages in their own right, or as the
  targets of type-based analyses.
\item A collection of compelling examples of algebraically indexed
  types, including a novel type system for geometry, a refined type
  system for information flow, based on logic, and a simple type
  system for basic continuity analysis.
\item Deduction of useful free theorems in each of our main examples.x
\item A refined relational interpretation in order to derive
  non-definability results. Fixing a minor problem in Kennedy's work.
\end{itemize}

%%% Local Variables:
%%% TeX-master: "paper"
%%% End:
