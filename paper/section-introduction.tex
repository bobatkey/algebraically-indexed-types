\section{Introduction}
\label{sec:introduction}

Possible angles of attack:
\begin{itemize}
\item Parametric polymorphic types allow us to prevent
  over-specification of the behaviour of programs. For instance, the
  type $\forall \alpha. [\alpha] \to [\alpha]$ is a generalisation of
  the types $[\mathsf{int}] \to [\mathsf{int}]$ and $[\mathsf{char}]
  \to [\mathsf{char}]$. Either of the latter two types over-specify
  the behaviour of the function.
\item There are other cases of programs that are over-specified. The
  leading example we just below is of geometric programs that
  manipulate coordinate data. Often, programs that manipulate
  coordinate data are insensitive to geometric transformations. For
  example, a program that computes the area of a triangle described by
  three points is insensitive to translations or rotations applied to
  all three points.
\item 
\end{itemize}

There are three main points to get across:
\begin{enumerate}
\item Why algebraically indexed types?
\item Why relational parametricity?
\item Why study them together?
\end{enumerate}

Key lessons:

\paragraph{Relational meanings are induced by indexes}

\paragraph{Indexes can usefully have algebraic structure}

\paragraph{Relational meanings can be non-compositional}

\subsection{Contributions}
\label{sec:contributions}

\begin{itemize}
\item Formulation of indexed parametricity that talks about
  terms, not as part of terms (FIXME)
\item Therefore, it can act as an analysis of programs, which can be
  imposed after the fact
\item Allow for more refined relational interpretations in order to
  derive non-definability results.
\item 
\end{itemize}

%%% Local Variables:
%%% TeX-master: "paper"
%%% End:
