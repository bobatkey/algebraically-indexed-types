\section{Discussion}
\label{sec:discussion}

We presented a general framework for algebraically indexed types and
instantiated it to yield novel type systems for geometry, logical
information flow and distance-indexed types. Our framework further
demonstrates the power of relational reasoning about typed
programs. From \thmref{thm:abstraction}, we derived interesting free
theorems, type isomorphisms and non-definability results. We conclude
with some observations and suggestions for further work.

\paragraph{Further Applications and Extensions} We have covered
several applications of algebraically indexed types in this paper, but
there are undoubtedly many more. Geometry for dimensions greater than
two is an obvious candidate, as are systems that are invariant
under different geometric groups (e.g.,~the Poincar\'{e} group for
relativity). Mathematical Physics is particularly rich in theories
that have some notion of invariance, and it will be exciting to pin
down the precise connections between these and type systems for which
%analogues of Reynolds' 
an abstraction theorem holds. 
Cardelli and Gardner describe a process calculus that builds in 3D affine geometry~\cite{Cardelli},
proving that process behaviour is invariant under affine transformations. Distinguishing points from vectors
provides the appropriate abstraction barrier, and the geometric group is determined by inspecting
term syntax. It would be interesting to recast their language in terms of our indexed types to
obtain purely type-based invariance theorems.

Geometric theorem
proving is another %possible 
application. Harrison
\cite{harrison09without} comments on the pervasiveness of % usefulness of
invariance properties in this area. Programs in our framework
automatically satisfy invariance properties, removing the need for
{\em ad hoc} proofs. % of these facts.

% In \autoref{sec:metric-types} we presented a type system with
% distance-indexed types, and noted the similarity with Reed and
% Pierce's system for non-expansivity \cite{reed10distance}. Exploring
% the relationship--particularly with respect to probability
% monads--also looks promising.

Type and effect analyses use types indexed by effect annotations with
algebraic structure (e.g.,~sets of read/write effect labels with an
idempotent monoid structure). Benton {\em et al.}~have used relational
interpretations to prove effect-dependent equivalences
\cite{benton06reading}. An extension of our framework with
type-indexed types should be able to express their effect-indexed monads
and prove their equivalences.

Extending our framework with type dependency would also allow for
further applications. For example, we could consider a type of lists
of length $n$, indexed by elements of the permutation group
$S_n$. Bernardy {\em et al.}~have presented a general framework for
relational reasoning and an abstraction theorem for dependent types
\cite{bernardy12proofs}. However, they work with pure type systems,
which define type equality via untyped rewriting, so it is not
immediately obvious how to integrate arbitrary equational theories
into their framework.

\paragraph{Semantic Equality} In general, the semantic equality
$\stackrel{sem}\sim$ in \defref{def:semantic-equality} is not an
equivalence relation. If the interpretations of all 
primitive types are partial equivalence relations then semantic
equality is indeed an equivalence relation. However, this excludes the
geometry and distance-indexed examples. More generally, we can
consider relational interpretations that are \emph{difunctional}. (A
relation is difunctional if whenever $(x,y)$, $(x',y')$ and $(x,y')$
are in the relation then so is $(x',y)$.) Difunctionality is weaker
than being a PER, but still suffices to prove that semantic equality
is an equivalence relation. Hofmann \cite{hofmann08correctness} has
used difunctional relations in the setting of effect
analyses. Difunctionality covers all our examples except
distance-indexed types. Note that for both PERs and difunctional
relations we need to close the relational interpretation of
existential types under the appropriate property to ensure that all
types are interpreted as PERs/difunctional relations. For
distance-indexed types it is possible that a new notion of
equivalence based on closeness is required.

%%% Local Variables:
%%% TeX-master: "paper"
%%% End:
