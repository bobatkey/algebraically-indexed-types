\documentclass[natbib,preprint]{sigplanconf}

\usepackage[usenames]{color}
\definecolor{citationcolour}{rgb}{0,0.4,0.2}
\definecolor{linkcolour}{rgb}{0,0,0.8}
\usepackage{hyperref}
\hypersetup{colorlinks=true,
            urlcolor=linkcolour,
            linkcolor=linkcolour,
            citecolor=citationcolour,
            pdftitle=Relational Parametricity for Algebraically-Indexed Types,
            pdfauthor={Robert Atkey, Neil Ghani, Patricia Johann, Andrew Kennedy},
            pdfkeywords={}}  
\def\sectionautorefname{Section}
\def\subsectionautorefname{Section}
\def\subsubsectionautorefname{Section}

\title{Abstraction and Invariance for Algebraically Indexed Types}

\authorinfo{Robert Atkey\and Patricia Johann}
           {University of Strathclyde}
           {\{Robert.Atkey,Patricia.Johann\}@strath.ac.uk}

\authorinfo{Andrew Kennedy}
           {Microsoft Research Cambridge}
           {akenn@microsoft.com}

\usepackage{mathpartir}
\usepackage{amsmath}
\usepackage{amssymb}
\usepackage{amsthm}
\usepackage{stmaryrd}
\usepackage[all]{xy}

\newcommand{\todo}[1]{[\textbf{TODO}: #1]}

\newcommand{\GL}[1]{\mathrm{GL}_#1}
\newcommand{\SynGL}[1]{\mathsf{GL}_#1}
\newcommand{\SE}[1]{\mathsf{SE}_#1}
\newcommand{\SynSE}[1]{\mathsf{SE}_#1}
\newcommand{\Orth}[1]{\mathrm{O}_#1}
\newcommand{\SynOrth}[1]{\mathsf{O}_#1}
\newcommand{\Transl}[1]{\mathrm{T}_#1}
\newcommand{\SynTransl}[1]{\mathsf{T}_#1}
\newcommand{\Scal}{\mathrm{Scal}}
\newcommand{\SynScal}{\mathsf{Scal}}

% \newcommand{\GLtwo}{\mathrm{GL}_2}
% \newcommand{\SynGLtwo}{\mathsf{GL}_2}
% \newcommand{\GLone}{\mathrm{GL}_1}
% \newcommand{\SynGLone}{\mathsf{GL}_1}

\newcommand{\sepbar}{\mathrel|}

\newcommand{\Rel}{\mathrm{Rel}}
\newcommand{\relArrow}{\mathrel{\widehat\to}}
\newcommand{\relTimes}{\mathrel{\widehat\times}}
\newcommand{\relSum}{\mathrel{\widehat+}}

\newcommand{\setOfIntegers}{\mathbb{Z}}
\newcommand{\setOfBooleans}{\{\tmTT, \tmFF\}}

\newcommand{\id}{\mathrm{id}}

\newcommand{\SortSet}{\mathit{Sort}}
\newcommand{\IndexOpSet}{\mathit{IndexOp}}
\newcommand{\PrimTypeSet}{\mathit{PrimType}}
\newcommand{\IndexAxiomSet}{\mathit{IndexAx}}

\newcommand{\indexOp}[1]{\texttt{#1}}
\newcommand{\idxTms}[2]{\mathrm{IdxTm}(#1 \vdash #2)}
\newcommand{\idxTmAS}[1]{\mathrm{IdxTm}(#1)}

\newcommand{\tyInt}{\texttt{int}}
\newcommand{\tyBool}{\texttt{bool}}
\newcommand{\tyUnit}{\texttt{unit}}
\newcommand{\tyReal}{\texttt{real}}
\newcommand{\tyPrim}[2]{\textup{\texttt{#1}}\langle #2 \rangle}
\newcommand{\tyPrimNm}[1]{\texttt{#1}}
\newcommand{\primTyArity}{\mathrm{tyArity}}
\newcommand{\indexOpArity}{\mathrm{opArity}}

\newcommand{\tyArr}{\to}
\newcommand{\tyProduct}{\times}
\newcommand{\tyX}[1]{\texttt{X}\langle #1 \rangle}
\newcommand{\isType}{\textsf{ type}}
\newcommand{\isCtxt}{\textsf{ ctxt}}
\newcommand{\Ty}{\textsf{Ty}}
\newcommand{\ty}{\textsf{ty}}

\newcommand{\tmTT}{\texttt{tt}}
\newcommand{\tmFF}{\texttt{ff}}

\newcommand{\relEnv}[1]{\mathcal{#1}}
\newcommand{\tySem}[1]{\llbracket #1 \rrbracket^{\mathcal{T}}}
\newcommand{\ctxtSem}[1]{\llbracket #1 \rrbracket^{\mathcal{C}}}
\newcommand{\tmSem}[1]{\llbracket #1 \rrbracket^{\mathit{tm}}}
\newcommand{\tyPrimSem}[1]{\llbracket #1 \rrbracket^{\mathcal{T}_0}}
\newcommand{\rsem}[3]{\llbracket #1 \rrbracket^{\mathcal{R}}_{#2}{#3}}
\newcommand{\extends}[2]{\mathsf{ext}(#1,#2)}

\newtheorem{lemma}{Lemma}
\newtheorem{theorem}{Theorem}
\newtheorem{definition}{Definition}
\newcommand{\lemref}[1]{\hyperref[#1]{Lemma \ref*{#1}}}
\newcommand{\thmref}[1]{\hyperref[#1]{Theorem \ref*{#1}}}
\newcommand{\defref}[1]{\hyperref[#1]{Definition \ref*{#1}}}
\newcommand{\propref}[1]{\hyperref[#1]{Proposition \ref*{#1}}}
\newcommand{\corref}[1]{\hyperref[#1]{Corollary \ref*{#1}}}
\newcommand{\conref}[1]{\hyperref[#1]{Conjecture \ref*{#1}}}
\newcommand{\exref}[1]{\hyperref[#1]{Example \ref*{#1}}}
\newcommand{\statementref}[1]{\hyperref[#1]{Statement \ref*{#1}}}

\newcommand{\sem}[1]{\llbracket #1 \rrbracket}
\newcommand{\isDefinedAs}{\stackrel{\mathit{def}}=}

\newcommand{\Geom}{\mathit{Geom}}

\newtheoremstyle{examplestyle}
  {\topsep}  % space above
  {\topsep}  % space below
  {\normalfont}% name of font to use in the body of the theorem
  {0em}% measure of space to indent
  {\bfseries}% name of head font
  {.}% punctuation between head and body
  {5pt plus 1pt minus 1pt}% space after theorem head; " " = normal interword space
  {}% Manually specify head
\theoremstyle{examplestyle}
\newtheorem{example}{Example}

\newtheoremstyle{restatementstyle}
  {\topsep}  % space above
  {\topsep}  % space below
  {\itshape}% name of font to use in the body of the theorem
  {0em}% measure of space to indent
  {\bfseries}% name of head font
  {.}% punctuation between head and body
  {5pt plus 1pt minus 1pt}% space after theorem head; " " = normal interword space
  {\thmname{#1} \thmnote{(Restatement of #3)}}% Manually specify head
\theoremstyle{restatementstyle}
\newtheorem{restateLemma}{Lemma}
\newtheorem{restateTheorem}{Theorem}

\newcommand{\fixme}[1]{\textbf{FIXME: #1}}

\begin{document}

\maketitle

\begin{abstract}
  Reynolds' relational parametricity provides a powerful way to reason
  about programs in terms of invariance under changes of data
  representation. A dazzling array of applications of Reynolds' theory
  exists, exploiting invariance to yield ``theorems for free'',
  uninhabitation results, and encodings of algebraic datatypes.

  Outside computer science, invariance under change is a common theme
  running through many areas mathematics and physics. For example, the
  area of a triangle is not altered by rotation or flipping. If we
  scale a triangle, then we scale its area, maintaining an invariant
  relationship between the triangle and its area. The transformations
  under which properties are invariant are often organised into
  groups, with the algebraic structure reflecting the composability
  and invertibility of transformations.

  In this paper, we investigate programming languages whose types are
  indexed by algebraic structures such as groups of geometric
  transformations. Other examples include types indexed by
  principals--for information flow security--and types indexed by
  distances--for analysis of analytic continuity properties. Following
  Reynolds, we prove a general abstraction theorem that covers all
  these instances. Consequences of the abstraction theorem include
  free theorems expressing invariance properties of programs, type
  isomorphisms based on invariance properties, and non-definability
  results, indicating when certain algebraically indexed types are
  uninhabited or only inhabited by trivial programs.
\end{abstract}

\category{D.1.1}{Programming techniques}{Applicative (functional)
  programming} \category{D.2.4}{Software Engineering}{Software/Program
  Verification} \category{D.3.3}{Programming Languages}{Language
  Constructs and Features---Data types and structures}

\terms
  Languages, Theory, Types

\keywords
  parametricity, units of measure, dimensional analysis, invariance, computational geometry

%%%%%%%%%%%%%%%%%%%%%%%%%%%%%%%%%%%%%%%%%%%%%%%%%%%%%%%%%%%%%%%%%%%%%%%%%%%%%%
%%%%%%%%%%%%%%%%%%%%%%%%%%%%%%%%%%%%%%%%%%%%%%%%%%%%%%%%%%%%%%%%%%%%%%%%%%%%%%
\section{Introduction}
\label{sec:introduction}

% Possible angles of attack:
% \begin{itemize}
% \item Parametric polymorphic types allow us to prevent
%   over-specification of the behaviour of programs. For instance, the
%   type $\forall \alpha. [\alpha] \to [\alpha]$ is a generalisation of
%   the types $[\mathsf{int}] \to [\mathsf{int}]$ and $[\mathsf{char}]
%   \to [\mathsf{char}]$. Either of the latter two types over-specify
%   the behaviour of the function.
% \item There are other cases of programs that are over-specified. The
%   leading example we just below is of geometric programs that
%   manipulate coordinate data. Often, programs that manipulate
%   coordinate data are insensitive to geometric transformations. For
%   example, a program that computes the area of a triangle described by
%   three points is insensitive to translations or rotations applied to
%   all three points.
% \item 
% \end{itemize}

% There are three main points to get across:
% \begin{enumerate}
% \item Why algebraically indexed types?
% \item Why relational parametricity?
% \item Why study them together?
% \end{enumerate}

Key lessons:

\paragraph{Relational meanings are induced by indexes}

\paragraph{Indexes can usefully have algebraic structure}

\paragraph{Relational meanings can be non-compositional}

\subsection{Contributions}
\label{sec:contributions}

\begin{itemize}
\item Formulation of a general class of type systems that can either
  be used as programming languages in their own right, or as the
  targets of type-based analyses.
\item A collection of compelling examples of algebraically indexed
  types, including a novel type system for geometry, a refined type
  system for information flow, based on logic, and a simple type
  system for basic continuity analysis.
\item Deduction of useful free theorems in each of our main examples.x
\item A refined relational interpretation in order to derive
  non-definability results. Fixing a minor problem in Kennedy's work.
\end{itemize}

%%% Local Variables:
%%% TeX-master: "paper"
%%% End:

\section{Geometry via Algebraically Indexed Types}
\label{sec:motivating-examples}

We motivate our investigation of algebraically indexed types and their
relational interpretations by developing a novel type system for
programs that manipulate two-dimensional geometric data. Geometry is
rich with operations that are invariant under transformation: affine
operations are invariant under change of origin
(\autoref{sec:affine-vector-ops}), vector space operations are
invariant under change of basis, and dot product is invariant under
orthogonal changes of basis
(\autoref{sec:motivation-generalising}). On the other hand, some
geometric operations are interestingly variant under transformation.
For example, cross products vary with scalings of the plane
(\autoref{sec:scale-invariance}). We incorporate (in)variance
information about geometric primitives into type systems via
algebraically indexed types.

\subsection{Origin Invariance and Representation Independence}\label{sec:oiri}

The basic data structure used in programs that manipulate geometric data is
the $n$-tuple of numbers. In the 2-dimensional case, tuples 
$\vec{v} = (x,y)$ serve double duty, representing
%are called upon to represent both 
both \emph{points}---offsets from some origin---and
\emph{vectors}---offsets in their own right.  Despite their common
representation, points and vectors are very different, and
distinguishing between them is the key feature of \emph{affine
  geometry} (see, for example, Chapter 2 of Gallier's book
\cite{gallier11geometric}). Nevertheless, computational geometry
libraries traditionally either leave it to the programmer to maintain
the distinction between points and vectors, or else use different
abstract types for points and vectors to enforce it.
%(one might say that the standard mathematical formulation of affine
%spaces uses the abstract types approach). 
In this paper we investigate a more sophisticated approach based on 
types indexed by change of origin transformations.

This approach regards the difference between points and vectors as a
change of data representation. For example, if $(0,0)$ and $(10,20)$
are two origins, then the tuple $(1,1)$ with respect to $(0,0)$ and the
tuple $(11,21)$ with respect to $(10,20)$ represent \emph{the same
  point} because they have the same displacement from these two
origins, respectively.
%It is merely an artifact of the representation of points as pairs of
%numbers that they appear to be different.
This suggests that programs that manipulate points %data representing points
should be invariant with respect to changes of
origin. Programs that manipulate vectors, on the other hand, should
not be invariant under change of origin. Different vectors represent
different offsets, and the vector $(0,0)$ always represents the zero
offset.

Invariance under change of representation immediately recalls
Reynolds' fable about two professors teaching the theory of complex
numbers \cite{reynolds83types}. One professor represents complex
numbers using rectangular coordinates ($x + iy$), while the other
represents them using polar coordinates ($\alpha\cos\theta +
i\alpha\sin\theta$). Happily, after learning the basic operations on
complex numbers in the two representations, the two classes can
interact because the theory of complex numbers is invariant under the
choice of representation. Reynolds formalises the idea of invariance
under changes of representation as preservation of relations.  For
example, if a binary relation $R$ relates the rectangular and polar
representations of complex numbers, then a program that manipulates
complex numbers at a level of abstraction above their specific
representation should preserve $R$. %More precisely, if
%$f$ is a program that is invariant under the choice of representation
%of complex numbers,
%%taking complex numbers to complex numbers that respects invariance
%%under change of representation. Then 
%$c$ is a complex number in rectangular form, $c'$ is a complex number
%in polar form, and $R$ is a relation that relates the rectangluar and
%polar representations of complex numbers, then $(c,c') \in R$ implies
%$(f(c), f(c')) \in R$.

Reynolds' relational approach can be applied in the geometric setting
to show how quantifying over all changes of origin ensures the
invariance of programs under any particular choice of origin. For
this, we first define a family of binary relations on $\mathbb{R}^2$
that is indexed by changes of origin. Changes of origin are
represented by vectors in $\mathbb{R}^2$, and form a group
$\Transl{2}$ of translations under addition. The $\Transl{2}$-indexed
family of binary relations $\{ R_{\vec{t}} \subseteq \mathbb{R}^2
\times \mathbb{R}^2 \}_{\vec{t} \in \Transl{2}}$ is then defined by
%\begin{displaymath}
$R_{\vec{t}} = \{ (\vec{v}, \vec{v'}) \sepbar \vec{v'} = \vec{v} + \vec{t} \}$.
%\end{displaymath}
We then consider a function $f$ that takes as input two tuples in
$\mathbb{R}^2$ and returns a single tuple in $\mathbb{R}^2$. We intend
that the tuples all represent points with respect to the same origin,
and that $f$ is invariant under the choice of origin.  Reynolds'
relational approach formalises this intention precisely. For any
$\vec{t} \in \Transl{2}$: % (i.e.~any change of origin), the function
%$f$
%should satisfy the following statement:
\begin{equation}\label{eq:f-preserve-rel-frame}
  \forall (\vec{v_1},\vec{v'_1}) \in R_{\vec{t}},
  (\vec{v_2},\vec{v'_2}) \in R_{\vec{t}}. (f(\vec{v_1}, \vec{v_2}),
  f(\vec{v'_1}, \vec{v'_2})) \in R_{\vec{t}}
\end{equation}
Unfolding the definition of $R_{\vec{t}}$ gives the equivalent formulation,
%\statementref{eq:f-preserve-rel-frame} is equivalent to: 
again for all $\vec{t} \in \Transl{2}$:
\begin{equation}\label{eq:f-invariant-frame}
  \forall \vec{v_1}, \vec{v_2}.\ f(\vec{v_1} + \vec{t},\vec{v_2} +
  \vec{t}) = f(\vec{v_1},\vec{v_2}) + \vec{t}.
\end{equation}
Thus, Reynolds' preservation of relations, when instantiated with the
family of relations $\{R_{\vec{t}}\}$, yields exactly the geometric
property of invariance under change of origin.

% An example function $f$ that satisfies
% \statementref{eq:f-invariant-frame} is the following function that
% computes a particular affine combination of two points by working
% directly on their coordinate representation:
% \begin{displaymath}
%   f(\vec{v_1}, \vec{v_2}) = \frac{1}{2}\vec{v_1} + \frac{1}{2}\vec{v_2}.
% \end{displaymath}
% In general, affine combinations of points $\lambda_1\vec{v_1} +
% \lambda_2\vec{v_2}$, where $\lambda_1 + \lambda_2 = 1$, satisfy
% \statementref{eq:f-invariant-frame}. Affine combination is one of the
% fundamental building blocks of affine geometry -- the properties of
% points invariant under invertible affine maps
% (\cite{gallier11geometric}, Chapter 2). If we drop the condition that
% $\lambda_1 + \lambda_2 = 1$, then we are dealing with linear
% combinations of vectors, and we are no longer invariant with respect
% to changes of frame. However, linear combinations are invariant with
% respect to change of basis. We can represent changes of basis as
% linear invertible maps $B : \mathbb{R}^2 \to \mathbb{R}^2$. The set of
% all such maps forms the general linear group $\GL(2)$. We now define
% another family of relations $\{R_{\texttt{vec}}(B) \subseteq
% \mathbb{R}^2 \times \mathbb{R}^2 \}_{B \in \GL(2)}$ that relates two
% points up to change of basis:
% \begin{displaymath}
%   R_{\texttt{vec}}(B) = \{ (\vec{v_1},\vec{v_2}) \sepbar B\vec{v_2} = \vec{v_1} \}.
% \end{displaymath}
% Now the functions $f_{\lambda_1\lambda_2}(\vec{v_1},\vec{v_2}) =
% \lambda_1\vec{v_1} + \lambda_2\vec{v_2}$, for arbitrary $\lambda_1$
% and $\lambda_2$, do preserve the relations $R_{\texttt{vec}}(B)$, for
% all $B \in \GL(2)$. Unfolding the definition of $R_{\texttt{vec}}(B)$
% in the analogous statement to \statementref{eq:f-preserve-rel-frame}
% for $R_{\texttt{vec}}$ instead of $R_{\texttt{pt}}$, we can see that
% preservation of the relations $R_{\texttt{vec}}$ characterises the
% functions that are invariant under change of basis. For all $B \in \GL(2)$, we have
% \begin{displaymath}
%   \forall \vec{v_1}, \vec{v_2}.\ f(B\vec{v_1},B\vec{v_2}) = B(f(\vec{v_1},\vec{v_2})),
% \end{displaymath}
% and this is exactly the property of invariance under change of basis
% we required above of programs manipulating vectors.

% Note that the family of relations $R_{\texttt{vec}}$ is just
% $R_{\texttt{pt}}$ when restricted to elements of the group
% $\GL(2)$. In the type system we introduce in the next section, we
% combine points and vectors into the same data type. Whether it
% represents a point or a vector depends on the group of geometric
% transformations that we expect it to be invariant under.

% FIXME: vectors are not invariant under change of origin, so they are
% represented by the relation $R_0$, which is just equality. But they,
% and points are invariant under change of basis. However, as we shall
% see below, not all operations are invariant under all changes of
% basis. In particular the dot product is only invariant under
% orthogonal transformations.

\subsection{A Type System for Change of Origin Invariance}
\label{sec:type-system-geom-intro}

Reynolds also showed how a type discipline can be used to establish
that (the denotational interpretations of) programs preserve
relations. For Reynolds, the type discipline of interest was that of
the polymorphic $\lambda$-calculus, which supports the construction of
new types by universal quantification over types.
%, in which new types can be
%constructed by universal quantification over all types.
In terms of relations, Reynolds interprets universal quantification
over types as quantification over binary relations between denotations
of types. By contrast, in our statements of geometric invariance in
\autoref{sec:oiri} we did not quantify over all
relations, but instead quantified over all changes of origin and used
a specific choice of origin to select a relation from the family
$\{R_{\vec{t}}\}$. This suggests introducing quantification over
changes of origin into the language of types. We use the notation
$\forall t \mathord: \SynTransl{2}. A$ for quantification over all
2-dimensional translations (i.e.,~choices of origin) $t$, and refer to
$\SynTransl{2}$ as the \emph{sort} of $t$. Note the difference in
fonts used to distinguish the semantic group $\Transl{2}$ from the
syntactic sort $\SynTransl{2}$. We use a similar convention
below, too.

Since the sort $\SynTransl{2}$ represents an abelian group, we can
combine its elements using the usual group operations. We write
operations additively, using $e_1 + e_2$ for the group operation, $-e$
for inverse and $0$ for the unit.  We also regard expressions built
from variables and the group operations up to the abelian group
axioms. For example, we regard $e_1 + (e_2 + e_3)$ and $(e_1 + e_2) +
e_3$ as equivalent.

Our language of types includes the unit type $\tyPrimNm{unit}$ and,
for all types $A$ and $B$, the function type $A \to B$, the sum type
$A + B$, and the tuple type $A \times B$. We also assume a primitive
type $\tyPrimNm{real}$, used to represent scalars. Although tuples of
real numbers represent points and vectors in geometric applications,
we cannot express this via the type
%central data structure in geometric applications is the tuple of real
%numbers for representing points and vectors, we cannot simply express
%this as the type 
$\tyPrimNm{real} \times \tyPrimNm{real}$. Indeed,
%because this type does not have the correct relational
%interpretation. (
two elements of type $\tyPrimNm{real}$ are related if and only if they
are equal and, by Reynolds' interpretation of tuple types, two
elements of $\tyPrimNm{real} \times \tyPrimNm{real}$ are also related
if and only if they are equal. But since this does not give the
correct relational interpretations for points and vectors, we
introduce a new type $\tyPrim{vec}{e}$, indexed by expressions $e$ of
sort $\SynTransl{2}$, to represent them.  The index $e$ represents the
displacement by change of origin of a point of this type.  Although we
have taken pains to distinguish geometric points and vectors, we use
the name $\tyPrimNm{vec}$ for both to recall the computer science
notion of vector as a homogeneous sequence of values with a known
length (in this case, 2).

As is standard for parametricity, every type has two interpretations:
an index-erasure interpretation that ignores the indexing expression,
and a relational interpretation as a binary relation on the
index-erasure interpretation. We denote the index-erasure and
relational interpretations with the notations $\tySem{\cdot}$ and
$\rsem{\cdot}{}$ respectively. To give such interpretations for the
types $\tyPrim{vec}{e}$ and $\forall t\mathord:\SynTransl{2}.A$, we
assume for now that we can map each expression $e$ of sort
$\SynTransl{2}$ to an element $\sem{e}\rho$ of the group $\Transl{2}$
using some environment $\rho$ that interprets $e$'s free
variables. The index-erasure and relational interpretations of
$\tyPrim{vec}{e}$ are:
\begin{displaymath}
  \begin{array}{l@{\hspace{0.5em}=\hspace{0.5em}}l}
    \tySem{\tyPrim{vec}{e}} & \mathbb{R}^2
    \\ \rsem{\tyPrim{vec}{e}}{}\rho & R_{\sem{e}\rho} = \{
    (\vec{v},\vec{v'}) \sepbar \vec{v'} = \vec{v} + \sem{e}\rho \}
  \end{array}
\end{displaymath}
The index-erasure and relational interpretations of 
$\forall t\mathord:\SynTransl{2}.A$ are:
%is simply the interpretation of the
%underlying type $A$. The relational interpretation intersects the
%relational interpretations of the type $A$ under all extensions of the
%relational environment $\rho$:
\begin{displaymath}
  \begin{array}{l@{\hspace{0.5em}=\hspace{0.5em}}l}
    \tySem{\forall t\mathord:\SynTransl{2}.A} & \tySem{A}
    \\ \rsem{\forall t\mathord:\SynTransl{2}.A}{}\rho & \bigcap\{
    \rsem{A}{}{(\rho,\vec{t})} \sepbar \vec{t} \in \Transl{2} \}
  \end{array}
\end{displaymath}
The index-erasure and relational interpretations are given
formally in Sections~\ref{sec:erasure-semantics}
%the relational interpretation is in general more complex, due to the
%possibility of non-compositional interpretations of free index
%variables. The relational interpretation will be presented in
and~\ref{sec:relational-semantics}. %, respectively.

At the end of \autoref{sec:oiri} we considered functions $f :
\mathbb{R}^2 \times \mathbb{R}^2 \to \mathbb{R}^2$ that preserve all
changes of origin. This property of $f$ can be expressed in terms of
types by
%\begin{displaymath}
$  \mathit{f} : \forall t \mathord: \SynTransl{2}.\ \tyPrim{vec}{t}
  \times \tyPrim{vec}{t} \to \tyPrim{vec}{t}$.
%\end{displaymath}
Spelling out the relational interpretation of this type using the
definitions above and the standard relational interpretations for
tuple and function types, we recover
\statementref{eq:f-invariant-frame} exactly.

\subsection{Affine and Vector Operations}
\label{sec:affine-vector-ops}

Invariance under change of origin is the key feature of affine
geometry, whose central operation is the affine combination of points:
$\lambda_1\vec{v_1} + \lambda_2\vec{v_2}$, where $\lambda_1 +
\lambda_2 = 1$.  Geometrically, this can be interpreted as describing
all the points on the unique line through the points represented by
$\vec{v_1}$ and $\vec{v_2}$ (assuming $\vec{v_1} \not= \vec{v_2}$).
We add affine combination of points to our calculus as
follows: % typing and intended denotation:for
\begin{displaymath}
  \begin{array}{l}
    \mathrm{affComb} :\forall t \mathord:
    \SynTransl{2}.\ \tyPrim{vec}{t} \to \tyPrimNm{real} \to
    \tyPrim{vec}{t} \to \tyPrim{vec}{t} \\ 
    \tmSem{\mathrm{affComb}}\ \vec{v_1}\ r\ \vec{v_2} = (1-r)\vec{v_1} +
    r\vec{v_2} 
\end{array}
\end{displaymath}
It can be verified by hand that the index-erasure interpretation %the intended denotation
$\tmSem{\mathrm{affComb}}$ is invariant under all changes of origin, as
dictated by its type.
% By defining the primitive function $\mathrm{affComb}$ to take a
% single $\tyPrimNm{real}$ parameter $t$, we can easily ensure that we
% are taking the affine combination of two representatives of points,
% and not just an arbitrary linear combination.

\begin{example}
  The evaluation of quadratic B\'{e}zier curves (B\'{e}zier curves
  with two endpoints and a single control point) can be expressed
  using the affine combination primitive as follows: % and three steps
  % of De Casteljau's algorithm:
  \begin{displaymath}
    \begin{array}{@{}l}
      \mathrm{quadB\acute{e}zier} : \forall t
      \mathord:\SynTransl{2}.\ \tyPrim{vec}{t} \mathord\to \tyPrim{vec}{t}\mathord\to
      \tyPrim{vec}{t} \mathord\to \tyPrimNm{real} \mathord\to \tyPrim{vec}{t}
      \\ \mathrm{quadB\acute{e}zier}\ [t]\ p_0\ p_1\ p_2\ s = \\ \quad
      \mathrm{affComb}\ [t]\ (\mathrm{affComb}\ [t]\ p_0\ s\ p_1)\ s\ (\mathrm{affComb}\ [t]\ p_1\ s\ p_2)
      \\
    \end{array}
  \end{displaymath}
  For two endpoints $p_0$ and $p_2$, a control point $p_1$, and $s \in
  [0,1]$, an application
  $\mathrm{quadB\acute{e}zier}\ p_0\ p_1\ p_2\ s$ gives the point on
  the curve at ``time'' $s$.  The type of
  $\mathrm{quadB\acute{e}zier}$ immediately tells us that it preserves
  all changes of origin.
\end{example}
% Affine combination only provides us with an operation on points. We
% are also interested in operations on the vectors representing offsets
% between points. We now examine the correct types to assign to the
% vector space operations of addition of vectors, negation of vectors,
% multiplication by a scalar and the zero vector. The typings of these
% operations will make use of the abelian group structure of change of
% origin translations.

The obvious type for vector addition is
%\begin{displaymath}
$  (+) : \tyPrim{vec}{0} \to \tyPrim{vec}{0} \to \tyPrim{vec}{0}$.
%\end{displaymath}
%(we write binary operators intended to be used infix in parentheses
%when not appearing in infix position, following the Haskell
%syntax.)
But we can reflect the fact that $(+)$ is not invariant under change
of origin by giving it a more precise type that reflects how it varies
with change of origin:
\begin{displaymath}
  (+) : \forall t_1, t_2 \mathord: \SynTransl{2}.\ \tyPrim{vec}{t_1}
  \to \tyPrim{vec}{t_2} \to \tyPrim{vec}{t_1 + t_2}
\end{displaymath}
Intuitively, this type says that if the first input vector has been
displaced by $t_1$ and the second by $t_2$, then their sum is
displaced by $t_1 + t_2$. We can also negate vectors, yielding a
vector which points in the opposite direction. Negation negates
translation arguments:
\begin{displaymath}
  \mathrm{negate} : \forall t \mathord:
  \SynTransl{2}.\ \tyPrim{vec}{t} \to \tyPrim{vec}{-t}
\end{displaymath}
Finally, with the primitive operations of addition and negation of
vectors we can define the derived operation of subtraction:
\begin{displaymath}
  \begin{array}{l}
    (-) : \forall t_1,t_2 \mathord:\SynTransl{2}.\ \tyPrim{vec}{t_1}
    \to \tyPrim{vec}{t_2} \to \tyPrim{vec}{t_1 - t_2}
    \\ (-)\ [t_1]\ [t_2]\ p_1\ p_2 = p_1 + \mathrm{negate}\ p_2
  \end{array}
\end{displaymath}

Given two points that are invariant with respect to the same change of
origin---i.e.,~two values of type $\tyPrim{vec}{t}$---we can use
subtraction to compute their offset:
\begin{displaymath}
  \begin{array}{l}
    \mathrm{offset} : \forall t
    \mathord:\SynTransl{2}.\ \tyPrim{vec}{t} \to \tyPrim{vec}{t} \to
    \tyPrim{vec}{0} \\ \mathrm{offset}\ [t]\ p_1\ p_2 = p_1 - p_2
  \end{array}
\end{displaymath}
The result is a vector expressed with respect to the null change of
origin: note how the algebraic structure on the indexing theory
induces type equalities that can be used to simplify the type of the
result of $\mathrm{offset}$ from $\tyPrim{vec}{t-t}$ to
$\tyPrim{vec}{0}$. The type of $(+)$ can also be specialised to the
case of moving a point by a vector:
\begin{displaymath}
  \begin{array}{l}
    \mathrm{moveBy} : \forall t
    \mathord:\SynTransl{2}.\ \tyPrim{vec}{t} \to \tyPrim{vec}{0} \to
    \tyPrim{vec}{t} \\ \mathrm{moveBy}\ [t]\ p\ v = p + v
  \end{array}
\end{displaymath}
The types we assign to the remaining vector space primitives, namely
$\mathrm{0} : \tyPrim{vec}{0}$ for
the zero vector and $(*) : \tyPrimNm{real} \to
    \tyPrim{vec}{0} \to \tyPrim{vec}{0}$ for
multiplication by a scalar, do not describe any
interesting effects on translations.
%\begin{displaymath}
%  \begin{array}{l@{\hspace{0.5em}:\hspace{0.5em}}l@{\hspace{4em}}l@{\hspace{0.5%em}:\hspace{0.5em}}l}
%    \mathrm{0} & \tyPrim{vec}{0}& (*) & \tyPrimNm{real} \to
%    \tyPrim{vec}{0} \to \tyPrim{vec}{0}
%  \end{array}
%\end{displaymath}

% \fixme{If we had dependent types, could the affComb operation be
%   derived from an appropriately dependently typed scaling operation?}

\begin{example}\label{ex:type-iso}
  The vector space operators and the properties that follow from their
  types allow us to establish a useful type isomorphism. Consider
  functions with types following the schema:
  \begin{displaymath}
    \tau_{n} \isDefinedAs \forall
    t\mathord:\SynTransl{2}.\ \underbrace{\tyPrim{vec}{t} \to ... \to
      \tyPrim{vec}{t}}_{n+1\textrm{ times}} \to \tyPrimNm{real}
  \end{displaymath}
  Just by looking at the types $\tau_{n}$, we know that their
  inhabitants will be invariant under change of origin because of the
  quantification over all $t$ in $\SynTransl{2}$. So we may as well
  choose one of the input points as the origin and assume that all the
  other points are defined with respect to it.  This formalises the
  common mathematical practice of stating that ``without loss of
  generality'' we can take some point in a description of a problem to
  be the origin provided the problem statement is invariant under
  translation. Each type $\tau_{n}$ is isomorphic to the corresponding
  type $\sigma_{n}$:
  \begin{displaymath}
    \sigma_{n} \isDefinedAs \underbrace{\tyPrim{vec}{0} \to
      ... \tyPrim{vec}{0}}_{n\textrm{ times}} \to \tyPrimNm{real}
  \end{displaymath}
  We demonstrate these isomorphisms formally in
  \autoref{sec:instantiations}, in the more general setting of types
  indexed by abelian groups.
  % The isomorphism between these two families of types can be witnessed
  % by the following two functions.
  % \begin{displaymath}
  %   \begin{array}{l}
  %     i_{n,A} : \tau_{n,A} \to \sigma_{n,A} \\
  %     i_{n,A}\ f\ [B]\ (p_1, p_2, ..., p_n) = f\ [B \ltimes 0]\ (0, p_1, ..., p_n) \\
  %     \\
  %     i^{-1}_{n,A} : \sigma_{n,A} \to \tau_{n,A} \\
  %     i^{-1}_{n,A}\ g\ [F]\ (p_0, ..., p_n) = g\ [\pi_L(F)]\ (p_1-p_0, ..., p_n-p_0)
  %   \end{array}
  % \end{displaymath}
  % The direction defined by the function $i_{n,A}$ treats the inputs
  % $p_1,...,p_n$ as vector offsets from the origin $0$. The direction
  % witnessed by $i^{-1}_{n,A}$ picks the first point to act as the
  % origin, and uses the operation of vector subtraction we defined
  % above to turn each of the other points into offsets from this
  % point. Note that this isomorphism is not unique: for each $n$ we can
  % pick any of the $n+1$ inputs to act as the distinguished
  % origin. Thus the families of types $\tau_{n,A}$ and $\sigma_{n,A}$
  % are isomorphic, but not uniquely isomorphic.
\end{example}

\begin{example}\label{ex:uninhabited-type}
  So far we have emphasised the derivation of properties, or ``free
  theorems'', of programs from their types. But using more refined
  relational interpretations of types we can also show that certain
  types are uninhabited. For example, the type $\forall t \mathord:
  \SynTransl{2}.\ \tyPrim{vec}{t + t} \to \tyPrim{vec}{t}$ has no
  inhabitants. Intuitively, this is because we cannot remove the extra
  $t$ in $\tyPrim{vec}{t+t}$ using the vector operations. We formalise
  this non-definability result in \autoref{sec:general-nondef} using a
  specialised relational interpretation.
\end{example}

\subsection{Change of Basis Invariance}
\label{sec:motivation-generalising}

Although vector addition, negation, and scaling are not invariant
under change of origin, they are invariant under change of basis. As
with origin invariance, we can express basis invariance as
preservation of relations indexed by changes of basis. Change of basis
is achieved by applying an invertible linear map, and the collection
of all such maps on $\mathbb{R}^2$ forms the \emph{General Linear}
group $\GL{2}$, which we represent in our language by a new indexing
sort $\SynGL{2}$ with (non-abelian) group structure that we will write
multiplicatively. We then extend $\tyPrimNm{vec}$ to allow indices of
sort $\SynGL{2}$, as well as $\SynTransl{2}$, so that
$\tyPrim{vec}{B,t}$ is a vector that varies with change of basis $B$
and change of origin $t$. Formally, the index-erasure and relational
semantics of $\tyPrim{vec}{B,t}$ are given by:
\begin{displaymath}
  \begin{array}{l@{\hspace{0.5em}=\hspace{0.5em}}l}
    \tySem{\tyPrim{vec}{e_B,e_t}} & \mathbb{R}^2
    \\ \rsem{\tyPrim{vec}{e_B,e_t}}{}\rho & \{ (\vec{v},\vec{v'})
    \sepbar \vec{v'} = (\sem{e_B}\rho)\vec{v} + \sem{e_t}\rho \}
  \end{array}
\end{displaymath}

\paragraph{Affine Geometry} An \emph{affine transformation} is
an invertible linear map together with a translation. We can assign
types to all the primitive affine and vector space operations
indicating how they they behave with respect to affine
transformations:
\begin{eqnarray*}
  \mathrm{affComb} & : &
  \begin{array}[t]{@{}l}
    \forall B \mathord: \SynGL{2}, t \mathord: \SynTransl{2}.\\
    \hspace{0.2cm} \tyPrim{vec}{B,t} \to \tyPrimNm{real} \to
    \tyPrim{vec}{B,t} \to \tyPrim{vec}{B,t}
  \end{array}
  \\
  (+) & : &
  \begin{array}[t]{@{}l}
    \forall B \mathord: \SynGL{2}, t_1,t_2 \mathord: \SynTransl{2}. \\
    \hspace{0.2cm}\tyPrim{vec}{B,t_1} \to \tyPrim{vec}{B,t_2} \to
    \tyPrim{vec}{B,t_1 + t_2}
  \end{array}
  \\
  \mathrm{negate} & : & \forall B \mathord: \SynGL{2}, t \mathord: \SynTransl{2}.\ \tyPrim{vec}{B,t} \to \tyPrim{vec}{B,-t} \\
  0 & : & \forall B \mathord: \SynGL{2}.\ \tyPrim{vec}{B,0} \\
  (*) & : & \forall B \mathord: \SynGL{2}.\ \tyPrimNm{real} \to \tyPrim{vec}{B,0} \to \tyPrim{vec}{B,0}
\end{eqnarray*}

\paragraph{Euclidean Geometry} Euclidean geometry extends affine
geometry with the \emph{dot product}, or \emph{inner product},
operation of two vectors. The dot product is defined 
%on the coordinate representation 
by
%\begin{displaymath}
$  (x_1,y_1) \cdot (x_2,y_2) = x_1x_2 + y_1y_2$.
%\end{displaymath}
% The dot product is used to define properties of vectors such as their
% \emph{norm} $||\vec{v}|| = \vec{v}\cdot\vec{v}$, which is the square
% of the length of the vector $\vec{v}$, and the notion of
% orthogonality: two vectors $\vec{v_1}$ and $\vec{v_2}$ are orthogonal
% if $\vec{v_1}\cdot\vec{v_2} = 0$.
To %incorporate the operation of dot product into our calculus, we
assign it a type we note that, although dot product is not
invariant under $\GL{2}$ or $\Transl{2}$, it
% that we have considered so far. However,e  the dot product 
is invariant under the subgroup $\Orth{2}$ of $\GL{2}$ of
\emph{orthogonal} linear transformations, i.e., the subgroup of
invertible linear maps whose matrix representations' transposes are
equal to their inverses. We thus introduce a new sort $\SynOrth{2}$ of
orthogonal transformations, and overload the multiplicative group
operations for inhabitants of $\SynOrth{2}$. Further assuming an
injection $\iota_O$ that takes $e : \SynOrth{2}$ to $\iota_O(e) :
\SynGL{2}$ we assign dot product this type:
%a type to the dot product, using quantification over $\SynOrth{2}$:
\begin{displaymath}
  (\cdot) : \forall O \mathord: \SynOrth{2}.\ \tyPrim{vec}{\iota_OO, 0} \to \tyPrim{vec}{\iota_OO, 0} \to \tyPrimNm{real}
\end{displaymath}

\begin{figure}[t]
  \centering
  \begin{eqnarray*}
    0   &:& \forall s \mathord:\SynGL{1}.\ \tyPrim{real}{s} \\
    1   &:& \tyPrim{real}{1} \\
    (+) &:& \forall s \mathord:\SynGL{1}.\ \tyPrim{real}{s} \to \tyPrim{real}{s} \to \tyPrim{real}{s} \\
    (-) &:& \forall s \mathord:\SynGL{1}.\ \tyPrim{real}{s} \to \tyPrim{real}{s} \to \tyPrim{real}{s} \\
    (*) &:& \forall s_1,s_2 \mathord:\SynGL{1}.\ \tyPrim{real}{s_1} \to \tyPrim{real}{s_2} \to \tyPrim{real}{s_1s_2} \\
    (/) &:&
    \begin{array}[t]{@{}l@{}l}
      \forall s_1,s_2 \mathord:\SynGL{1}.\ \tyPrim{real}{s_1}\ & \to \tyPrim{real}{s_2} \\
      &\to \tyPrim{real}{s_1s_2^{-1}} + \tyPrimNm{unit} \\
    \end{array}\\
    \mathrm{abs} &:& \forall s \mathord:\SynGL{1}.\ \tyPrim{real}{s} \to \tyPrim{real}{\abs{s}} %\\
%    \mathrm{sqrt} &:& \tyPrim{real}{1} \to \tyPrim{real}{1}
  \end{eqnarray*}
  \caption{Operations on scaled real numbers}
  \label{fig:real-ops}
\end{figure}

The cross product of two vectors is defined on coordinate
representations as $(x_1,y_1) \times (x_2,y_2) = x_1y_2 - x_2y_1$.
Geometrically, the cross product is the signed area of the
parallelogram described by the pair of input vectors.  Under change of
basis by an invertible linear transformation $B$, the cross product of
two vectors varies with the determinant of $B$. This corresponds to
scaling the plane by the change of basis transformation, so we augment
our calculus with a new sort $\SynGL{1}$ of scale factors
(i.e.,~$1$-dimensional invertible linear maps). Semantically,
$\SynGL{1}$ ranges over the non-zero real numbers and forms an abelian
group which we write multiplicatively. We also add two new operations:
determinant, $\det B$, which takes an inhabitant of $\SynGL{2}$ to its
determinant in $\SynGL{1}$, and absolute value, $\abs{e}$, which takes
scaling factors to scaling factors. We also refine the type
$\tyPrimNm{real}$ of real numbers so that it is indexed by the sort
$\SynGL{1}$: $\tyPrim{real}{e}$. The old type $\tyPrimNm{real}$ is
then just $\tyPrim{real}{1}$, and the full collection of operations on
real numbers indexed by scaling factors is shown in
\autoref{fig:real-ops}. We can thus assign cross product the type:
%With the additional sort $\SynGL{1}$ and the refined type of real
%numbers, we can assign a type to the cross product:
\begin{displaymath}
  (\times) : \forall B \mathord: \SynGL{2}.\ \tyPrim{vec}{B,0} \to
  \tyPrim{vec}{B,0} \to \tyPrim{real}{\det B} 
\end{displaymath}
Since the absolute value of the determinant of an orthogonal
transformation is always $1$, we assume $\abs{\det (\iota_O O)} = 1$
to hold for any $O \in \SynOrth{2}$.

\begin{example}\label{ex:area-of-triangle-1}
  We can use the operations of this subsection to
  compute the area of a triangle. We have:
  \begin{displaymath}
    \begin{array}{@{}l}
      \mathrm{area} : \forall B\mathord:\SynGL{2},
      t\mathord:\SynTransl{2}.\ \\
      \hspace{0.8cm}\tyPrim{vec}{B, t} \to \tyPrim{vec}{B, t} \to
      \tyPrim{vec}{B, t} \to \tyPrim{real}{\abs{\det B}}
      \\ \mathrm{area}\ [B]\ [t]\ p_1\ p_2\ p_3 = \frac{1}{2} *
      \mathrm{abs}\ ((p_2 - p_1) \times (p_3 - p_1))
    \end{array}
  \end{displaymath}
  The calculation is performed in several steps, each of which removes
  some of the symmetry described by the type of
  $\mathrm{area}$. First, the two offset vectors $p_2 - p_1$ and $p_3
  - p_1$ are computed. These operations remove the effect of
  translations on the result in exactly the same way as the type
  isomorphism in \exref{ex:type-iso}. Next, we compute the cross
  product of the two vectors, which gives the area of the
  parallelogram described by the sides of the triangle and has type
  $\tyPrim{real}{\det B}$. This removes some of the symmetry due to
  invertible linear maps, but the cross product still varies with the
  sign of the determinant. We remove this symmetry as well using
  $\mathrm{abs}$. This gives a value of type $\tyPrim{real}{\abs{\det B}}$
  which we multiply by $\frac{1}{2}$ to recover the area of the
  triangle rather than that of the whole parallelogram. If we
  specialise $\mathrm{area}$ to just orthogonal transformations, the
  assumption $|\det (\iota_OO)| = 1$ gives the following type:
  \begin{displaymath}
    \begin{array}{@{}l}
      \mathrm{area} : \forall O\mathord:\SynOrth{2},
      t\mathord:\SynTransl{2}.\ \\
      \hspace{0.8cm}\tyPrim{vec}{\iota_OO, t} \to
      \tyPrim{vec}{\iota_OO, t} \to \tyPrim{vec}{\iota_OO, t} \to
      \tyPrim{real}{1}
    \end{array}
  \end{displaymath}
  This type shows that the area of a triangle is invariant under
  orthogonal transformations and translations. Combinations of such
  transformations are {\em isometries}, i.e., distance preserving
  maps.
\end{example}

% \begin{enumerate}
% \item Possibly do a $n$-body gravity simulator as an example?
% \item Have a special sort and type for rotations. An operation for
%   computing the rotation between two points, and for applying a
%   rotation to a vector. This will allow for more free theorems...
% \end{enumerate}

\subsection{Scale Invariance and Dimensional Analysis}
\label{sec:scale-invariance}

Indexing types by scaling factors brings us to the original
inspiration for the current work: Kennedy's interpretation of his
units of measure type system via scaling invariance
\cite{kennedy97relational}. Kennedy shows how interpreting types in
terms of scaling invariance brings the techniques of dimensional
analysis to bear on programming. The types of the real number
arithmetic operations in \autoref{fig:real-ops} are exactly the types
Kennedy assigns in his units of measure system, except for that of the
absolute value operation. Semantically, our type indexes by
%our assigned type is that, semantically, we are indexing by
non-zero scaling factors, whereas Kennedy's indexes by strictly
positive ones. 

In our two-dimensional setting we can add to Kennedy's one-dimensional
scaling invariance an operation $\iota_1$ that, semantically, takes
scale factors in $\GL{1}$ to invertible linear maps in $\GL{2}$,
i.e., takes numbers $s$ to matrices $\left(
  \begin{smallmatrix}s & 0 \\ 0 & s\end{smallmatrix}\right)$.  This
operation satisfies the equation $\det (\iota_1 s) = s^2$, indicating
that scaling the plane by $s$ in both directions scales areas by
$s^2$. %, as we now illustrate.

\begin{example}\label{ex:area-of-triangle-2}
  Just as we specialised the type of the $\mathrm{area}$ function to
  orthogonal transformations in \exref{ex:area-of-triangle-1}, we can
  also specialise $\mathrm{area}$'s type to scaling
  transformations. This yields the type:
  \begin{displaymath}
    \begin{array}{@{}l}
      \mathrm{area} : \forall s\mathord:\SynGL{1}, t\mathord:\SynTransl{2}.\ \\
      \hspace{0.8cm} \tyPrim{vec}{\iota_1s, t} \to \tyPrim{vec}{\iota_1s, t} \to \tyPrim{vec}{\iota_1s, t} \to \tyPrim{real}{s^2}
    \end{array}
  \end{displaymath}
  As expected, the area of a triangle varies with the square of
  scalings of the plane, and this is reflected in the type.
\end{example}

Linear maps of the form $\left(
  \begin{smallmatrix}s & 0 \\ 0 & s\end{smallmatrix}\right)$, as
generated by $\iota_1$, commute with all other invertible linear
maps. We thus require
$(\iota_1 s)B = B(\iota_1 s)$ to hold.
The scaling maps $\left(
  \begin{smallmatrix}s & 0 \\ 0 & s\end{smallmatrix}\right)$
are precisely the elements of $\GL{2}$ that commute with all others;
these form the \emph{centre} of $\GL{2}$. If we keep track of
scalings, then we can assign the more precise types to
scalar multiplication and dot product. These
 are shown in \autoref{fig:vec-ops}, which summarises the most general types 
of all the vector operations that we have described.

\begin{example}
  With the operations in \autoref{fig:real-ops}, it is not possible to
  write a term with the following type that is not constantly zero:
  \begin{displaymath}
    \forall s \mathord: \SynGL{2}.\ \tyPrim{real}{s^2} \to \tyPrim{real}{s}
  \end{displaymath}
  This was shown by Kennedy for his units of measure system
  \cite{kennedy97relational}.  In particular, it is not possible to
  write a square root function with the above type.  The
  non-definability of square root is similar to the uninhabitation of the
  type in \exref{ex:uninhabited-type}.

  In \autoref{sec:types-indexed-abelian-groups-indef} we revisit
  Kennedy's result and show that even if we add square root as a
  primitive operation---with the type above---then it is still not
  possible to construct the cube root function. The non-definability of
  cube root is related to the impossibility of trisecting an arbitrary
  angle by ruler and compass constructions.
%  is of interest due to its relevance to the classical problem of
%  trisecting an angle by ruler and compass constructions.
\end{example}


\begin{figure}[t]
  \centering
  \begin{align*}
    0 & : \forall B \mathord: \SynGL{2}.\ \tyPrim{vec}{B,0} \\
    (+) & :
    \begin{array}[t]{@{}l}
      \forall B \mathord: \SynGL{2}, t_1,t_2 \mathord: \SynTransl{2}. \\
      \tyPrim{vec}{B,t_1} \to \tyPrim{vec}{B,t_2} \to
      \tyPrim{vec}{B,t_1 + t_2}
    \end{array}
    \\
    \mathrm{negate} & : \forall B \mathord: \SynGL{2}, t \mathord: \SynTransl{2}.\ \tyPrim{vec}{B,t} \to \tyPrim{vec}{B,-t} \\
  (*) & : 
  \begin{array}[t]{@{}l}
    \forall s \mathord: \SynGL{1}, B \mathord:\SynGL{2}.\ \\
    \tyPrim{real}{s} \to \tyPrim{vec}{B,0} \to \tyPrim{vec}{\iota_1(s)B,0}
  \end{array}\\
    \mathrm{affComb} &:
    \begin{array}[t]{@{}l}\forall t \mathord:
    \SynTransl{2}.\ \tyPrim{vec}{t} \to \tyPrim{real}{1} \to
    \tyPrim{vec}{t} \to \tyPrim{vec}{t}
    \end{array} \\
    (\cdot) & :
\begin{array}[t]{@{}l}
    \forall s \mathord: \SynGL{1}, O \mathord: \SynOrth{2}.\ \tyPrim{vec}{\iota_1(s)\iota_O(O),0} \to\\
    \hspace{2cm}\tyPrim{vec}{\iota_1(s)\iota_O(O),0}\to \tyPrim{real}{s^2}
\end{array}\\
  (\times) & : \forall B \mathord: \SynGL{2}.\ \tyPrim{vec}{B,0} \to
  \tyPrim{vec}{B,0} \to \tyPrim{real}{\det B} 
\end{align*}
  \caption{Operations on vectors}
  \label{fig:vec-ops}
\end{figure}


%%% Local Variables:
%%% TeX-master: "paper"
%%% End:


\section{A General Framework}
\label{sec:a-general-framework}

%In this section 
We now present our framework for algebraically
indexed types. % and representation independence. %We define the syntax
%and semantics of algebraically indexed types, then their semantics
%and finally we present a general programming language for
%algebraically indexed types, together with an abstraction theorem.

\subsection{Algebraically-Indexed Types}
\label{sec:algebraically-indexed-types}

The index expressions and types of an instantiation of our general
framework are derived from the following data:
\begin{enumerate}
\item A collection $\SortSet$ of index sorts. We use the
  meta-syntactic variables $s,s_1,s_2,...$ for arbitrary sorts taken
  from $\SortSet$.
\item A collection $\IndexOpSet$ of index operations, with a function
  $\indexOpArity : \IndexOpSet \to \SortSet^* \times \SortSet$. (We use
  the notation $A^*$ to denote the set of lists of elements of some
  set $A$.)
\item A collection $\PrimTypeSet$ of primitive types, with a function
  $\primTyArity : \PrimTypeSet \to \SortSet^*$. %, describing the sorts
%  of the arguments of each primitive type.
\end{enumerate}

\begin{example}[Two-Dimensional Geometry]
  \label{ex:two-dim-geo-operations}
  The two-dimensional geometry system has a sort for each of the
  geometric groups mentioned in \autoref{sec:motivating-examples}, so
  $\SortSet = \{\SynTransl{2}, \SynGL{2}, \SynOrth{2}, \SynGL{1} \}$.
  %For the index operations, 
  We have additive group structure on $\SynTransl{2}$, multiplicative
  group structure on $\SynGL{1}$, $\SynGL{2}$, and $\SynOrth{2}$,
  injections from $\SynOrth{2}$ and $\SynGL{1}$ into $\SynGL{2}$,
  determinant, and absolute value. Thus, $\IndexOpSet = \{ 0, +, -,
  1_G, -\cdot_G-, -^{-1_G}, \iota_O, \iota_1, \det, \abs{\cdot} \}$, where
  $G \in \{\SynGL{1}, \SynGL{2}, \SynOrth{2}\}$, and
  %  with the following assignment of arities:
  \begin{displaymath}
    \begin{array}{@{}l@{\hspace{0em}=\hspace{0em}}l@{\hspace{0.5em}}l@{\hspace{0em}=\hspace{0em}}l}
      \indexOpArity(0) & ([], \SynTransl{2}) &
      \indexOpArity(1_G) & ([], G) \\
      \indexOpArity(+) & ([\SynTransl{2}, \SynTransl{2}], \SynTransl{2}) &
      \indexOpArity(\cdot_G) & ([G,G],G) \\
      \indexOpArity(-) & ([\SynTransl{2}], \SynTransl{2}) &
      \indexOpArity(^{-1_G}) & ([G], G) \\
      \indexOpArity(\iota_o) & ([\SynOrth{2}], \SynGL{2}) &
      \indexOpArity(\iota_1) & ([\SynGL{1}], \SynGL{2}) \\
      \indexOpArity(\det) & ([\SynGL{2}], \SynGL{1}) &
      \indexOpArity(\abs{\cdot}) & ([\SynGL{1}], \SynGL{1})
    \end{array}
  \end{displaymath}
%  The intended interpretations of the top three pairs of operations
%  are group unit, group combination and group negation,
%  respectively. 
%  When we discuss equational theories on index expressions 
%in \autoref{sec:type-equality} 
%we will impose the (abelian) group laws. For this example, 
We also have $\PrimTypeSet = \{ \tyPrimNm{vec},
  \tyPrimNm{real} \}$, with $\primTyArity(\tyPrimNm{vec}) =
           [\SynGL{2}, \SynTransl{2}]$ and
           $\primTyArity(\tyPrimNm{real}) = [\SynGL{1}]$.
\end{example}

We assume a countably infinite collection of index variable names $i,
i_1, i_2,\ldots$. \emph{Index contexts} $\Delta = i_1 \mathord: s_1,
..., i_n \mathord: s_n$ are lists of variable/sort pairs such that
all the variable names are distinct.
\begin{figure*}[t]
  \centering
  % \textbf{Index contexts}\\

  % $\Delta = i_1 \mathord: s_1, ..., i_n \mathord: s_n$, where each $s_i \in \SortSet$ and no variable name is repeated.

  % \bigskip

  {\small
  \textbf{Well-sorted index expressions}
  \begin{mathpar}
    \inferrule* [right=IVar]
    {i : s \in \Delta}
    {\Delta \vdash i : s}
    
    \inferrule* [right=IOp]
    {\indexOp{f} \in \mathit{IndexOp} \\
      \indexOpArity(\indexOp{f}) = ([s_1,...,s_n], s) \\
      \{\Delta \vdash e_j : s_j\}_{1 \leq j \leq n}}
    {\Delta \vdash \indexOp{f}(e_1, ..., e_n) : s}
  \end{mathpar}

  \smallskip

  \textbf{Well-indexed types}
  \begin{mathpar}
    \inferrule* [right=TyPrim]
    {\tyPrimNm{X} \in \mathit{PrimType} \\\\
      \primTyArity(\tyPrimNm{X}) = [s_1,...,s_n] \\
      \{\Delta \vdash e_j : s_j\}_{1\leq j \leq n}}
    {\Delta \vdash \tyPrim{X}{e_1,...,e_n} \isType}

    \inferrule* [right=TyUnit]
    { }
    {\Delta \vdash \tyUnit \isType}

    \inferrule* [right=TyArr]
    {\Delta \vdash A \isType \\ \Delta \vdash B \isType}
    {\Delta \vdash A \tyArr B \isType}

    \inferrule* [right=TyTuple]
    {\Delta \vdash A \isType \\ \Delta \vdash B \isType}
    {\Delta \vdash A \tyProduct B \isType}

    \inferrule* [right=TySum]
    {\Delta \vdash A \isType \\ \Delta \vdash B \isType}
    {\Delta \vdash A + B \isType}
    
    \inferrule* [right=TyForall] %FIXME: macroize forall
    {\Delta, i \mathord: s \vdash A \isType}
    {\Delta \vdash \forall i \mathord: s. A \isType}

    \inferrule* [right=TyEx]
    {\Delta, i \mathord: s \vdash A \isType}
    {\Delta \vdash \exists i \mathord: s. A \isType}
  \end{mathpar}}
  \caption{Index expressions and types}
  \label{fig:indexes-and-types}
\end{figure*}
%Given the above data, 
The rules in \autoref{fig:indexes-and-types}
generate two judgements: well-sorted index expressions $\Delta \vdash
e : s$ and well-indexed types $\Delta \vdash A \isType$. Since index
variables may appear in types, types are judged to be well-indexed
with respect to an index context $\Delta$. The rules for well-sorted
index expressions are particularly simple. %: either an index expression
%is a variable that appears in the context (rule \TirName{IVar}), or it
%is an application of an index operation taken from $\IndexOpSet$ to
%other index expressions (rule \TirName{IOp}). 
The rules for well-indexed types include the usual ones
%rules 
for
%constructing types of 
the simply-typed $\lambda$-calculus with unit, sum and tuple types
(rules \TirName{TyUnit}, \TirName{TyArr}, \TirName{TyTuple} and
\TirName{TySum}). 
We use $\tyPrimNm{bool}$ as an abbreviation for $\tyPrimNm{unit} + \tyPrimNm{unit}$.
The rule \TirName{TyPrim} %allows us to 
forms, from a primitive type $\tyPrimNm{X}$ and appropriately sorted
index expressions $e_1,...,e_n$, the well-indexed type
$\tyPrim{X}{e_1,...,e_n}$. The rule \TirName{TyForall} 
%permits the formation of 
forms universally quantified types, where the universal
quantification ranges over all index expressions of some
sort. Existential types, %are constructed 
formed using the \TirName{TyEx} rule,
%and 
allow for abstraction by hiding. %FIXME: think about the
%introduction of existentials more, perhaps forward ref to their use
We say that a type is \emph{quantifier-free} if it does not contain
ether universal or existential quantification.

\subsubsection{Simultaneous Substitution of Index Expressions}
\label{sec:simultaneous-substitution}

It %will be technically 
is convenient to express substitution of index
expressions %in our framework 
in terms of simultaneous substitutions.
Given a pair of index contexts $\Delta$ and $\Delta' = i_1 \mathord:
s_1, ..., i_n \mathord: s_n$, a \emph{simultaneous substitution}
$\Delta \vdash \sigma \Rightarrow \Delta'$ is a sequence of
expressions $\sigma = (e_1,...,e_n)$ such that $\Delta \vdash e_j :
s_j$ for all $1 \leq j \leq n$. Given a simultaneous substitution
$\Delta \vdash \sigma = (e_1,...,e_n) \Rightarrow \Delta'$ and a
variable $i_j \mathord: s_j$ in $\Delta'$, we write $\sigma(i_j)$ for
the index expression $e_j$. We write $\Delta \Rightarrow \Delta'$ for
the set of all simultaneous substitutions $\sigma$ such that $\Delta
\vdash \sigma \Rightarrow \Delta'$.
%
%It will be useful to 
We can think of any sequence of sorts as an index
context. In particular, we will make use of simultaneous substitutions
of the form $\Delta \vdash \sigma \Rightarrow
\primTyArity(\tyPrimNm{X})$, since these are exactly sequences of
index arguments suitable for the primitive type $\tyPrimNm{X}$. By
further abuse of notation, we write $\Delta \Rightarrow
\primTyArity(\tyPrimNm{X})$ for the set of all simultaneous
substitutions $\sigma$ such that $\Delta \vdash \sigma \Rightarrow
\primTyArity(\tyPrimNm{X})$.

For a simultaneous substitution $\Delta \vdash \sigma \Rightarrow
\Delta'$, where $\Delta' = i_1\mathord:s_1,...,i_n\mathord:s_n$, and a
variable/sort pair $i\mathord:s$ such that $i$ does not appear in
either $\Delta$ or $\Delta'$, we can form the \emph{lifted}
simultaneous substitution $\Delta,i\mathord:s \vdash
\sigma_{i\mathord:s} = (\sigma(i_1), ..., \sigma(i_n), i) \Rightarrow
\Delta',i\mathord:s$. %Note that we have implicity used the fact that
%well-sortedness of index expressions is preserved by addition of extra
%items to the context.
%
Application of a simultaneous substitution $\Delta \vdash \sigma
\Rightarrow \Delta'$ to a well-sorted index expression $\Delta' \vdash
e : s$ yields a well-sorted index expression $\Delta \vdash \sigma^*e
: s$. The expression $\sigma^*e$ is defined on variables as $\sigma^*i
\isDefinedAs \sigma(i)$, and on operation symbols as
$\sigma^*(\indexOp{f}(e_1,...,e_n)) \isDefinedAs
\indexOp{f}(\sigma^*e_1, ..., \sigma^*e_n)$.  %Similarly, 
Given
%a well-indexed type 
$\Delta' \vdash A \isType$, 
%can apply $\sigma$ to $A$ to produce a new well-indexed type 
we have $\Delta \vdash \sigma^*A
\isType$. The key clauses defining $\sigma^*A$ are for primitive types
and the universal and existential quantifiers:
\begin{displaymath}
  \begin{array}{c}
    \sigma^*(\tyPrim{X}{e_1,...,e_n}) \isDefinedAs \tyPrim{X}{\sigma^*e_1,...,\sigma^*e_n}
    \\
    \begin{array}{c@{\hspace{2em}}c}
      \sigma^*(\forall i\mathord:s.A) \isDefinedAs \forall i\mathord:s.\sigma_{i\mathord:s}^*A
      &
      \sigma^*(\exists i\mathord:s.A) \isDefinedAs \exists i\mathord:S. \sigma_{i\mathord:s}^*A
    \end{array}
  \end{array}
\end{displaymath}
% \begin{lemma}
%   Let $\Delta \vdash \sigma \Rightarrow \Delta'$ be a simultaneous
%   substitution.
%   \begin{enumerate}
%   \item If $\Delta' \vdash e : s$, then $\Delta \vdash \sigma^*e : s$; and
%   \item If $\Delta' \vdash A \isType$, then $\Delta \vdash \sigma^*A
%     \isType$.
%   \end{enumerate}
% \end{lemma}
The \emph{identity} simultaneous substitution $\Delta \vdash
\id_\Delta \Rightarrow \Delta$ is %just the sequence of variables in
%$\Delta$: 
$\id_\Delta = (i_1,...,i_n)$ where $\Delta =
i_1\mathord:s_1,...,i_n\mathord:s_n$. The \emph{composition} of 
%two simultaneous substitutions 
$\Delta \vdash \sigma \Rightarrow \Delta'$
and $\Delta' \vdash \sigma' \Rightarrow \Delta''$, where $\sigma' =
(e'_1,...,e'_n)$, is defined as $\Delta \vdash \sigma' \circ \sigma
\isDefinedAs (\sigma^*e'_1, ..., \sigma^*e'_n) \Rightarrow \Delta''$.
%
Given a context $\Delta = i_1\mathord:s_1,...,i_n\mathord:s_n$, and a
variable/sort pair $i\mathord:s$ such that $i$ does not appear in
$\Delta$, we define the \emph{projection} simultaneous substitution
$\Delta, i\mathord:s \vdash \pi_{i\mathord:s} \Rightarrow \Delta$ as
$\pi_{i\mathord:s} = (i_1,...,i_n)$. %The subscript on
%$\pi_{i\mathord:s}$ is the variable/sort pair that is being discarded.

\subsubsection{Index Expression Equality and Type Equality}
\label{sec:type-equality}

Much of the power of indexing types by the expressions of an algebraic
theory comes from the equations of the theory. 
%In the %two-dimensional geometry example of 
For example, in 
\autoref{sec:motivating-examples} the types
$\tyPrim{vec}{B,t_1 + t_2}$ and $\tyPrim{vec}{B, t_2 + t_1}$ are
considered equal by the type system
%due to the commutativity of the $+$ operation.
because $+$ is commutative.
%
In the general framework, the equations between types are derived from
a set $\IndexAxiomSet$ of axioms $\Delta \vdash e \stackrel{ax}\equiv
e' : s$ that %We assume that all axioms $(\Delta \vdash e
%\stackrel{ax}\equiv e' : s)$ % \in \IndexAxiomSet$ 
are well-sorted, in the sense that both $\Delta \vdash e : s$ and $\Delta \vdash e' : s$
hold.

Given a set $\IndexAxiomSet$ of axioms, we generate the equality
judgment between index expressions $\Delta \vdash e \equiv e' : s$ by
a set of rules. The following rule lets us use substitution
instances of axioms:
\begin{displaymath}
  \inferrule*
  {(\Delta' \vdash e \stackrel{ax}\equiv e' : s) \in \IndexAxiomSet \\
    \Delta \vdash \sigma \Rightarrow \Delta'}
  {\Delta \vdash \sigma^*e \equiv \sigma^*e' : s}
\end{displaymath}
We also assume the standard congruence, reflexivity, symmetry and
transitivity rules for the equality judgment.

\begin{example}[Two-Dimensional Geometry]
  \label{ex:two-dim-geo-axioms}
  In \autoref{sec:motivating-examples} we assumed 
  %that 
  various equational axioms %hold 
  for indexing expressions standing for
  elements of geometric groups. Assuming the abelian group axioms for
  translations we can formalise this in our %general 
  framework:
  %We can now make this assumption formal
  %in our general framework. Semantically, translations form an abelian
  %group under addition, so we assume the abelian group axioms for
  %translations:
  \begin{displaymath}
    \begin{array}{l}
      t : \SynTransl{2} \vdash t + 0 \stackrel{ax}\equiv t : \SynTransl{2} \\
      t_1, t_2, t_3 : \SynTransl{2} \vdash t_1 + (t_2 + t_3) \stackrel{ax}\equiv (t_1 + t_2) + t_3 : \SynTransl{2} \\
      t : \SynTransl{2} \vdash t + (-t) \stackrel{ax}\equiv 0 : \SynTransl{2} \\
      t_1, t_2 : \SynTransl{2} \vdash t_1 + t_2 \stackrel{ax}\equiv t_2 + t_1 : \SynTransl{2} \\
    \end{array}
  \end{displaymath}
  Similarly, the sort of scale factors $\SynGL{1}$ forms an abelian
  group under multiplication, and the sorts $\SynGL{2}$ and
  $\SynOrth{2}$ form (non-abelian) multiplicative groups, so we assume
  the appropriate axioms. We also assume that the operations $\iota_O,
  \iota_1, \det$ and $\abs{\cdot}$ are group homomorphisms, and that
  expressions of the form $\iota_1(s)$ commute with group
  multiplication in the sort $\SynGL{2}$. The absolute value of the
  determinant of an orthgonal transformation is always $1$, so we also
  assume $\abs{\det(\iota_OO)} \stackrel{ax}\equiv 1$. Likewise, scaling
  maps have a determined determinant: $\det(\iota_1(s))
  \stackrel{ax}\equiv s \cdot s$.
\end{example}

The equality judgment $\Delta \vdash e \equiv e' : s$ on index
expressions generates the equality judgment $\Delta \vdash A \equiv
B \isType$ on types. The basic rule generating equality judgments on
types equates %states that %two 
applications of primitive types %are equal 
if
their %index expression 
arguments are equal:
\begin{displaymath}
  \inferrule*
  {\{ \Delta \vdash e_j \equiv e'_j : s_j\}_{1\leq j \leq n}}
  {\Delta \vdash \tyPrim{X}{e_1,...,e_n} \equiv \tyPrim{X}{e'_1,...,e'_n} \isType}
\end{displaymath}
The rest of the rules for equality on types ensure that it is a
congruence relation 
and an equivalence relation.
%on types. %and that it is an equivalence relation
%(i.e.,~equality on types is reflexive, symmetric and transitive).
% For example, for universally quantified types we have
% the following congruence rule:
% \begin{displaymath}
%   \inferrule*
%   {\Delta, i : s \vdash A \equiv B \isType}
%   {\Delta \vdash \forall i\mathord:s.A \equiv \forall i\mathord:s.B \isType}
% \end{displaymath}
% The congruence rules for the other type formers are similar.

% FIXME: think about integrating this lemma into the running text
% \begin{lemma}
%   Let $\Delta \vdash \sigma \Rightarrow \Delta'$ be a simultaneous
%   substitution.
%   \begin{enumerate}
%   \item If $\Delta' \vdash e \equiv e' : s$ then $\Delta \vdash
%     \sigma^*e \equiv \sigma^*e' : s$; and
%   \item If $\Delta' \vdash A \equiv B \isType$ then $\Delta \vdash
%     \sigma^*A \equiv \sigma^* B \isType$.
%   \end{enumerate}
% \end{lemma}

%A pair of 
The simultaneous substitutions $\Delta \vdash \sigma \Rightarrow
\Delta'$ and $\Delta \vdash \sigma' \Rightarrow \Delta'$ are defined
to be equal if their component expressions are equal in the context
$\Delta$: i.e.,~ if $\Delta \vdash e_j \equiv e'_j : s_j$, for all
$j$. We write $\Delta \vdash \sigma \equiv \sigma' \Rightarrow
\Delta'$ when two simultaneous substitutions are equal.

%%%%%%%%%%%%%%%%%%%%%%%%%%%%%%%%%%%%%%%%%%%%%%%%%%%%%%%%%%%%%%%%%%%%%%%%%%%%%%
%%%%%%%%%%%%%%%%%%%%%%%%%%%%%%%%%%%%%%%%%%%%%%%%%%%%%%%%%%%%%%%%%%%%%%%%%%%%%%
\subsection{Semantics of Algebraically-Indexed Types}
\label{sec:semantics-algebraically-indexed-types}

%\fixme{Make sure that the semantics is motivated}
%Having defined the language of algebraically-indexed types, we now
%turn to their denotational interpretation. 
%We now give the denotational interpretation of algebraically indexed
%types.  In \autoref{sec:index-erasure-semantics} we define an
%\emph{index-erasure} interpretation of types that interprets every
%well-indexed type as a set, ignoring the indexing
%expressions. %Building on the index-erasure semantics, we 
%We then define in
%\autoref{sec:relational-semantics} the relational
%interpretation of types.

\autoref{sec:index-erasure-semantics} gives an \emph{index-erasure}
interpretation of types, interpreting every well-indexed type as a
set, ignoring the indexing expressions.
\autoref{sec:relational-semantics} gives the relational interpretation
of types.

% For both of our semantics of types, we state and prove two properties:
% types that are syntactically equal have equal denotations, and that
% substitution of index terms is interpreted via composition. These
% properties ensure that our semantics of types is well-behaved.

\subsubsection{The Index-Erasure Interpretation of Types}
\label{sec:index-erasure-semantics}

The defining feature of the index erasure interpretation is that 
semantics of a well-indexed type $\tyPrim{X}{e_1,...,e_n}$ is
determined solely by the primitive type $\tyPrimNm{X}$ and not by the
index expressions $e_1,...,e_n$. %Accordingly, 
We thus assume each
primitive type $\tyPrimNm{X} \in \PrimTypeSet$ is assigned a set
$\tyPrimSem{\tyPrimNm{X}}$ and %We 
extend this assignment to %every
well-indexed types by: % induction on the type structure:
\begin{displaymath}
  \begin{array}{@{}c@{\hspace{0em}}c@{}}
    \begin{array}{@{}l}
      \tySem{\tyUnit} \isDefinedAs \{*\} \\
      \tySem{A \tyProduct B} \isDefinedAs \tySem{A} \times \tySem{B} \\
      \tySem{A \tyArr B} \isDefinedAs \tySem{A} \to \tySem{B} \\
    \end{array}
    &
    \begin{array}{l}
      \tySem{A + B} \isDefinedAs \tySem{A} + \tySem{B} \\
      \tySem{\tyPrim{X}{e_1,...,e_n}} \isDefinedAs \tyPrimSem{\tyPrimNm{X}} \\
      \tySem{\forall i\mathord:s. A} \isDefinedAs \tySem{A} \\
      \tySem{\exists i\mathord:s. A} \isDefinedAs \tySem{A}
    \end{array}
  \end{array}
\end{displaymath}
% The interpretation of the unit type is a chosen one element set, and
% the interpretations of the function, tuple and sum types are simply
% the corresponding constructions on sets. 
Note that 
the interpretations of the
universal and existential quantifiers 
do indeed
ignore the indexing. %: the
%interpretation of the type $\forall i\mathord:s.A$ is exactly the
%interpretation of the type $A$, and likewise for $\exists
%i\mathord:s.A$.
We then have:

% The index-erasure interpretation completely ignores index expressions
% and type equality is defined as an extension of index
% equality. Therefore, it is straightforward to prove that equal types
% have equal denotations when interpreted in the index-erasure
% semantics, and that substitution of index terms has no effect on the
% index-erasure interpretation of types:
\begin{lemma}\label{lem:tyeqsubst-erasure}
  \begin{enumerate}
  \item If $\Delta \vdash A \equiv B \isType$ then $\tySem{A} =
    \tySem{B}$; and
  \item If $\Delta' \vdash A \isType$ and $\Delta \vdash \sigma
    \Rightarrow \Delta'$, then $\tySem{\sigma^*A} = \tySem{A}$.
  \end{enumerate}
\end{lemma}

\begin{example}[Two-Dimensional Geometry]
  \exref{ex:two-dim-geo-axioms}
%  The two-dimensional geometry instantiation of the general framework
  uses the assignment $\tyPrimSem{\tyPrimNm{vec}} = \mathbb{R}^2$ and
  $\tyPrimSem{\tyPrimNm{real}} = \mathbb{R}$.
\end{example}

\subsubsection{The Relational Interpretation of Types}
\label{sec:relational-semantics}

%We now define 
The relational semantics of the well-indexed type
$\Delta \vdash A \isType$ is a binary relation on the index-erasure
interpretation of $A$. We write $\Rel(X)$ for the set
of binary relations $R \subseteq X \times X$ on the set $X$.

For the unit, tuple, sum and function types we define the
relational interpretation as a standard logical relation. The
relational interpretations of primitive types with index arguments and
the universally quantified types require an interpretation of index
contexts. %We assign interpretations to index contexts 
These are given in terms of sets
of \emph{relational environments}.

The definition of relational environments is the most subtle part of
our framework. 

%The straightforward approach is to assume an
%interpretation of indexing expressions in which sorts are
%interpreted as 
%sets and index operations are functions, and then define the
%relational interpretation of a primtive type $\tyPrim{X}{e_1,...,e_n}$
%in terms of the interpretations of $e_1,...,e_n$. This approach
%suffices to prove free theorems and type isomorphisms. However, to
%prove non-definability results, like the one in
%\exref{ex:uninhabited-type}, we need a more refined interpretation
%that accounts for the definable programs permitted by the algebraic
%indexing. Here, we identify the properties we require of relational
%environments for the abstraction theorem to hold. In
%\autoref{sec:constr-rel-env}, we give two generic constructions of
%relational environments.

A relational environment for a context $\Delta$ is a %(dependently typed)
function $\rho$ that, for each primitive type $\tyPrimNm{X}$ and
instantiation of its arguments $\sigma$, assigns a binary relation
$\rho\ \tyPrimNm{X}\ \sigma$ on the index-erasure interpretation of
$\tyPrimNm{X}$:
\begin{displaymath}
  \rho : (\tyPrimNm{X} \in \PrimTypeSet) \to (\Delta \Rightarrow \primTyArity(\tyPrimNm{X})) \to \Rel(\tyPrimSem{\tyPrimNm{X}})
\end{displaymath}
%Every relation environment must respect index expression equality: for
%any $\tyPrimNm{X} \in \PrimTypeSet$ and pair of simultaneous
%substitutions $\Delta \vdash \sigma \Rightarrow
%\primTyArity(\tyPrimNm{X})$ and $\Delta \vdash \sigma' \Rightarrow
%\primTyArity(\tyPrimNm{X})$, if $\Delta \vdash \sigma \equiv \sigma'
%\Rightarrow \primTyArity(\tyPrimNm{X})$, then $\rho\ \tyPrimNm{X}\
%\sigma = \rho\ \tyPrimNm{X}\ \sigma'$.
Relational environments must respect index expression equality: for
all $\tyPrimNm{X} \in \PrimTypeSet$,
$\Delta \vdash \sigma \Rightarrow
\primTyArity(\tyPrimNm{X})$, and $\Delta \vdash \sigma' \Rightarrow
\primTyArity(\tyPrimNm{X})$, if $\Delta \vdash \sigma \equiv \sigma'
\Rightarrow \primTyArity(\tyPrimNm{X})$, then $\rho\ \tyPrimNm{X}\
\sigma = \rho\ \tyPrimNm{X}\ \sigma'$.
%
We write $\mathrm{RelEnv}(\Delta)$ for the set of all relational
environments for a context $\Delta$. Given %a relational environment
$\rho \in \mathrm{RelEnv}(\Delta)$ and %a simultaneous substitution
$\Delta \vdash \sigma \Rightarrow \Delta'$, we can derive the composed
relational environment (with an abuse of notation) $\rho \circ \sigma
\in \mathrm{RelEnv}(\Delta')$ as $(\rho \circ \sigma)\ \tyPrimNm{X}\
\sigma' = \rho\ \tyPrimNm{X}\ (\sigma' \circ \sigma)$.

We interpret index contexts $\Delta$ as certain
%sets of relation environments, i.e.,~
subsets of $\mathrm{RelEnv}(\Delta)$. Depending on the primitive
operations that we assume for our system, we may have different sets
of relational environments that enforce different invariants. 
% We
% require that the sets of relational environments are \emph{valid} in
% the following sense:

\begin{definition}
  \label{defn:valid-rel-env-family}
  An index context-indexed family of sets of relational environments
  $\{ \relEnv{E}(\Delta) \subseteq \mathrm{RelEnv}(\Delta) \}_\Delta$
  is \emph{substitutive} if:
  \begin{enumerate}
  \item it is closed under composition with simultaneous substitution:
    if $\rho \in \relEnv{E}(\Delta)$ and $\Delta \vdash \sigma
    \Rightarrow \Delta'$, then $\rho \circ \sigma \in
    \relEnv{E}(\Delta')$; and
%  \item If we have a pair of relation environments $\rho_1 \in
%    \relEnv{E}(\Delta', i\mathord:s')$ and $\rho_2 \in
%    \relEnv{E}(\Delta)$, along with a simultaneous substitution
%    $\Delta' \vdash \sigma \Rightarrow \Delta$, such that $\rho_1
%    \circ \pi_{i\mathord:s'} = \rho_2 \circ \sigma$, then there exists
%    a $\rho \in \relEnv{E}(\Delta, i\mathord:s')$ such that $\rho
%    \circ \sigma_{i\mathord:s'} = \rho_1$ and $\rho \circ
%    \pi_{i\mathord:s'} = \rho_2$.
  \item if $\rho_1 \in \relEnv{E}(\Delta', i\mathord:s)$ and $\rho_2
    \in \relEnv{E}(\Delta)$, and $\Delta \vdash \sigma \Rightarrow
    \Delta'$ is such that $\rho_1 \circ \pi_{i\mathord:s} = \rho_2
    \circ \sigma$, then there exists a $\rho \in \relEnv{E}(\Delta,
    i\mathord:s)$ such that $\rho \circ \sigma_{i\mathord:s} =
    \rho_1$ and $\rho \circ \pi_{i\mathord:s} = \rho_2$.
    % such that the outer edges of the
    % following diagram commute, for all primitive types $\tyPrimNm{X} \in
    % \mathit{PrimType}$:
    % % FIXME: check that these compositions are the right way round
    % % FIXME: re-do this as a set of equations
    % % FIXME: give a better English description of why this is needed
    % \begin{displaymath}
    %   \xymatrix{
    %   {\Delta \Rightarrow \primTyArity(\tyPrimNm{X})} \ar[r]^(.45){- \circ \pi_{i\mathord:s'}} \ar[d]_{- \circ \sigma}
    %   &
    %   {\Delta,i\mathord:s' \Rightarrow \primTyArity(\tyPrimNm{X})} \ar[d]^{- \circ \sigma_{s'}} \ar@/^/[rdd]^{\rho_1 \tyPrimNm{X}}
    %   \\
    %   {\Delta' \Rightarrow \primTyArity(\tyPrimNm{X})} \ar[r]^(.45){- \circ \pi_{i\mathord:s'}} \ar@/_/[rrd]_{\rho_2\ \tyPrimNm{X}}
    %   &
    %   {\Delta,i\mathord:s' \Rightarrow \primTyArity(\tyPrimNm{X})} \ar@{.>}[dr]_{\rho\ \tyPrimNm{X}}
    %   \\
    %   &
    %   &
    %   {\Rel(\tyPrimSem{\tyPrimNm{X}})}
    % }
    % \end{displaymath}
    % Then there exists a relation environment $\rho \in
    % \relEnv{E}(\Delta,i\mathord:s')$ (the dotted arrow) such that the
    % two triangles in the bottom right of the diagram commute.
    % FIXME: mention that we already know that the square commutes
  \end{enumerate}
\end{definition}
%
% For the general
% framework, we assume that, for each index context $\Delta$, we are
% given a set of relation environments $\relEnv{E}(\Delta)$. Note that
% this assignment of sets of relation environments to index contexts is
% not assumed to be compositional: it is not necessarily the case that
% the set of relational environments $\relEnv{E}(\Delta_1,\Delta_2)$ is
% defined in terms of $\relEnv{E}(\Delta_1)$ and
% $\relEnv{E}(\Delta_2)$. However, it must satisfy the following two
% properties:
%
\noindent
These conditions ensure that the relational interpretation of types we
define below behaves correctly with respect to simultaneous
substitution of index expressions (see
\lemref{lem:tyeqsubst-relational}, below).

% The second condition may
% seem mysterious at first, but it is essential for proving that the
% relational interpretation of universal quantification types behaves
% correctly with respect to application of simultaneous substitution.
%
Given a relational environment $\rho \in \relEnv{E}(\Delta)$, %we define
the set of extensions %$\extends{\rho}{i\mathord:s}$ 
of $\rho$ by %an
%additional index variable 
$i\mathord:s$ %to be 
is $\extends{\rho}{i\mathord:s} \isDefinedAs \{ \rho' \in
\relEnv{E}(\Delta,i\mathord:s) \sepbar \rho' \circ \pi_{i\mathord:s} =
\rho \}$. The set of extensions of a relational environment will be
used in interpreting the universal and existential type formers.  We
assign a relational interpretation to all well-indexed types $\Delta
\vdash A \isType$ by induction on their derivations, parameterised by
%relational environments 
$\rho \in \relEnv{E}(\Delta)$:
\begin{eqnarray*}
  \rsem{\tyUnit}{\relEnv{E}}\rho & \isDefinedAs & \{(*,*)\} \\
  \rsem{\tyPrim{X}{e_1,...,e_n}}{\relEnv{E}}\rho & \isDefinedAs & \rho\ {\tyPrimNm{X}}\ (e_1,...,e_n) \\
  \rsem{A \tyArr B}{\relEnv{E}}\rho & \isDefinedAs & \rsem{A}{\relEnv{E}}\rho \relArrow \rsem{B}{\relEnv{E}}\rho \\
  \rsem{A \tyProduct B}{\relEnv{E}}\rho & \isDefinedAs & \rsem{A}{\relEnv{E}}\rho \relTimes \rsem{B}{\relEnv{E}}\rho \\
  \rsem{A + B}{\relEnv{E}}\rho & \isDefinedAs & \rsem{A}{\relEnv{E}}\rho \relSum \rsem{B}{\relEnv{E}}\rho \\
  \rsem{\forall i\mathord:s.A}{\relEnv{E}}\rho & \isDefinedAs & \bigcap\{ \rsem{A}{\relEnv{E}}{\rho'} \sepbar \rho' \in \extends{\rho}{i\mathord:s} \} \\
  \rsem{\exists i\mathord:s.A}{\relEnv{E}}\rho & \isDefinedAs & \bigcup\{ \rsem{A}{\relEnv{E}}{\rho'} \sepbar \rho' \in \extends{\rho}{i\mathord:s} \}
\end{eqnarray*}
In this definition we have made use of the following %three
constructions on binary relations: if $R \in \Rel(X)$ and $S \in
\Rel(Y)$, then $R \relArrow S \in \Rel(X \to Y)$ is %defined as 
$\{(f_1,f_2) \sepbar \forall (a_1,a_2) \in R.\ (f_1a_1,f_2a_2) \in S
\}$, and %. With the same assumptions on $R$ and $S$, the relation 
$R
\relTimes S \in \Rel(X \times Y)$ is %defined as 
$\{((a_1,b_1),(a_2,b_2)) \sepbar (a_1,a_2) \in R \land (b_1,b_2) \in S
\}$, and %. Finally, the relation 
$R \relSum S \in \Rel(X + Y)$ is %defined as
$\{ (\mathrm{inl}\ x, \mathrm{inl}\ x') \sepbar (x,x') \in R \} \cup
\{ (\mathrm{inr}\ y, \mathrm{inr}\ y') \sepbar (y,y') \in S \}$.

% The following lemma states that the relation interpretation of types
% that we have defined in this section behaves well: the first part of
% the lemma states that two types that are judgmentally equal are given
% equal relational interpretations, and the second part states that
% substitution of index expressions in types can be interpreted by the
% composition of relational environments with simultaneous
% substitutions.
\begin{lemma}\label{lem:tyeqsubst-relational}
\par
  \begin{enumerate}
  \item If $\Delta \vdash A \equiv B \isType$, then for all $\rho \in
    \relEnv{E}(\Delta)$, $\rsem{A}{\relEnv{E}}{\rho} =
    \rsem{B}{\relEnv{E}}{\rho}$;
  \item If $\Delta' \vdash A \isType$ and $\relEnv{E}$ is
    substitutive, then for all $\Delta \vdash \sigma \Rightarrow
    \Delta'$ and $\rho \in \relEnv{E}(\Delta)$,
    $\rsem{\sigma^*A}{\relEnv{E}}\rho = \rsem{A}{\relEnv{E}}(\rho
    \circ \sigma)$.
  \end{enumerate}
\end{lemma}
\noindent
Note that the equations in both parts of
\lemref{lem:tyeqsubst-relational} are well-typed by virtue of the
corresponding parts of \lemref{lem:tyeqsubst-erasure}.


%%%%%%%%%%%%%%%%%%%%%%%%%%%%%%%%%%%%%%%%%%%%%%%%%%%%%%%%%%%%%%%%%%%%%%%%%%%%%%
%%%%%%%%%%%%%%%%%%%%%%%%%%%%%%%%%%%%%%%%%%%%%%%%%%%%%%%%%%%%%%%%%%%%%%%%%%%%%%
\subsection{Well-typed Programs and the Abstraction Theorem}
\label{sec:well-typed-programs}

We now present the rules for well-typed programs over the 
%collection of 
types %we defined 
in \autoref{sec:algebraically-indexed-types}. Each
well-typed program is assigned an index-erasure semantics, building on
the index-erasure semantics of types %we defined 
in \autoref{sec:index-erasure-semantics}. Our main result
(\thmref{thm:abstraction}) is that the index-erasure semantics of
every well-typed program is related to itself in the relational
interpretation of its type: this is the abstraction theorem for every
instantiation of our general framework.

Well-typed programs are defined with respect to well-indexed typing
contexts, which are in turn defined with respect to an index
context. Well-indexed typing contexts with respect to an index context
$\Delta$ are sequences of variable/type pairs with no repeated
variable names such that each type is well-indexed with respect to
$\Delta$. Formally, well-indexed typing contexts are 
given by
%defined by the following two rules:
\begin{mathpar}
  \inferrule*
  { }
  {\Delta \vdash \epsilon \isCtxt}

  \inferrule*
  {\Delta \vdash \Gamma \isCtxt \\ \Delta \vdash A \isType \\ x \not\in \Gamma}
  {\Delta \vdash \Gamma, x : A \isCtxt}
\end{mathpar}
Application of simultaneous substitutions extends to typing contexts
by applying the simultaneous substitution to each type.

%The typing rules for our framework define 
Well-typed programs 
are defined
with respect to an index context $\Delta$ and a type context $\Delta \vdash
\Gamma \isCtxt$. The judgment $\Delta; \Gamma \vdash M : A$ is defined
in \autoref{fig:programs}. The equational theory on types is
incorporated into the type system via the rule \TirName{TyEq}, which
allows for a program that has %is judged to have 
type $A$ to also have any
%other 
equal type $B$ as well.
\begin{figure*}[t]
  \centering
  {\small
  \begin{mathpar}
    \inferrule* [right=Var]
    {\Delta \vdash \Gamma \isCtxt \\ x : A \in \Gamma}
    {\Delta; \Gamma \vdash x : A}

    \inferrule* [right=TyEq]
    {\Delta; \Gamma \vdash M : A \\ \Delta \vdash A \equiv B \isType}
    {\Delta; \Gamma \vdash M : B}

    \inferrule* [right=Unit]
    { }
    {\Delta; \Gamma \vdash * : 1}

    \inferrule* [right=Pair]
    {\Delta; \Gamma \vdash M : A \\
      \Delta; \Gamma \vdash N : B}
    {\Delta; \Gamma \vdash (M, N) : A \tyProduct B}

    \inferrule* [right=Proj1]
    {\Delta; \Gamma \vdash M : A \tyProduct B}
    {\Delta; \Gamma \vdash \pi_1 M : A}

    \inferrule* [right=Proj2]
    {\Delta; \Gamma \vdash M : A \tyProduct B}
    {\Delta; \Gamma \vdash \pi_2 M : B}

    \inferrule* [right=Inl]
    {\Delta; \Gamma \vdash M : A}
    {\Delta; \Gamma \vdash \mathrm{inl}\ M : A + B}

    \inferrule* [right=Inr]
    {\Delta; \Gamma \vdash M : B}
    {\Delta; \Gamma \vdash \mathrm{inr}\ M : A + B}

    \inferrule* [right=Case]
    {\Delta; \Gamma \vdash M : A + B \\\\
      \Delta; \Gamma, x : A \vdash N_1 : C \\\\
      \Delta; \Gamma, y : B \vdash N_2 : C}
    {\Delta; \Gamma \vdash \textrm{case}\ M\ \textrm{of}\ \textrm{inl}\ x.N_1; \textrm{inr}\ y.N_2 : C}

    \inferrule* [right=Abs]
    {\Delta; \Gamma, x : A \vdash M : B}
    {\Delta; \Gamma \vdash \lambda x.M : A \tyArr B}

    \inferrule* [right=App]
    {\Delta; \Gamma \vdash M : A \tyArr B \\
      \Delta; \Gamma \vdash N : A}
    {\Delta; \Gamma \vdash M N : B}

    \inferrule* [right=UnivAbs]
    {\Delta, i \mathord: s; \pi_{i\mathord:s}^*\Gamma \vdash M : A}
    {\Delta; \Gamma \vdash \Lambda i. M : \forall i\mathord:s. A}

    \inferrule* [right=UnivApp]
    {\Delta; \Gamma \vdash M : \forall i\mathord:s. A \\ \Delta \vdash e : s}
    {\Delta; \Gamma \vdash M [e] : (\id_\Delta, e)^*A}

    \inferrule* [right=ExPack]
    {\Delta; \Gamma \vdash M : (\id_\Delta, e)^*A \\ \Delta, i\mathord:s \vdash A \isType}
    {\Delta; \Gamma \vdash \langle[e], M\rangle: \exists i\mathord:s. A}

    \inferrule* [right=ExUnpack]
    {\Delta; \Gamma \vdash M : \exists i\mathord:s. A \\\\
      \Delta, i\mathord:s; \pi_{i\mathord:s}^*\Gamma, x : A \vdash N : \pi_{i\mathord:s}^*B}
    {\Delta; \Gamma \vdash \mathrm{let}\langle[i],x\rangle = M\ \mathrm{in}\ N : B}
  \end{mathpar}}
  
  \caption{Well-typed Programs}
  \label{fig:programs}
\end{figure*}

\paragraph{Index-Erasure Interpretation of Programs}

% FIXME: do we need to prove something about the effect
% of substitution on the erasure semantics of contexts?

We assign an index-erasure semantics to any well-indexed typing
context $\Delta \vdash \Gamma \isCtxt$ by induction:
$\ctxtSem{\epsilon} = \{*\}$ and $\ctxtSem{\Gamma, x : A} =
\ctxtSem{\Gamma} \times \tySem{A}$. For a well-typed program $\Delta;
\Gamma \vdash M : A$, we define the \emph{erasure interpretation} as a
function $\tmSem{M} : \ctxtSem{\Gamma} \to \tySem{A}$ that completely
ignores the indexing information. In light of
\lemref{lem:tyeqsubst-erasure}, we do this %we define this function is defined
directly on the syntax of well-typed programs, rather than on typing
derivations. The definition of $\tmSem{M}$ is completely standard,
except for the clauses for universal and existential types:
\begin{displaymath}
  \begin{array}{@{}l}
    \begin{array}{@{}l@{\hspace{0.5em}}c@{\hspace{0.5em}}l@{\hspace{2em}}l@{\hspace{0.5em}}c@{\hspace{0.5em}}l@{}}
      \tmSem{\Lambda i.\ M}\eta & \isDefinedAs & \tmSem{M}\eta
      &
      \tmSem{M[e]}\eta & \isDefinedAs & \tmSem{M}\eta \\
      \tmSem{\langle [e], M\rangle}\eta & \isDefinedAs & \tmSem{M}\eta
    \end{array} \\
    \tmSem{\mathrm{let}\langle[i], x\rangle = M\ \mathrm{in}\ N}\eta \isDefinedAs \tmSem{N}(\eta, \tmSem{M}\eta)
  \end{array}
\end{displaymath}

We use our index-erasure semantics to define when a pair of programs
are contextually equivalent with respect to syntactically defined
contexts, following Hofmann \cite{hofmann08correctness}.

\begin{definition}\label{defn:ctxt-equiv}
  Let $\Gamma_{\mathit{ops}}$ be a typing context with no free index
  variables that describes the types of the primitive operations, and
  let $\eta_{\mathit{ops}} \in \ctxtSem{\Gamma_{\mathit{ops}}}$ be an interpretation
  of these primitive operations.

  Two programs $\Delta; \Gamma_{\mathit{ops}}, \Gamma \vdash M_1 : A$
  and $\Delta; \Gamma_{\mathit{ops}}, \Gamma \vdash M_2 : A$ are
  \emph{contextually equivalent} ($\Delta; \Gamma_{\mathit{ops}},
  \Gamma \vdash M_1 \stackrel{ctx}\approx M_2 : A$) if for all
  programs $;\Gamma_{\mathit{ops}} \vdash T : (\forall \Delta. \Gamma
  \to A) \to \tyPrimNm{bool}$, it is the case that $\tmSem{T\ (\Lambda
    \Delta. \lambda \Gamma. M_1)}\eta_{\mathit{ops}} = \tmSem{T\
    (\Lambda \Delta. \lambda \Gamma. M_2)}\eta_{\mathit{ops}}$.
\end{definition}

\paragraph{The Abstraction Theorem}
%\label{sec:abstraction-theorem}

We now state the abstraction theorem for well-typed programs. To state
and prove this theorem for open programs, we extend the
relational interpretation of types to typing contexts. The relational
interpretation of contexts is defined by: % the following clauses: %, where
%we have used the relational lifting of cartesian product $-\relTimes-$
%defined in \autoref{sec:relational-semantics}.
\begin{displaymath}
  \begin{array}{@{}l@{\hspace{0.1em}\isDefinedAs\hspace{0.1em}}l@{\hspace{1.5em}}l@{\hspace{0.1em}\isDefinedAs\hspace{0.1em}}l}
    \rsem{\epsilon}{\relEnv{E}}{\rho} & \{(*,*)\} &
    \rsem{\Gamma, x : A}{\relEnv{E}}{\rho} & \rsem{\Gamma}{\relEnv{E}}\rho \relTimes \rsem{A}{\relEnv{E}}\rho
  \end{array}
\end{displaymath}
% The relational interpretation of contexts inherits from the relational
% interpretation of types the property of interpreting the application
% of simultaneous substitutions as composition:
% \begin{lemma}\label{lem:ctxtsubst-rel}
%   If $\Delta' \vdash \Gamma \isCtxt$ and $\Delta \vdash \sigma
%   \Rightarrow \Delta'$, then for all $\rho \in \relEnv{E}(\Delta)$,
%   $\rsem{\sigma^*\Gamma}{\relEnv{E}}\rho =
%   \rsem{\Gamma}{\relEnv{E}}{(\rho \circ \sigma)}$.
% \end{lemma}

\begin{theorem}[Abstraction]\label{thm:abstraction}
  Assume $\relEnv{E}$ is substitutive. If $\Delta; \Gamma \vdash M :
  A$, then for all $\rho \in \relEnv{E}(\Delta)$ and $\eta_1, \eta_2
  \in \ctxtSem{\Gamma}$ such that $(\eta_1, \eta_2) \in
  \rsem{\Gamma}{\relEnv{E}}\rho$, we have $(\tmSem{M}\eta_1,
  \tmSem{M}\eta_2) \in \rsem{A}{\relEnv{E}}\rho$.
\end{theorem}

We refer to programs satisfying the conditions in the second part of
\thmref{thm:abstraction} as being in the \emph{quantifier-free}
fragment of our framework. These conditions are very similar to those
used in the fragment of System F covered by Hindley-Milner type
inference.

Semantic equivalence of programs is defined in terms of the relational
interpretation. As a consequence of \thmref{thm:abstraction}, semantic
equivalence is a sound approximation of contextual equivalence.
\begin{definition}\label{def:semantic-equality}
  Let $\Gamma_{\mathit{ops}}$ and $\eta_{\mathit{ops}}$ be a context
  of primitive operations and its interpretation as in
  \defref{defn:ctxt-equiv}. Let $\relEnv{E}_{\mathit{ops}}$ be a
  substitutive family of sets of relational environments such that for
  all $\Delta$ and $\rho \in \relEnv{E}_{\mathit{ops}}(\Delta)$,
  $(\eta_{\mathit{ops}}, \eta_{\mathit{ops}}) \in
  \rsem{\Gamma_{\mathit{ops}}}{\relEnv{E}_{\mathit{ops}}}\rho$.

  Two programs $\Delta; \Gamma_{\mathit{ops}}, \Gamma \vdash M_1 : A$
  and $\Delta; \Gamma_{\mathit{ops}}, \Gamma \vdash M_2 : A$ are
  \emph{semantically equal} ($\Delta; \Gamma_{\mathit{ops}}, \Gamma
  \vdash M_1 \stackrel{sem}\sim M_2 : A$) if for all $\rho \in
  \relEnv{E}_{\mathit{ops}}$, and all $(\eta_1,\eta_2) \in
  \rsem{\Gamma}{\relEnv{E}_{\mathit{ops}}}\rho$, we have
  $(\tmSem{M_1}(\eta_{\mathit{ops}},
  \eta_1),\tmSem{M_2}(\eta_{\mathit{ops}}, \eta_2)) \in
  \rsem{A}{\relEnv{E}_{\mathit{ops}}}\rho$.
\end{definition}

\begin{theorem}[Soundness]\label{thm:soundness}
  $\Delta; \Gamma_{\mathit{ops}}, \Gamma \vdash M_1 \stackrel{sem}\sim M_2 : A$ implies
  $\Delta; \Gamma_{\mathit{ops}}, \Gamma \vdash M_1 \stackrel{ctx}\approx M_2 : A$
\end{theorem}

%%%%%%%%%%%%%%%%%%%%%%%%%%%%%%%%%%%%%%%%%%%%%%%%%%%%%%%%%%%%%%%%%%%%%%%%%%%%%%
%%%%%%%%%%%%%%%%%%%%%%%%%%%%%%%%%%%%%%%%%%%%%%%%%%%%%%%%%%%%%%%%%%%%%%%%%%%%%%
%%%%%%%%%%%%%%%%%%%%%%%%%%%%%%%%%%%%%%%%%%%%%%%%%%%%%%%%%%%%%%%%%%%%%%%%%%%%%%
\subsection{Construction of Relational Environments}
\label{sec:constr-rel-env}

\newcommand{\Gen}{\mathrm{Gen}}
\newcommand{\Free}{\mathrm{Free}}
\newcommand{\semSort}[1]{\llbracket #1 \rrbracket^{\mathcal{S}}}
\newcommand{\semIndexExp}[1]{\llbracket #1 \rrbracket^{\mathcal{I}}}

We give construction of families $\relEnv{E}$ of sets of
relational environments that allow us to derive useful consequences of
\thmref{thm:abstraction}. 
To satisfy the hypothesis of
\thmref{thm:abstraction} we still need to prove, for each
instantiation of our general framework, that the program-level
primitive operations preserve all the relations in our
constructions. This is the content of Lemmas
\ref{lem:geom-environments-1}, \ref{lem:geom-environments-2},
\ref{lem:monoid-ops-related}, \ref{lem:environments-information-flow}
and \ref{lem:metric-environments}.

%\paragraph{Model-based Relational Environments}
Our relational environments are defined in terms of
models. To each sort $s \in \SortSet$ we assign a carrier set
$\semSort{s}$. We extend this assignment to index contexts and other
sequences of sorts (e.g., type arities) using the cartesian product of
sets: $\semSort{i_1\mathord:s_1,...,i_n\mathord:s_n} = \semSort{s_1}
\times ... \times \semSort{s_n}$. For each index operation $\texttt{f}
\in \IndexOpSet$ with $\indexOpArity(\texttt{f}) = ([s_1,...,s_n],s)$,
we assume a function $\semIndexExp{\texttt{f}} : \semSort{s_1,...,s_n}
\to \semSort{s}$. For each well-sorted index expression $\Delta \vdash
e : s$, we assign a function $\semIndexExp{e} : \semSort{\Delta} \to
\semSort{s}$ by recursion on the structure of $e$. We further assume
that for each axiom $\Delta \vdash e \stackrel{ax}\equiv e' : s \in
\IndexAxiomSet$, we have $\semIndexExp{e} = \semIndexExp{e'}$. A
\emph{model} is a pair $\mathcal{M} = (\semSort{\cdot}_{\mathcal{M}},
\semIndexExp{\cdot}_{\mathcal{M}})$ satisfying all the axioms in
$\IndexAxiomSet$.

Now fix a model $\mathcal{M}$ and assume, for each primitive type
$\tyPrimNm{X}$, a relational interpretation parameterised by elements
from the model: $R_{\tyPrimNm{X}} :
\semSort{\primTyArity(\tyPrimNm{X})}_{\mathcal{M}} \to
\Rel(\tyPrimSem{\tyPrimNm{X}})$. Our family of sets of relational
environments is $\relEnv{E}_{\mathcal{M},R}(\Delta) = \{ \rho^\delta
\sepbar \delta \in \semSort{\Delta}_{\mathcal{M}} \}$, where
$\rho^\delta\ \tyPrimNm{X}\ (e_1,...,e_n) =
R_{\tyPrimNm{X}}(\semIndexExp{e_1}_{\mathcal{M}}\delta, ...,
\semIndexExp{e_n}_{\mathcal{M}}\delta)$.

\begin{theorem}
  The families $\relEnv{E}_{\mathcal{M},R}$ are substitutive.
\end{theorem}


%%% Local Variables:
%%% TeX-master: "paper"
%%% End:

\section{Instantiations of the General Framework}
\label{sec:instantiations}

\subsection{Multiplicative Group Examples}
\label{sec:multiplicative-examples}

\subsection{Additive Group Examples}
\label{sec:additive-examples}

\subsection{Logic Examples}
\label{sec:logic-examples}


\section{Monoid Indexed Types}
\label{sec:monoid-indexed-types}

\newcommand{\Mon}{\textit{Mon}}

In the last section we looked at types indexed by abelian groups. We
now examine an instantiation of our general framework with a monoid
type indexed by elements of a commutative monoid. Our main example of
monoid indexed types is strings indexed by ``taint'' markers: when
strings are appended their taint markers are combined. Thanks to the
information in the indexes, we are able to prove an indefinability
result (\thmref{thm:monoid-indefinability}).

We assume a single indexing sort $\mathsf{M}$, with the monoid
operations of unit and multiplication along with the commutative
monoid axioms. There is a single primitive type $\tyPrimNm{M}$, with
$\primTyArity(\tyPrimNm{M}) = [\mathsf{M}]$. For the index-erasure
semantics, we assume some (not necessarily commutative) monoid $M$,
which we also write multiplicatively. Thus, we set
$\tyPrimSem{\tyPrimNm{M}} = M$. The collection of primitive operations
$\Gamma_{\mathit{Mon}}$ is simply the two monoid operations:
\begin{displaymath}
  \begin{array}{r@{\hspace{0.5em}:\hspace{0.5em}}l@{\hspace{2em}}r@{\hspace{0.5em}:\hspace{0.5em}}l}
    1 & \tyPrim{M}{1}, &
    (\cdot) & \forall a,b\mathord:\mathsf{M}.\ \tyPrim{M}{a} \to \tyPrim{M}{b} \to \tyPrim{M}{a b}
  \end{array}
\end{displaymath}
which we interpret using the monoid structure of $M$, giving an
environment $\eta_\Mon \in \ctxtSem{\Gamma_\Mon}$.

For the sets of relational environments, we make use of the
construction in \autoref{sec:constr-rel-env}, setting the chosen model
$\mathcal{M}$ to be the \emph{centre} $Z(M)$ of $M$ (recall that the
centre of a monoid is the set of all elements that commute with every
other element). We then set $R_{\tyPrimNm{M}}(m) = \{ (n,mn) \sepbar n
\in M \}$ and $R^\bullet_{\tyPrimNm{M}} = \emptyset$. By setting the
model $\mathcal{M}$ to only be the elements of $M$ that commute with
everything else, we ensure that the following lemma holds:
\begin{lemma}
  For all $\Delta$ and $\rho \in \relEnv{E}_\Mon(\Delta)$,
  $(\eta_\Mon,\eta_\Mon) \in \rsem{\Gamma_\Mon}{\relEnv{E}}\rho$.
\end{lemma}

\newcommand{\mscup}{\stackrel{ms}\cup}

\begin{example}
  By selecting a suitable monoid $M$, we can simulate a programming
  language that dynamically annotates strings with taint markers. Let
  $\Sigma$ be a set of characters from some alphabet, and let $T$ be a
  set of taint markers. Set $M = \Sigma^* \times
  \mathcal{MS}_{\mathit{fin}}(T)$, where $\mathcal{MS}_{\mathit{fin}}$
  denotes the finite power set operator. Since both $\Sigma^*$ and
  $\mathcal{MS}_{\mathit{fin}}(T)$ are monoids (the latter with
  multiset union as the monoid multiplication, which we write as
  $\mscup$), $M$ is itself a monoid. The centre of $M$, $Z(M)$,
  consists of all elements of the form $([], t)$ for multisets of
  taint markers.

  As an example of a free theorem for this instantiation of our
  general framework, consider programs of the following type:
  \begin{displaymath}
    ; \Gamma_\Mon \vdash M : \forall a,b\mathord:\mathsf{M}.\ \tyPrim{M}{a} \to \tyPrim{M}{b} \to \tyPrim{M}{aa}
  \end{displaymath}
  From this type we can derive the following free theorem about
  $M$. For all $a, b \in \mathcal{MS}_{\mathit{fin}}(T)$, $(s_1,t_1),
  (s_2,t_2) \in \Sigma^*$,
  \begin{displaymath}
    \begin{array}{l}
      \pi_2(\tmSem{M}\ (s_1,t_1 \mscup a)\ (s_2,t_2 \mscup b)) = \\
      \hspace{2em}\pi_2(\tmSem{M}\ (s_1,t_1)\ (s_2,t_2)) \mscup a \mscup a
    \end{array}
  \end{displaymath}
  We read this result as follows: if we taint the first input by $a$
  and the second by $b$, then the taint of the output varies according
  to $M$'s type. In this case the output contains two copies of $a$
  and none of $b$.
\end{example}

% if $b$ does not appear in the output, then it cannot have been used in
% the program. Turn this into a theorem...

% If forall i1...,in. M<e0> \to ... \to M<em> -> M<e>
% and e0 is not a factor of e, then we can get rid of the first input

% It ought to be invariant under its first input... Set R^\bullet to
% be the everywhere relation....

% prove this by using a relational environment that excludes 

\begin{theorem}
  \begin{equation}
    \label{eq:fo-monoid-type-insensitive}
    ; \Gamma_\Mon \vdash M : \forall i_1,\dots,i_m\mathord:\mathsf{M}.\ \tyPrim{M}{e_1} \to \dots \to \tyPrim{M}{e_n} \to \tyPrim{M}{e}
  \end{equation}
  
\end{theorem}

\begin{theorem}
  \label{thm:monoid-indefinability}
  Let $m$ and $n$ be natural numbers. For all $n$-by-$m$-matricies of
  natural numbers $A$ and $m$-by-$1$ matricies of natural numbers $b$,
  there exists a program $M$ such that
  \begin{displaymath}
    \label{eq:fo-monoid-type}
    ; \Gamma_\Mon \vdash M : \forall i_1,\dots,i_m\mathord:\mathsf{M}.\ \tyPrim{M}{e_1} \to \dots \to \tyPrim{M}{e_n} \to \tyPrim{M}{e}
  \end{displaymath}
  where $e_j = i_1^{A_{j1}}\dots i_m^{A_{jm}}$ and $e =
  i_1^{b_1}\dots i_m^{b_m}$, if and only if there is a solution
  $\vec{x} = (x_1,\cdots,x_n)$ in the natural numbers to the equation
  $A x^\top = b$.
\end{theorem}

\begin{example}
  By \thmref{thm:monoid-indefinability}, it is impossible to write a
  program with the type $\forall m_1,m_2 \mathord: \mathsf{M}.\
  \tyPrim{M}{m_1m_2} \to \tyPrim{M}{m_1}$, showing that it is
  impossible to convert a doubly tainted value to a singly tainted
  one.

  In contrast to the case of abelian group indexed types, the lack of
  inverses in the monoid indexed setting means that the type $\forall
  m_1,m_2\mathord:\mathsf{M}.\ \tyPrim{M}{m_1m_2} \to \tyPrim{M}{m_2}
  \to \tyPrim{M}{m_1}$ is also uninhabited. There are no naturals
  $a_1,a_2$ such that $(m_1m_2)^{a_1}m_2^{a_2} \equiv m_1$.
\end{example}

% \fixme{Finish this} We can further assume that there are other
% operations that distribute over the multiplication of the monoid, for
% example semi-ring operations (e.g.~relational algebra).

\section{Metric Spaces and Simple Continuity Analysis}
\label{sec:continuity-analysis}

Our final motivating example of algebraically indexed types and
representation independence makes use of the metric space structure of
the real numbers. In the geometry example, we used relational
interpretations that related values via geometrical
transformations. An alternative relational interpretation of the real
numbers is to relate values that are close, in the metric space
sense. By indexing types by a measure of closeness, we can actually
state continuity of functions as a typing problem.

We introduce an indexing sort $\mathsf{R}^{>0}$, representing positive
distances between real numbers. Our type of real numbers is indexed by
expressions of sort $\mathsf{R}^{>0}$, and we use the following
index-erasure and relational interpretations:
\begin{displaymath}
  \begin{array}{l@{\hspace{0.5em}=\hspace{0.5em}}l}
    \tySem{\tyPrim{real}{e}} & \mathbb{R} \\
    \rsem{\tyPrim{real}{e}}{}\rho & \{ (x,x') \sepbar |x-x'| < \sem{e}\rho \}
  \end{array}
\end{displaymath}
Thus, two real numbers are related if they only differ by the positive
real number assigned to $e$. With this interpretation, and a
straightforward interpretation of existential types, we can state the
standard $\epsilon$-$\delta$ definition of continuity as a type:
\begin{displaymath}
  \forall \epsilon \mathord: \mathsf{R}^{>0}.\ \exists \delta\mathord: \mathsf{R}^{>0}.\ \tyPrim{real}{\delta} \to \tyPrim{real}{\epsilon}
\end{displaymath}
Chaudhuri, Gulwani and Lublinerman \cite{chaudhuri10continuity} have
presented a program logic based approach to verifying the continuity
of programs. Algebraically indexed types give us a way of expressing
and verifying continuity properties of functions in a type based way.

\begin{displaymath}
  \begin{array}{r@{\hspace{0.5em}:\hspace{0.5em}}l}
    \underline{c} & \forall \epsilon\mathord:\mathsf{R}^{>0}.\ \tyPrim{real}{\epsilon} \\
    (+) & \forall \epsilon_1, \epsilon_2\mathord:\mathsf{R}^{>0}.\ \tyPrim{real}{\epsilon_1} \to \tyPrim{real}{\epsilon_2} \to \tyPrim{real}{\epsilon_1 + \epsilon_2} \\
    (-) & \forall \epsilon_1, \epsilon_2\mathord:\mathsf{R}^{>0}.\ \tyPrim{real}{\epsilon_1} \to \tyPrim{real}{\epsilon_2} \to \tyPrim{real}{\epsilon_1 + \epsilon_2} \\
    (*) & \forall \epsilon_1, \epsilon_2\mathord:\mathsf{R}^{>0}.\ \tyPrim{real}{\epsilon_1} \to \tyPrim{real}{\epsilon_2} \to \tyPrim{real}{\epsilon_1\epsilon_2}
  \end{array}
\end{displaymath}

\begin{enumerate}
\item State the operations and equations (rational powers, addition
  and multiplication (remember: positive reals))
\item Derive subtyping by addition of $0$
\item Do the probability monad as a special case, with a specific
  relational interpretation that relates things according to the
  differential privacy notion. Need to think about the
  measure-theoretic problems...
\item Work out the details of the continuity example, especially the
  implementation of an ADT for continuous functions that can handle
  polynomials.
\end{enumerate}

\section{Singleton Types}
\label{sec:singleton-types}

\newcommand{\Sing}{\mathit{Sing}}

Singleton types differ from the ``variation'' types that we have
considered so far because they nail down exactly the underlying value
instead of describing how it varies.

Indexing theory contains all the integers as constants, along with
$+$, $-$ and $*$. Normal arithmetic laws for these
operations. Program-level operations:
\begin{displaymath}
  \begin{array}{@{}c@{\hspace{0.5em}:\hspace{0.5em}}l}
    \underline{z} & \tyPrim{int}{z} \\ % FIXME: only need zero and one
    (+) & \forall z_1,z_2\mathord:\mathsf{integer}.\ \tyPrim{int}{z_1} \to \tyPrim{int}{z_2} \to \tyPrim{int}{z_1 + z_2} \\
    (-) & \forall z_1,z_2\mathord:\mathsf{integer}.\ \tyPrim{int}{z_1} \to \tyPrim{int}{z_2} \to \tyPrim{int}{z_1 - z_2} \\
    (*) & \forall z_1,z_2\mathord:\mathsf{integer}.\ \tyPrim{int}{z_1} \to \tyPrim{int}{z_2} \to \tyPrim{int}{z_1 * z_2}
  \end{array}
\end{displaymath}

\begin{enumerate}
\item Given an algebraic theory, and a model, we can define a
  singleton type system.
\item Need to carefully define this in general, and give the
  appropriate relational environments
\item Give the integers (freely generated monoid) as an example
\end{enumerate}

\begin{theorem}
  There exists a program
  \begin{displaymath}
    -; \Gamma_\Sing \vdash M : \forall i_1,...,i_m\mathord:\mathsf{S}.\ \tyPrim{S}{e_1} \to ... \to \tyPrim{S}{e_n} \to \tyPrim{S}{e}
  \end{displaymath}
  if and only if there exists an index expression
  $i'_1\mathord:\mathsf{S}, ..., i'_n\mathord:\mathsf{S} \vdash e' :
  \mathsf{S}$ such that $(e_1,...,e_n)^*e' \equiv e$.
\end{theorem}

\begin{proof}
  (If) Use the index expression $e'$ to define $M$, replacing each
  index operation with its program-level counterpart.

  (Only if) \fixme{use a special relational environment...}
\end{proof}

% \subsection{Predicate Types}
% \label{sec:predicate-types}

% \begin{enumerate}
% \item Let the indexing theory be predicates on the underlying value
% \item With the laws of boolean algebra
% \item Singleton types are a special case (i.e. with equality to a fixed value)
% \item But also have $\land$, $\lor$, $\lnot$ (and derived $\Rightarrow$)
% \item A pair of things is related if they both satisfy the predicate
%   (and are equal?)
% \item Need a special type for comparisons?
%   \begin{displaymath}
%     (<) : \forall z_1,z_2:\mathsf{integer}.\ \tyPrim{int}{z_1} \to \tyPrim{int}{z_2} \to (\tyPrim{int}{< z_1} \times 
%   \end{displaymath}
% \end{enumerate}

\section{Logical Information Flow}
\label{sec:information-flow}

\begin{enumerate}
\item This is now a special case of the preceeding subsection
\item Also makes use of the relational aspect, the previous stuff
  could make do with unary logical predicates as the interpretations
  of types, but the information flow properties make use of the
  relational interpretation.
\end{enumerate}

\fixme{
  \begin{enumerate}
  \item Make the definition of control of information flow clearer
  \item Make the example a bit more compelling
  \item Put in some examples of composite principals formed from
    boolean logic, and the use of boolean algebra to do reasoning in
    the types. List the laws of boolean algebra, and the laws of
    Heyting algebra. Need to explain the difference between 
  \item Possibly use singleton types to be able to do more interesting
    security policies?
  \item Put in the formal statement of the information flow property
    and the proof.
  \end{enumerate}
}

An extreme example of invariance under change of representation is the
tracking of information flow through programs. If a program does not
depend upon a particular piece of information, then it will be
invariant under \emph{all} changes of representation of this
information. As thoroughly described by Sabelfeld and Sands
\cite{sabelfeld01per}, information flow can be captured semantically
through the use of partial equivalence relations (PERs). Interpreting
types as PERs forms an instance of Reynolds' approach to abstraction
and representation independence. Abadi, Banerjee, Heintze and Riecke
\cite{abadi99core} built a \emph{Core Calculus for Dependency}, using
a type system based around a security level indexed monad $T_lA$. Tse
and Zdancewic \cite{tse04translating} showed that it is possible to
translate Abadi et al.'s calculus into System F, translating the
monadic type $T_lA$ to $\alpha_l \to A$ for some free type variable
$\alpha_l$, and making direct use of Reynolds' abstraction theorem to
prove information flow properties. We now show how a refined variant
of Tse and Zdancewic's translation can be expressed via
algebraically-indexed types.

We assume a single indexing sort $\mathsf{principal}$, intended to
represent (possibly composite) principals in a system. Principals are
combined using the connectives of boolean logic ($\top$, $\bot$,
$\lor$, $\land$, $\lnot$). For the equational theory of principals, we
assume all the axioms of boolean algebra, and semantically we
interpret composite principals as boolean values: either true ($\top$)
or false ($\bot$). We introduce a single primitive type, indexed by
expressions of sort $\mathsf{principal}$, $\tyPrim{T}{e}$, with the
following interpretations:
\begin{displaymath}
  \begin{array}{l@{\hspace{0.5em}=\hspace{0.5em}}l}
    \tySem{\tyPrim{T}{e}} & \{*\} \\
    \rsem{\tyPrim{T}{e}}{}\rho & \left\{
      \begin{array}{ll}
        \{(*,*)\} & \textrm{if }\sem{e}\rho = \top \\
        \{\}      & \textrm{if }\sem{e}\rho = \bot
      \end{array}
      \right.
  \end{array}
\end{displaymath}
The relational interpretation of $\tyPrim{T}{e}$ forces this type to
only be inhabited when the boolean expression $e$ evaluates to true
under the environment $\rho$. To write programs, we assume the
following pair of primitive operations:
\begin{eqnarray*}
  \mathrm{proj} & : & \forall p, q\mathord:\mathsf{principal}.\ \tyPrim{T}{p \land q} \to \tyPrim{T}{p} \\
  \mathrm{combine} & : & \forall p, q\mathord:\mathsf{principal}.\ \tyPrim{T}{p} \to \tyPrim{T}{q} \to \tyPrim{T}{p \land q}
\end{eqnarray*}
We can now adapt Tse and Zdancewic's translation of Abadi et al.'s
monadic type to our setting. We define a type synonym, for each type
$A$ and expression $e$ of sort $\mathsf{principal}$:
\begin{displaymath}
  T_eA = \tyPrim{T}{e} \to A
\end{displaymath}
For every principal $e$, we can endow the types $T_e-$ with the
structure of a monad. This is due to the fact that it is an instance
of the ``environment'' (or ``reader'') monad \cite{jones95functional}.

\begin{example}
  Suppose we have a program with the following type:
  \begin{displaymath}
    \forall p,q\mathord:\mathsf{principal}. \tyPrim{T}{e} \to T_p(\tyPrimNm{unit} + \tyPrimNm{unit}) \to T_q(\tyPrimNm{unit} + \tyPrimNm{unit})
  \end{displaymath}
  where $e$ is some boolean algebra expression involving $p$ and
  $q$. We will show in \autoref{sec:instantiations} that programs with
  this type are non-constant (i.e.~depend on their input) if and only
  if the formula $e \Rightarrow q \Rightarrow p$ is valid in boolean
  logic.
\end{example}

\begin{enumerate}
\item Treat the information flow thing as a type isomorphism, in the
  same way that Tse and Zdancewic do
  \begin{displaymath}
    \begin{array}{l}
    \forall \vec{p}\mathord:\mathsf{principal}.\ T_{e_1}\tyPrimNm{bool} \to T_{e_2}\tyPrimNm{bool}\\
    \hspace{4em}\cong \\
    \left\{
      \begin{array}{ll}
        \tyPrimNm{bool} \tyArr \tyPrimNm{bool} & \textrm{if }e_2 \vdash e_1 \\
        \tyPrimNm{bool} & \textrm{if }e_2 \not\vdash e_1
      \end{array}
    \right.
  \end{array}
  \end{displaymath}
  where $\tyPrimNm{bool} = \tyUnit + \tyUnit$.
\item mention the issue with bogus tokens if we allow for a
  call-by-name style of non-termination
\item if we axiomatise intuitionistic logic instead of
  boolean/classical logic, we get a different type isomorphism.
\item The isomorphism relies upon $e_1 \vdash e_2 \Leftrightarrow e_1
  \lor e_2 = e_1$.
\item Relevant logic? Look at the Wikipedia page to see how to do
  residuated (commutative) monoids on lattices with just
  equations. With this, we will be able to do substructural logics
  too. As long as we know that a particular axiomatisation is sound
  and complete for some reasoning system, then we can state and prove
  the non-interference isomorphism above.
\item Compare to Tse and Zdancewic's addition of axioms to the logic.
\item Internalising equality? The logical structure can then be used
  on types as well as everything else. Need the special rules for
  equality... (see the Fomega paper).
\end{enumerate}

% Go a step further by considering equivalence relations on word32, and
% their lattice structure. Link this to quantative information
% flow. Will also need to include singleton types.

\subsection{Abelian Group Indexed Types}
\label{sec:abelian-group-indexed-types}

\newcommand{\Grp}{\mathit{Grp}}

Primitive operations $\Gamma_\Grp$:
\begin{displaymath}
  \begin{array}{r@{\hspace{0.5em}:\hspace{0.5em}}l}
    1 & \tyPrim{G}{1} \\
    (\cdot) & \forall a,b\mathord:\mathsf{G}.\ \tyPrim{G}{a} \to \tyPrim{G}{b} \to \tyPrim{G}{a b} \\
    \ ^{-1} & \forall a\mathord:\mathsf{G}.\ \tyPrim{G}{a} \to \tyPrim{G}{a^{-1}}
  \end{array}
\end{displaymath}

For the index-erasure interpretation, we assume a group $(G, 1, \cdot,
\ ^{-1})$. Let $Z(G)$ be the centre of the $G$: i.e.,~the set of
elements that commute with every other element in $G$. We set
$\tyPrimSem{G} = G$. The primitive operations in $\Gamma_\Grp$ are
interpreted just as the unit, multiplication and inverse operations
of $G$. For each index context $\Delta$, the set
$\relEnv{E}_\Grp(\Delta)$ of relational environments is defined to be
$\{ \rho_{(H,h)} \sepbar H\textrm{ is a subgroup of }(\Delta
\Rightarrow \mathsf{G})/\equiv, h\textrm{ is a group homomorphism }H
\to Z(G) \}$, where
\begin{displaymath}
  \rho_{(H,h)}\ \tyPrimNm{G}\ (e) = \left\{
    \begin{array}{ll}
      \{ \} & \textrm{if}\ e\not\in H \\
      \{ (x, h(e)\cdot x) \sepbar x \in G \} & \textrm{if}\ e\in H
    \end{array}
  \right.
\end{displaymath}

\begin{lemma}
  For all $\Delta$ and $\rho \in \relEnv{E}_\Grp(\Delta)$,
  $(\eta_\Grp,\eta_\Grp) \in \rsem{\Gamma_\Grp}{\relEnv{E}}\rho$.
\end{lemma}

\begin{theorem}
  Let $m$ and $n$ be natural numbers. For all $n$-by-$m$-matricies of
  natural numbers $A$ and $m$-by-$1$ matricies of integers $b$,
  there exists a program $M$ such that
  \begin{equation}
    \label{eq:fo-group-type}
    \Gamma_\Grp \vdash M : \forall i_1,\dots,i_m\mathord:\mathsf{G}.\ \tyPrim{G}{e_1} \to \dots \to \tyPrim{G}{e_n} \to \tyPrim{G}{e}
  \end{equation}
  where $e_j = i_1^{A_{j1}}\dots i_m^{A_{jm}}$ and $e = i_1^{b_1}\dots
  i_m^{b_m}$, if and only if there is a solution $\vec{x} =
  (x_1,\cdots,x_n)$ in the integers to the equation $A x^\top = b$.
\end{theorem}

\fixme{Define shorthand notation $x^z$}

\begin{proof}
  (If) The program $\Lambda i_1\dots i_m.\ \lambda g_1\dots g_n.\
  g_1^{x_1}\dots g_n^{x_n}$ satisfies the typing judgment
  (\ref{eq:fo-group-type}).

  (Only if) Assume that there is a program $M$ satisfying the typing
  judgment (\ref{eq:fo-group-type}). By \thmref{thm:abstraction} we
  know that for all relational environments $\rho \in
  \relEnv{E}_\Grp(i_1,\dots,i_m)$, and all $g_j,g'_j \in G$,
  \begin{displaymath}
    \begin{array}{l}
      (g_1,g'_1) \in \rho\ \tyPrimNm{G}\ (e_1) \land \dots \land (g_n,g'_n) \in \rho\ \tyPrimNm{G}\ (e_n) \Rightarrow \\
      \quad (\tmSem{M}g_1\dots g_n, \tmSem{M}g'_1\dots g'_n) \in \rho\ \tyPrimNm{G}\ (e)
    \end{array}
  \end{displaymath}
  We select the relational environments
  \begin{displaymath}
    \rho\ \tyPrimNm{G}\ (e) = \left\{
      \begin{array}{ll}
        \{(g,g) \sepbar g \in \mathcal{G} \} &
        \begin{array}[t]{@{}l}
          \textrm{if}\ e = e_1^{x_1}\dots e_n^{x_n} \\
          \textrm{ for some }x_1,\dots,x_n \in \mathbb{Z}
        \end{array}
        \\
        \{\} & \textrm{otherwise}
      \end{array}
    \right.
  \end{displaymath}
  Now, for any $g_1,\dots,g_m \in \mathcal{G}$, we know that, for all
  $j$, $(g_j,g_j) \in \rho\ \tyPrimNm{G}\ (e_j)$, so we know that
  $(\tySem{M}g_1\dots g_n, \tySem{M}g_1 \dots g_n) \in \rho\
  \tyPrimNm{G}\ (e)$. Thus there exist integers $x_1,...,x_n$ such
  that $e = e_1^{x_1}\dots e_n^{x_n}$. But we also know that $e =
  i_1^{b_1}...i_m^{b_m}$. By the cancellation property of free groups,
  we learn that $x_1,...,x_n$ is the solution we need.
\end{proof}

\begin{enumerate}
\item Can add any other operations that distribute over the group
  multiplication.
\item What about partial division operations? Make division just
  return $0$.
\item Adding square root as an operation. Need to check that all the
  other operations preserve the new relational environments. Square
  root is also supported by the plain relational environments, so the
  type isomorphism result still holds.
\end{enumerate}



\subsubsection{Type isomorphisms}
\label{sec:abelian-group-type-isos}

\begin{enumerate}
\item This only needs the simpler relational environments, but needs
  the group used for the index-erasure semantics to be abelian.
\item One direction is easy, but the other requires the use of the
  free theorem for the type.
\item There are two version: one with some fixed ground type as the
  result, and one with the same type as the result, but with the same
  parameter.
\item Integrates into the Smith Normal Form thing too, since
  invertible integer-valued matrices can be turned into type
  isomorphisms.
\item Partiality?
\end{enumerate}

\fixme{Define what a type isomorphism is}

\begin{theorem}
  For all natural numbers $n$, the types
  \begin{displaymath}
    \tau_{n} \isDefinedAs \forall a\mathord:\mathsf{G}.\ \underbrace{\tyPrim{G}{a} \to ... \to \tyPrim{G}{a}}_{n+1\textrm{ times}} \to \tyPrim{G}{a}
  \end{displaymath}
  and
  \begin{displaymath}
    \sigma_{n} \isDefinedAs \underbrace{\tyPrim{G}{1} \to ... \tyPrim{G}{1}}_{n\textrm{ times}} \to \tyPrim{G}{1}
  \end{displaymath}
  are isomorphic.
\end{theorem}

\begin{proof}
  We define programs $\Gamma_\Grp \vdash M : \tau_n \to \sigma_n$ and
  $\Gamma_\Grp \vdash M^{-1} : \sigma_n \to \tau_n$ as follows:
  \begin{displaymath}
    M = \lambda f. \lambda g_1 \dots g_n.\ f\ [1]\ 1\ g_1\ \dots\ g_n
  \end{displaymath}
  and
  \begin{displaymath}
    M^{-1} = \lambda f. \Lambda a.\ \lambda g_0 \dots g_n.\ (f\ (g_1 g^{-1}_0)\ \dots\ (g_n g^{-1}_0)) g_0
  \end{displaymath}
  In one direction, the proof that these are mutually inverse is
  straightfoward, using only the fact that $G$ is a group. For any
  program $\Gamma_\Grp \vdash f : \sigma_n$, we have
  \begin{displaymath}
    \begin{array}{cl}
        & \tmSem{M\ (M^{-1}\ f)}\eta_\Grp \\
      = & (\tmSem{M}\eta_{\Grp})\ (\lambda g_0 \dots g_n.\ (\tmSem{f}\eta_\Grp\ (g_1 g^{-1}_0)\ \dots\ (g_n g^{-1}_0)) g_0) \\
      = & \lambda g_1\dots g_n.\ \tmSem{f}\eta_\Grp\ (g_1 1^{-1})\ \dots\ (g_n 1^{-1})) 1) \\
      = & \lambda g_1\dots g_n.\ \tmSem{f}\eta_\Grp\ g_1\ \dots\ g_n \\
      = & \tmSem{f}\eta_\Grp
    \end{array}
  \end{displaymath}
  as required. In the other direction, we need to use the abstraction
  theorem. For any program $\Gamma_\Grp \vdash f : \tau_n$, we have
  \begin{displaymath}
    \begin{array}{cl}
        & \tmSem{M^{-1}\ (M\ f)}\eta_\Grp \\
      = & (\tmSem{M^{-1}}\eta_\Grp)\ (\lambda g_1\dots g_n.\ (\tmSem{f}\eta_\Grp\ 1\ g_1\ \dots\ g_n)) \\
      = & \lambda g_0\dots g_n.\ (\tmSem{f}\eta_\Grp\ 1\ (g_1g^{-1}_0)\ \dots\ (g_ng^{-1}_0))g_0
    \end{array}
  \end{displaymath}
  By \thmref{thm:abstraction}, we know that the following holds of
  $\tmSem{f}$:
  \begin{displaymath}
    \forall a.\ \forall g_0 \dots g_n.\ \tmSem{f}\eta_\Grp\ (g_0a)\ \dots\ (g_na) = (\tmSem{f}\eta_\Grp\ g_0\ \dots\ g_n)a
  \end{displaymath}
  As a special case, we set $a = g^{-1}_0$, and learn that
  \begin{displaymath}
    \forall g_0 \dots g_n.\ \tmSem{f}\eta_\Grp\ 1\ (g_1g^{-1}_0) \dots\ (g_ng^{-1}_0) = (\tmSem{f}\eta_\Grp\ g_0\ \dots\ g_n)g^{-1}_0.
  \end{displaymath}
  By functional extensionality, we are done.
\end{proof}

%%% Local Variables:
%%% TeX-master: "paper"
%%% End:

\section{Discussion}
\label{sec:discussion}

We have presented a general framework for algebraically indexed types
and instantiated it to yield novel type systems for geometry, logical
information flow and distance-indexed types. Our framework further
demonstrates the power of relational reasoning about typed
programs. From \thmref{thm:abstraction}, we have derived interesting
free theorems, type isomorphisms and non-definability results. We
conclude with some observations and suggestions for further work.

\paragraph{Further Applications and Extensions} We have covered
several applications of algebraically indexed types in this paper, but
there are undoubtedly many more. Geometry for dimensions greater than
two is an obvious candidate, as well as systems that are invariant
under different geometric groups (e.g.,~the Poincar\'{e} group for
relativity). Mathematical Physics is particularly rich in theories
that have some notion of invariance, and it will be exciting to pin
down the precise connections between these and type systems for which
analogues of Reynolds' abstraction theorem hold. Geometric theorem
proving is another possible application area. Harrison
\cite{harrison09without} comments on the pervasive useful of
invariance properties in this area. Programs in our framework
automatically satisfy invariance properties, removing the need for
ad-hoc proofs of these facts.

In \autoref{sec:metric-types} we presented a type system with
distance-indexed types, and noted the similarity with Reed and
Pierce's system for non-expansivity \cite{reed10distance}. Exploring
the relationship--particularly with respect to probability
monads--also looks promising.

Type and effect analyses use types indexed by effect annotations that
have algebraic structure (e.g.~sets of read/write effect labels, with
the structure of an idempotent monoid). Benton et al.~have used
relational interpretations to prove effect-dependent equivalences
\cite{benton06reading}. We expect that an extension of our framework
with type-indexed types should be able to express their effect-indexed
monads and prove their equivalences.

Extending our framework with type dependency would also allow for
further applications. For example, we could consider a type of lists
of length $n$, indexed by elements of the permutation group
$S_n$. Bernardy et al.~have presented a general framework for
relational reasoning and an abstraction theorem for dependent types
\cite{bernardy12proofs}. However, they work with pure type systems,
which define type equality via untyped rewriting, so it is not
immediately obvious how to integrate arbitrary equational theories
into their framework.

\paragraph{Extension of the non-definability results} Our
non-definabilty results are currently restricted to the quantifier
free fragment of our framework, as defined in the statement of
\thmref{thm:abstraction}. This is a consequence of the
non-compositionality of the relational environments that we use to
prove non-definabilty results. Lifting this restriction is a key item
of future work.

\paragraph{Semantic Equality} In general, the semantic equality
$\stackrel{sem}\sim$ in \defref{def:semantic-equality} is not an
equivalence relation. If we assume that the interpretations of all the
primitive types are partial equivalence relations then semantic
equality is indeed an equivalence relation. However, this excludes the
geometry and distance-indexed examples. More generally, we can
consider relational interpretations that are \emph{difunctional}. (A
relation is difunctional if whenever $(x,y)$, $(x',y')$ and $(x,y')$
are in the relation then so is $(x',y)$.) Difunctionality is weaker
than being a PER, but still suffices to prove that semantic equality
is an equivalence relation. Hofmann \cite{hofmann08correctness} has
used difunctional relations in the setting of effect
analyses. Difunctionality covers all our examples except
distance-indexed types. Note that for both PERs and difunctional
relations we need to close the relational interpretation of
existential types under the appropriate property to ensure that all
types are interpreted as PERs/difunctional relations. For
distance-indexed types it is possible that a new notion of
equivalence, based on closeness, is required.

%%% Local Variables:
%%% TeX-master: "paper"
%%% End:


%%%%%%%%%%%%%%%%%%%%%%%%%%%%%%%%%%%%%%%%%%%%%%%%%%%%%%%%%%%%%%%%%%%%%%%%%%%%%%
%%%%%%%%%%%%%%%%%%%%%%%%%%%%%%%%%%%%%%%%%%%%%%%%%%%%%%%%%%%%%%%%%%%%%%%%%%%%%%
\bibliographystyle{abbrvnat}
\bibliography{paper}

\newpage
\appendix

\section{Details of proofs}

\begin{proof}\textbf{[Of \lemref{lem:tyeq-erasure}]}
  By induction on the derivation of $\Delta \vdash A \equiv B
  \isType$:
  \begin{description}
  \item[Case \TirName{TyEqUnit}] In this case, $A = B = \tyUnit$, so
    $\tySem{A} = \tySem{B}$.
  \item[Case \TirName{TyEqPrimIdx}] We have $A = \tyPrim{X}{e_1}$ and
    $B = \tyPrim{X}{e_2}$, for some $e_1$ and $e_2$ of the same
    sort. The erasure semantics ignores the index arguments, so
    $\tySem{A} = \tySem{\tyPrim{X}{e_1}} = \tyPrimSem{X} =
    \tySem{\tyPrim{X}{e_2}} = \tySem{B}$, as required.
  \item[Case \TirName{TyEqPrimNoIdx}] We have $A = \tyPrimNm{X}$ and
    $B = \tyPrimNm{X}$. So $\tySem{A} = \tyPrimSem{X} = \tySem{B}$, as
    required.
  \item[Case \TirName{TyEqArr}] Here, $A = A_1 \tyArr A_2$ and $B =
    B_1 \tyArr B_2$, and by the induction hypothesis we know that
    $\tySem{A_1} = \tySem{B_1}$ and $\tySem{A_2} = \tySem{B_2}$. Hence
    $\tySem{A} = \tySem{A_1 \tyArr A_2} = \tySem{A_1} \to \tySem{A_2}
    = \tySem{B_1} \to \tySem{B_2} = \tySem{B_1 \tyArr B_2} =
    \tySem{B}$.
  \item[Case \TirName{TyEqProd}] In this case, $A = A_1 \tyProduct
    A_2$ and $B = B_1 \tyProduct B_2$, and by the induction hypothesis
    we know that $\tySem{A_1} = \tySem{B_1}$ and $\tySem{A_2} =
    \tySem{B_2}$. Hence $\tySem{A} = \tySem{A_1 \tyProduct A_2} =
    \tySem{A_1} \times \tySem{A_2} = \tySem{B_1} \times \tySem{B_2} =
    \tySem{B_1 \tyProduct B_2} = \tySem{B}$.
  \item[Case \TirName{TyEqSum}] In this case, $A = A_1 + A_2$ and $B =
    B_1 + B_2$, and by the induction hypothesis we know that
    $\tySem{A_1} = \tySem{B_1}$ and $\tySem{A_2} = \tySem{B_2}$. Hence
    $\tySem{A} = \tySem{A_1 + A_2} = \tySem{A_1} + \tySem{A_2} =
    \tySem{B_1} + \tySem{B_2} = \tySem{B_1 + B_2} = \tySem{B}$.
  \item[Case \TirName{TyEqForall}] We have $A = \forall
    i\mathord:s.A'$ and $B = \forall i\mathord:s.B'$. By the induction
    hypothesis we know that $\tySem{A'} = \tySem{B'}$, and hence
    $\tySem{A} = \tySem{\forall i\mathord:s. A'} = \tySem{A'} =
    \tySem{B'} = \tySem{\forall i\mathord:s. B'} = \tySem{B}$.
  \end{description}
\end{proof}

\begin{proof}\textbf{[Of \lemref{lem:tysubst-erasure}]}
  By induction on the derivation of $\Delta \vdash A \isType$.
  \begin{description}
  \item[Case \TirName{TyUnit}] Directly from the definition of
    substitution on types of the form $\tyUnit$, we have
    $\tySem{\sigma^*\tyUnit} = \tySem{\tyUnit}$, as required.
  \item[Case \TirName{TyPrimIdx}] Since the index-erasure semantics
    ignores all index expessions, we have
    $\tySem{\sigma^*\tyPrim{X}{e}} = \tySem{\tyPrim{X}{\sigma^*e}} =
    \tyPrimSem{X} = \tySem{\tyPrim{X}{e}}$.
  \item[Case \TirName{TyPrimNoIdx}] Similar to the previous case.
  \item[Case \TirName{TyArr}] We have $A = A_1 \tyArr A_2$, and by the
    induction hypothesis, we know that $\tySem{\sigma^*A_1} =
    \tySem{A_1}$ and $\tySem{\sigma^*B_1} = \tySem{B_1}$. Thus, by the
    definition of substitution on arrow types, we have
    $\tySem{\sigma^*(A_1 \tyArr A_2)} = \tySem{\sigma^*A_1 \tyArr
      \sigma^*A_2} = \tySem{\sigma^*A_1} \to \tySem{\sigma^*A_2} =
    \tySem{A_1} \to \tySem{A_2} = \tySem{A_1 \tyArr A_2}$, as
    required.
  \item[Case \TirName{TyProd}] We have $A = A_1 \tyProduct A_2$, and
    by the induction hypothesis, we know that $\tySem{\sigma^*A_1} =
    \tySem{A_1}$ and $\tySem{\sigma^*A_2} = \tySem{A_2}$. Thus, by the
    definition of substitution on product types, we have
    $\tySem{\sigma^*(A_1 \tyProduct A_2)} = \tySem{\sigma^*A_1
      \tyProduct \sigma^*A_2} = \tySem{\sigma^*A_1} \times
    \tySem{\sigma^*A_2} = \tySem{A_1} \times \tySem{A_2} = \tySem{A_1
      \tyProduct A_2}$, as required.
  \item[Case \TirName{TySum}] We have $A = A_1 + A_2$, and by the
    induction hypothesis, we know that $\tySem{\sigma^*A_1} =
    \tySem{A_1}$ and $\tySem{\sigma^*A_2} = \tySem{A_2}$. Thus, by the
    definition of substitution on sum types, we have
    $\tySem{\sigma^*(A_1 + A_2)} = \tySem{\sigma^*A_1 + \sigma^*A_2} =
    \tySem{\sigma^*A_1} + \tySem{\sigma^*A_2} = \tySem{A_1} +
    \tySem{A_2} = \tySem{A_1 + A_2}$, as required.
  \item[Case \TirName{TyForall}] We have $A = \forall i\mathord:s.A'$,
    and by the induction hypothesis we know that, for all $\sigma'$,
    $\tySem{\sigma'^*A'} = \tySem{A'}$. Therefore, from the definition
    of substitution for $\forall$-types, we have
    $\tySem{\sigma^*\forall i\mathord:s.A'} = \tySem{\forall
      i\mathord:s.\sigma_s^{*}A'} = \tySem{\sigma_s^{*}A'} = \tySem{A'}
    = \tySem{\forall i\mathord:s.A'}$, where we have relied on the
    fact that the index-erasure semantics of types ignores the index
    quantification.
  \end{description}
\end{proof}

\begin{proof}\textbf{[Of \lemref{lem:tyeq-rel}]}
  By induction on the derivation of $\Delta \vdash A \equiv B \isType$.
  \begin{description}
  \item[Case \TirName{TyEqUnit}] In this case $A = B = \tyUnit$, so
    the statement is trivially statisfied.
  \item[Case \TirName{TyEqPrimIdx}] In this case we have $A =
    \tyPrim{X}{e_1}$ and $B = \tyPrim{X}{e_2}$, with $\Delta \vdash
    e_1 \equiv e_2 : s$. The index expressions $e_1$ and $e_2$ are in
    the same equivalence class, so by the definition of relation
    environments we have that $\rho_{\tyPrimNm{X}}(e_1) =
    \rho_{\tyPrimNm{X}}(e_2)$ and hence
    $\rsem{\tyPrim{X}{e_1}}{\relEnv{E}}\rho =
    \rsem{\tyPrim{X}{e_2}}{\relEnv{E}}\rho$.
  \item[Case \TirName{TyEqPrimNoIdx}] This case is similar to the case
    for \TirName{TyEqUnit}.
  \item[Case \TirName{TyEqArr}] We have $A = A_1 \tyArr A_2$ and $B =
    B_1 \tyArr B_2$, with $\Delta \vdash A_1 \equiv B_1 \isType$ and
    $\Delta \vdash A_2 \equiv B_2 \isType$. By the induction
    hypothesis, we know that $\rsem{A_1}{\relEnv{E}}\rho =
    \rsem{B_1}{\relEnv{E}}\rho$ and $\rsem{A_2}{\relEnv{E}}\rho =
    \rsem{B_2}{\relEnv{E}}\rho$. Thus we may reason as follows:
    $\rsem{A_1 \tyArr A_2}{\relEnv{E}}\rho =
    \rsem{A_1}{\relEnv{E}}\rho \relArrow \rsem{A_2}{\relEnv{E}}\rho =
    \rsem{B_1}{\relEnv{E}}\rho \relArrow \rsem{B_2}{\relEnv{E}}\rho =
    \rsem{B_1 \tyArr B_2}{\relEnv{E}}\rho$, as required.
  \item[Case \TirName{TyEqProd}] We have $A = A_1 \tyProduct A_2$ and
    $B = B_1 \tyProduct B_2$, with $\Delta \vdash A_1 \equiv B_1
    \isType$ and $\Delta \vdash A_2 \equiv B_2 \isType$. By the
    induction hypothesis, we know that $\rsem{A_1}{\relEnv{E}}\rho =
    \rsem{B_1}{\relEnv{E}}\rho$ and $\rsem{A_2}{\relEnv{E}}\rho =
    \rsem{B_2}{\relEnv{E}}\rho$. Therefore, we can reason as follows:
    $\rsem{A_1 \tyProduct A_2}{\relEnv{E}}\rho =
    \rsem{A_1}{\relEnv{E}}\rho \relTimes \rsem{A_2}{\relEnv{E}}\rho =
    \rsem{B_1}{\relEnv{E}}\rho \relTimes \rsem{B_2}{\relEnv{E}}\rho =
    \rsem{B_1 \tyProduct B_2}{\relEnv{E}}\rho$, as required.
  \item[Case \TirName{TyEqSum}] We have $A = A_1 + A_2$ and $B = B_1 +
    B_2$, with $\Delta \vdash A_1 \equiv B_1 \isType$ and $\Delta
    \vdash A_2 \equiv B_2 \isType$. By the induction hypothesis, we
    know that $\rsem{A_1}{\relEnv{E}}\rho =
    \rsem{B_1}{\relEnv{E}}\rho$ and $\rsem{A_2}{\relEnv{E}}\rho =
    \rsem{B_2}{\relEnv{E}}\rho$. Therefore, we can reason as follows:
    $\rsem{A_1 + A_2}{\relEnv{E}}\rho = \rsem{A_1}{\relEnv{E}}\rho
    \relSum \rsem{A_2}{\relEnv{E}}\rho = \rsem{B_1}{\relEnv{E}}\rho
    \relSum \rsem{B_2}{\relEnv{E}}\rho = \rsem{B_1 +
      B_2}{\relEnv{E}}\rho$, as required.
  \item[Case \TirName{TyEqForall}] We have $A = \forall
    i\mathord:s. A'$ and $B = \forall i\mathord:s.B'$, with $\Delta, i
    : s \vdash A' \equiv B' \isType$. By the induction hypothesis,
    we know that for all $\rho'$, $\rsem{A'}{\relEnv{E}}\rho' =
    \rsem{B'}\rho'$. We reason as follows:
    \begin{eqnarray*}
      &    & (x_1,x_2) \in \rsem{\forall i\mathord:s. A'}{\relEnv{E}}\rho \\
      &\iff& \forall \rho' \in \extends{\rho}{i:s}.\ (x_1,x_2) \in \rsem{A'}{\relEnv{E}}\rho' \\
      &\iff& \forall \rho' \in \extends{\rho}{i:s}.\ (x_1,x_2) \in \rsem{B'}{\relEnv{E}}\rho' \\
      &\iff& (x_1,x_2) \in \rsem{\forall i\mathord:s. B'}{\relEnv{E}}\rho
    \end{eqnarray*}
    Thus, by extensionality, we have $\rsem{\forall
      i\mathord:s.A'}\rho \equiv \rsem{\forall
      i\mathord:s.B'}{\relEnv{E}}\rho$, as required.
  \end{description}
\end{proof}

\begin{proof}\textbf{[Of \lemref{lem:tysubst-rel}]}
  The first part of the lemma statement is proved by induction on the
  derivation of $\Delta \vdash A \isType$. We analyse each case in
  turn:
  \begin{description}
  \item[Case \TirName{TyUnit}] Directly from the definition of
    substitution on types of the form $\tyUnit$, and the fact that
    $\rsem{\tyUnit}{\relEnv{E}}\rho_1 =
    \rsem{\tyUnit}{\relEnv{E}}\rho_2$ for any pair $\rho_1,\rho_2$ of
    relation environments, we have
    $\rsem{\sigma^*\tyUnit}{\relEnv{E}}\rho =
    \rsem{\tyUnit}{\relEnv{E}}{(\rho \circ \sigma^*)}$, as required.
  \item[Case \TirName{TyPrimIdx}] We have $A = \tyPrim{X}{e}$, therefore
    $\rsem{\sigma^*A}{\relEnv{E}}\rho =
    \rsem{\sigma^*\tyPrim{X}{e}}{\relEnv{E}}\rho =
    \rsem{\tyPrim{X}{\sigma^*e}}{\relEnv{E}}\rho =
    \rho_{\tyPrimNm{X}}(\sigma^*e) =
    \rsem{\tyPrim{X}{e}}{\relEnv{E}}{(\rho \circ \sigma^*)}$, as
    required.
  \item[Case \TirName{TyPrimNoIdx}] Similar to the case for
    \TirName{TyUnit}.
  \item[Case \TirName{TyArr}] In this case, $A = A_1 \tyArr A_2$ and
    by the induction hypothesis, we know that $\rsem{\sigma^*A_1}{\relEnv{E}}\rho
    = \rsem{A_1}{\relEnv{E}}{(\rho \circ \sigma^*)}$ and $\rsem{\sigma^*A_2}{\relEnv{E}}\rho =
    \rsem{A_2}{\relEnv{E}}{(\rho \circ \sigma^*)}$. Accordingly, we reason as
    follows: $\rsem{\sigma^*(A_1 \tyArr A_2)}{\relEnv{E}}\rho = \rsem{\sigma^*A_1
      \tyArr \sigma^*A_2}{\relEnv{E}}\rho = \rsem{\sigma^*A_1}{\relEnv{E}}\rho \relArrow
    \rsem{\sigma^*A_2}{\relEnv{E}}\rho = \rsem{A_1}{\relEnv{E}}{(\rho \circ \sigma^*)}
    \relArrow \rsem{A_2}{\relEnv{E}}{(\rho \circ \sigma^*)} = \rsem{A_1 \tyArr
      A_2}{\relEnv{E}}{(\rho \circ \sigma^*)}$, as required.
  \item[Case \TirName{TyProd}] In this case, $A = A_1 \tyProduct A_2$
    and by the induction hypothesis, we know that
    $\rsem{\sigma^*A_1}{\relEnv{E}}\rho = \rsem{A_1}{\relEnv{E}}{(\rho
      \circ \sigma^*)}$ and $\rsem{\sigma^*A_2}{\relEnv{E}}\rho =
    \rsem{A_2}{\relEnv{E}}{(\rho \circ \sigma^*)}$. Accordingly, we
    reason as follows: $\rsem{\sigma^*(A_1 \tyProduct
      A_2)}{\relEnv{E}}\rho = \rsem{\sigma^*A_1 \tyProduct
      \sigma^*A_2}{\relEnv{E}}\rho =
    \rsem{\sigma^*A_1}{\relEnv{E}}\rho \relTimes
    \rsem{\sigma^*A_2}{\relEnv{E}}\rho = \rsem{A_1}{\relEnv{E}}{(\rho
      \circ \sigma^*)} \relTimes \rsem{A_2}{\relEnv{E}}{(\rho \circ
      \sigma^*)} = \rsem{A_1 \tyProduct A_2}{\relEnv{E}}{(\rho \circ
      \sigma^*)}$, as required.
  \item[Case \TirName{TySum}] In this case, $A = A_1 + A_2$
    and by the induction hypothesis, we know that
    $\rsem{\sigma^*A_1}{\relEnv{E}}\rho = \rsem{A_1}{\relEnv{E}}{(\rho
      \circ \sigma^*)}$ and $\rsem{\sigma^*A_2}{\relEnv{E}}\rho =
    \rsem{A_2}{\relEnv{E}}{(\rho \circ \sigma^*)}$. Accordingly, we
    reason as follows: $\rsem{\sigma^*(A_1 +
      A_2)}{\relEnv{E}}\rho = \rsem{\sigma^*A_1 +
      \sigma^*A_2}{\relEnv{E}}\rho =
    \rsem{\sigma^*A_1}{\relEnv{E}}\rho \relSum
    \rsem{\sigma^*A_2}{\relEnv{E}}\rho = \rsem{A_1}{\relEnv{E}}{(\rho
      \circ \sigma^*)} \relSum \rsem{A_2}{\relEnv{E}}{(\rho \circ
      \sigma^*)} = \rsem{A_1 + A_2}{\relEnv{E}}{(\rho \circ
      \sigma^*)}$, as required.
  \item[Case \TirName{TyForall}] We have $A = \forall
    i\mathord:s.A'$. We prove the required equation by demonstrating
    two inclusions. From left-to-right, we know that
    \begin{equation}
      \label{eq:tysubst-rel-forall1}
      (x_1,x_2) \in \rsem{\sigma^*\forall i\mathord:s. A'}{\relEnv{E}}\rho
    \end{equation}
    and we must show that $(x_1, x_2) \in \rsem{\forall
      i\mathord:s. A'}{\relEnv{E}}{(\rho \circ \sigma^*)}$. From
    (\ref{eq:tysubst-rel-forall1}) and the definition of substitution,
    we know that
    \begin{displaymath}
      \forall \rho' \in \extends{\rho}{i:s}.\ (x_1,x_2) \in \rsem{\sigma_s^*A'}{\relEnv{E}}\rho'
    \end{displaymath}
    By the induction hypothesis, we can deduce that
    \begin{equation}
      \label{eq:tysubst-rel-forall2}
      \forall \rho' \in \extends{\rho}{i:s}.\ (x_1,x_2) \in \rsem{A'}{\relEnv{E}}{(\rho' \circ \sigma_s^{*})}
    \end{equation}
    To show the required inclusion, we must show that for all $\rho_1
    \in \extends{\rho \circ \sigma^*}{i:s}$, it is the case that
    $(x_1,x_2) \in \rsem{A'}{\relEnv{E}}{\rho_1}$. We now make use of
    the pushout property of our family $\relEnv{E}$ of relation
    environments: we have a $\rho \in \relEnv{E}(\Delta)$ and $\rho_1
    \in \relEnv{E}(\Delta, i:s)$ such that $\rho_1 \circ \pi_{i:s} =
    \rho \circ \sigma^*$ (by the assumption that $\rho_1 \in
    \extends{\rho \circ \sigma^*}{i:s}$), so we can obtain a $\rho'
    \in \relEnv{E}(\Delta',i:s)$ such that $\rho' \circ \pi_{i:s} =
    \rho$ (so $\rho' \in \extends{\rho}{i:s}$), and $\rho' \circ
    \sigma^* = \rho_1$. Therefore we use
    (\ref{eq:tysubst-rel-forall2}) to deduce that $(x_1,x_2) \in
    \rsem{A'}{\relEnv{E}}{\rho_1}$, and the inclusion from
    left-to-right is shown.

    From right-to-left, we must now show
    (\ref{eq:tysubst-rel-forall2}) under the assumption that
    \begin{equation}
      \label{eq:tysubst-rel-forall3}
      \forall \rho_1 \in \extends{\rho \circ \sigma^*}{i:s}. (x_1,x_2) \in \rsem{A'}{\relEnv{E}}{\rho_1}.
    \end{equation}
    Given $\rho' \in \extends{\rho}{i:s}$, we let $\rho_1 = \rho'
    \circ \sigma^*$. Now, $\rho_1 \circ \pi_{i:s} = \rho' \circ
    \sigma^* \circ \pi_{i:s} = \rho' \circ \pi_{i:s} \circ \sigma_s^{*}
    = \rho \circ \sigma_s^{*}$. Thus $\rho_1 \in \extends{\rho \circ
      \sigma^*}{i:s}$ and we may use (\ref{eq:tysubst-rel-forall3}) to
    deduce that $(x_1,x_2) \in \rsem{A'}{\relEnv{E}}{(\rho' \circ \sigma_s^{*})}$,
    as required.
  \end{description}
  The second part of the lemma statement follows directly by induction
  on the derivation of $\Delta \vdash \Gamma \isCtxt$ and the first
  part. The step case of the induction is very similar to the
  \TirName{TyProd} case above.
\end{proof}


\end{document}
