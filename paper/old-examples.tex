%%%%%%%%%%%%%%%%%%%%%%%%%%%%%%%%%%%%%%%%%%%%%%%%%%%%%%%%%%%%%%%%%%%%%%%%%%%%%%
%%%%%%%%%%%%%%%%%%%%%%%%%%%%%%%%%%%%%%%%%%%%%%%%%%%%%%%%%%%%%%%%%%%%%%%%%%%%%%
\section{Example Systems}\label{sec:example-systems}

\subsection{Units of Measure}\label{sec:units-example}

We now show that our general framework encompasses Kennedy's
units-of-measure system \cite{kennedy97relational}. For our indexing
algebraic theory, we have a single sort $\mathsf{unit}$ intended to
represent composite units such as $\mathit{m}\ \mathit{s}^{-1}$ or
$\mathit{kg}\ \mathit{m}^{-2}$. Units of measure form an abelian
group, so we have three operations:
\begin{mathpar}
  1 : \mathsf{unit},

  \cdot : \mathsf{unit}, \mathsf{unit} \to \mathsf{unit},

  -^{-1} : \mathsf{unit} \to \mathsf{unit},
\end{mathpar}
subject to the normal abelian group axioms. We will elide the group
multiplication operation and just write multiplication as
juxtaposition: $u_1u_2$.

We let $\mathit{PrimType}(\mathsf{unit}) = \{\tyPrimNm{num}$, and
$\mathit{PrimType}(\cdot) = \{\tyPrimNm{rational}\}$. The intended
reading is that the type $\tyPrim{num}{u_1u^2_2}$ represents
quantities in units of $u_1u^2_2$, while values of type
$\tyPrimNm{rational}$ represent dimensionless constants. We make a
distinction between $\tyPrim{num}{1}$ and $\tyPrimNm{rational}$ for
technical reasons arising from the relational interpretation below: we
want the relation on $\tyPrim{num}{1}$ to be the identity relation,
and the existence of $\mathit{toRat}$ and $\mathit{fromRat}$ forces
this\footnote{Kennedy assumes that all relational environments set the
  relation on $\tyPrim{num}{1}$ to be the identity relation, but we
  cannot do this in general, because we do not necessarily have a
  group structure in the indexing theory.}. To write programs using
the $\tyPrimNm{num}$ type, we assume an typing context
$\Gamma_{\textrm{units}}$ consisting of the following typed
operations:
\begin{eqnarray*}
  \mathit{toRat} & : & \tyPrim{num}{1} \to \tyPrimNm{rational} \\
  \mathit{fromRat} & : & \tyPrim{num}{1} \to \tyPrimNm{rational} \\
  0 & : & \forall u. \tyPrim{num}{u} \\
  1 & : & \tyPrim{num}{1} \\
  + & : & \forall u. \tyPrim{num}{u} \to \tyPrim{num}{u} \to \tyPrim{num}{u} \\
  - & : & \forall u. \tyPrim{num}{u} \to \tyPrim{num}{u} \to \tyPrim{num}{u} \\
  * & : & \forall u_1 u_2. \tyPrim{num}{u_1} \to \tyPrim{num}{u_2} \to \tyPrim{num}{u_1u_2} \\
  / & : & \forall u_1 u_2. \tyPrim{num}{u_1} \to \tyPrim{num}{u_2} \to \tyPrim{num}{u_1u_2^{-1}} + 1 \\
  < & : & \forall u. \tyPrim{num}{u} \to \tyPrim{num}{u} \to \tyBool
\end{eqnarray*}
There are constants for zero and one, but only the constant for zero
is polymorphic in the units. In \autoref{sec:units-semantics} below,
we will following Kennedy by interpreting units of measure in terms of
\emph{scalings} by rational numbers, where it will be apparent that
the constant one cannot be unit-polymorphic: we can scale $0$ and
still get $0$, but we cannot scale $1$ by an arbitrary rational number
and expect to get $1$.

\subsection{Euclidean Plane Geometry}\label{sec:2d-example}

We now consider a type system for writing programs that are invariant
under the translational symmetries of the Euclidean plane. We assume a
single sort $\mathsf{translation}$. The sort $\mathsf{translation}$ is
intended to represent translations in the plane, and comes equipped
with abelian group structure that we write additively:
\begin{mathpar}
  0 : \mathsf{translation}

  + : \mathsf{translation},\mathsf{translation} \to \mathsf{translation}

  - : \mathsf{translation} \to \mathsf{translation}
\end{mathpar}
In indexing expressions involving variables of sort
$\mathsf{translation}$, we use the shorthand $t_1 - t_2$ to stand for
$t_1 + (-t_2)$.

There are two primitive types:
$\mathit{PrimType}(\mathsf{translation}) = \{\tyPrimNm{vec}\}$, and
$\mathit{PrimType}(\cdot) = \{\tyPrimNm{real}\}$. The types
$\tyPrim{vec}{t}$ represent two-dimensional vectors, indistinguishable
up to some translation $t$. Values of type $\tyPrimNm{real}$ are just
real numbers. To write programs using these types we assume a typing
context $\Gamma_{\mathit{2D}}$ consisting of the following operations:
\begin{eqnarray*}
  + & : & \forall t_1 t_2\mathord:\mathsf{translation}.\tyPrim{vec}{t_1} \to \tyPrim{vec}{t_2} \to \tyPrim{vec}{t_1 + t_2} \\
  - & : & \forall t\mathord:\mathsf{translation}. \tyPrim{vec}{t} \to \tyPrim{vec}{-t} \\
  0 & : & \tyPrim{vec}{0} \\
  * & : & \tyPrimNm{real} \to \tyPrim{vec}{0} \to \tyPrim{vec}{0} \\
  \mathrm{getX} & : & \tyPrim{vec}{0} \to \tyPrimNm{real} \\
  \mathrm{getY} & : & \tyPrim{vec}{0} \to \tyPrimNm{real} \\
  \mathrm{vec}  & : & \tyPrimNm{real} \to \tyPrimNm{real} \to \tyPrim{vec}{0}
\end{eqnarray*}
The first three operations express the abelian group structure of
2-dimensional real vectors. Note how the indexing tracks the
operations on vectors: e.g.~addition on vectors is reflected by
addition on the indicies. The final four operations are operations
that are not invariant under arbitrary translation: multiplication by
a scalar, and projection from and constructon of vectors. For the
purposes of examples, we also assume a standard set of arithmetic and
other operations on real numbers.

As an example of the use of the above operations, consider the
following type of a function for computing the area of a triangle
described by three points. Since the area of a triangle is invariant
under translations that are applied to all three points, we can
quantify over all translations $t$:
\begin{displaymath}
  \mathit{area} : \forall t\mathord:\mathsf{translation}.\ \tyPrim{vec}{t} \to \tyPrim{vec}{t} \to \tyPrim{vec}{t} \to \tyPrimNm{real}
\end{displaymath}
Since the result is invariant under translation, it safe to assume
that one of the vertexes of the triangle is the origin, and only
specify two points to describe the triangle. So we might expect that
the type of $\mathit{area}$ is isomorphic to this type:
\begin{displaymath}
  \mathit{area'} : \tyPrim{vec}{0} \to \tyPrim{vec}{0} \to \tyPrimNm{real}
\end{displaymath}
Given $\mathit{area'}$, we can define $\mathit{area}$ using the
operations in $\Gamma_{\mathit{2D}}$:
\begin{displaymath}
  \mathit{area} = \Lambda t. \lambda v_1 v_2 v_3.\ \mathit{area'}\ (v_2 - v_1)\ (v_3 - v_1)
\end{displaymath}
and vice versa, we can define $\mathit{area'}$ in terms of
$\mathit{area}$:
\begin{displaymath}
  \mathit{area'} = \lambda v_1 v_2. \mathit{area}\ [0]\ 0\ v_1\ v_2
\end{displaymath}
In \autoref{sec:2d-semantics} below, we will show that these two
translations form a type isomorphism between the types of
$\mathit{area}$ and $\mathit{area'}$.

\paragraph{Alternative Transformation Groups?}
We should also consider other groups of transformations for indexing:
isometries, orthogonal transformations, rotations and reflections.

\subsection{Security Typing and Non-interference}

This example shows how our framework can accommodate non-interference
properties common in security type systems. This example has a
different flavour to the two previous examples because it does not
concern invariance under the action of some group. We assume a single
sort $\mathsf{principal}$, intended to represent (possibly composite)
principals in a system. Principals are combined using the connectives
of boolean logic:
\begin{mathpar}
  \top : \mathsf{principal}

  \bot : \mathsf{principal}

  \land : \mathsf{principal},\mathsf{principal} \to \mathsf{principal}

  \lor : \mathsf{principal},\mathsf{principal} \to \mathsf{principal}

  \Rightarrow : \mathsf{principal},\mathsf{principal} \to \mathsf{principal}

  \lnot : \mathsf{principal} \to \mathsf{principal}
\end{mathpar}
For the equational theory of principals, we assume all the axioms of
boolean algebra.

Let $\mathit{PrimType}(\mathsf{principal}) = \{\tyPrimNm{T}\}$. The
intended reading of a type $\tyPrim{T}{p}$ for some principal $p$ is
that the existence of a value of this type indicates that the
principal $p$ is ``happy''. To write programs, we assume the following
typing context $\Gamma_{\mathit{sec}}$ of operations:
\begin{eqnarray*}
  \mathrm{proj} & : & \forall p_1 p_2\mathord:\mathsf{principal}.\ \tyPrim{T}{p_1 \land p_2} \to \tyPrim{T}{p_1} \\
  \mathrm{combine} & : & \forall p_1 p_2\mathord:\mathsf{principal}.\ \tyPrim{T}{p_1} \to \tyPrim{T}{p_2} \to \tyPrim{T}{p_1 \land p_2}
\end{eqnarray*}

To illustrate the use of this application of our framework, we define
a type synonym, for each type $A$ and principal $p$:
\begin{displaymath}
  T_pA = \tyPrim{T}{p} \to A
\end{displaymath}
For every principal $p$, we can endow the types $T_p-$ with the
structure of a monad. This is due to the fact that it is an instance
of the ``environment'' (or ``reader'') monad. Explicitly, the
$\mathit{return}$ and $>>=$ of these monads are given by:
\begin{eqnarray*}
  \mathit{return} & : & A \to T_pA \\
  \mathit{return} & = & \lambda a\ t.\ a \\
  \\
  (>>=) & : & T_pA \to (A \to T_pB) \to T_pB \\
  (>>=) & = & \lambda a\ f\ t.\ f\ (a\ t)\ t
\end{eqnarray*}
The intended reading of a value of type $T_pA$ is as a piece of data
of type $A$ that can only be accessed if the principal $p$ is
happy. The existence of such a monad should allow us to draw parallels
between our system and the Dependency Core Calculus (DCC) system of
Abadi \emph{et al.} \cite{abadi99core}. The definition of a DCC monad
as an instance of an environment monad follows Tse and Zdancewic's
interpretation of DCC in System F, where they use System F
parametricity to ensure non-interference properties.

There is also a coercion operator between instances of $T_pA$ for
different principals. Reading the type in terms of happiness, this
says that if we have a piece of data that is protected by $p$, and
happiness of $q$ implies happiness of $p$, then this data can also be
accessed by $q$.
\begin{eqnarray*}
  \mathit{coerce} & : & \tyPrim{T}{q \Rightarrow p} \to T_pA \to T_qA \\
  \mathit{coerce} & = & \lambda t_{qp}\ a\ t_q.\ a\ (\mathrm{combine}\ t_{qp}\ t_q)
\end{eqnarray*}
Here we have used the fact that the result of the application of
$\mathrm{combine}$ has type $\tyPrim{T}{(q \Rightarrow p) \land
  q}$. By the laws of boolean algebras---which we have assumed for our
indexing theory---this is equal to the type $\tyPrim{T}{p}$. Thus we
get access to the data stored in $a$, which is protected by $p$.

Now suppose we have a term $M$ with the following typing:
\begin{displaymath}
  p,q:\mathsf{principal}; t_{qp} : \tyPrim{T}{q \Rightarrow p}, x : T_q (1 + 1) \vdash M : T_p (1 + 1)
\end{displaymath}
Intuitively, given the operations $\Gamma_{\mathit{sec}}$ we have
defined above, it should be the case that $M$ can only be equivalent
to the program which is constantly $\lambda t. \mathrm{inl}(*)$ or
$\lambda t. \mathrm{inr}(*)$, because there is no useful data to be
gained from the $t_{qp}$ argument, and a value of $\tyPrim{T}{q}$
cannot be generated in order to get access to the data within the $x$
argument. The application of our framework that we have presented in
this section has a ``non-interference'' property: the value of $x$
cannot interfere with the output. In \autoref{sec:sec-semantics} we
show that this is indeed the case.

\paragraph{Alternative Logics?}
Above, we assumed that the indexing theory of principals was the
theory of boolean algebras. As a consequence, a value of type $T_{p
  \lor \lnot p}A$ can always be accessed, due to the exlcuded middle
property of boolean logic. An interesting question is: what happens we
we consider alternative logics. If we take the theory of Heyting
algebras for our theory of principals, then data typed as $T_{p \lor
  \lnot p}A$ can only be accessed if either $p$ or $\lnot p$ can be
demonstrated. Considering an alternative logical theory alters the
non-interference property of the type system.

\paragraph{Alternative Presentations?}
In the above we presented the theory of boolean logic by means of the
equational theory of boolean algebras. This isn't necessarily the most
convenient way of presenting a logic. An alternative approach would be
to consider a trivial equational theory with no axioms, and
incorporate the rules of logic consequence as operations in
$\Gamma_{\mathit{sec}}$. For example, as a Hilbert system:
\begin{eqnarray*}
  \mathit{MP} & : & \forall p,q.\ \tyPrim{T}{p \Rightarrow q} \to \tyPrim{T}{p} \to \tyPrim{T}{q} \\
  \mathit{k} & : & \forall p,q.\ \tyPrim{T}{p \Rightarrow q \Rightarrow p} \\
  \mathit{s} & : & \forall p,q,r.\ \tyPrim{T}{(p \Rightarrow q \Rightarrow r) \Rightarrow (p \Rightarrow q) \Rightarrow p \Rightarrow r}
\end{eqnarray*}
Plus axioms for the other connectives.  Or possibly some kind of
encoding of a natural deduction system?


%%%%%%%%%%%%%%%%%%%%%%%%%%%%%%%%%%%%%%%%%%%%%%%%%%%%%%%%%%%%%%%%%%%%%%%%%%%%%%
%%%%%%%%%%%%%%%%%%%%%%%%%%%%%%%%%%%%%%%%%%%%%%%%%%%%%%%%%%%%%%%%%%%%%%%%%%%%%%
\section{Consequences of Parametricity}

We now give instances of our semantic framework for each of the three
examples from \autoref{sec:example-systems}. For each of the three
examples, we defined an indexing theory, a collection of primitive
types, and a collection of primitive operations $\Gamma_X$. Below, we
assign the ``obvious'' index-erasure semantics to each of the
primitive types and operations: so we have $\tyPrimSem{X}$ for each
primitive type $\tyPrimNm{X}$, and a $\eta_X \in \ctxtSem{\Gamma_X}$.

The key question we wish to answer below is whether there is a family
of sets of relation environments $\{\relEnv{E}_X(\Delta)\}_{\Delta}$
that satisfies the following completeness property identified by
Kennedy.
\begin{equation}\label{eq:kennedy-completeness}
  (\forall \Delta. \forall \rho \in \relEnv{E}'(\Delta). (\eta_X, \eta_X) \in \rsem{\Gamma_X}{\relEnv{E}'}{\rho}) \Leftrightarrow \relEnv{E}'\subseteq \relEnv{E}_X
\end{equation}
where the subset relation on families of sets of relation environments
is defined as $\relEnv{E}_1 \subseteq \relEnv{E}_2$ if $\forall
\Delta. \relEnv{E}_1(\Delta) \subseteq \relEnv{E}_2(\Delta)$.

Such a completeness result allows us to derive inexpressibility
results for a given instance of our general framework. Kennedy
demonstrates this for the units of measure system by showing that a
putative square root function of type $\forall u.\ \tyPrim{num}{u^2}
\to \tyPrim{num}{u}$ can only ever be the constantly zero function (or
non-terminating).

\subsection{Units of Measure}\label{sec:units-semantics}

For the units of measure example, we set $\tyPrimSem{num} =
\tyPrimSem{rational} = \mathbb{Q}$, the set of rational numbers. We
define $\eta_{\mathit{units}} \in \Gamma_{\mathit{units}}$ are defined
to be the standard arithmetical and comparison operations on rational
numbers, and $\mathit{toRat}$ and $\mathit{fromRat}$ are defined to be
the identity functions. The division function (of type $/ : \forall
u_1,u_2. \tyPrim{num}{u_1} \to \tyPrim{num}{u_2} \to
\tyPrim{num}{u_1u^{-1}_2} + 1$) returns $\mathrm{inr}(*)$ only when
its second argument is zero.

Kennedy defines the following family of sets of relational
environments. Note that, since we have assumed the abelian group
axioms, the set $\idxTms{\Delta}{\mathsf{unit}}$ of indexing
expressions quotiented by the equational theory, is an abelian group.
\begin{displaymath}
  \relEnv{E}_{\mathit{units}}(\Delta) = \left\{ \rho^{G,h} \sepbar
    \begin{array}{l}
      G\textrm{ is a subgroup of }\idxTms{\Delta}{\mathsf{unit}} \\
      h : G \to \mathbb{Q}^{>0} \\
      h\textrm{ is a group homomorphism}
    \end{array}
  \right\}
\end{displaymath}
where $\mathbb{Q}^{>0} = \{ q \in \mathbb{Q} \sepbar q > 0 \}$ and
\begin{displaymath}
  \rho^{G,h}_{\tyPrimNm{num}}(e) = \left\{
    \begin{array}{ll}
      \{(0,0)\} & \textrm{if }e \not\in G \\
      \{(r,h(e)r) \sepbar r \in \mathbb{Q} \} & \textrm{if }u \in G
    \end{array}
  \right.
\end{displaymath}

\begin{theorem}
  The family of sets of relational environments
  $\relEnv{E}_{\mathit{units}}$ satisfies the completeness property
  \hyperref[eq:kennedy-completeness]{(\ref*{eq:kennedy-completeness})}
  above. That is, for all $\relEnv{E}'$:
  \begin{displaymath}
    (\forall \Delta.\forall \rho \in \relEnv{E}'(\Delta). (\eta_{\mathit{units}}, \eta_{\mathit{units}}) \in \rsem{\Gamma_{\mathit{units}}}{\relEnv{E}'}{\rho}) \Leftrightarrow \relEnv{E}'\subseteq \relEnv{E}_{\mathit{units}}
  \end{displaymath}
\end{theorem}

\begin{proof}
  (Following Kennedy).  From right to left, we have to check that for
  each $\rho \in \relEnv{E}'(\Delta)$, where $\relEnv{E}' \subseteq
  \relEnv{E}_{\mathit{units}}$, that each of the semantic
  interpretations of the operations satisfies
  $(\eta_{\mathit{units}}(\mathit{op}),
  \eta_{\mathit{units}}(\mathit{op})) \in
  \rsem{\Gamma(\mathit{op})}{\relEnv{E}'}\rho$. For instance,
  multiplication has to satisfy: for all subgroups $G \subseteq
  \idxTms{\Delta,u_1,u_2}{\mathsf{unit}}$ and all group homomorphisms
  $h : G \to \mathbb{Q}^{>0}$, then $(q_1,q'_1) \in \rho^{G,h}(u_1)$
  and $(q_2,q'_2) \in \rho^{G,h}(u_2)$ implies that $(q_1q_2,q'_1q'_2)
  \in \rho^{G,h}(u_1u_2)$. There are four cases, depending on whether
  $u_1 \in G$ and whether $u_2 \in G$. If $u_1 \not\in G$, then $q_1 =
  q'_1 = 0$, so $q_1q_2 = 0$ and $q'_1q'_2 = 0$. Since $(0,0) \in
  \rho^{G,h}(u')$ for any $u'$, $G$ and $h$, we are done. If $u_1 \in
  G$ and $u_2 \in G$, then we need to show that $(q_1q_2,
  h(u_1)q_1h(u_2)q_2) \in \rho^{G,h}(u_1u_2)$. By $h$ being a
  homomorphism, we have $(q_1q_2, h(u_1)q_1h(u_2)q_2) = (q_1q_1,
  h(u_1u_2)q_1q_2)$, which is in $\rho^{G,h}(u_1u_2)$. The cases for
  the other operations are similar.

  From left to right, we have the following properties of any
  relational environment $\rho_{\tyPrimNm{num}} \in
  \relEnv{E}'(\Delta)$, derived from the hypothesis that all the
  operations in $\eta_{\mathit{units}}$ satisfy their types'
  relational semantics.
  \begin{enumerate}
  \item\label{item:obsv-poly} By the existence of the function
    $\mathit{toRat}$, the relation $\rho_{\tyPrimNm{num}}(1)$ is the
    identity relation.
  \item\label{item:zero-poly} For any unit expression $e$, $(0,0) \in
    \rho_{\tyPrimNm{num}}(e)$.
  \item \label{item:one-poly} $(1,1) \in \rho_{\tyPrimNm{num}}(1)$.
  \item \label{item:mult-poly} For any unit expressions $e_1,e_2$,
    such that $(q_1,q'_1) \in \rho_{\tyPrimNm{num}}(e_1)$ and
    $(q_2,q'_2) \in \rho_{\tyPrimNm{num}}(e_2)$, we have $(q_1q_2,
    q'_1q'_2) \in \rho_{\tyPrimNm{num}}(e_1e_2)$.
  \item\label{item:div-poly} For any unit expressions $e_1,e_2$, such
    that $(q_1,q'_1) \in \rho_{\tyPrimNm{num}}(e_1)$ and $(q_2,q'_2)
    \in \rho_{\tyPrimNm{num}}(e_2)$, we have that either $q_2 = q'_2 =
    0$, or we have that $(q_1/q_2,q'_1,q'_2) \in
    \rho_{\tyPrimNm{num}}(e_1e_2^{-1})$.
  \item\label{item:compare-poly} For any unit expression $e$, such
    that $(q_1,q'_1) \in \rho_{\tyPrimNm{num}}(e)$ and $(q_2,q'_2) \in
    \rho_{\tyPrimNm{num}}(e)$, we have $q_1 < q_2$ iff $q'_1 < q'_2$.
  \end{enumerate}
  We now show that if $(q,q') \in \rho_{\tyPrimNm{num}}(e)$ for some
  unit expression $e$, then either $q = q' = 0$, or $q' = rq$ for some
  $r \in \mathbb{Q}^{>0}$, determined by $e$. We know from
  \ref{item:zero-poly} above that $q = q' = 0$ is a possibility. Now
  consider what happens if $q \not= 0$: in this case, by
  \ref{item:zero-poly} and \ref{item:compare-poly}, we know that $0 <
  q$ iff $0 < q'$, so $q' \not= 0$, and $q'$ has the same sign as
  $q$. So $q/q'$ exists and is in $\mathbb{Q}^{>0}$. Hence $q = rq'$
  for $r = q/q'$. So now we know that if $(q,q') \in
  \rho_{\tyPrimNm{num}}(e)$, and $q \not= 0$ or $q' \not= 0$ then
  there exists an $r$ such that $q = rq'$.

  We now show that the unit expression $e$ determines the value of
  $r$. Assume that we have $(q_1,r_1q_1) \in \rho_{\tyPrimNm{num}}(e)$
  and $(q_2,r_2q_2) \in \rho_{\tyPrimNm{num}}(e)$. By
  \ref{item:div-poly}, we know that $(q_1/q_2, (r_1q_1)/(r_2q_2)) \in
  \rho_{\tyPrimNm{num}}(1)$. By \ref{item:obsv-poly},
  $\rho_{\tyPrimNm{num}}(1)$ is the equality relation, so we can
  deduce that $r_1 = r_2$.

  Now let $G$ be the subgroup of $\idxTms{\Delta}{\mathsf{unit}}$ such
  that $e \in G$ implies that $\rho_{\tyPrimNm{num}}(e) \not=
  \{(0,0)\}$. By the argument above, for each $e \in G$, there is a
  unique $r \in \mathbb{Q}^{>0}$ such that $\rho_{\tyPrimNm{num}}(e) =
  \{ (q,rq) \sepbar q \in \mathbb{Q} \}$. Thus there is a function $h
  : G \to \mathbb{Q}^{>0}$. By \ref{item:one-poly} and
  \ref{item:mult-poly}, $h$ is a group homomorphism. Thus we have
  shown that $\rho_{\tyPrimNm{num}}$ has the form of
  $\rho^{G,h}_{\tyPrimNm{num}}$ for some $G$ and $h$. Since
  $\rho_{\tyPrimNm{num}}$ was an arbitrary element of
  $\relEnv{E}'(\Delta)$, and $\Delta$ was arbitrary, we have shown
  that $\relEnv{E}' \subseteq \relEnv{E}_{\mathit{units}}$.
\end{proof}

Some notes towards a generalisation:
\begin{itemize}
\item By inspecting Kennedy's proof, we can see that the operations of
  addition and subtraction do not play any role, save for an
  obligation to prove that they satisfy the relational semantics of
  their types. This follows just by distributivity of addition over
  multiplication.
\item Kennedy's proof does not rely on any particular features of
  $\mathbb{Q}$ other than the fact that it is an ordered field. By
  inspecting the proof, it is possible to see that if we remove the
  comparison operation $< : \forall u. \tyPrim{num}{u} \to
  \tyPrim{num}{u} \to \tyPrimNm{bool}$ from $\Gamma_{\mathit{units}}$,
  then the appropriate family of sets of relational environments uses
  homomorphisms into the set of non-zero rationals:
  \begin{displaymath}
    \begin{array}{l}
      \relEnv{E}_{\mathit{units}\backslash\{<\}}(\Delta) = \\
      \quad\left\{ \rho^{G,h} \sepbar
        \begin{array}{l}
          G\textrm{ is a subgroup of }\idxTms{\Delta}{\mathsf{unit}} \\
          h : G \to \mathbb{Q}^{\not=0} \\
          h\textrm{ is a group homomorphism}
        \end{array}
      \right\}
    \end{array}
  \end{displaymath}
  To complete the step in the proof where the comparison operation was
  used, we use the division operation to show that if $(q,q') \in
  \rho_{\tyPrimNm{num}}(e)$ for some $e$, and $q \not= 0$, then $q'
  \not= 0$ (but they need not necessarily have the same sign).

  Just as Kennedy's proof generalises to an arbitrary ordered field,
  if we remove the comparison operation we can generalise to an
  arbitrary field as the underlying semantics of $\tyPrimNm{num}$ and
  $\tyPrimNm{rational}$.
\end{itemize}

\subsection{Euclidean Plane Geometry}\label{sec:2d-semantics}

We set $\tyPrimSem{vec} = \mathbb{R}^2$, and $\tyPrimSem{real} =
\mathbb{R}$. The semantic interpretation $\eta_{\mathit{2D}}$ of the
operations in $\Gamma_{\mathit{2D}}$ consists of the standard vector
zero, addition, negation and scaling in $\mathbb{R}^2$, along with
projections to interpret $\mathit{getX}$ and $\mathit{getY}$, and
pairing to interpret $\mathit{vec}$.

Let
\begin{displaymath}
  \begin{array}{l}
    \relEnv{E}_{\mathit{2D}}(\Delta) = \\
    \quad\left\{ \rho^{G,h} \sepbar
      \begin{array}{l}
        G\textrm{ is a subgroup of }\idxTms{\Delta}{\mathsf{translation}} \\
        h : G \to \mathbb{R}^2 \\
        h\textrm{ is a group homomorphism}
      \end{array}
    \right\}
  \end{array}
\end{displaymath}
where
\begin{displaymath}
  \rho^{G,h}_{\tyPrimNm{vec}}(t) = \left\{
    \begin{array}{ll}
      \{\} & \textrm{if }t \not\in G \\
      \{(v,h(t) + v) \sepbar v \in \mathbb{R}^2 \} & \textrm{if }t \in G
    \end{array}
  \right.
\end{displaymath}
Note that, unlike the units of measure example above, the pair $(0,0)$
need not appear in every relation. This is because we do not have a
translation polymorphic constant like $0 : \forall u. \tyPrim{num}{u}$
in the units of measure example.

\begin{theorem}
  The family of sets of relational environments
  $\relEnv{E}_{\mathit{units}}$ satisfies the completeness property
  \hyperref[eq:kennedy-completeness]{(\ref*{eq:kennedy-completeness})}
  above. That is, for all $\relEnv{E}'$:
  \begin{displaymath}
    (\forall \Delta.\forall \rho \in \relEnv{E}'(\Delta). (\eta_{\mathit{2D}}, \eta_{\mathit{2D}}) \in \rsem{\Gamma_{\mathit{2D}}}{\relEnv{E}'}{\rho}) \Leftrightarrow \relEnv{E}'\subseteq \relEnv{E}_{\mathit{2D}}
  \end{displaymath}
\end{theorem}

\begin{proof}
  From right to left, we must check that each operation satisfies the
  relational interpretation of its type. This is very similar to the
  case for the units of measure example above.

  From left to right, we have the following properties of any
  relational environment $\rho_{\tyPrimNm{vec}} \in
  \relEnv{E}'(\Delta)$, derived from the hypothesis that all the
  operations in $\eta_{\mathit{2D}}$ satisfy their types'
  relational semantics.
  \begin{enumerate}
  \item\label{item:obsv-vec-poly} By the existence of the functions
    $\mathit{getX}$ and $\mathit{getY}$, the relation
    $\rho_{\tyPrimNm{vec}}(0)$ is the identity relation.
  \item \label{item:zero-vec-poly} $(0,0) \in \rho_{\tyPrimNm{vec}}(0)$.
  \item \label{item:add-vec-poly} For any translation expressions $t_1,t_2$,
    such that $(v_1,v'_1) \in \rho_{\tyPrimNm{vec}}(t_1)$ and
    $(v_2,v'_2) \in \rho_{\tyPrimNm{vec}}(t_2)$, we have $(v_1 + v_2,
    v'_1 + v'_2) \in \rho_{\tyPrimNm{vec}}(t_1 + t_2)$.
  \item\label{item:neg-vec-poly} For any translation expression $t$,
    such that $(v,v') \in \rho_{\tyPrimNm{vec}}(t)$, we have that
    $(-v,-v') \in \rho_{\tyPrimNm{vec}}(-t)$.
  \end{enumerate}
  We now show that if $(v,v') \in \rho_{\tyPrimNm{vec}}(t)$ for some
  translation expression $t$, then $v' = w + v$ for some $w \in
  \mathbb{R}^2$, where $w$ is determined by $t$. If we are given
  $(v,v') \in \rho_{\tyPrimNm{vec}}(t)$, then we can set $w = v' -
  v$. This is uniquely determined by $t$: if we have $(v, w + v) \in
  \rho_{\tyPrimNm{vec}}(t)$ and $(v, w' + v) \in
  \rho_{\tyPrimNm{vec}}(t)$, then by \ref{item:add-vec-poly} and
  \ref{item:neg-vec-poly}, we have $(v - v, (w + v) - (w' + v)) \in
  \rho_{\tyPrimNm{vec}}(0)$. By \ref{item:obsv-vec-poly}, we know that
  $\rho_{\tyPrimNm{vec}}(0)$ is the identity relation, so we can
  deduce that $w = w'$.

  Now define $G \subseteq \idxTms{\Delta}{\mathsf{translation}}$ to
  have $t \in G$ if $\rho_{\tyPrimNm{vec}}(t) \not= \{\}$. Since the
  translation factor $w$ is determined by $t$, we can define a
  function $h : G \to \mathbb{R}^2$. By \ref{item:zero-vec-poly} and
  \ref{item:add-vec-poly}, $h$ is a group homomorphism, and $(v,v')
  \in \rho_{\tyPrimNm{vec}}$ iff $v = h(t) + v'$. Thus
  $\rho_{\tyPrimNm{vec}}$ is of the form
  $\rho^{G,h}_{\tyPrimNm{vec}}$, as required.
\end{proof}

Some more notes:
\begin{enumerate}
\item The proof is slightly simpler than the case for the units of
  measure example above because the underlying semantics of
  $\tyPrimNm{vec}$ forms an abelian group, so we can just use inverses
  to determine the difference factor $w$, and we don't have to deal
  with the possibility of division by zero.
\item The proof above should generalise to any abelian group.
\item Generalising to non-abelian groups does not look
  straightforward: the step that shows that the translation factor $w$
  is uniquely determined by the translation expression $t$ fails. We
  end up with
  \begin{displaymath}
    0 = w + v + (-w') + (-v)
  \end{displaymath}
  from which it is not possible to conclude that $w = w'$. (Maybe
  something using normal subgroups would work?)
\item A slightly different example is obtained by considering
  rotations rather than translations (we consider rotations as an
  additive abelian group). In this case, we could have the following
  typed operations:
  \begin{eqnarray*}
    0 & : & \forall r.\ \tyPrim{vec}{r} \\
    + & : & \forall r.\ \tyPrim{vec}{r} \to \tyPrim{vec}{r} \to \tyPrim{vec}{r} \\
    - & : & \forall r.\ \tyPrim{vec}{r} \to \tyPrim{vec}{r} \\
    * & : & \forall r.\ \tyPrimNm{real} \to \tyPrim{vec}{r} \to \tyPrim{vec}{r} \\
    \mathit{norm} & : & \forall r. \tyPrim{vec}{r} \to \tyPrimNm{real} \\
    \mathit{angleOf} & : & \forall r. \tyPrim{vec}{r} \to \tyPrim{rot}{r} + 1 \\
    \mathit{rotateBy} & : & \forall r_1 r_2. \tyPrim{vec}{r_1} \to \tyPrim{rot}{r_2} \to \tyPrim{vec}{r_1 + r_2} \\
    0_{\mathit{rot}} & : & \tyPrim{rot}{0} \\
    +_{\mathit{rot}} & : & \forall r_1 r_2. \tyPrim{rot}{r_1} \to \tyPrim{rot}{r_2} \to \tyPrim{rot}{r_1 + r_2} \\
    -_{\mathit{rot}} & : & \forall r. \tyPrim{rot}{r} \to \tyPrim{rot}{-r}
  \end{eqnarray*}
  where $\tyPrimSem{vec} = \mathbb{R}^2$, as before, and
  $\tyPrimSem{rot} = \mathrm{SO}(2)$, the group of rotations of the
  plane. The operation $\mathit{angleOf}$ is intended to return
  $\mathrm{inr}(*)$ when the provided vector is zero.

  In this case, rotations form an abelian group, so we can re-use the
  proof above to give a family of sets of relational environments that
  is complete for the group operations on rotations. The fact that
  rotations are linear admits the typing of the first four
  operations. The existence of the $\mathit{angleOf}$ and
  $\mathit{rotateBy}$ functions allow us to determine ``rotation
  factors'' between arbitrary pairs of vectors.

  I conjecture that the following definition of
  $\relEnv{E}_{\mathit{2Drot}}$ satisfies the completeness criterion,
  with the standard interpretation of the operations above.
  \begin{displaymath}
    \begin{array}{l}
      \relEnv{E}_{\mathit{2Drot}}(\Delta) = \\
      \quad\left\{ \rho^{G,h} \sepbar
        \begin{array}{l}
          G\textrm{ is a subgroup of }\idxTms{\Delta}{\mathsf{rotation}} \\
          h : G \to \mathrm{SO}(2) \\
          h\textrm{ is a group homomorphism}
        \end{array}
      \right\}
    \end{array}
  \end{displaymath}
  where
  \begin{displaymath}
    \rho^{G,h}_{\tyPrimNm{vec}}(r) = \left\{
      \begin{array}{ll}
        \{(0,0)\} & \textrm{if }t \not\in G \\
        \{(v,h(r) v) \sepbar v \in \mathbb{R}^2 \} & \textrm{if }t \in G
      \end{array}
    \right.
  \end{displaymath}
  and
  \begin{displaymath}
    \rho^{G,h}_{\tyPrimNm{rot}}(r) = \left\{
      \begin{array}{ll}
        \{\} & \textrm{if }t \not\in G \\
        \{(s,h(r) s) \sepbar s \in \mathrm{SO}(2) \} & \textrm{if }t \in G
      \end{array}
    \right.
  \end{displaymath}
\end{enumerate}

\subsection{Security Typing}\label{sec:sec-semantics}

In this case, we let $\tyPrimSem{T} = \{*\}$, and the operations
$\mathit{proj}$ and $\mathit{combine}$ are given the only possible
interpretations.

I conjecture that the follow definition works:
\begin{displaymath}
  \begin{array}{l}
    \relEnv{E}_{\mathit{sec}}(\Delta) = \\
    \quad\left\{ \rho^{h} \sepbar
      \begin{array}{l}
        h : \idxTms{\Delta}{\mathsf{principal}} \to \{\top,\bot\} \\
        h\textrm{ is a boolean algebra homomorphism}
      \end{array}
    \right\}
  \end{array}
\end{displaymath}
where
\begin{displaymath}
  \rho^{h}_{\tyPrimNm{T}}(p) = \left\{
    \begin{array}{ll}
      \{\} & \textrm{if }h(p) = \bot \\
      \{(*,*) \} & \textrm{if }h(p) = \top
    \end{array}
  \right.
\end{displaymath}

One can compare this definition to the previous ones by considering
the boolean algebra homomorphism as defining a sub-boolean algebra of
$\idxTms{\Delta}{\mathsf{principal}}$.
