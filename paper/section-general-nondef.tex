\section{Non-definability for the General Framework}
\label{sec:general-nondef}

The non-definability results in the previous section required the use
of specially crafted models and relational interpretations. It is
reasonable to ask whether or not there is a general method for
constructing suitable models and relational interpretations to prove
non-definability results. In this section, we show that this is
possible for a large class of first-order types in any instance of our
general framework.

In \exref{ex:sqrt-root-nondef}, we showed that the type $\forall s
\mathord: \SynGL{1}.~\tyPrim{real}{s^2} \to \tyPrim{real}{s}$ only has
trivial inhabitants. Intuitively, this is because the index of the
result type ($s$) cannot be obtained from the index of the input type
($s^2$) using the abelian group operations and axioms. This
observation can be used to give a sufficient condition for
non-inhabitation for types of the form:
\begin{displaymath}
  \forall i_1\mathord:s_1,\ldots,i_m\mathord:s_m.~\tyPrim{X}{e_1} \to \ldots \to \tyPrim{X}{e_n} \to \tyPrim{X}{e}
\end{displaymath}
Roughly speaking, if this type is inhabited, then it must be the case
that the index expression $e$ can be generated from the set of index
expressions $\{e_1,...,e_n\}$.

We assume that we are working with an instantiation of the general
framework from \autoref{sec:a-general-framework} with a closed typing
context $\Gamma_{\mathit{ops}}$ describing the types of the primitive
operations and a chosen index-erasure semantics. We will use a special
relational interpretation built from the syntax of the indexing
expressions to prove our general non-definability result. For
simplicity, we assume that there is only one primitive type,
$\tyPrimNm{X}$, but the technique we describe here extends to the
general case. We also assume that $\tySem{\tyPrimNm{X}}$ is non-empty.

To state our general non-definability condition, we need to define the
set of index expressions generated by some finite set of index
expressions. Given a set $S$ of index expressions that are all
well-indexed in some index context $\Delta$, we define
$\mathrm{Gen}_0(S)$ to be the set of expressions that are built from
the elements of $S$ and the primitive index operations. To account for
the equations between indexing terms, we close under index expression
equivalence to get the set $\Gen(S) = \{ e \sepbar \exists s.\ \exists
e' \in \mathrm{Gen}_0(S).\ \Delta \vdash e \equiv e' : s \}$. Now a
type is \emph{well-generated} if it is closed and of the form:
\begin{displaymath}
  \forall i_1\mathord:s_1,\ldots,i_m\mathord:s_m.~\tyPrim{X}{e_1} \to \ldots \to \tyPrim{X}{e_n} \to \tyPrim{X}{e}
\end{displaymath}
and $e \in \Gen(\{e_1,...,e_n\})$.
\begin{theorem}\label{thm:general-nondef}
  Assume that the members of $\Gamma_{\mathit{ops}}$ are all of
  well-generated types. If there exists an $M$ with typing:
  \begin{displaymath}
    \Gamma_{\mathit{ops}} \vdash M : \forall i_1\mathord:s_1,...,i_m\mathord:s_m.~\tyPrim{X}{e_1} \to ... \to \tyPrim{X}{e_n} \to \tyPrim{X}{e}
  \end{displaymath}
  then $e \in \Gen(\{e_1,...,e_n\})$.
\end{theorem}
This theorem is usually applied in the contrapositive: if $e \not\in
\Gen(\{e_1,...,e_n\})$ then no such $M$ can exist. Note that if
$\Gamma_{\mathit{ops}}$ contains operations corresponding to each of
the index-level operations (as, for example, in
\autoref{fig:real-ops}), then this theorems yields a characterisation
of definable terms, since the construction of $e$ from $e_1,...,e_n$
can be replicated at the term level.
\begin{proof}
  Let $\Delta = i_1\mathord:s_1,...,i_m\mathord:s_m$ be the index
  context constructed from the universally quantified type variables
  in the type of $M$.  To interpret the index expressions, we use the
  free model over the variables in $\Delta$ constructed from the
  syntax. This model assigns to each sort $s$ the set of index
  expressions $\{ e \sepbar \Delta \vdash e : s \}/\mathord\equiv$,
  quotiented by index expression equality. Index operations are
  interpreted by the corresponding syntactic operation on equivalence
  classes: $\semIndexExp{\texttt{\textup{f}}}([e_1],...,[e_n]) =
  [\texttt{\textup{f}}(e_1,...,e_n)]$.

  We take the relational interpretation of the primitive type
  $\tyPrimNm{X}$ as:
  \begin{displaymath}
    \semPrimType{\tyPrimNm{X}}(e) = \{ (x,x) \sepbar x \in \tySem{\tyPrimNm{X}} \land e \in \mathrm{Gen}(\{e_1,...,e_n\}) \}
  \end{displaymath}
  It is straightforward to check that for any index-erasure
  interpretation of the primitive operations $\eta_{\mathit{ops}} \in
  \Gamma_{\mathit{ops}}$, we have $(\eta_{\mathit{ops}},
  \eta_{\mathit{ops}}) \in \rsem{\Gamma_{\mathit{ops}}}*$ due to the
  fact that all members of $\Gamma_{\mathit{ops}}$ have well-generated
  types. Hence, by \thmref{thm:abstraction}, we know that for all
  terms (i.e. elements of the free model over $\Delta$)
  $e'_1\mathord:s_1,...,e'_m\mathord:s_m$:
  \begin{displaymath}%FIXME: check order of index environments
    \begin{array}{@{}l}
      \forall (x_1,x_1') \in \rsem{\tyPrim{X}{e_1}}(e'_1,...,e'_m), ..., \\
      \quad(x_n, x'_n) \in \rsem{\tyPrim{X}{e_n}}(e'_1,...,e'_m). \\
      \quad (\tmSem{M} \eta_{\mathit{ops}}~x_1 ~...~x_n, \tmSem{M}\eta_{\mathit{ops}}~x'_1~...~x'_n) \in \rsem{\tyPrim{X}{e}}(e'_1,...,e'_m)
    \end{array}
  \end{displaymath}
  By setting $e'_j = i_j$, and using an arbitrary element $x \in
  \tySem{\tyPrimNm{X}}$ (which we have assumed non-empty), we have,
  for all $k$, $(x,x) \in \rsem{\tyPrim{X}{e_k}}(i_1,...,i_m)$ since
  each $e_k$ is a member of the set we are using to generate
  terms. Now $(\tmSem{M} \eta_{\mathit{ops}}~x...x,
  \tmSem{M}\eta_{\mathit{ops}}~x...x) \in \rsem{\tyPrimNm{X}}(e)$ and
  so $e \in \Gen(\{e_1,...,e_n\})$.
\end{proof}

% When applying \thmref{thm:general-nondef}, it is useful to observe
% that the free model constructed from the index variables $i_1,...,i_m$
% often has a well-known structure. Moreover, the set $\Gen(S)$ is
% always a sub-model of the free model. These observations can make it
% easy to apply the theorem in specific cases.

\paragraph{Application to \exref{ex:uninhabited-type}}
\thmref{thm:general-nondef} can be directly applied to show that the
type $\forall t\mathord:\SynTransl{2}.~\tyPrim{vec}{t + t} \to
\tyPrim{vec}{t}$ has no inhabitants. The free model over the single
index variable $t$ is (isomorphic to) the integers, and the sub-model
generated by the index expression $t + t$ corresponds to the even
integers. The result now follows simply because $1$ (i.e. the
interpretation of $t$) is not an even number.

\paragraph{Abelian Group Indexed Types} Kennedy
\cite{kennedy97relational} has given a general characterisation of
definability at first-order in the case of abelian group indexing in
terms of integer solutions to a set of linear equations. Specialising
\thmref{thm:general-nondef} to the case of abelian group indexing
yields Kennedy's characterisation.

\paragraph{Polymorphic Constants} \thmref{thm:general-nondef} does not
apply in the case when we have polymorphic constants. This is the case
with the polymorphic $0 : \forall
s\mathord:\SynGL{1}.~\tyPrim{real}{s}$ in
\autoref{fig:real-ops}. \thmref{thm:general-nondef} does not apply
because the index expression $s$ is not generated by the empty set:
$0$'s type is not well-generated. Nevertheless, it is easy to adapt
the proof of \thmref{thm:general-nondef} to handle a polymorphic
constant like $0$ by setting the relational interpretation of
$\tyPrimNm{X}$ to be:
\begin{displaymath}
  \rsem{X}(e) = \{ (x,x) \sepbar x \in \tySem{\tyPrimNm{X}} \land (x = 0 \lor e \in \Gen(\{e_1,...,e_n\})) \}
\end{displaymath}
The conclusion of the theorem now states that either we have $e \in
\Gen(\{e_1,...,e_n\})$ \emph{or} $\tmSem{M}$ is the constant $0$
function. This extended theorem can now be used to give an alternative
proof for \exref{ex:sqrt-root-nondef}. As this example illustrates,
there may be many different models that can be used to prove a
non-definability result.

\paragraph{Adding Index Operations} \thmref{thm:general-nondef} also
does not directly apply in the case of \exref{ex:cube-root-nondef},
again because assumption that the types of the primitive operations
are all well-generated is not satisfied. In this case, the assumed
square root operation has type $\forall
s\mathord:\SynGL{1}.~\tyPrim{real}{s^2} \to \tyPrim{real}{s}$, and as
we observed at the start of this section, $s$ is not in the set
generated by $s^2$. However, to enable the application of the theorem,
we can assume an additional index operation $-^{1/2}$, acting like
square root at the index level. Now the free model produced in the
proof of the theorem is isomorphic to the dyadic numbers with addition
and halving, and the generated sub-model consists of the dyadic
rationals of the form $\frac{3k}{2^n}$. Again there is a diversity of
models that can be used to prove a single non-definability result.

%%% Local Variables:
%%% TeX-master: "paper"
%%% End:
