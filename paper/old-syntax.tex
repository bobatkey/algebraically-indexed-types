%%%%%%%%%%%%%%%%%%%%%%%%%%%%%%%%%%%%%%%%%%%%%%%%%%%%%%%%%%%%%%%%%%%%%%%%%%%%%%
%%%%%%%%%%%%%%%%%%%%%%%%%%%%%%%%%%%%%%%%%%%%%%%%%%%%%%%%%%%%%%%%%%%%%%%%%%%%%%
\section{Type System}

We now introduce our type system. We parameterise our development by
an arbitrary multi-sorted algebraic theory, in the sense of universal
algebra. We present the precise form of the theories we allow in
\autoref{sec:index-exp}, and define some auxiliary notions of
substitution and terms-up-to-equality. Several examples of useful
algebraic theories will be given in
\autoref{sec:example-systems}. Once an algebraic theory has been
fixed, we define the well-formed types and terms with respect to this
theory in \autoref{sec:types-and-terms}. 

Data:
\begin{enumerate}
\item A set $\mathit{Sort}$ of sorts
\item A collection of primitive index operations $\mathit{IndexOp}$
\item A map $\indexOpArity : \mathit{IndexOp} \to \mathit{Sort}^* \times \mathit{Sort}$
\item A set $\mathit{PrimType}$ of primitive types
\item A map $\primTyArity : \mathit{PrimType} \to \mathit{Sort}^*$
\end{enumerate}

Then generate:
\begin{enumerate}
\item Well-sorted index expressions
\item Well-indexed types
\end{enumerate}

More data:
\begin{enumerate}
\item A set of primitive operations $\mathit{PrimOps}$
\end{enumerate}

This generates the well-typed terms. 

For the semantics:
\begin{enumerate}
\item A ``relational'' meaning for index contexts. Can this be made
  separate? Make the point that it is non-compositional, which is
  useful. 
\item Underlying semantics for each type
\item Assume that the underlying semantics for types is not affected
  by the index environment (could it be?)
\item For each index context and primitive type, we demand a
  relational meaning for that type.
\item Relational meanings are closed under some operations.
\end{enumerate}

\subsection{Algebraic Theories for Indexing}\label{sec:index-exp}

\begin{figure*}[t]
  \centering
  Well-sorted terms
  \begin{mathpar}
    \inferrule*
    {i : s \in \Delta}
    {\Delta \vdash x : s}
    
    \inferrule*
    {\indexOp{f} : s_1,...,s_n \to s \\
      \Delta \vdash e_1 : s_1 \\
      \dots \\
      \Delta \vdash e_n : s_n}
    {\Delta \vdash \indexOp{f}(e_1, ..., e_n) : s}
  \end{mathpar}

  \medskip

  Equational rules
  \begin{mathpar}
    \inferrule*
    {e_1 \stackrel{ax}= e_2 : s \in \mathit{Ax}(\Delta)}
    {\Delta \vdash e_1 \equiv e_2 : s}

    \inferrule*
    {\indexOp{f} : s_1,...,s_n \to s \\
      \Delta \vdash e_1 \equiv e'_1 : s_1 \\
      \dots \\
      \Delta \vdash e_n \equiv e'_n : s_n}
    {\Delta \vdash \indexOp{f}(e_1, ..., e_n) \equiv \indexOp{f}(e'_1, ..., e'_n) : s}

    \inferrule*
    {\Delta \vdash e : s}
    {\Delta \vdash e \equiv e : s}

    \inferrule*
    {\Delta \vdash e_1 \equiv e_2 : s}
    {\Delta \vdash e_2 \equiv e_1 : s}

    \inferrule*
    {\Delta \vdash e_1 \equiv e_2 : s \\ \Delta \vdash e_2 \equiv e_3 : s}
    {\Delta \vdash e_1 \equiv e_3 : s}
  \end{mathpar}
  \caption{Algebraic Theories for Indexing}
  \label{fig:alg-theories}
\end{figure*}

We assume a set $\mathit{Sort}$ of \emph{index sorts}, and we will use
$s, s_1, s_2, ...$ to range over elements of $\mathit{Sort}$. An index
context $\Delta$ is list of index variable and sort pairs, $i : s$,
and we use the symbol $\epsilon$ to denote the empty index context (we
will also use this symbol below to denote the empty typing
context). We use letters like $i$ and $j$, sometimes with subscripts,
to range over index variables. We also assume a collection
$\mathit{IndexOp}$ of sorted index operations $\indexOp{f} :
s_1,...,s_n \to s$. From the index operations we generate the
well-sorted index expressions by the rules in
\autoref{fig:alg-theories}.

For each index context $\Delta$, we assume a set $\mathit{Ax}(\Delta)$
of axioms $e_1 \stackrel{ax}= e_2 : s$ where both $\Delta \vdash e_1 :
s$ and $\Delta \vdash e_2 : s$. From this collection of axioms, we
generate the equational theory of index expressions by the rules in
\autoref{fig:alg-theories} by closing under congruence, reflexivity,
symmetry and transitivity.

% For a pair of index contexts $\Delta$ and $\Delta'$, a substitution
% $\sigma \in \subst{\Delta}{\Delta'}$ is a function $\sigma$ mapping
% variables in $\Delta$ to well-sorted index expressions in $\Delta'$.

Index context morphisms $\sigma : \Delta \Rightarrow \Delta'$ consist
of a sequence of terms, one for each element of $\Delta'$, all
well-sorted in the index context $\Delta$.
\begin{mathpar}
  \inferrule*
  { }
  {\cdot : \Delta \Rightarrow \epsilon}

  \inferrule*
  {\sigma : \Delta \Rightarrow \Delta' \\ \Delta \vdash e : s}
  {\sigma, e : \Delta \Rightarrow \Delta', i:s}
\end{mathpar}
Directly from the definition of index context morphisms, we can see
that given an index context morphism $\sigma : \Delta \Rightarrow
\Delta'$, then for any index variable $i : s \in \Delta'$ there is an
index expression $\sigma(i)$ such that $\Delta \vdash \sigma(i) : s$.

We write $\id_\Delta : \Delta \Rightarrow \Delta$ for the index
context morphism comprised of the terms built from the variables of
$\Delta$, in order. We write $\pi_{i\mathord:s} : \Delta, i : s
\Rightarrow \Delta$ for the context morphism again comprised of the
terms built from the variables of $\Delta$, but this time typed in a
larger context. For an index context morphism $\sigma : \Delta \to
\Delta'$ we write $\sigma_s : \Delta, i\mathord:s \Rightarrow \Delta',
i\mathord:s$ for the index context morphism generated by lifting each
term to the larger context $\Delta, i\mathord:s$, and mapping the
variable $i\mathord:s$ to the term $i$.

Given an index expression $\Delta' \vdash e : s$, we define
re-indexing by an index context morphism $\sigma : \Delta \Rightarrow
\Delta'$ by the following clauses:
\begin{eqnarray*}
  \sigma^*i & = & \sigma(i) \\
  \sigma^*\indexOp{f}(e_1, ..., e_n) & = & \indexOp{f}(\sigma^*e_1,...,\sigma^*e_n)
\end{eqnarray*}
It is easy to check that these clauses preserve well-formedness of
index expressions, so that we obtain $\Delta \vdash \sigma^*e :
s$. Moreover re-indexing by index context morphisms preserves
equality:
\begin{lemma}\label{lem:indexeq-subst}
  If $\Delta \vdash e_1 \equiv e_2 : s$, then for any $\sigma :
  \Delta' \Rightarrow \Delta$, we have $\Delta' \vdash \sigma^*e_1
  \equiv \sigma^*e_2 : s$.
\end{lemma}

Also, projection commutes with lifting of substitutions. This is used
in the proof of \lemref{lem:tysubst-rel}.
\begin{lemma}
  $\sigma^* \circ \pi^*_{i\mathord:s} = \pi^*_{i\mathord:s} \circ \sigma^*_s$
\end{lemma}

For an index context $\Delta$ and sort $s$, we write
$\idxTms{\Delta}{s}$ for the set of index expressions $e$ such that
$\Delta \vdash e : s$, quotiented by the equivalence relation $\Delta
\vdash - \equiv - : s$. In light of \lemref{lem:indexeq-subst}, we can
see that re-indexing by an index context morphism $\sigma : \Delta
\Rightarrow \Delta'$ can be used as a function $\sigma^* :
\idxTms{\Delta'}{s} \to \idxTms{\Delta}{s}$, for any sort $s$.

\subsection{Types and Terms}\label{sec:types-and-terms}

\begin{figure*}[t]
  \centering
  Well-indexed Types

  \medskip
  Equality of well-indexed types
  \begin{mathpar}
    \inferrule* [right=TyEqPrim]
    {\primTyArity(\tyPrimNm{X}) = (s_1,...,s_n) \\
      \{\Delta \vdash e_i \equiv e'_i : s_i\}_{1\leq i \leq n}}
    {\Delta \vdash \tyPrim{X}{e_1,...,e_n} \equiv \tyPrim{X}{e'_1,..,e'_n} : \sortType}

    \inferrule* [right=TyEqUnit]
    { }
    {\Delta \vdash \tyUnit \equiv \tyUnit : \sortType}

    \inferrule* [right=TyEqArr]
    {\Delta \vdash A \equiv A' : \sortType \\ \Delta \vdash B \equiv B' : \sortType}
    {\Delta \vdash A \tyArr B \equiv A' \tyArr B' : \sortType}

    \inferrule* [right=TyEqProd]
    {\Delta \vdash A \equiv A' : \sortType \\ \Delta \vdash B \equiv B' : \sortType}
    {\Delta \vdash A \tyProduct B \equiv A' \tyProduct B' : \sortType}

    \inferrule* [right=TyEqSum]
    {\Delta \vdash A \equiv A' : \sortType \\ \Delta \vdash B \equiv B' : \sortType}
    {\Delta \vdash A + B \equiv A' + B' : \sortType}

    \inferrule* [right=TyEqForall]
    {\Delta, i : s \vdash A \equiv A' : \sortType}
    {\Delta \vdash \forall i \mathord: s. A \equiv \forall i \mathord: s. A' : \sortType}
  \end{mathpar}

  \medskip
  Well-indexed contexts
  \begin{mathpar}
    \inferrule*
    { }
    {\Delta \vdash \epsilon \isCtxt}

    \inferrule*
    {\Delta \vdash \Gamma \isCtxt \\ \Delta \vdash A : \sortType \\ x \not\in \Gamma}
    {\Delta \vdash \Gamma, x : A \isCtxt}
  \end{mathpar}
  \caption{Well-indexed Types and Contexts}
  \label{fig:types}
\end{figure*}

The rules for building the types of a system in our generic framework
are given in \autoref{fig:types}. The types of our system are built
from a set of primitive types $\mathit{PrimType}$. 

For each sort $s \in \mathit{Sort}$, we assume we are given a set of
names of primitive types $\mathit{PrimType}(s)$. We assume that the
sets $\mathit{PrimType}(s)$ are disjoint for non-equal sorts $s_1,
s_2$. For example, in the units of measure example in
\autoref{sec:units-example}, there is a single sort $\mathsf{unit}$,
and $\mathit{PrimType}(\mathsf{unit}) = \{\tyPrimNm{num}\}$. For each
primitive type $\tyPrimNm{X} \in \mathit{PrimType}(s)$, and index
expression $\Delta \vdash e : s$, we form the type $\tyPrim{X}{e}$, by
the \TirName{TyPrimIdx} rule in \autoref{fig:types}. Thus, in the
units-of-measure example, the type $\tyPrim{num}{u}$ represents
numbers measured by the units $u$.

We also assume that there is an additional set of primitive types
$\mathit{PrimType}(\cdot)$ that is disjoint from all the other sets of
primitive types. These primitive types do not need to be paired with
an index expression to form well-indexed types. Such types are created
by the \TirName{TyPrimNoIdx} rule. An example of such a primitive type
is the type $\tyPrimNm{real}$ of real-valued scalars in the Euclidean
plane geometry example in \autoref{sec:2d-example}.

The rules for generating the rest of the types of the system,
including unit, product, sum, function and universal types, are given
in \autoref{fig:types}. We use capital roman letters like $A$ and $B$
to range over well-indexed types. Well-indexed types come with an
equational theory, derived from the equational theory on index
expressions. This equational theory is also presented in
\autoref{fig:types}.

We define substitution by an index context morphism $\sigma : \Delta
\Rightarrow \Delta'$ on types by structural induction on the structure
of the type: given a type $\Delta' \vdash A : \sortType$, we obtain a
type $\Delta \vdash \sigma^*A : \sortType$. The only interesting case
is for \TirName{TyPrimIdx}:
\begin{displaymath}
  \sigma^*\tyPrim{X}{e} = \tyPrim{X}{\sigma^*e}
\end{displaymath}

Well-formed contexts are defined to be lists of variables paired with
well-formed types with respect to some index context $\Delta$, with
the condition that no variable names are repeated. Formally, contexts
are defined by the two rules shown in
\autoref{fig:types}. Substitution by an index context morphism extends
to contexts just by doing substitution on each individual type.

The typing rules for our system define the well-typed terms with
respect to an index context $\Delta$ and a type context $\Delta \vdash
\Gamma \isCtxt$. The judgement $\Delta; \Gamma \vdash M : A$ is
defined in \autoref{fig:terms}. The equational theory on types is
incorporated into the type system via the rule \TirName{TyEq}, which
allows for a term that is judged to have type $A$ to also have any
other equal type $B$.
\begin{figure*}[t]
  \centering
  \begin{mathpar}
    \inferrule* [right=Var]
    {\Delta \vdash \Gamma \isCtxt \\ x : A \in \Gamma}
    {\Delta; \Gamma \vdash x : A}

    \inferrule* [right=TyEq]
    {\Delta; \Gamma \vdash M : A \\ \Delta \vdash A \equiv B : \sortType}
    {\Delta; \Gamma \vdash M : B}

    \inferrule* [right=Unit]
    { }
    {\Delta; \Gamma \vdash * : 1}

    \inferrule* [right=Pair]
    {\Delta; \Gamma \vdash M : A \\
      \Delta; \Gamma \vdash N : B}
    {\Delta; \Gamma \vdash (M, N) : A \tyProduct B}

    \inferrule* [right=Proj1]
    {\Delta; \Gamma \vdash M : A \tyProduct B}
    {\Delta; \Gamma \vdash \pi_1 M : A}

    \inferrule* [right=Proj2]
    {\Delta; \Gamma \vdash M : A \tyProduct B}
    {\Delta; \Gamma \vdash \pi_2 M : B}

    \inferrule* [right=Inl]
    {\Delta; \Gamma \vdash M : A}
    {\Delta; \Gamma \vdash \mathrm{inl}\ M : A + B}

    \inferrule* [right=Inr]
    {\Delta; \Gamma \vdash M : B}
    {\Delta; \Gamma \vdash \mathrm{inr}\ M : A + B}

    \inferrule* [right=Cond]
    {\Delta; \Gamma \vdash M : A + B \\
      \Delta; \Gamma, x : A \vdash N_1 : C \\
      \Delta; \Gamma, y : B \vdash N_2 : C}
    {\Delta; \Gamma \vdash \textrm{case}\ M\ \textrm{of}\ \textrm{inl}\ x.N_1; \textrm{inr}\ y.N_2 : C}

    \inferrule* [right=Abs]
    {\Delta; \Gamma, x : A \vdash M : B}
    {\Delta; \Gamma \vdash \lambda x.M : A \tyArr B}

    \inferrule* [right=App]
    {\Delta; \Gamma \vdash M : A \tyArr B \\
      \Delta; \Gamma \vdash N : A}
    {\Delta; \Gamma \vdash M N : B}

    \inferrule* [right=UnivAbs]
    {\Delta, i : s; \pi_{i\mathord:s}^*\Gamma \vdash M : A}
    {\Delta; \Gamma \vdash \Lambda i. M : \forall i\mathord:s. A}

    \inferrule* [right=UnivApp]
    {\Delta; \Gamma \vdash M : \forall i\mathord:s. A \\ \Delta \vdash e : s}
    {\Delta; \Gamma \vdash M [e] : (\id_\Delta, e)^*A}
  \end{mathpar}
  
  \caption{Well-typed terms}
  \label{fig:terms}
\end{figure*}
