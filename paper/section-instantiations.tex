\section{Geometric Consequences of Abstraction}
\label{sec:instantiations}

\fixme{explain the general setup, with a set of operations for each
  instantiation.}

\begin{example}[Two-dimensional Geometry] We define the ``standard''
  model of the indexing theory for the two-dimensional geometry
  example as follows. Each of the sorts are interpreted just as their
  semantic counterparts:
  \begin{displaymath}
    \begin{array}{@{}llll}
      \semSort{\SynTransl{2}} = \Transl{2}
      &
      \semSort{\SynGL{2}} = \GL{2}
      &
      \semSort{\SynOrth{2}} = \Orth{2}
      &
      \semSort{\SynGL{1}} = \GL{1}
    \end{array}
  \end{displaymath}
  Each of the index operations (e.g.,~the group structure and
  determinant) are interpreted by the intended semantic
  operations. These clearly satisfy the axioms we stated in
  \exref{ex:two-dim-geo-axioms}. The assignment of relations for the
  primitive types $\tyPrimNm{vec}$ and $\tyPrimNm{real}$ are given by
  the following definitions:
  \begin{displaymath}
    \begin{array}{ll}
      R^{\bullet}_{\tyPrimNm{vec}} = \{( \vec{0}, \vec{0} ) \}&
      R_{\tyPrimNm{vec}}(B,\vec{t}) = \{ (\vec{v}, B\vec{v} + \vec{t}) \sepbar \vec{v} \in \mathbb{R}^2 \} \\
      R^{\bullet}_{\tyPrimNm{real}} = \{(0,0)\} &
      R_{\tyPrimNm{real}}(k) = \{ (x, kx) \sepbar x \in \mathbb{R} \}
    \end{array}
  \end{displaymath}
  The definitions of $R_{\tyPrimNm{vec}}$ and $R_{\tyPrimNm{real}}$
  are as may be expected given the informal definitions we gave in
  \autoref{sec:motivating-examples}. We have set
  $R^\bullet_{\tyPrimNm{vec}}$ and $R^\bullet_{\tyPrimNm{real}}$ to
  relate $\vec{0}$ and $0$ to themselves to account for the
  polymorphic zero constants we have assumed.
\end{example}

\begin{lemma}
  The thing about $\Gamma_\Geom$.
\end{lemma}

\subsection{Theorems for Free}
\label{sec:theorems-for-free}

Consider the type of the area of a triangle function we defined in
\exref{ex:area-of-triangle-1}:
\begin{displaymath}
  \begin{array}{@{}l}
    \mathrm{area} : \forall B\mathord:\SynGL{2}, t\mathord:\SynTransl{2}.\ \\
    \hspace{0.8cm}\tyPrim{vec}{B, t} \to \tyPrim{vec}{B, t} \to \tyPrim{vec}{B, t} \to \tyPrim{real}{|\det B|}
  \end{array}
\end{displaymath}
By \thmref{thm:abstraction}, we can derive the following free theorem. For all $B \in \GL{2}$, $\vec{t} \in \Transl{2}$, and $\vec{x}, \vec{y}, \vec{z} \in \mathbb{R}^2$, we have 
\begin{displaymath}
  |\det B|(\tmSem{\mathrm{area}}\ \vec{x}\ \vec{y}\ \vec{z}) = \tmSem{\mathrm{area}}\ (B\vec{x} + \vec{t})\ (B\vec{y} + \vec{t})\ (B\vec{z} + \vec{t})
\end{displaymath}
Thus, directly from the type of the $\mathrm{area}$ function, we can
see that its index-erasure semantics is (a) invariant under
translations, and (b) if the inputs are subjected to a linear
transformation $B$, the output varies with the absolute value of the
determinant of $B$.



% Wadler's influential ``Theorems for Free!''  \cite{wadler89theorems}
% emphasised a particular aspect of Reynolds' theory of relational
% parametricity: the fact that we can read off theorems about a program
% simply by looking at the relational interpretation of its type. We now
% state some free theorems that are derivable in each of the three main
% examples we introduced in \autoref{sec:motivating-examples}.

\subsection{Isomorphisms for Types Indexed by Abelian Groups}
\label{sec:types-indexed-abelian-groups}

\begin{enumerate}
\item Two cases: when we have a total ``division'' operation, and when
  we don't.
\item Type isomorphisms, based on Smith Normal Form
% \item Non-definability results, based on the non-compositional
%   relational environments
% \item Alteration of the non-definability results in the light of
%   addition of a square root operation
% \item Mention the variation we can obtain by changing some types,
%   e.g. the type of $\mathrm{abs}$. Also, the complex numbers example.
\end{enumerate}

\fixme{Copy the stuff over from the other file}

\subsection{Indefinability}

\fixme{Copy the stuff over from the other file}

%%% Local Variables:
%%% TeX-master: "paper"
%%% End:
