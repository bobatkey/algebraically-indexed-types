\section{Consequences of the Abstraction Theorem}
\label{sec:instantiations}

\fixme{explain the general setup, with a set of operations for each
  instantiation.}

\subsection{Theorems for Free}
\label{sec:theorems-for-free}

Wadler's influential ``Theorems for Free!''  \cite{wadler89theorems}
emphasised a particular aspect of Reynolds' theory of relational
parametricity: the fact that we can read off theorems about a program
simply by looking at the relational interpretation of its type. We now
state some free theorems that are derivable in each of the three main
examples we introduced in \autoref{sec:motivating-examples}.

\subsubsection{Two-Dimensional Geometry}
\label{sec:geometry-consequences}

(In)variance properties

\subsubsection{Logical Information Flow}
\label{sec:information-flow-consequences}

Non-interference properties

\subsubsection{Metric Spaces}
\label{sec:metric-space-consequences}

Continuity via $\epsilon$-$\delta$ proofs.

\subsection{Types Indexed by Abelian Groups}
\label{sec:types-indexed-abelian-groups}

\begin{enumerate}
\item Two cases: when we have a total ``division'' operation, and when
  we don't.
\item Type isomorphisms, based on Smith Normal Form
\item Non-definability results, based on the non-compositional
  relational environments
\item Alteration of the non-definability results in the light of
  addition of a square root operation
\item Mention the variation we can obtain by changing some types,
  e.g. the type of $\mathrm{abs}$. Also, the complex numbers example.
\end{enumerate}




%%% Local Variables:
%%% TeX-master: "paper"
%%% End:
