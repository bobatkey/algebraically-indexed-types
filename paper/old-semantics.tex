%%%%%%%%%%%%%%%%%%%%%%%%%%%%%%%%%%%%%%%%%%%%%%%%%%%%%%%%%%%%%%%%%%%%%%%%%%%%%%
%%%%%%%%%%%%%%%%%%%%%%%%%%%%%%%%%%%%%%%%%%%%%%%%%%%%%%%%%%%%%%%%%%%%%%%%%%%%%%
%%%%%%%%%%%%%%%%%%%%%%%%%%%%%%%%%%%%%%%%%%%%%%%%%%%%%%%%%%%%%%%%%%%%%%%%%%%%%%
\section{Semantics and Parametricity Theorem}

There is an index-erasure semantics, and a relational semantics. The
relational semantics allows us to derive free theorems about the
index-erasure semantics.

\subsection{Index Erasure Semantics}

We now define a semantics of types and terms that ignores all indexing
information. By defining this semantics we can easily see that the
indexing information can be discarded before execution.

\subsubsection{Index-Erasure Semantics of Types}

For each primitive type $\tyPrimNm{X} \in \mathit{PrimType}(s)$, for
some sort $s$, and for the primitive types $\tyPrimNm{X} \in
\mathit{PrimType}(\cdot)$, we assume that $\tyPrimNm{X}$ is assigned a
set $\tyPrimSem{X}$. We extend this to assign a set to every
well-formed type by induction on the type structure. To interpret the
universal quantifier, we simply ignore the indexing:
\begin{eqnarray*}
  \tySem{\tyUnit} & = & \{*\} \\
  \tySem{\tyPrim{X}{e}} & = & \tyPrimSem{X} \\
  \tySem{\tyPrim{X}} & = & \tyPrimSem{X} \\
  \tySem{A \tyProduct B} & = & \tySem{A} \times \tySem{B} \\
  \tySem{A \tyArr B} & = & \tySem{A} \to \tySem{B} \\
  \tySem{A + B} & = & \tySem{A} + \tySem{B} \\
  \tySem{\forall i\mathord:s. A} & = & \tySem{A}
\end{eqnarray*}

Since our definition has completely ignored the indexing, and type
equality is defined as an extension of index equality, we can easily
see that equal types have equal denotations in our index-erasure
semantics:
\begin{lemma}\label{lem:tyeq-erasure}
  If $\Delta \vdash A \equiv B : \sortType$ then $\tySem{A} =
  \tySem{B}$.
\end{lemma}

Moreover, the index-erasure semantics is insensitive to index
substitutions.
\begin{lemma}\label{lem:tysubst-erasure}
  If $\Delta \vdash A : \sortType$ and $\sigma : \Delta' \Rightarrow
  \Delta$, then $\tySem{\sigma^*A} = \tySem{A}$.
\end{lemma}

For any typing context $\Delta \vdash \Gamma \isCtxt$, we define its
index-erasure semantics $\ctxtSem{\Gamma}$ by induction on its
structure: $\ctxtSem{\epsilon} = 1$ and $\ctxtSem{\Gamma, x : A} =
\ctxtSem{\Gamma} \times \tySem{A}$, where $1$ is the canonical
one-element set.

\subsubsection{Index-Erasure Semantics of Terms}

For a well-formed term $\Delta; \Gamma \vdash e : A$, we define a
function $\tmSem{e} : \ctxtSem{\Gamma} \to \tySem{A}$, completely
ignoring the indexing information. In light of
\lemref{lem:tyeq-erasure} and \lemref{lem:tysubst-erasure}, we may
define this function directly on the syntax of well-typed terms,
rather than on the typing derivations. We define $\tmSem{e}$ by the
following clauses:
\begin{eqnarray*}
  \tmSem{x}\eta & = & \eta_x \\
  \tmSem{(M,N)}\eta & = & (\tmSem{M}\eta, \tmSem{N}\eta) \\
  \tmSem{\pi_1M}\eta & = & \pi_1(\tmSem{M}\eta) \\
  \tmSem{\pi_2M}\eta & = & \pi_2(\tmSem{M}\eta) \\
  \tmSem{\mathrm{inl}\ M}\eta & = & \mathrm{inl}\ (\tmSem{M}\eta) \\
  \tmSem{\mathrm{inr}\ M}\eta & = & \mathrm{inr}\ (\tmSem{M}\eta) \\
  \left\llbracket
    \begin{array}{l}
      \textrm{case}\ M\ \textrm{of}\\
      \textrm{inl}\ x.N_1;\\
      \textrm{inr}\ y.N_2
    \end{array}\right\rrbracket^{\mathit{tm}}\eta & = &
  \left\{
    \begin{array}{ll}
      \tmSem{N_1}(\eta,a) & \textrm{if }\tmSem{M}\eta = \mathrm{inl}(a) \\
      \tmSem{N_2}(\eta,b) & \textrm{if }\tmSem{M}\eta = \mathrm{inr}(b)
    \end{array}
  \right. \\
  \tmSem{\lambda x. M}\eta & = & \lambda v. \tmSem{M}(\eta, v) \\
  \tmSem{M N}\eta & = & (\tmSem{M}\eta) (\tmSem{N}\eta) \\
  \tmSem{\Lambda i.\ M}\eta & = & \tmSem{M}\eta \\
  \tmSem{M[e]}\eta & = & \tmSem{M}\eta
\end{eqnarray*}
In the above, we have written $\eta_x$ to denote the appropriate
projection from $\eta$ to get the position corresponding to the
variable $x$ in the context.

\subsection{Relational Semantics}

\subsubsection{Relation Environments}

For a set $X$, let $\Rel(X)$ be the set of binary relations on
$X$. For an index context $\Delta$, a \emph{relation environment} is a
family of functions $\rho = \{ \rho_{\tyPrimNm{X}} :
\idxTms{\Delta}{s} \to \Rel(\tyPrimSem{X}) \}_{s \in \mathit{Sort},
  \tyPrimNm{X} \in \mathit{PrimType}(s)}$.

We define our relational semantics of types with respect to a given
collection of relation environments: we assume that, for each index
context $\Delta$, we are given a set of relation environments
$\relEnv{E}(\Delta)$. This family must satisfy the following two
properties:
\begin{enumerate}
\item Closure under re-indexing by index context morphisms: if $\rho
  \in \relEnv{E}(\Delta)$ and $\sigma : \Delta' \Rightarrow \Delta$,
  then $\rho \circ \sigma^* \in \relEnv{E}(\Delta')$.
\item If we have a pair of relation environments $\rho_1 \in
  \relEnv{E}(\Delta', i\mathord:s')$ and $\rho_2 \in
  \relEnv{E}(\Delta)$, along with an index context morphism $\sigma :
  \Delta' \Rightarrow \Delta$, such that the outer edges of the
  following diagram commute, for all primitive types $\tyPrimNm{X} \in
  \mathit{PrimType}$:
  \begin{displaymath}
    \xymatrix{
      {\idxTms{\Delta}{s}} \ar[r]^(.45){\pi^*_{i\mathord:s'}} \ar[d]_{\sigma^*}
      &
      {\idxTms{\Delta,i\mathord:s'}{s}} \ar[d]^{\sigma_{s'}^*} \ar@/^/[rdd]^{\rho_1}
      \\
      {\idxTms{\Delta'}{s}} \ar[r]^(.45){\pi^*_{i\mathord:s'}} \ar@/_/[rrd]_{\rho_2}
      &
      {\idxTms{\Delta',i\mathord:s'}{s}} \ar@{.>}[dr]_\rho
      \\
      &
      &
      {\Rel(\tyPrimSem{X})}
    }
  \end{displaymath}
  Then there exists a relation environment $\rho \in
  \relEnv{E}(\Delta,i\mathord:s')$ such that the two triangles in the
  bottom right of the diagram commute.
\end{enumerate}

Given a relation environment $\rho \in \relEnv{E}(\Delta)$, we define
the set of extensions $\extends{\rho}{i\mathord:s}$ of $\rho$ by an
additional index variable $i\mathord:s$ to be the following:
\begin{displaymath}
  \extends{\rho}{i\mathord:s} = \{ \rho' \in \relEnv{E}(\Delta,i\mathord:s) \sepbar \rho' \circ \pi^*_{i\mathord:s} = \rho \}
\end{displaymath}

\subsubsection{Relational Semantics of Types}

%FIXME: re-word this
A given relation context $\rho \in \relEnv{E}(\Delta)$ is extended to
a relation interpretation of any type $\Delta \vdash A : \sortType$ by
induction its derivation. Thus, we define a relation
$\rsem{A}{\relEnv{E}}{\rho} \in \Rel(\tySem{A})$ for each well-indexed
type.
\begin{eqnarray*}
  \rsem{\tyUnit}{\relEnv{E}}\rho & = & \{(*,*)\} \\
  \rsem{\tyPrim{X}{e_1,...,e_n}}{\relEnv{E}}\rho & = & \rho\ {\tyPrimNm{X}}\ (e_1,...,e_n) \\
  \rsem{A \tyArr B}{\relEnv{E}}\rho & = & \rsem{A}{\relEnv{E}}\rho \relArrow \rsem{B}{\relEnv{E}}\rho \\
  \rsem{A \tyProduct B}{\relEnv{E}}\rho & = & \rsem{A}{\relEnv{E}}\rho \relTimes \rsem{B}{\relEnv{E}}\rho \\
  \rsem{A + B}{\relEnv{E}}\rho & = & \rsem{A}{\relEnv{E}}\rho \relSum \rsem{B}{\relEnv{E}}\rho \\
  \rsem{\forall i\mathord:s.A}{\relEnv{E}}\rho & = & \bigcap\{ \rsem{A}{\relEnv{E}}\rho' \sepbar \rho' \in \extends{\rho}{i:s} \}
\end{eqnarray*}
In this definition we have made use of the following three
constructions on binary relations. If $R \in \Rel(X)$ and $S \in
\Rel(Y)$, then $R \relArrow S \in \Rel(X \to Y)$ is defined as $\{
(f_1,f_2) \sepbar \forall (a_1,a_2) \in R.\ (f_1a_1,f_2a_2) \in S
\}$. With the same assumptions on $R$ and $S$, the relation $R
\relTimes S \in \Rel(X \times Y)$ is defined as $\{
((a_1,b_1),(a_2,b_2)) \sepbar (a_1,a_2) \in R \land (b_1,b_2) \in S
\}$. Finally, the relation $R \relSum S \in \Rel(X + Y)$ is defined as
$\{ (\mathrm{inl}\ x, \mathrm{inl}\ x') \sepbar (x,x') \in R \} \cup
\{ (\mathrm{inr}\ y, \mathrm{inr}\ y') \sepbar (y,y') \in S \}$.

We extend the relational semantics to contexts of well-formed types by
the following clauses:
\begin{eqnarray*}
  \rsem{\epsilon}{\relEnv{E}}{\rho} & = & \id_{\ctxtSem{\epsilon}} \\
  \rsem{\Gamma, x : A}{\relEnv{E}}{\rho} & = & \rsem{\Gamma}{\relEnv{E}}\rho \relTimes \rsem{A}{\relEnv{E}}\rho
\end{eqnarray*}

\subsubsection{Parametricity}

This relies on \lemref{lem:tyeq-erasure} to make sure that it is
well-typed.

\begin{lemma}\label{lem:tyeq-rel}
  If $\Delta \vdash A \equiv B : \sortType$, then for all $\rho \in
  \relEnv{E}(\Delta)$, $\rsem{A}{\relEnv{E}}\rho = \rsem{B}{\relEnv{E}}\rho$.
\end{lemma}

This next lemma relies on \lemref{lem:tysubst-erasure} to make sure
that it is well-typed.

\begin{lemma}\label{lem:tysubst-rel}
  If $\Delta \vdash A : \sortType$ and $\sigma : \Delta' \Rightarrow
  \Delta$, then for all $\rho \in \relEnv{E}(\Delta')$,
  $\rsem{\sigma^*A}{\relEnv{E}}\rho = \rsem{A}{\relEnv{E}}{(\rho \circ \sigma^*)}$. Likewise,
  if $\Delta \vdash \Gamma \isCtxt$ and $\sigma : \Delta' \Rightarrow
  \Delta$, then for all $\rho \in \relEnv{E}(\Delta')$,
  $\rsem{\sigma^*\Gamma}{\relEnv{E}}\rho = \rsem{\Gamma}{\relEnv{E}}{(\rho \circ \sigma^*)}$.
\end{lemma}

