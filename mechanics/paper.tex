\documentclass[preprint]{sigplanconf}

\newcommand{\sepbar}{\mathrel|}
\usepackage{mathpartir}
\usepackage{amsmath}
\usepackage{amssymb}
\usepackage{amsthm}
\usepackage{stmaryrd}

\title{From Parametricity to Conservation Laws, via Noether's Theorem}

\authorinfo{}{}{}
% \authorinfo{Robert Atkey}
%            {}
%            {bob.atkey@gmail.com}

\newtheorem{lemma}{Lemma}
\newtheorem{theorem}{Theorem}
\newtheorem{definition}{Definition}
\newtheorem{property}{Property}
\newtheorem{proposition}{Proposition}
\newcommand{\lemref}[1]{\hyperref[#1]{Lemma~\ref*{#1}}}
\newcommand{\thmref}[1]{\hyperref[#1]{Theorem~\ref*{#1}}}
\newcommand{\defref}[1]{\hyperref[#1]{Definition~\ref*{#1}}}
\newcommand{\propref}[1]{\hyperref[#1]{Proposition~\ref*{#1}}}
\newcommand{\corref}[1]{\hyperref[#1]{Corollary~\ref*{#1}}}
\newcommand{\conref}[1]{\hyperref[#1]{Conjecture~\ref*{#1}}}
\newcommand{\exref}[1]{\hyperref[#1]{Example~\ref*{#1}}}
\newcommand{\statementref}[1]{\hyperref[#1]{Statement~\ref*{#1}}}

\newcommand{\smoothrightarrow}{\stackrel{C^\infty}\longrightarrow}
\newcommand{\typeOfCartSp}[1]{\lbag #1 \rbag}
\newcommand{\elab}[1]{\lfloor #1 \rfloor}

\newtheoremstyle{examplestyle}
  {\topsep}  % space above
  {\topsep}  % space below
  {\normalfont}% name of font to use in the body of the theorem
  {0em}% measure of space to indent
  {\bfseries}% name of head font
  {.}% punctuation between head and body
  {5pt plus 1pt minus 1pt}% space after theorem head; " " = normal interword space
  {}% Manually specify head
\theoremstyle{examplestyle}
\newtheorem{example}{Example}
\newtheorem*{example*}{Example}

\newcommand{\fomega}{F$\omega$}
\newcommand{\sem}[1]{\llbracket #1 \rrbracket}
\newcommand{\Set}{\mathsf{Set}}
\newcommand{\Rel}{\mathsf{Rel}}
\newcommand{\Pow}{\mathcal{P}}
\newcommand{\frel}[1]{\langle #1 \rangle}
\newcommand{\coq}[1]{\textsf{#1}}
\newcommand{\CIC}{\textsc{CIC}}

\newcommand{\semKU}[1]{\llbracket #1 \rrbracket^O}
\newcommand{\semKR}[1]{\llbracket #1 \rrbracket^R}
\newcommand{\semKI}[1]{\llbracket #1 \rrbracket^{\mathrm{id}}}

\begin{document}

\maketitle

\begin{abstract}
  Invariance is of paramount importance in programming languages and
  in physics. In programming languages, John Reynolds' theory of
  relational parametricity demonstrates that parametric polymorphic
  programs are invariant under change of data representation, a
  property that yields ``free'' theorems about programs just from
  their types. In physics, Emmy Noether showed that if the action of a
  physical system is invariant under change of coordinates, then the
  physical system has a conserved quantity: a quantity that remains
  constant for all time. Knowledge of conserved quantities can reveal
  deep properties of physical systems. For example, the conservation
  of energy, which by Noether's theorem is a consequence of a system's
  invariance under time-shifting.

  In this paper, we link Reynolds' relational parametricity with
  Noether's theorem for deriving conserved quantities. We propose an
  extension of System F$\omega$ with new kinds, types and term
  constants for writing programs that describe classical mechanical
  systems in terms of their Lagrangians. We show, by constructing a
  relationally parametric model of our extension of F$\omega$, that
  relational parametricity is enough to satisfy the hypotheses of
  Noether's theorem, and so to derive conserved quantities for free,
  directly from the polymorphic types of Lagrangians expressed in our
  system.
\end{abstract}

\category{CR-number}{subcategory}{third-level}

\terms
term1, term2

\keywords
keyword1, keyword2

\section{Introduction}

Reynolds' theory of relational parametricity \cite{reynolds82} tells
us that parametrically polymorphic programs automatically satisfy
invariance properties. Such invariance properties are often called
\emph{Free Theorems}, after Wadler \cite{wadler89theorems}, since they
follow ``for free'' from the types of programs, rather than through
detailed study of the program text itself. A classic example,
presented by Wadler, is the free theorem for programs $f$ with the
following type:
\begin{displaymath}
  f : \forall \alpha.~\mathsf{List}~\alpha \to \mathsf{List}~\alpha
\end{displaymath}
Such programs takes lists of $\alpha$s, for any type $\alpha$, to
lists of $\alpha$s. Using Reynolds' theory of relational
parametricity, Wadler showed that \emph{any} $f$ with this type
satisfies the following property:
\begin{displaymath}
  \forall \alpha, \beta, g : \alpha \to \beta, l : \mathsf{List}~\alpha.~\mathrm{map}~g~(f~[\alpha]~l) = f~[\beta]~(\mathrm{map}~g~l)
\end{displaymath}
Thus, any $f$ with the type given above commutes with mapping some
arbitrary function $g$ over lists. Thinking in terms of abstract data
types and change of data representation, this free theorem states that
$f$ is invariant under change of data representation from an arbitrary
type $\alpha$ to another arbitrary type $\beta$, via $g$.

Using invariance under change to derive useful properties is a concept
much older than programming languages. In physics, Noether's theorem
\cite{noether} provides a general way to derive conserved properties
of physical systems from 




FIXME: example of translation invariance, with the coupled spring example
\begin{equation}\label{eq:coupled-spring-langrangian}
  L(t,q_1,q_2,\dot{q_1},\dot{q_2}) = \frac{1}{2}m(\dot{q_1}^2 + \dot{q_2}^2) - \frac{1}{2}k(q_1 - q_2)^2
\end{equation}
As a general rule, Lagrangians for classical mechanical systems take
the form $L = T - V$, where $T$ is the kinetic energy of the system
and $V$ is the potential energy. From the Lagrangian, using the
principle of stationary action, we can derive the following two
equations of motion for this system. (We describe the principle of
stationary action, and the process for deriving the equations of
motion fully in Section~\ref{sec:conservation-laws-from-symmetry}.)
For this system, the equations of motion are a pair of ordinary
differential equations (ODEs) that describe how the positions and
velocities of the particles evolve over time:
\begin{displaymath}
  FIXME
\end{displaymath}
We could now proceed to solve these ODEs to further analyse the
behaviour of this system. However, Noether's theorem gives us a
powerful way of gaining insight into properties of these ODEs ``for
free'', without necessarily having to find solutions to them.

The Lagrangian (\ref{eq:coupled-spring-langrangian}) does not refer to
any fixed point in space; only the relative distance between the two
particles, along with their velocities, is relevant. Therefore, the
Langrangian (\ref{eq:coupled-spring-langrangian}) is invariant under
translation in space by some arbitrary displacement $x$:
\begin{equation}\label{eq:coupled-spring-translation-invariance}
  L(t,q_1,q_2,\dot{q_1},\dot{q_2}) = L(t,q_1+x,q_2+x,\dot{q_1},\dot{q_2})
\end{equation}
By Noether's theorem, invariance under spatial translation implies
that the linear momentum of the whole system is constant for all
time. For this system, conservation of linear momentum is stated
mathematically like so:
\begin{equation}\label{eq:coupled-spring-linear-momentum}
  FIXME
\end{equation}
In general, Noether's theorem gives us a way of deriving conserved
properties like (\ref{eq:coupled-spring-linear-momentum}) from
invariance properties like
(\ref{eq:coupled-spring-translation-invariance}). In this case, we
have used invariance under translation in space to derive conservation
of linear momentum. Other common examples include invariance under
translation in time, yielding conservation of energy, and invariance
under rotation, yield conservation of angular momentum. We will see
examples of each of these kinds of invariance, and consequent
conservation laws, in Section~\ref{sec:examples}.

The invariance property stated in
Equation~(\ref{eq:coupled-spring-translation-invariance}) is highly
reminisicent of the

In this paper, we present a way of using a generalised version of
Reynolds' theory of relational parametricity to prove the geometric
invariance properties required for the hypotheses of Noether's
theorem. We build an extension of System F$\omega$ suitable for
writing invariant Lagrangians. In our system, the Lagrangian
(\ref{eq:coupled-spring-langrangian}) we gave above for describing a
system of two particles coupled by a spring will have the type:
\begin{displaymath}
  \forall x \mathord: \mathsf{T}(1).~C^\infty(\mathbb{R}\langle 1, 0 \rangle \times \mathbb{R}\langle 1, x \rangle \times \mathbb{R}\langle 1, x \rangle \times \mathbb{R}\langle 1, 0 \rangle \times \mathbb{R}\langle 1, 0 \rangle, \mathbb{R}\langle 1, 0 \rangle)
\end{displaymath}
where... FIXME

Atkey, Johann and Kennedy \cite{atkey13abstraction} have already
presented a polymorphic type system for expressing geometric
invariance properties similar to the 

FIXME: area of triangle type

By extending System F$\omega$, we get to 

\paragraph{Contributions}

Our major contributions are threefold:
\begin{enumerate}
\item FIXME: reformulate AJK in terms of the reflexive graph model of
  relational parametricity in order to unify that paper with the
  natural extension of Reynolds' relational parametricity to indexed
  types.
\item FIXME: Present an extension of System F$\omega$ suitable for
  writing invariant Lagrangians. This requires special attention to
  ensure that all the functions that we can write are actually
  continuous and differentiable. Hence we include a special type of
  smooth, invariant functions between cartesian spaces.
\item FIXME: Present many examples of 
\end{enumerate}

\paragraph{Outline}

\begin{itemize}
\item FIXME: Recall the necessary background for Noether's theorem
  (Section~\ref{sec:conservation-laws-from-symmetry}).
\item FIXME: Show how to get symmetry properties from types
  (Section~\ref{sec:symmetry-from-types}).
\item FIXME: In Section~\ref{sec:types-for-classical-mech}, we
  construct a type system for classical mechanics,
\item FIXME: Section~\ref{sec:examples} we present examples of using
  our extension of System F$\omega$ for writing invariant Lagrangians
  that describe physical systems.
\item FIXME: field theory
\end{itemize}

\section{Conservation Laws from Symmetry}
\label{sec:conservation-laws-from-symmetry}

Make sure to state up front the necessary background required. To
understand: basic(ish) calculus. We will be using the Calculus of
Variations, but only by stating the main results. We go for standard
physics notation, rather than anything more complicated.

\subsection{The Principle of Stationary Action}

Introduce by stating Newton's second law, and saying that it can be
derived from the principle of stationary action.

Analytical mechanics, Lagrangian and Hamiltonian mechanics. Basically
different mathematical reformulations of Newtonian mechanics that
yield new insights into the underlying structure of the kinds of
physical systems studied in classical mechanics. 

\paragraph{Lagrangian and Action}

Lagrangian mechanics reformulates Newtonian mechanics in terms of 

Two points: the Langrangian and generalised coordinates. Do
generalised coordinates as a secondary thing.

\begin{displaymath}
  \mathcal{S}[q;a;b] = \int_a^b L(t,q(t),\dot{q}(t)) \mathit{dt}
\end{displaymath}

\begin{example}
  A single particle under a potential field. Set the potential field
  to be (downward) gravity.
\end{example}

\paragraph{The Principle of Stationary Action}

The principle of stationary action 

\begin{displaymath}
  \frac{d}{dt}\frac{\partial L}{\partial \dot{q_i}} - \frac{\partial L}{\partial q_i} = 0
\end{displaymath}

Explain the different derivatives going on here

Example: Newton's second law...

\subsection{Noether's Theorem} Noether's theorem provides us with deep
insights into the properties of the solutions of the Euler-Lagrange
equations. FIXME: motivate invariance.

FIXME: Define one-parameter groups

\paragraph{Invariance of the Action} Let
\begin{displaymath}
  \mathcal{S}[q;a;b] = \int_a^b L(t,q(t),\dot{q}(t)) \mathit{dt}
\end{displaymath}
be the action of some physical system described by the Lagrangian
$L$. Assume a differentiable invertible function $\tau : \mathbb{R}
\to \mathbb{R}$ that transforms time in some way and a function
$\sigma : \mathbb{R}^n \to \mathbb{R}^n$ that transforms the vector of
generalised coordinates $\vec{q}$ to another vector of generalised
coordinates $\sigma(\vec{q})$.

We say that the action $\mathcal{S}$ is invariant under the
transformations $\tau$ and $\sigma$ if it is the case that for all
$a$, $b$ and $q$ that:
\begin{equation}\label{eq:invariance}
    \int_a^b L(t,q(t),\dot{q}(t)) \mathit{dt}
    = \int_{\tau(a)}^{\tau(b)} L(s, q^*(s), \dot{q^*}(s)) \mathit{ds}
\end{equation}
where $q^*(s) = (\sigma \circ q \circ \tau^{-1})(s)$ is the path $q$
transformed by $\tau$ and $\sigma$.  FIXME: Remark on the use of $t$
and $s$ in the integrals. By change of variables, the right hand
integral in the above equation is equal to the following integral:
\begin{displaymath}
  \int_a^b L(\tau(t), q^*(\tau(t)), \dot{q^*}(\tau(t)))\cdot \dot{\tau}(t) \mathit{dt}
\end{displaymath}
Since the endpoints $a$ and $b$ are arbitrary 


\begin{example}
  Continue a simple example from above to show that its action is
  invariant under (e.g.) time translation.
\end{example}

\paragraph{Noether's Theorem} Noether's theorem applies to actions
that are \emph{continuously} invariant. FIXME: not quite right: write
out the maths first.

(a) parameterise $\tau$ and $\sigma$ by a real value $\epsilon$ such
that we get a continuous symmetry of the action; (b) state Noether's
theorem as we will use it.

\begin{theorem}
  Noether's theorem
\end{theorem}

\begin{example}
  Continue the simple example from above to show that the invariance
  under time translation yields conservation of energy.
\end{example}

State that we will do more examples later on, after we discuss how to
derive symmetry from types.

\section{Symmetry from Types}

The first part here will be at least partly taken from Atkey
2012. Work out a way of emphasising that we do not need to do anything
to extend the basic underlying semantic structures involved: reflexive
graphs are enough. This presentation differs from Atkey 2012 in that
we emphasise the reflexive graph structure more, in order to be able
to better expose the connection with the groupoid kinds in the section
below.

\subsection{Relational Parametricity for System F$\omega$}
\label{sec:refl-graphs-for-fomega}

Reflexive graphs give us a way of capturing the higher-dimensional
structure of types in a 

Relational parametricity (Reynolds, BFSS), reflexive graphs
(Robinson/Rosolini, Hasegawa, me).

Explain the semantic setting. Explain the ``property'' thing.

\paragraph{Syntax of System F$\omega$} Present the basic syntax of the
kinds, types and terms of (predicative) System F$\omega$.

Small kinds. We will extend the kinds, types and terms later to
encompass the necessary geometrical constructs for classical
mechanics.

\begin{displaymath}
  \kappa ::= * \sepbar \kappa_1 \to \kappa_2 \sepbar \kappa_1 \times \kappa_2 \sepbar \cdots
\end{displaymath}

\begin{mathpar}
  \inferrule*
  {\kappa_1~\mathrm{small} \\ \kappa_2~\mathrm{small}}
  {\kappa_1 \to \kappa_2~\mathrm{small}}

  \inferrule*
  {\kappa_1~\mathrm{small} \\ \kappa_2~\mathrm{small}}
  {\kappa_1 \times \kappa_2~\mathrm{small}}
\end{mathpar}

Type equality... $\Theta \vdash A \equiv B : \kappa$

\begin{figure}[t]
  \centering
  \begin{mathpar}
    \inferrule*
    {\alpha : \kappa \in \Theta}
    {\Theta \vdash \alpha : \kappa}

    \inferrule*
    {\Theta, \alpha : \kappa_1 \vdash A : \kappa_2}
    {\Theta \vdash \lambda \alpha : \kappa_1.\ A : \kappa_1 \to \kappa_2}

    \inferrule*
    {\Theta \vdash F : \kappa_1 \to \kappa_2 \\ \Theta \vdash A : \kappa_1}
    {\Theta \vdash F A : \kappa_2}

    \inferrule*
    {\Theta \vdash A : \kappa_1 \\ \Theta \vdash B : \kappa_2}
    {\Theta \vdash \langle A, B \rangle : \kappa_1 \times \kappa_2}

    \inferrule* [right=${i \in \{1,2\}}$]
    {\Theta \vdash A : \kappa_1 \times \kappa_2}
    {\Theta \vdash \pi_i~A : \kappa_i}

    \inferrule*
    {\Theta \vdash A : * \\ \Theta \vdash B : *}
    {\Theta \vdash A \times B : *}

    \inferrule*
    {\Theta \vdash A : * \\ \Theta \vdash B : *}
    {\Theta \vdash A \to B : *}

    \inferrule*
    {\Theta, \alpha : \kappa \vdash A : * \\ \kappa\textrm{ small}}
    {\Theta \vdash \forall\alpha\mathord:\kappa.\ A : *}
  \end{mathpar}
  \caption{Types and their Kinds}
  \label{fig:types}
\end{figure}

\begin{figure}[t]
  \centering
  \begin{mathpar}
    \inferrule*
    {x : A \in \Gamma}
    {\Theta \sepbar \Gamma \vdash x : A}

    \inferrule*
    {\Theta \sepbar \Gamma \vdash e : A \\ \Theta \vdash A \equiv B : *}
    {\Theta \sepbar \Gamma \vdash e : B}
  \end{mathpar}
  \begin{mathpar}
    \inferrule*
    {\Theta \sepbar \Gamma \vdash e_1 : A_1 \\\\
      \Theta \sepbar \Gamma \vdash e_2 : A_2}
    {\Theta \sepbar \Gamma \vdash (e_1, e_2) : A_1 \times A_2}

    \inferrule* [right=${i \in \{1,2\}}$]
    {\Theta \sepbar \Gamma \vdash e : A_1 \times A_2}
    {\Theta \sepbar \Gamma \vdash \pi_i e : A_i}
  \end{mathpar}
  \begin{mathpar}
    \inferrule*
    {\Theta \sepbar \Gamma, x : A \vdash e : B}
    {\Theta \sepbar \Gamma \vdash \lambda x : A.\ e : A \to B}

    \inferrule*
    {\Theta \sepbar \Gamma \vdash e_1 : A \to B \\
      \Theta \sepbar \Gamma \vdash e_2 : A}
    {\Theta \sepbar \Gamma \vdash e_1 e_2 : B}

    \inferrule*
    {\Theta, \alpha : \kappa \sepbar \Gamma \vdash e : A \\ \alpha \not\in \mathit{fv}(\Gamma)}
    {\Theta \sepbar \Gamma \vdash \Lambda \alpha \mathord: \kappa.\ e : \forall \alpha \mathord:\kappa. A}

    \inferrule*
    {\Theta \sepbar \Gamma \vdash e : \forall \alpha\mathord:\kappa. A \\ \Theta \vdash B : \kappa}
    {\Theta \sepbar \Gamma \vdash e\ [B] : A\{B/\alpha\}}
  \end{mathpar}
  \caption{Terms and their Types}
  \label{fig:terms}
\end{figure}


\paragraph{Reflexive Graphs and the Interpretation of Kinds} We will
interpret every kind $\kappa$ is as a reflexive graph, which we now
define. A \emph{reflexive graph} is a triple $(O, R, \mathrm{id})$,
where $O$ is a large set of objects, $R : O \times O \to \Set$ assigns
a small set of directed `edges' to each pair of objects, and
$\mathrm{id} : \forall o \in O.~R(o,o)$ assigns a distinguished
`identity' edge from every object to itself. We think of the edges of
a reflexive graph as abstract ``relations'' between the
objects. Indeed, in the interpretation of the kind of types, $*$,
below, the edges will be exactly relations.

A \emph{small} reflexive graph is a reflexive graph $(O, R,
\mathrm{id})$ where $O$ is a small set of objects. We use small
reflexive graphs as the semantic interpretation of small kinds.

The interpretation of kinds as reflexive graphs, and small kinds as
small reflexive graphs is a key property of our semantics that we will
maintain as we add additional kinds in
Sections~\ref{sec:discrete-kinds} and~\ref{sec:groupoid-kinds},
below. We state this as Property~\ref{property:semantic-kinds}:

\begin{property}\label{property:semantic-kinds}
  Each kind $\kappa$ is interpreted as a reflexive graph
  $\sem{\kappa}$. If $\kappa~\mathrm{small}$, then $\sem{\kappa}$ is a
  small reflexive graph.
\end{property}

An appealing interpretation of reflexive graphs is as ``categories
without composition''. Following this intuition, we define morphisms
of reflexive graphs as ``functors'', without the preservation of
composition condition. A \emph{morphism of reflexive graphs}
$(O_1,R_1,\mathrm{id}_1)$ and $(O_2, R_2, \mathrm{id}_2)$ is a pair of
mappings $f : O_1 \to O_2$ and $r : \forall o, o' \in
O_1.~R_1(o,o') \to R_2(f~o, f~o')$ such that identities are
preserved: $r~o~o~(\mathrm{id}_1~o) = \mathrm{id}_2~(f~o)$. We use
morphisms of reflexive graphs below to interpret well-kinded types.

We will use the notation $-^O$, $-^R$ and $-^{\mathrm{id}}$ for the
first, second and third projections out of tuples representing
reflexive graphs. Similarly, we use $-^f$ and $-^r$ for the first and
second projections out of tuples representing reflexive graph
morphisms.

We now define the interpretations of the basic kinds of System
F$\omega$ we defined above, making sure that we maintain
Property~\ref{property:semantic-kinds}. At base kind, the collection
of objects is simply the type of all small sets; edges between $A$ and
$B$ are binary relations on $A$ and $B$ (i.e., subsets of $A \times
B$); and the distinguished identity edge is exactly the equality
relation:
\begin{displaymath}
  \sem{*} = (\Set, \Rel, \equiv)
\end{displaymath}
The reflexive graph $\sem{*}$ is not small, due to the collection of
all small sets $\Set$ not forming a small set.

For higher kinds $\kappa_1 \to \kappa_2$, the collection of objects
consists of reflexive graph morphisms from the interpretation of
$\kappa_1$ to the interpretation of $\kappa_2$; the edges between
morphisms $(f, r)$ and $(f', r')$ are edge transformers; and the
distinguished identity relation for $(f,r)$ is just $r$:
\begin{displaymath}
  \begin{array}{@{}l}
    \sem{\kappa_1 \to \kappa_2} = \\
    \begin{array}[t]{l@{}l@{}}
      (& \{ (f,r) \sepbar (f, r) : \sem{\kappa_1} \to \sem{\kappa_2} \},\\
      &((f,r),(f',r')) \mapsto \forall o,o'. \semKR{\kappa_1}(o,o') \to \semKR{\kappa_2}(f~o, f'~o'), \\
      &(f,r) \mapsto r\ )
    \end{array}
  \end{array}
\end{displaymath}
By the assumption that our collection of small sets is closed under
the formation of function spaces and set comprehension, if
$\sem{\kappa_1}$ and $\sem{\kappa_2}$ are small reflexive graphs, then
so is $\sem{\kappa_1 \to \kappa_2}$.

Product kinds are interpreted by taking the product of their
interpretations as reflexive graphs:
\begin{displaymath}
  \begin{array}{@{}l}
    \sem{\kappa_1 \times \kappa_2} = \\
    \begin{array}[t]{l@{}l}
      (& \semKU{\kappa_1} \times \semKU{\kappa_2}, \\
      & ((o_1,o_2),(o_1',o_2')) \mapsto \semKR{\kappa_1}(o_1,o_1') \times \semKR{\kappa_2}(o_2,o_2'), \\
      & (o_1,o_2) \mapsto (\semKI{\kappa_1}(o_1), \semKI{\kappa_2}(o_2)))
    \end{array}
  \end{array}
\end{displaymath}
The objects of the interpretation of a product kind is the product of
the underlying collections of objects of the two parts, and the
relational component is simply the product of the relational
components. This naturally leads to the identity component being
defined as the tuple of the identity components of the two
parts. Again, by assumption that our collection of small sets is
closed under products, if $\sem{\kappa_1}$ and $\sem{\kappa_2}$ are
small reflexive graphs, then so is $\sem{\kappa_1 \times
  \kappa_2}$. Product kinds generalise to the interpretation of
kinding contexts $\Theta = \alpha_1 : \kappa_1, ..., \alpha_n :
\kappa_n$, which are interpreted as the product of the reflexive graph
interpretations of $\kappa_1, ..., \kappa_n$:
\begin{displaymath}
  \begin{array}{l}
    \sem{\alpha_1 : \kappa_1, ..., \alpha_n : \kappa_n} = \\
    \begin{array}[t]{l@{}l}
      (&\semKU{\kappa_1} \times ... \times \semKU{\kappa_n}, \\
      &(\theta,\theta') \mapsto \semKR{\kappa_1}(\pi_1\theta, \pi_1\theta') \times ... \times \semKR{\kappa_n}(\pi_n\theta, \pi_n\theta'), \\
      &\theta \mapsto (\semKI{\kappa_1}(\pi_1 \theta), ..., \semKI{\kappa_n}(\pi_n \theta)))
    \end{array}
  \end{array}
\end{displaymath}

In Sections~\ref{sec:discrete-kinds} and~\ref{sec:groupoid-kinds}
below, we will extend System F$\omega$ with additional kinds, and
assign them reflexive graph interpretations, making sure that we
maintain Property~\ref{property:semantic-kinds}.

\begin{figure*}[t]
  \begin{displaymath}
    \begin{array}{l@{\hspace{0.3em}}c@{\hspace{0.3em}}l}
      \sem{\Theta \vdash \alpha_i : \kappa_i}^f \theta & = & \pi_i \theta \\
      \sem{\Theta \vdash \alpha_i : \kappa_i}^r\theta\theta' \rho & = & \pi_i \rho \\
      \\
      \sem{\Theta \vdash \lambda \alpha \mathord: \kappa_1.~A : \kappa_1 \to \kappa_2}^f \theta & = & (
      \begin{array}[t]{@{}l}
        \lambda o \in \semKU{\kappa_1}.\ \sem{A}^f(\theta, o), 
        \lambda o, o' \in \semKU{\kappa_1}, r \in \semKR{\kappa_1}(o,o').\ \sem{A}^r(\theta,o)(\theta,o')(\semKI{\Theta}\theta, r))
      \end{array}\\
      \sem{\Theta \vdash \lambda \alpha \mathord: \kappa_1.~A : \kappa_1 \to \kappa_2}^r \theta\theta' \rho & = &
      \lambda o,o' \in \semKU{\kappa_1}, r \in \semKR{\kappa_1}(o,o').\ \sem{A}^r(\theta,o)(\theta',o')(\rho, r) \\
      \\
      \sem{\Theta \vdash F A : \kappa_2}^f \theta & = & \pi_1 (\sem{F}^f\ \theta)\ (\sem{A}^f\ \theta)\\
      \sem{\Theta \vdash F A : \kappa_2}^r\theta\theta'\rho & = & \sem{F}^r\theta\theta'\rho\ (\sem{A}^f \theta)(\sem{A}^f \theta')(\sem{A}^r\theta\theta'\rho) \\
      \\
      \sem{\Theta \vdash \langle A, B \rangle : \kappa_1 \times \kappa_2}^f \theta & = & (\sem{A}^f \theta, \sem{B}^f \theta) \\
      \sem{\Theta \vdash \langle A, B \rangle : \kappa_1 \times \kappa_2}^r \theta \theta' \rho & = & (\sem{A}^r \theta \theta' \rho, \sem{B}^r \theta \theta' \rho) \\
      \\
      \sem{\Theta \vdash \pi_i A : \kappa_i}^f \theta & = & \pi_i (\sem{A}^f \theta) \\
      \sem{\Theta \vdash \pi_i A : \kappa_i}^r \theta \theta' \rho & = & \pi_i (\sem{A}^r \theta \theta' \rho)
    \end{array}
  \end{displaymath}
  \caption{Interpretation of type-level $\lambda$-calculus as reflexive graph morphisms}
  \label{fig:type-level-lambda-calc}
\end{figure*}

\begin{figure*}[t]
  \begin{displaymath}
    \begin{array}{l@{\hspace{0.3em}}c@{\hspace{0.3em}}l}
      \sem{\Theta \vdash A \to B : *}^f \theta & = & \sem{A}^f\theta \to \sem{B}^f\theta \\
      \sem{\Theta \vdash A \to B : *}^r\theta\theta'\rho & = &
      \{ (f,f') \sepbar \forall (a,a') \in \sem{A}^r\theta\theta'\rho.\ (f~a, f'~a') \in \sem{B}^r\theta\theta'\rho \} \\
      \\
      \sem{\Theta \vdash A \times B : *}^f \theta & = & \sem{A}^f\theta \times \sem{B}^f\theta \\
      \sem{\Theta \vdash A \times B : *}^r\theta\theta'\rho & = &
      \{ ((a,b),(a',b')) \sepbar (a,a') \in \sem{A}^r\theta\theta'\rho, (b,b') \in \sem{B}^r\theta\theta'\rho \} \\
      \\
      \sem{\Theta \vdash \forall \alpha\mathord:\kappa.~A : *}^f \theta &=& \{
      \begin{array}[t]{@{}l}
        x \in (\forall o \in \semKU{\kappa}.\ \sem{A}^f(\theta, o)) \sepbar
        \forall o,o',r \in \semKR{\kappa}(o,o').\ (x~o, x~o') \in \sem{A}^r(\theta,o)(\theta,o')(\semKI{\Theta} \theta, r) \}
      \end{array} \\
      \sem{\Theta \vdash \forall \alpha\mathord:\kappa.~A : *}^r\theta\theta'\rho &=& \{ (x,x') \sepbar \forall o, o', r \in \semKR{\kappa}(o,o').\ (x~o, x'~o') \in \sem{A}^r(\theta,o)(\theta',o')(\rho, r) \}
    \end{array}
  \end{displaymath}
  \caption{Interpretation of basic types as reflexive graph morphisms}
  \label{fig:basic-type-interpretation}
\end{figure*}

\paragraph{Interpretation of Types} Well-kinded types $\Theta \vdash A
: \kappa$ from Figure~\ref{fig:types} are interpreted as reflexive
graph morphisms $\sem{A} : \sem{\Theta} \to \sem{\kappa}$. We sum this
up as a property of our semantics:
\begin{property}\label{property:semantic-types}
  Each well-kinded type $\Theta \vdash A : \kappa$ is interpreted as a
  reflexive graph morphism $\sem{A} : \sem{\Theta} \to \sem{\kappa}$,
  such that if $\Theta \vdash A \equiv B : \kappa$, then $\sem{A} =
  \sem{B}$.
\end{property}
%
% That is, every well-kinded type is interpreted as a pair $(\sem{A}^f,
% \sem{A}^r)$ of functions $\sem{A}^f : \sem{\Theta}^O \to
% \sem{\kappa}^O$ and $\sem{A}^r : \forall o, o'.~\sem{\Theta}^R(o,o')
% \to \sem{\kappa}^R(\sem{A}^f~o, \sem{A}^f~o')$, such that identity
% edges are preserved.
%
The interpretation of the $\lambda$-calculus fragment (i.e.,
variables, $\lambda$-abstraction and application, and products) of the
language of well-kinded types is displayed in
Figure~\ref{fig:type-level-lambda-calc}. The interpretations are
unsurprising given the reflexive graph interpretation of the kinds
$\kappa_1 \to \kappa_2$ and $\kappa_1 \times \kappa_2$ we gave above.

Figure~\ref{fig:basic-type-interpretation} shows the interpretations
of the basic type constructors for function and product types, and
universal quantification. Each of these constructs builds an object of
kind $*$, so the object-level interpretation is a small set in $\Set$,
and the relation-level interpretation is an actual relation. In the
cases of the function and product types, the object-level
interpretation is just as set-theoretic function and product
respectively, and the relation-level interpretation uses the standard
logical relations interpretations of these type
constructors. Universal quantification, $\forall
\alpha\mathord:\kappa.~A$, is interpreted at the object level by
taking the dependent product over the objects of the interpretation of
$\kappa$ (this product exists because we have stipulated that $\kappa$
must be small), and then restricting to those elements of the
dependent product that preserve relations. This restriction is
required for this interpretation to preserve identity edges, and so be
a reflexive graph morphism. The relation-level interpretation of
universal quantification is the standard relational interpretation of
such types, albeit here generalised to kinds interpreted as arbitrary
reflexive graphs.

Well-kinded typing contexts $\Theta \vdash \Gamma$ are interpreted as
reflexive graph morphisms $\sem{\Gamma} : \sem{\Theta} \to \sem{*}$ by
taking the product of the interpretations of their constituent types,
similar to the interpretation of the product types $A \times B$.

We will extend basic System F$\omega$ with additional types and type
equalities in Sections~\ref{sec:discrete-kinds},
\ref{sec:groupoid-kinds}, and~\ref{sec:types-for-classical-mech},
below. These new types will also be assigned interpretations as
reflexive graph morphisms, and we will ensure that
Property~\ref{property:semantic-types} is maintained.

\paragraph{Interpretation of Terms} We omit the straightforward and
relatively uninteresting interpretation of well-typed terms $\Theta
\sepbar \Gamma \vdash e : A$, and just state that there is a
well-defined function interpreting each well-typed term, with the
property that it takes related environments to related results:

\begin{property}
  For all well-typed terms $\Theta \sepbar \Gamma \vdash e : A$ there
  is a function $\sem{e} \in (\forall \theta \in \semKU{\Theta}.\
  \sem{\Gamma}^f\theta \to \sem{A}^f\theta)$, such that, for all
  $\theta, \theta' \in \semKU{\Theta}$, $\rho \in
  \semKR{\Theta}(\theta,\theta')$, $\gamma \in \sem{\Gamma}^f\theta$
  and $\gamma' \in \sem{\Gamma}^f\theta'$, if $(\gamma, \gamma') \in
  \sem{\Gamma}^r\theta\theta'\rho$ then $(\sem{e}\theta\gamma,
  \sem{e}\theta'\gamma') \in \sem{A}^r\theta\theta'\rho$.  Moreover,
  this interpretation is sound for the $\beta\eta$ equational theory
  of terms.
\end{property}

\subsection{Discrete Kinds}
\label{sec:discrete-kinds}

In Section~\ref{sec:refl-graphs-for-fomega}, we only had a single base
kind: the kind $*$ of proper types, with a specific intepretation as
the reflexive graph of sets and relations. We now describe two
families of base kinds with interpretations that are particular sorts
of reflexive graphs. In this section, we look at discrete kinds; kinds
whose reflexive graph interpretations are such that the reflexive
edges are the only edges between objects. In the following section
(Section~\ref{sec:groupoid-kinds}), we look at groupoid kinds, where
edges are composable and invertible.

Discrete kinds can be seen as the natural way of lifting base types up
to the kind level. 


\begin{displaymath}
  \kappa ::= \cdots \sepbar \mathsf{nat}
\end{displaymath}

\begin{displaymath}
  \inferrule*
  { }
  {\mathsf{nat}~\mathrm{small}}
\end{displaymath}

\begin{displaymath}
  \sem{\mathsf{nat}} = (\mathbb{N}, (n_1,n_2) \mapsto \{ * \sepbar n_1 = n_2 \}, n \mapsto *)
\end{displaymath}

\begin{mathpar}
  \inferrule*
  { }
  {\Theta \vdash \mathit{zero} : \mathsf{nat}}

  \inferrule*
  {\Theta \vdash A : \mathsf{nat}}
  {\Theta \vdash \mathit{succ}~A : \mathsf{nat}}

  \inferrule*
  {\Theta \vdash A : \mathsf{nat} \\
    \Theta \vdash B : \kappa \\
    \Theta \vdash C : \kappa \to \kappa}
  {\Theta \vdash \mathit{natrec}~A~B~C : \kappa}
\end{mathpar}

Make the point that any set can be used as the interpretation of a
kind. Also, the utility of singleton types.

\subsection{Groupoid Kinds}
\label{sec:groupoid-kinds}

There is a full and faithful functor from the category of groupoids to
the category of reflexive graphs, that preserves smallness.

\paragraph{The additive group of integers}

\paragraph{Matrix groups}

GL and O, with group operations

$\mathit{ortho}_n : \mathsf{O}(n) \to \mathsf{GL}(n)$
$\mathit{scale}_n : \mathsf{GL}(1) \to \mathsf{GL}(n)$

\paragraph{Translation groups}

$\mathit{exp} : \mathsf{T}(1) \to \mathsf{GL}(1)$
$\underline{c}\cdot- : \mathsf{Z} \to \mathsf{T}(1)$

\paragraph{Groupoid of Cartesian Spaces}


\subsection{Why not Groupoids Everywhere?}

Explain why we aren't using groupoids as kinds: don't need to, doesn't
directly generalise relational parametricity, see Robinson's
paper. Distance-indexed types.

\section{A Type System for Classical Mechanics}
\label{sec:types-for-classical-mech}

FIXME: rewrite this para to refer back concretely

We have a way of getting symmetry properties from types, and a way of
using symmetry properties to derive conservation laws for physical
systems. We now put the two together by building an applied version
System F$\omega$ with new kind-, type- and term-level constants that
will deliver us the exact symmetry properties we require to be able to
apply Noether's theorem.

%\subsection{Extending System F$\omega$ for Classical Mechanics}

Let us enumerate our requirements for an extension of System F$\omega$
for writing invariant Lagrangians\footnote{Make sure that I say this
  is the plan back in Sec 2}.

\subsection{Extending System F$\omega$ with Smooth Functions}

\begin{enumerate}
\item We have already done the matrix kinds and the CartSp kind back
  in the previous section. Add $n-ary$ products for CartSp.
  \begin{displaymath}
    \begin{array}{l@{\hspace{0.3em}}c@{\hspace{0.3em}}l}
      C^\infty &:& \mathsf{CartSp} \to \mathsf{CartSp} \to * \\
      \mathbb{R}^n &:& \mathsf{GL}(n) \to \mathsf{T}(n) \to \mathsf{CartSp} \\
%      (\times) &:& \mathsf{CartSp} \to \mathsf{CartSp} \to \mathsf{CartSp} \\
      \mathit{vec} &:& \mathsf{nat} \to \mathsf{CartSp} \to \mathsf{CartSp}
    \end{array}
\end{displaymath}
\item Note that the $n$ in $\mathbb{R}^n$ is static
\item Add the type former for the smooth function space, and all the
  relevant constants
\item Give a (very) small example, and complain that it is very
  difficult to write anything in this point-free style.
\end{enumerate}

Administrative things: composition, products

\begin{displaymath}
  \begin{array}{l@{\hspace{0.3em}}c@{\hspace{0.3em}}l}
    (+) &:&
    \begin{array}[t]{@{}l}
      \forall g \mathord: \mathsf{GL}(n), t_1,t_2 \mathord: \mathsf{T}(n). \\
      \hspace{1.5cm} C^\infty(\mathbb{R}^n\langle g, t_1 \rangle \times \mathbb{R}^n\langle g, t_2 \rangle, \mathbb{R}^n\langle g, t_1 + t_2 \rangle)\\
    \end{array} \\
    (-) &:&
    \begin{array}[t]{@{}l}
      \forall g \mathord: \mathsf{GL}(n), t_1,t_2 \mathord: \mathsf{T}(n). \\
      \hspace{1.5cm} C^\infty(\mathbb{R}^n\langle g, t_1 \rangle \times \mathbb{R}^n\langle g, t_2 \rangle, \mathbb{R}^n\langle g, t_1 - t_2 \rangle)\\
    \end{array} \\
    \sin &:& \forall z \mathord: \mathsf{Z}.~C^\infty(\mathbb{R}\langle 1, 2\pi \cdot z \rangle, \mathbb{R} \langle 1, 0 \rangle) \\
    \cos &:& \forall z \mathord: \mathsf{Z}.~C^\infty(\mathbb{R}\langle 1, 2\pi \cdot z \rangle, \mathbb{R} \langle 1, 0 \rangle) \\
    \mathrm{exp} &:& \forall t \mathord: \mathsf{T}(1).~C^\infty(\mathbb{R}\langle 1, t \rangle, \mathbb{R}\langle \mathit{exp}~t, 0 \rangle)
  \end{array}
\end{displaymath}

\subsection{A Surface Syntax for Smooth Functions}

\begin{enumerate}
\item Kind of cartesian spaces, parameterised by a matrix group and a
  translation group
\item Smooth functions between spaces: motivate this by looking at the
  requirements of the Euler-Lagrange equations.
\item Primitives for dealing with the geometry of the spaces (inner
  product, vector space operations, cross product(?), trigonometric
  functions (at least sin and cos)).
\end{enumerate}

\begin{displaymath}
  \kappa ::= \cdots \sepbar \mathsf{CartSp} \sepbar \mathsf{T}(n) \sepbar \mathsf{GL}(n) \sepbar \mathsf{O}(n) \sepbar \mathsf{Z}
\end{displaymath}

\begin{itemize}
\item $\mathsf{CartSp}$ is the kind of cartesian spaces
  ($\mathbb{R}^n$, for some $n$) and diffeomorphisms between them,
  with the standard definition of diffeomorphism on these spaces. The
  identity edge is just the identity function.
\item $\mathbb{R}^n~t~g$ is the cartesian space with the
  diffeomorphism defined by the action of the translation and matrix
  group.
\item $X \times Y$ takes the product of two cartesian spaces, and the
  corresponding product of the diffeomorphisms.
\item $X \smoothrightarrow Y$ consists of the infinitely
  differentiable functions from $X$ to $Y$ that preserve the
  diffeomorphisms.
\end{itemize}

\begin{enumerate}
\item Need coercions between the different groups, as required by the
  examples.
\item Need operations on vectors, e.g., the usual linear space ones,
  and inner product. Perhaps also the cross product one.
\item What operations on the smooth function space do we need?
  composition, products, identities.
\end{enumerate}

\begin{figure*}[t]
  \centering
  \textbf{Administrative rules}
  \begin{mathpar}
    \inferrule*
    {x : \typeOfCartSp{X} \in \Gamma}
    {\Theta \sepbar \Gamma; \Delta \vdash x : X}

    \inferrule*
    {x : X \in \Delta}
    {\Theta \sepbar \Gamma; \Delta \vdash x : X}

    \inferrule*
    {\Theta \sepbar \Gamma; \Delta \vdash e_1 : X \\
      \Theta \sepbar \Gamma; \Delta, x : X \vdash e_2 : Y}
    {\Theta \sepbar \Gamma; \Delta \vdash \mathrm{let}~x = e_1~\mathrm{in}~e_2 : Y}

    \inferrule*
    {\Theta \sepbar \Gamma \vdash e_1 : C^\infty(X, Y) \\\\
      \Theta \sepbar \Gamma; \Delta \vdash e_2 : X}
    {\Theta \sepbar \Gamma; \Delta \vdash e_1 (e_2) : Y}
  \end{mathpar}
  \textbf{Vector space operations, and dot product}
  \begin{mathpar}
    \inferrule*
    {\vec{x} \in \mathbb{R}^n}
    {\Theta \sepbar \Gamma; \Delta \vdash \vec{x} : \mathbb{R}^n\langle 1, 0 \rangle}

    \inferrule*
    {\Theta \vdash g : \mathsf{GL}(n)}
    {\Theta \sepbar \Gamma; \Delta \vdash 0 : \mathbb{R}^n\langle g, 0 \rangle}

    \inferrule*
    {\Theta \sepbar \Gamma; \Delta \vdash e_1 : \mathbb{R}^n\langle g, t_1 \rangle \\
      \Theta \sepbar \Gamma; \Delta \vdash e_2 : \mathbb{R}^n\langle g, t_2 \rangle}
    {\Theta \sepbar \Gamma; \Delta \vdash e_1 + e_2 : \mathbb{R}^n\langle g, t_1 + t_2 \rangle}

    \inferrule*
    {\Theta \sepbar \Gamma; \Delta \vdash e_1 : \mathbb{R}^n\langle g, t_1 \rangle \\
      \Theta \sepbar \Gamma; \Delta \vdash e_2 : \mathbb{R}^n\langle g, t_2 \rangle}
    {\Theta \sepbar \Gamma; \Delta \vdash e_1 - e_2 : \mathbb{R}^n\langle g, t_1 - t_2 \rangle}

    \inferrule*
    {\Theta \sepbar \Gamma; \Delta \vdash e_1 : \mathbb{R}\langle g_1, 0 \rangle \\
      \Theta \sepbar \Gamma; \Delta \vdash e_2 : \mathbb{R}^n\langle g_2, 0 \rangle}
    {\Theta \sepbar \Gamma; \Delta \vdash e_1 * e_2 : \mathbb{R}^n\langle \mathit{scale}_n(g_1)g_2, 0 \rangle}

    \inferrule*
    {\Theta \sepbar \Gamma; \Delta \vdash e_1 : \mathbb{R}^n\langle \mathit{scale}_n(s)\mathit{ortho}_n(o), 0 \rangle \\
      \Theta \sepbar \Gamma; \Delta \vdash e_2 : \mathbb{R}^n\langle \mathit{scale}_n(s)\mathit{ortho}_n(o), 0 \rangle}
    {\Theta \sepbar \Gamma; \Delta \vdash e_1 \cdot e_2 : \mathbb{R}\langle\mathit{scale}_n(s)^n, 0 \rangle}
  \end{mathpar}
  \textbf{Transcendental functions}
  \begin{mathpar}
    \inferrule*
    {\Theta \sepbar \Gamma; \Delta \vdash e : \mathbb{R}\langle 1, t \rangle}
    {\Theta \sepbar \Gamma; \Delta \vdash \mathrm{exp}~e : \mathbb{R}\langle \mathit{exp}~t, 0 \rangle}

    \inferrule*
    {\Theta \sepbar \Gamma; \Delta \vdash e : \mathbb{R}\langle 2\pi z, 0 \rangle}
    {\Theta \sepbar \Gamma; \Delta \vdash \mathrm{sin}~e : \mathbb{R}\langle 1, 0 \rangle}

    \inferrule*
    {\Theta \sepbar \Gamma; \Delta \vdash e : \mathbb{R}\langle 2\pi z, 0 \rangle}
    {\Theta \sepbar \Gamma; \Delta \vdash \mathrm{cos}~e : \mathbb{R}\langle 1, 0 \rangle}
  \end{mathpar}
  FIXME: products, division, n-ary products (sum, zip, reduce), constants
  \caption{Type system for smooth functions}
  \label{fig:smooth-terms}
\end{figure*}

\begin{theorem}
  If $\Theta \sepbar \Gamma; \Delta \vdash e : X$ then $\Theta \sepbar
  \Gamma \vdash \elab{e} : C^\infty(\elab{\Delta}, X)$.
\end{theorem}

\section{Examples}

$- \sepbar V : \forall ... . C^\infty(\mathbb{R}^3\langle o, t \rangle, \mathbb{R}\langle 1, 0 \rangle); t : \mathbb{R}\langle 1, t_t \rangle, q : \mathbb{R}^3\langle o, t_q \rangle, \dot{q} : \mathbb{R}^3\langle o, 0 \rangle \vdash \frac{1}{2} * m * (\dot{q} \cdot \dot{q}) - V(q) : \mathbb{R}\langle 1, 0 \rangle$

\begin{enumerate}
\item free particle
\item $2$-body
\item $n$-body (use length-indexed vectors)
\item pendulums
\item oscillators
\end{enumerate}

\section{Basic Field Theory}

\subsection{Principle of Stationary Action for Fields}

\subsection{Noether's Theorem for Fields}

\subsection{The Klein-Gordon Equation}

\section{Conclusions}

\begin{enumerate}
\item Gauge theory
\item Quantum field theory
\item Dependent types
\item Homotopy type theory
\item Distance indexed types and differential privacy.
\item Languages with function spaces with more structure (Alex's
  Enriched Effect Calculus) (Idea for differential privacy)
\end{enumerate}

\end{document}
