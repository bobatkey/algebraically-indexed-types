\documentclass{article}

\usepackage[usenames]{color}
\definecolor{citationcolour}{rgb}{0,0.4,0.2}
\definecolor{linkcolour}{rgb}{0,0,0.8}
\usepackage{hyperref}
\hypersetup{colorlinks=true,
            urlcolor=linkcolour,
            linkcolor=linkcolour,
            citecolor=citationcolour,
            pdftitle=Geometric Invariance and Relational Parametricity,
            pdfauthor={Robert Atkey},
            pdfkeywords={}}
\def\sectionautorefname{Section}
\def\subsectionautorefname{Section}

\usepackage{amssymb}
\usepackage{stmaryrd}

\newcommand{\sepbar}{\mathrel|}

\newcommand{\GA}{\mathrm{GA}}
\newcommand{\GL}{\mathrm{GL}}
\newcommand{\Is}{\mathrm{Is}}
\newcommand{\Orth}{\mathrm{O}}
\newcommand{\SOrth}{\mathrm{SO}}
\newcommand{\Transl}{\mathrm{T}}
\newcommand{\Scale}{\mathrm{Scale}}

\newcommand{\sem}[1]{\llbracket #1 \rrbracket}
\newcommand{\isDefinedAs}{\stackrel{\mathit{def}}=}

\newcommand{\statementref}[1]{\hyperref[#1]{Statement \ref*{#1}}}

\newenvironment{example}{\vspace{0.1in}\noindent\textbf{Example}\quad}{}

\newcommand{\ignore}[1]{}

\title{Geometric Invariance and Relational Parametricity}
\author{Robert Atkey}

\begin{document}

\maketitle

\section{Coordinate-free Geometry and Invariance}

When writing programs that manipulate $n$-dimensional geometric data,
the basic data structure is the $n$-tuple of numbers. In the
two-dimensional case, tuples $\vec{v} = (x,y)$ can be called upon to
represent both points -- as offsets from some origin -- and vectors --
as offsets in their own right. Geometrically, the two notions of
\emph{point} and \emph{vector} are very different despite their common
representation. By representing a point concretely as an offset from
some privileged origin, we inadvertently fix a particular origin,
$(0,0)$. However, as made clear by Gallieo, many of the laws of
physics are invariant under change of the frame of reference: if we
change where we think the origin is, then we do not change the
fundamental laws. Putting this in concrete terms, if we have two
frames of reference, one with $(0,0)$ as the origin, and one with
$(10,20)$ as the origin, then the tuple $(1,1)$ with respect to the
first frame and $(11,21)$ with respect to the second frame represent
the \emph{same} point. It is merely an artifact of our representation
of points as pairs of numbers that these appear to be
different. Therefore, when we write programs that manipulate data that
represents points, then we ought to follow the general principle that
our programs are invariant with respect to changes in the choice of
origin.

Vectors are not intended to be regarded as invariant under change of
choice of origin. Vectors represent offsets, or movements, and the
vector $(0,0)$ always represents the zero offset. However, they are
intended to be invariant under change of basis: the vector $(2,3)$,
with respect to the basis $(1,0),(0,1)$, and the vector $(1,1)$, with
respect to the basis $(2,0),(0,3)$, should be regarded as the same
vector. Invariance under change of basis also applies to the
representation of points. We therefore refine the principle of
invariance under change of origin for points to also include invariant
under change of basis for points and vectors. A particular choice of
origin and basis is called a \emph{frame} (\cite{gallier11geometric},
Chapter 2). In general, the technique of maintaining geometric
invariants such as frame invariance has been referred to as
\emph{coordinate-free geometry}: we should aim to think in terms of
abstract concepts such as points and vectors, rather than their actual
representation as tuples of numbers.

Invariance under change of representation immediately recalls
Reynolds' famous story of two professors teaching the theory of
complex numbers to students \cite{reynolds83types}. One professor
chooses rectangular coordinates ($x + iy$), while the other chooses
polar coordinates ($\alpha\cos\theta + i\alpha\sin\theta$). Happily,
the two classes can interact because the theory of complex numbers is
invariant with respect to the choice of representation. Reynolds
formalises the idea of invariance with respect to changes of
representation by using the notion of preservation of relations. If a
binary relation $R$ relates two representations of a concept---for
example the rectangular and polar representations of the same complex
number are related---then a program that manipulates complex numbers
at a level of abstraction above their specific representation should
preserve the relation $R$. Concretely, let $f$ be a program taking
complex numbers to complex numbers that is intended to respect
invariance under change of representation. Then if $c$ is a complex
number in rectangular form, and $c'$ is a complex number in polar form
such that $R$ relates them, which we write as $(c,c') \in R$, then it
should be the case that $(f(c), f(c')) \in R$.

We now apply Reynolds' relational approach to defining invariance
under changes of representation to the geometric setting. Following
Reynolds, we express the fact that two tuples in $\mathbb{R}^2$ are
representatives of the same point or vector by defining binary
relations over $\mathbb{R}^2$. We wish to ensure that invariance with
respect to all choices of frame is respected, so we parameterise our
relations by the particular choice made. A neat way to represent a
choice of frame is as an invertible affine map from $\mathbb{R}^2$ to
$\mathbb{R}^2$. Affine maps are of the form $F(\vec{v})^\top =
A\vec{v}^\top + b$, where $A$ is an invertible real $2 \mathord\times
2$-matrix and $b$ is a vector in $\mathbb{R}^2$. The matrix $A$
represents the change of basis, and $b$ represents the change of
origin. The map $F$ tells us how to translate the representative of a
point in our chosen frame to the representative of the same point in
the ``standard frame'' with origin represented by $(0,0)$ and basis
$(1,0),(0,1)$. The set of all such maps forms the general affine group
$\GA(2)$.

We define a $\GA(2)$-parameterised family of binary relations $\{
R_{\texttt{pt}}(F) \subseteq \mathbb{R}^2 \times \mathbb{R}^2 \}_{F
  \in \GA(2)}$. We intend that two tuples $\vec{v_1} = (x_1,y_1)$ and
$\vec{v_2} = (x_2,y_2)$ are related for $F \in \GA(2)$, if, when the
former is interpreted with respect to the standard frame and the
latter is interpreted with respect to the frame described by $F$, the
two tuples represent the same point.
\begin{displaymath}
  R_{\texttt{pt}}(F) = \{ (\vec{v_1}, \vec{v_2}) \sepbar F\vec{v_2} = \vec{v_1} \}
\end{displaymath}

Now consider a function $f$ that, for example, takes two tuples in
$\mathbb{R}^2$ and returns a single tuple in $\mathbb{R}^2$. We intend
that all these tuples represent points in the same frame, but that the
function itself should be invariant with respect to the choice of
frame. We use Reynolds' idea of preservation of relations to formalise
this: for any $F \in \GA(2)$ (i.e.~any choice of frame), the function
$f$ should satisfy the following statement:
\begin{equation}\label{eq:f-preserve-rel-frame}
  \forall (\vec{v_1},\vec{v'_1}) \in R_{\texttt{pt}}(F), (\vec{v_2},\vec{v'_2}) \in R_{\texttt{pt}}(F).\ (f(\vec{v_1}, \vec{v_2}), f(\vec{v'_1}, \vec{v'_2})) \in R_{\texttt{pt}}(F).
\end{equation}
Unfolding the definition of $R_{\texttt{pt}}$, we can see that this
statement is equivalent to the following statement expressing
invariance of $f$ with respect to changes of frame: for all $F \in
\GA(2)$,
\begin{equation}\label{eq:f-invariant-frame}
  \forall \vec{v_1}, \vec{v_2}.\ f(F\vec{v_1},F\vec{v_2}) = F(f(\vec{v_1},\vec{v_2})).
\end{equation}
So Reynolds' idea of preservation of relations, when instantiated with
the family of relations $R_{\texttt{pt}}$, yields exactly the
geometric property of invariance under change of frame.

An example function $f$ that satisfies
\statementref{eq:f-invariant-frame} is the following function that
computes a particular affine combination of two points by working
directly on their coordinate representation:
\begin{displaymath}
  f(\vec{v_1}, \vec{v_2}) = \frac{1}{2}\vec{v_1} + \frac{1}{2}\vec{v_2}.
\end{displaymath}
In general, affine combinations of points $\lambda_1\vec{v_1} +
\lambda_2\vec{v_2}$, where $\lambda_1 + \lambda_2 = 1$, satisfy
\statementref{eq:f-invariant-frame}. Affine combination is one of the
fundamental building blocks of affine geometry -- the properties of
points invariant under invertible affine maps
(\cite{gallier11geometric}, Chapter 2). If we drop the condition that
$\lambda_1 + \lambda_2 = 1$, then we are dealing with linear
combinations of vectors, and we are no longer invariant with respect
to changes of frame. However, linear combinations are invariant with
respect to change of basis. We can represent changes of basis as
linear invertible maps $B : \mathbb{R}^2 \to \mathbb{R}^2$. The set of
all such maps forms the general linear group $\GL(2)$. We now define
another family of relations $\{R_{\texttt{vec}}(B) \subseteq
\mathbb{R}^2 \times \mathbb{R}^2 \}_{B \in \GL(2)}$ that relates two
points up to change of basis:
\begin{displaymath}
  R_{\texttt{vec}}(B) = \{ (\vec{v_1},\vec{v_2}) \sepbar B\vec{v_2} = \vec{v_1} \}.
\end{displaymath}
Now the functions $f_{\lambda_1\lambda_2}(\vec{v_1},\vec{v_2}) =
\lambda_1\vec{v_1} + \lambda_2\vec{v_2}$, for arbitrary $\lambda_1$
and $\lambda_2$, do preserve the relations $R_{\texttt{vec}}(B)$, for
all $B \in \GL(2)$. Unfolding the definition of $R_{\texttt{vec}}(B)$
in the analogous statement to \statementref{eq:f-preserve-rel-frame}
for $R_{\texttt{vec}}$ instead of $R_{\texttt{pt}}$, we can see that
preservation of the relations $R_{\texttt{vec}}$ characterises the
functions that are invariant under change of basis. For all $B \in \GL(2)$, we have
\begin{displaymath}
  \forall \vec{v_1}, \vec{v_2}.\ f(B\vec{v_1},B\vec{v_2}) = B(f(\vec{v_1},\vec{v_2})),
\end{displaymath}
and this is exactly the property of invariance under change of basis
we required above of programs manipulating vectors.

Note that the family of relations $R_{\texttt{vec}}$ is just
$R_{\texttt{pt}}$ when restricted to elements of the group
$\GL(2)$. In the type system we introduce in the next section, we
combine points and vectors into the same data type. Whether it
represents a point or a vector depends on the group of geometric
transformations that we expect it to be invariant under.

\section{A Type System for Geometric Invariance}
\label{sec:type-system-geom-intro}

In the last section, we saw how Reynolds' notion of preservation of
relations can be used to define when it is the case that programs
manipulating representatives of points and vectors make geometric
sense. Reynolds further insight was to see how preservation of
relations can automatically be established by following a typing
discipline. In Reynolds' case, the typing discipline was the
second-order polymorphic $\lambda$-calculus, also known as System F,
where new types can be constructed by universal quantification over
all types. Universal type quantification is commonly written $\forall
\alpha. A$, which denotes a type that may be instantiated with any
other type $B$ by replacing the type variable $\alpha$ with $B$ in
$A$.

In terms of relations, Reynolds interprets universal quantification
over types as quantification over binary relations between denotations
of types. In the statements of geometric invariance that we stated in
the previous section (e.g., \statementref{eq:f-invariant-frame}), we
did not quantify over all relations. Instead, we quantified over all
elements of a geometric group such as $\GA(2)$ or $\GL(2)$, and used
those elements to select relations from the families $R_{\texttt{pt}}$
and $R_{\texttt{vec}}$. Driven by this observation, we introduce
quantification over elements of geometric groups into the language of
types. We use the notation $\forall X \mathord: G. A$ for
quantification over elements of geometric groups, where $G$ is the
name of some geometric group, and $X$ is a variable name standing for
an element of $G$. We refer to $G$ as the \emph{sort} of $X$. As we
proceed, we will introduce the sorts representing geometric groups
that we will be using in addition to $\GA(2)$ and $\GL(2)$.

The sorts $\GA(2)$ and $\GL(2)$ represent groups, so we can combine
their elements using the usual group operations. We write operations
multiplicatively, using juxtaposition ($e_1e_2$) for multiplication,
$e^{-1}$ for inverse and $1$ for the unit. We overload the same
notation to stand for the group operations in both $\GA(2)$ and
$\GL(2)$. We also regard expressions built from variables and the
group operations up to the group axioms. So, for example,
$F_1(F_2F_3)$ and $(F_1F_2)F_3$ are to be regarded as equivalent.

Our language of types also includes the common type constructors for
function types, $A \to B$, sum types $A + B$, unit type
$\mathsf{Unit}$ and tuple types $A \times B$. We also assume a
primitive type $\mathsf{real}$, used to represent scalars. As we noted
at the start of the previous section, the central data structure in
geometric applications is the tuple of real numbers for representing
points and vectors. However, we cannot simply express this as the type
$\mathsf{real} \times \mathsf{real}$ because it will not have the
correct relational interpretation. (Two elements of type
$\mathsf{real}$ will be related if and only if they are equal, and
hence two elements of $\mathsf{real} \times \mathsf{real}$ will be
related if and only if they are equal, by Reynolds' definition of the
relational interpretation of tuple types.) Therefore, we introduce a
new type parameterised by expressions $e$ of sort $\GA(2)$ to
represent points (and later, vectors): $\mathsf{vec}\langle e
\rangle$. We have used the name $\mathsf{vec}$, even though we have
taken pains to separate geometric points and vectors, to recall the
computer science notion of vector as a sequence of values of
homogeneous type which has a known length (in this case two).

% FIXME: Denotational semantics of $\mathsf{vec}$? and transformation
% quantification?

Equipped with quantification over the group $\GA(2)$ of invertible
affine maps, and the parameterised types $\mathsf{vec}\langle \cdot
\rangle$, we can now express geometric invariance properties as
types. In the previous section, we gave examples of functions $f :
\mathbb{R}^2 \times \mathbb{R}^2 \to \mathbb{R}^2$ that have the
property that they preserve all invertible affine maps. We can now
express this property as a type:
\begin{displaymath}
  \mathrm{f} : \forall F \mathord: \GA(2).\ \mathsf{vec}\langle F \rangle \times \mathsf{vec}\langle F \rangle \to \mathsf{vec}\langle F \rangle
\end{displaymath}
We use the variable name $F$ for elements of $\GA(2)$ to suggest the
intuition that this typing indicates that $\mathrm{f}$ takes a pair of
points in the $F$rame $F$ and returns a point in the same frame.

\subsection{Affine Combination of Points}
\label{sec:affine-combination}

As we stated in the previous section, the primitive operation of
affine geometry is affine combination of points: $\lambda_1\vec{v_1} +
\lambda_2\vec{v_2}$, where $\lambda_1 + \lambda_2 = 1$. Geometrically,
affine combination can be interpreted as interpolation between the
points represented by $\vec{v_1}$ and $\vec{v_2}$. We add affine
combination of points to our calculus, with the following typing and
intended denotation.
\begin{displaymath}
  \begin{array}{l}
    \mathrm{affineCombination} : \forall F \mathord: \GA(2).\ \mathsf{vec}\langle F \rangle \to \mathsf{real} \to \mathsf{vec}\langle F \rangle \to \mathsf{vec}\langle F \rangle \\
    \sem{\mathrm{affineCombination}}\ (x_1,y_1)\ t\ (x_2,y_2) = ((1-t)x_1 + tx_2, (1-t)y_1 + ty_2)
  \end{array}
\end{displaymath}
By defining the primitive function $\mathrm{affineCombination}$ to
take a single $\mathsf{real}$ parameter $t$, we can easily ensure that
we are taking the affine combination of two representatives of points,
and not just an arbitrary linear combination. Note that the intended
denotation in the second line does not take any argument corresponding
to the universally quantified $F$ in the type. In our calculus,
geometric group expressions are erased at runtime.

% In Linear Algebra notation, we could have written the right-hand side
% as $(1-t)\vec{v_1} + t\vec{v_2}$, where $\vec{v_1} = (x_1,y_1)$ and
% $\vec{v_2} = (x_2,y_2)$. 

\begin{example}
  With just the single primitive operation of affine combination, we
  can already express some useful programs. For example, the
  evaluation of quadratic B\'{e}zier curves (B\'{e}zier curves with
  two endpoints and a single control point) can be expressed using the
  affine combination primitive and three steps of De Casteljau's
  algorithm:
  \begin{displaymath}
    \begin{array}{l}
      \mathrm{quadraticB\acute{e}zier} : \forall F \mathord:\GA(2).\ \mathsf{vec}\langle F \rangle \to \mathsf{vec}\langle F \rangle \to \mathsf{vec}\langle F \rangle \to \mathsf{real} \to \mathsf{vec}\langle F \rangle \\
      \mathrm{quadraticB\acute{e}zier}\ [F]\ p_0\ p_1\ p_2\ t = \\
      \quad
      \begin{array}{l@{\hspace{-0.002cm}}l}
        \mathrm{affineCombination}\ [F]\ & (\mathrm{affineCombination}\ [F]\ p_0\ t\ p_1) \\
        & t \\
        & (\mathrm{affineCombination}\ [F]\ p_1\ t\ p_2)
      \end{array}
    \end{array}
  \end{displaymath}
  For two endpoints $p_0$ and $p_2$ and a control point $p_1$,
  $\mathrm{quadraticB\acute{e}zier}\ p_0\ p_1\ p_2\ t$ for $t \in
  [0,1]$ gives us the point on the curve at ``time'' $t$. Note that
  the type of $\mathrm{quadraticB\acute{e}zier}$ tells us immediately
  that it preserves all invertible affine maps.
  %FIXME: forward ref to ``free theorem''.
\end{example}

\subsection{Vector Space Operations}
\label{sec:vector-space-ops}

Affine combination only provides us with an operation on points. For
the purposes of computational geometry, we are also interested in
operations on the vectors representing offsets between points. We now
examine the correct types to assign to the vector space operations of
addition of vectors, negation of vectors, multiplication by a scalar
and the zero vector.

In our discussion in the previous section, we observed that vector
operations are those that are preserved by change-of-basis operations
-- i.e.~under the action of elements of the group $\GL(2)$ of
invertible linear maps. Since $\GL(2)$ is a subgroup of $\GA(2)$ we
may form the type $\mathsf{vec}\langle e \rangle$ when $e$ is of sort
$\GL(2)$ as well as $\GA(2)$. The obvious type assignment for vector
addition is now the following:
\begin{displaymath}
  (+) : \forall B\mathord:\GL(2).\ \mathsf{vec}\langle B \rangle \to \mathsf{vec}\langle B \rangle \to \mathsf{vec}\langle B \rangle
\end{displaymath}
(we write binary operators intended to be used infix in parentheses
when not appearing in infix position, following the Haskell
syntax.) The use of $B$ for names of elements of $\GL(2)$ is meant to
suggest that the type $\mathsf{vec}\langle B \rangle$ denotes vectors
with respect to the $B$asis $B$.

However, it is possible to give vector addition a more precise type
that describes its effect on the chosen origin as well as its
invariance under change of basis. We now introduce a new sort
$\Transl(2)$ of translations. Translations form an abelian group,
which we write additively, using $+$,$-$ and $0$, and we use lower
case $t$, $t_1$, etc.~for $t$ranslation variables. Invertible affine
maps are constructed from invertible linear maps and translations, so
given a $B : \GL(2)$ and a $t : \Transl(2)$, we can form an invertible
affine map $B \ltimes t : \GA(2)$. Conversely, given $F : \GA(2)$, we
have an invertible linear map $\pi_L(F) : \GL(2)$, and a translation
$\pi_T(F) : \Transl(2)$. Thinking of elements of $\GA(2)$ as frames,
$\pi_L$ returns the basis of the frame, and $\pi_T$ returns the
origin.

We can now give vector addition a more general type, which
demonstrates how adding vectors has an effect on translations:
\begin{displaymath}
  (+) : \forall B \mathord: \GL(2), t_1, t_2 \mathord: T(2).\ \mathsf{vec}\langle B \ltimes t_1\rangle \to \mathsf{vec}\langle B \ltimes t_2 \rangle \to \mathsf{vec}\langle B \ltimes (t_1 + t_2)\rangle
\end{displaymath}
We can also negate vectors, yielding a vector which points in the
opposite direction. Negation is invariant under invertible linear maps
but negates translations:
\begin{displaymath}
  \mathrm{negate} : \forall B \mathord: \GL(2), t \mathord: T(2).\ \mathsf{vec}\langle B \ltimes t\rangle \to \mathsf{vec}\langle B\ltimes(-t)\rangle
\end{displaymath}
With addition and negation of vectors, we can define the derived
operation of vector subtraction with the following type:
\begin{displaymath}
  \begin{array}{l}
    (-) : \forall B \mathord:\GL(2), t_1,t_2 \mathord:T(2).\ \mathsf{vec}\langle B \ltimes t_1\rangle \to \mathsf{vec}\langle B \ltimes t_2\rangle \to \mathsf{vec}\langle B \ltimes (t_1 - t_2)\rangle \\
    (-)\ [B]\ [t_1]\ [t_2]\ p_1\ p_2 = p_1 + \mathrm{negate}\ p_2
\end{array}
\end{displaymath}

Given two points in the same frame---i.e.~two values of type
$\mathsf{vec}\langle F \rangle$---we can compute their offset, which
is a vector expressed with respect to the basis part of $F$, as a
special case of vector subtraction:
\begin{displaymath}
  \begin{array}{l}
    \mathrm{offset} : \forall F \mathord:\GA(2).\ \mathsf{vec}\langle F \rangle \to \mathsf{vec}\langle F \rangle \to \mathsf{vec}\langle \pi_L(F) \ltimes 0 \rangle \\
    \mathrm{offset}\ [F]\ p_1\ p_2 = p_1 - p_2
  \end{array}
\end{displaymath}
The subtraction operation in the body of $\mathrm{offset}$ is
instantiated with $B = \pi_L(F)$, $t_1 = \pi_T(F)$ and $t_2 =
\pi_T(F)$. The translation $\pi_T(F)$ cancels itself out, and we are
left with a vector in the basis $\pi_L(F)$ of the frame $F$.

Similarly, the type of the vector addition operation can be
specialised to the case of moving a point by an offset represented by
a vector:
\begin{displaymath}
  \begin{array}{l}
    \mathrm{moveBy} : \forall F \mathord:\GA(2).\ \mathsf{vec}\langle F \rangle \to \mathsf{vec}\langle \pi_L(F) \ltimes 0 \rangle \to \mathsf{vec}\langle F \rangle \\
    \mathrm{moveBy}\ [F]\ p\ v = p + v
  \end{array}
\end{displaymath}
The addition operation in the body of $\mathrm{moveBy}$ is
instantiated with $B = \pi_L(F)$, $t_1 = \pi_T(F)$ and $t_2 =
0$. Since $t_1 + t_2 = \pi_T(F) + 0 = \pi_T(F)$ in this case, we are
left with a value of type $\mathsf{vec}\langle \pi_L(F) \ltimes
\pi_T(F) \rangle$, which is equivalent to the type
$\mathsf{vec}\langle F \rangle$.

The types we assign to the remaining vector space primitives, the zero
vector and multiplication by a scalar, do not describe any interesting
effect on translations%\footnote{but they could...}
. The zero vector has the following polymorphic type, indicating that
for any basis $B$, there is a zero vector.
\begin{displaymath}
  \mathrm{0} : \forall B \mathord: \GL(2).\ \mathsf{vec}\langle B \ltimes 0 \rangle
\end{displaymath}
Intuitively, there cannot be a zero vector for any frame $F$, because
this would entail picking a privileged origin. For multiplication of a
vector by a scalar, we assign the following type:
\begin{displaymath}
  (*) : \forall B \mathord: \GL(2).\ \mathsf{real} \to \mathsf{vec}\langle B \ltimes 0 \rangle \to \mathsf{vec}\langle B \ltimes 0 \rangle
\end{displaymath}
This type indicates that scaling a vector is preserved by change of
basis.

\begin{example}
  The vector space primitives we have defined in this section, along
  with the geometric invariance properties that follow from their
  types, allow us to establish a useful type isomorphism. Consider
  functions with types conforming to the following schema:
  \begin{displaymath}
    \tau_{n,A} \isDefinedAs \forall F\mathord:\GA(2).\ \mathsf{vec}\langle F \rangle^{n+1} \to A[F]
  \end{displaymath}
  We take $n$ to be a natural number, and $A^n$ for some type $A$
  denotes the $n$-ary tupling of $A$ with itself: i.e.~$A \times
  ... \times A$, with $A$ repeated $n$ times. The notation $A[-]$
  denotes a type with a ``hole'' that can be filled with an
  expression. In this case, we require that $A[e]$ be a well-formed
  type when $e$ has sort $\GA(2)$.

  Just by looking at the types $\tau_{n,A}$, we know that their
  inhabitants will be invariant under change of frame, due to the
  quantification over all $F$ in $\GA(2)$. Therefore, we may as well
  pick one of the input points as the origin and assume that all the
  other points are defined with respect to this chosen origin. This is
  similar to the common mathematical step of stating that, ``without
  loss of generality'', we may pick some point in a description of a
  problem to be the origin, as long as the problem statement is
  invariant under translation.

  The types $\tau_{n,A}$ are isomorphic to the corresponding types
  $\sigma_{n,A}$, defined below. Inhabitants of the types
  $\sigma_{n,A}$ take one fewer points (now seen as vector offsets
  from some origin), but are invariant only under change of basis.
  \begin{displaymath}
    \sigma_{n,A} \isDefinedAs \forall B\mathord:\GL(2).\ \mathsf{vec}\langle B \rangle^n \to A[B \ltimes 0]
  \end{displaymath}
  The isomorphism between these two families of types can be witnessed
  by the following two functions.
  \begin{displaymath}
    \begin{array}{l}
      i_{n,A} : \tau_{n,A} \to \sigma_{n,A} \\
      i_{n,A}\ f\ [B]\ (p_1, p_2, ..., p_n) = f\ [B \ltimes 0]\ (0, p_1, ..., p_n) \\
      \\
      i^{-1}_{n,A} : \sigma_{n,A} \to \tau_{n,A} \\
      i^{-1}_{n,A}\ g\ [F]\ (p_0, ..., p_n) = g\ [\pi_L(F)]\ (p_1-p_0, ..., p_n-p_0)
    \end{array}
  \end{displaymath}
  The direction defined by the function $i_{n,A}$ treats the inputs
  $p_1,...,p_n$ as vector offsets from the origin $0$. The direction
  witnessed by $i^{-1}_{n,A}$ picks the first point to act as the
  origin, and uses the operation of vector subtraction we defined
  above to turn each of the other points into offsets from this
  point. Note that this isomorphism is not unique: for each $n$ we can
  pick any of the $n+1$ inputs to act as the distinguished
  origin. Thus the families of types $\tau_{n,A}$ and $\sigma_{n,A}$
  are isomorphic, but not uniquely isomorphic.
\end{example}

\subsection{Euclidean Geometry}
\label{sec:euclidean-geom}

\subsubsection{Dot Product}

Euclidean geometry extends affine geometry with the operation of
\emph{dot product} (also known as the \emph{inner product}) of two
vectors, which produces a real number. The standard Euclidean dot
product is defined on the coordinate representation in the following
way:
\begin{displaymath}
  (x_1,y_1) \cdot (x_2,y_2) = x_1x_2 + y_1y_2
\end{displaymath}
The dot product is used to define properties of vectors such as their
\emph{norm} $||\vec{v}|| = \vec{v}\cdot\vec{v}$, which is the square
of the length of the vector $\vec{v}$, and the notion of
orthogonality: two vectors $\vec{v_1}$ and $\vec{v_2}$ are orthogonal
if $\vec{v_1}\cdot\vec{v_2} = 0$.

To incorporate the operation of dot product into our calculus, we must
give it a type. It is not the case that the dot product is invariant
under the action of any of the groups $\GA(2)$, $\GL(2)$ or
$\Transl(2)$ that we have considered so far. In fact, the dot product
is invariant under the subgroup $\Orth(2)$ of $\GL(2)$ of
\emph{orthogonal} linear transformations. The group $\Orth(2)$
consists of invertible linear maps whose determinant is either $1$ or
$-1$. We introduce orthogonal transformations as a new sort
$\Orth(2)$, and overload the multiplicative group operations syntax
for inhabitants of $\Orth(2)$. We also assume a function $\iota_O$
that takes $e : \Orth(2)$ to $\iota_O(e) : \GL(2)$.

We may now assign the following type to the dot product primitive:
\begin{displaymath}
  (\cdot) : \forall O \mathord: \Orth(2).\ \mathsf{vec}\langle \iota_O(O) \ltimes 0 \rangle \to \mathsf{vec}\langle \iota_O(O) \ltimes 0 \rangle \to \mathsf{real}
\end{displaymath}
With dot product as a primitive operation, we can define several other
useful derived operations. First, the norm of a vector, which in the
case of standard Euclidean geometry is the square of the length of the
vector. The norm is invariant under orthogonal changes of basis, but
not arbitrary changes of basis. Hence, we quantify over all orthogonal
transformations in the type of $\mathrm{norm}$:
\begin{displaymath}
  \begin{array}{l}
    \mathrm{norm} : \forall O \mathord: \Orth(2).\ \mathsf{vec}\langle O \rangle \to \mathsf{real} \\
    \mathrm{norm}\ [O]\ v = v \cdot v
  \end{array}
\end{displaymath}
Given the norm function, we can compute the distance between a pair of
points in a frame, provided that the frame has an orthogonal basis
described by some orthogonal transformation $O$:
\begin{displaymath}
  \begin{array}{l}
    \mathrm{distance} : \forall O \mathord: \Orth(2), t \mathord: \Transl(2).\ \mathsf{vec}\langle O \ltimes t \rangle \to \mathsf{vec}\langle O \ltimes t \rangle \to \mathsf{real} \\
    \mathrm{distance}\ [O]\ [t]\ p_1\ p_2 = \mathrm{sqrt}\ (\mathrm{norm}\ (\mathrm{offset}\ p_1\ p_2))
  \end{array}
\end{displaymath}
In the definition of $\mathrm{distance}$, we have made use of a square
root function on the reals $\mathrm{sqrt} : \mathsf{real} \to
\mathsf{real}$.

The dot product is also used to determine whether two vectors are
orthogonal to one another. In the case of the standard dot product in
$2$-dimensional space, two vectors are orthogonal if they are at right
angles to one another.
\begin{displaymath}
  \begin{array}{l}
    \mathrm{orthogonal} : \forall O \mathord: \Orth(2).\ \mathsf{vec}\langle O \rangle \to \mathsf{vec}\langle O \rangle \to \mathsf{bool} \\
    \mathrm{orthogonal}\ [O]\ v_1\ v_2 = v_1 \cdot v_2 \equiv 0
  \end{array}
\end{displaymath}

\subsubsection{Cross Product}

\begin{figure}[t]
  \centering
  \begin{eqnarray*}
    0   &:& \forall S \mathord:\Scale.\ \mathsf{real}\langle S \rangle \\
    1   &:& \mathsf{real}\langle 1 \rangle \\
    (+) &:& \forall S \mathord:\Scale.\ \mathsf{real}\langle S \rangle \to \mathsf{real}\langle S \rangle \to \mathsf{real}\langle S \rangle \\
    (-) &:& \forall S \mathord:\Scale.\ \mathsf{real}\langle S \rangle \to \mathsf{real}\langle S \rangle \to \mathsf{real}\langle S \rangle \\
    (*) &:& \forall S_1,S_2 \mathord:\Scale.\ \mathsf{real}\langle S_1 \rangle \to \mathsf{real}\langle S_2 \rangle \to \mathsf{real}\langle S_1S_2 \rangle \\
    (/) &:& \forall S_1,S_2 \mathord:\Scale.\ \mathsf{real}\langle S_1 \rangle \to \mathsf{real}\langle S_2 \rangle \to \mathsf{real}\langle S_1S_2^{-1} \rangle + 1 \\
    \mathrm{abs} &:& \forall S \mathord:\Scale.\ \mathsf{real}\langle S \rangle \to \mathsf{real}\langle |S| \rangle \\
    \mathrm{sqrt} &:& \mathsf{real}\langle 1 \rangle \to \mathsf{real}\langle 1 \rangle
  \end{eqnarray*}
  \caption{Operations on scaled real numbers}
  \label{fig:real-ops}
\end{figure}

The cross product is another operation that takes pairs of vectors and
produces a real number. Algebraically, the cross product is defined on
coordinate representations as follows:
\begin{displaymath}
  (x_1,y_1) \times (x_2,y_2) = x_1y_2 - x_2y_1
\end{displaymath}
Geometrically, the cross product of two vectors is the signed area of
the parallelogram described by the pair of input vectors. It turns out
that the value of the cross product of two vectors varies with the
determinant of the matrix describing the basis of the vector space. We
therefore augment our calculus with a new sort $\Scale$ of scale
factors. Semantically, $\Scale$ ranges over the non-zero real numbers
and forms an abelian group, which we write multiplicatively. We also
add two new operations: determinant, $\det B$, which takes an
inhabitant of $\GL(2)$ to its determinant in $\Scale$, and the
operation of absolute value that takes scaling factors to scaling
factors which we write as $|e|$. We also refine the type
$\mathsf{real}$ of real numbers to take a parameter of sort $\Scale$:
$\mathsf{real}\langle e \rangle$, where the old type $\mathsf{real}$
is just $\mathsf{real}\langle 1 \rangle$. The collection of operations
on real numbers with scaling factors is shown in
\autoref{fig:real-ops}.

With the additional sort $\Scale$, and the refined type of real
numbers, we can state the type of the cross product.
\begin{displaymath}
  (\times) : \forall B \mathord: \GL(2).\ \mathsf{vec}\langle B \rangle \to \mathsf{vec}\langle B \rangle \to \mathsf{real}\langle \det B \rangle
\end{displaymath}

For $O : \Orth(2)$, the absolute value of the determinant is always
$1$, so we assume the equation $|\det O| = 1$ to hold.

\begin{example}
  Using the structure we have defined in this subsection, we can
  define the following function that computes the area of a
  triangle. By the type of this function, we can see that the area of
  a triangle is invariant under orthognal transformations and
  arbitrary translations.
  \begin{displaymath}
    \begin{array}{l}
      \mathrm{area} : \forall O\mathord:\Orth(2), t\mathord:\Transl(2).\ \mathsf{vec}\langle O \ltimes t \rangle \to \mathsf{vec}\langle O \ltimes t \rangle \to \mathsf{vec}\langle O \ltimes t \rangle \to \mathsf{real}\langle 1 \rangle \\
      \mathrm{area}\ [O]\ [t]\ p_1\ p_2\ p_3 = \frac{1}{2} * \mathrm{abs}\ ((p_2 - p_1) \times (p_3 - p_1))
    \end{array}
  \end{displaymath}
  In the body of this function, we have performed the calculation in
  several steps, each of which removes some of the symmetry described
  by the types. First we have computed the two offset vectors $p_2 -
  p_1$ and $p_3 - p_1$. These operations remove the effect of
  translations on the result, in exactly the same way as the type
  isomorphism we demonstrated above. We then compute the cross product
  of the two vectors to determine the area of the parallelogram
  described by the sides of the triangle. This value has type
  $\mathsf{real}\langle \det O \rangle$. We know that the determinant
  of an orthogonal transformation is always either $-1$ or $1$, so we
  have a result which varies according to possible reflections of the
  space. To remove this variation, we apply the $\mathrm{abs}$
  function to leave us with a value of type $\mathsf{real}\langle
  |\det O| \rangle$. This type is equal to the type
  $\mathsf{real}\langle 1 \rangle$. Finally, we multiply by
  $\frac{1}{2}$ to recover the area of the triangle, rather than the
  whole parallelogram.
\end{example}

% \subsection{Transformations Within and Between Frames}
% \label{sec:runtime-rep-frames}

% \begin{enumerate}
% \item A translation with respect to some frame is simply a vector with
%   respect to that frame's basis. We can apply it to points just by 
% \item 
% \end{enumerate}

% Singleton types

% Representing frames, a type representing translations:
% \begin{displaymath}
%   \mathsf{translation}\langle T \rangle
% \end{displaymath}
% And representing bases, as linear maps
% \begin{displaymath}
%   \mathsf{invLinearMap}\langle B \rangle
% \end{displaymath}

%%%%%%%%%%%%%%%%%%%%%%%%%%%%%%%%%%%%%%%%%%%%%%%%%%%%%%%%%%%%%%%%%%%%%%%%%%%%%%%%
%%%%%%%%%%%%%%%%%%%%%%%%%%%%%%%%%%%%%%%%%%%%%%%%%%%%%%%%%%%%%%%%%%%%%%%%%%%%%%%%
%%%%%%%%%%%%%%%%%%%%%%%%%%%%%%%%%%%%%%%%%%%%%%%%%%%%%%%%%%%%%%%%%%%%%%%%%%%%%%%%
% FIXME: write down all the groups, and the functions between them

%%%%%%%%%%%%%%%%%%%%%%%%%%%%%%%%%%%%%%%%%%%%%%%%%%%%%%%%%%%%%%%%%%%%%%%%%%%%%%%%
%%%%%%%%%%%%%%%%%%%%%%%%%%%%%%%%%%%%%%%%%%%%%%%%%%%%%%%%%%%%%%%%%%%%%%%%%%%%%%%%
%%%%%%%%%%%%%%%%%%%%%%%%%%%%%%%%%%%%%%%%%%%%%%%%%%%%%%%%%%%%%%%%%%%%%%%%%%%%%%%%
\bibliographystyle{plain}
\bibliography{paper}


\end{document}
